%%%%%%%%%%%%%%%%%%%%%%%%%%%%%%%%%%%%%%%%%%%%%%%%%%%%%%%%%%%%%%%%%%%%%%%%%%%

\documentclass[a4paper,oneside,12pt]{article}
\usepackage{mystyle}

\begin{document}

\title{\Large\bf Polar coordinate system}
\author{%%
  Minh Van Nguyen \\
  \url{mvngu@gmx.com}
}
\date{\today}
\maketitle


%%%%%%%%%%%%%%%%%%%%%%%%%%%%%%%%%%%%%%%%%%%%%%%%%%%%%%%%%%%%%%%%%%%%%%%%%%%

\section{Degrees and radians}

You already know that a pair $\tuple{a}{b}$ of real numbers can be
represented as a point in the Cartesian coordinate system.  The pair
$\tuple{a}{b}$ can also be represented as a point in another
coordinate system called the \emph{polar coordinate system}.  Before
discussing the polar coordinate system, you need to know about
\emph{radians}.

One way to measure angles is by using degrees.  Another way to measure
angles is to use radians, which are defined as follows.  A
\emph{unit circle} is a circle whose radius is $r = 1$ so the
circumference of a unit circle is
%%
\begin{align*}
2 \pi r
&=
2 \pi \times 1 \\[4pt]
&=
2 \pi.
\end{align*}
%%
The value of $2 \pi$ is the distance around the unit circle.  The
value of $2 \pi$ is also used as the angle around any circle and you
say that any circle has $2 \pi$ radians just as you would say that any
circle has $360$ degrees.  You use the symbol $\radian{}$ for radians
so that $\radian{2\pi}$ means $2\pi$ radians.  Since any circle has
$360$ degrees~(or $\degree{360}$), then
$\degree{360} = \radian{2\pi}$.  Divide both sides of the last
equation by $2$ and you get $\degree{180} = \radian{\pi}$, which means
that $\degree{180}$ is equivalent to $\pi$ radians.

\begin{exercise}
Explain how you would define one radian.
\end{exercise}

\ifbool{showSolution}{
\begin{solution}
Since $\degree{180} = \radian{\pi}$, divide both sides by $\pi$ to get
$\degree{180} / \pi = \radian{1}$.  Therefore one radian is
approximately $57.2958$ degrees, rounded to four decimal places.
\end{solution}
}{}

\begin{example}
\label{ex:90_degrees_to_radians}
Convert $\degree{90}$ to radians.
\end{example}

\begin{solution}
Start with the expression $\degree{180} = \radian{\pi}$.  Divide both
sides by $2$ to get $\degree{90} = \frac{\radian{\pi}}{2}$.  Therefore
$\degree{90}$ is equivalent to $\pi / 2$ radians.
\end{solution}

\begin{example}
\label{ex:convert_270_degrees_to_radians}
Convert $\degree{270}$ to radians.
\end{example}

\begin{solution}
You know that $\degree{180} + \degree{90} = \degree{270}$.  Since
$\degree{180} = \radian{\pi}$ and
$\degree{90} = \frac{\radian{\pi}}{2}$, then you can write
%%
\begin{align*}
\degree{270}
&=
\radian{\pi} + \frac{\radian{\pi}}{2} \\[4pt]
&=
\frac{\radian{2\pi}}{2} + \frac{\radian{\pi}}{2} \\[4pt]
&=
\frac{\radian{3\pi}}{2}.
\end{align*}
%%
Therefore $\degree{270}$ is equivalent to $\frac{3\pi}{2}$ radians.
\end{solution}

\begin{exercise}
Convert $\degree{45}$ to radians.
\end{exercise}

\ifbool{showSolution}{
\begin{solution}
You know that $\degree{90} / 2 = \degree{45}$ and
$\degree{90} = \frac{\radian{\pi}}{2}$.  Then
%%
\begin{align*}
\degree{45}
&=
\frac{\radian{\pi}}{2} \times \frac{1}{2} \\[4pt]
&=
\frac{\radian{\pi}}{4}.
\end{align*}
%%
In other words, $\degree{45}$ is equivalent to $\pi / 4$ radians.
\end{solution}
}{}

\begin{exercise}
Convert $\degree{135}$ to radians.
\end{exercise}

\ifbool{showSolution}{
\begin{solution}
You know that $\degree{135} = \degree{90} + \degree{45}$.  Since
$\degree{90} = \frac{\radian{\pi}}{2}$ and
$\degree{45} = \frac{\radian{\pi}}{4}$, it follows that
%%
\begin{align*}
\degree{135}
&=
\frac{\radian{\pi}}{2} + \frac{\radian{\pi}}{4} \\[4pt]
&=
\frac{\radian{2\pi}}{4} + \frac{\radian{\pi}}{4} \\[4pt]
&=
\frac{\radian{3\pi}}{4}.
\end{align*}
%%
In other words, $\degree{135}$ is equivalent to $\frac{3\pi}{4}$
radians.
\end{solution}
}{}

\begin{exercise}
Convert $\pi / 6$ radians to degrees.
\end{exercise}

\ifbool{showSolution}{
\begin{solution}
You have $\degree{180} = \radian{\pi}$.  Divide both sides of the last
equation by $6$ to get
\[
\frac{\degree{180}}{6}
=
\frac{\radian{\pi}}{6}
\]
which simplifies to $\degree{30} = \frac{\radian{\pi}}{6}$.  That is,
$\frac{\pi}{6}$ radians is equivalent to $\degree{30}$.
\end{solution}
}{}

\begin{exercise}
Convert $\pi / 3$ radians to degrees.
\end{exercise}

\ifbool{showSolution}{
\begin{solution}
Start with the expression $\degree{180} = \radian{\pi}$.  Divide both
sides by $3$ to get
\[
\frac{\degree{180}}{3}
=
\frac{\radian{\pi}}{3}
\]
which simplifies to $\degree{60} = \frac{\radian{\pi}}{3}$.  In other
words, $\pi / 3$ radians is equivalent to $\degree{60}$.
\end{solution}
}{}


%%%%%%%%%%%%%%%%%%%%%%%%%%%%%%%%%%%%%%%%%%%%%%%%%%%%%%%%%%%%%%%%%%%%%%%%%%%

\section{Sine and cosine}

You will use the \emph{sine} and \emph{cosine} functions a lot so it
is important that you become familiar with these functions.  Let
$\varphi$ be an angle in radians.  The functions
$\sin\varphi$~(pronounced ``sine of $\varphi$'') and
$\cos\varphi$~(pronounced ``cosine of $\varphi$'') are two common
trigonometric functions.  To define the values of $\sin\varphi$ and
$\cos\varphi$, consider \Figure{fig:right_angled_triangle_angle_phi}.
The value of $\sin\varphi$ is defined as the side opposite to
$\varphi$~(i.e.~$b$) over the hypotenuse $c$:
%%
\begin{equation}
\label{eqn:define_sine}
\sin\varphi
=
\frac{b}{c}.
\end{equation}
%%
The value of $\cos\varphi$ is defined as the side adjacent to
$\varphi$~(i.e.~$a$) over the hypotenuse:
%%
\begin{equation}
\label{eqn:define_cosine}
\cos\varphi
=
\frac{a}{c}.
\end{equation}
%%
\Figure{fig:cosine_sine} shows the plots of $\sin\varphi$ and
$\cos\varphi$ for angles from $\varphi = -2\pi$ radians to
$\varphi = 2\pi$ radians.  Whenever you see the functions
$\sin\varphi$ or $\cos\varphi$, you may assume that the angle
$\varphi$ is in radians and not degrees.  For the moment, do not worry
about how to calculate the values of $\sin\varphi$ and $\cos\varphi$
for any given value of $\varphi$.  You will only consider special
values of $\varphi$ that allow you to calculate exact values of
$\sin\varphi$ and $\cos\varphi$.

\begin{figure}[!htbp]
\centering
\includegraphics[scale=1]{image/04/right-triangle.pdf}
\caption{%%
  A right-angled triangle whose base and height are $a$ and $b$,
  respectively.  The hypotenuse is $c$.  The angle between $a$ and $c$
  is $\varphi$ radians.
}
\label{fig:right_angled_triangle_angle_phi}
\end{figure}

\begin{figure}[!htbp]
\centering
\includegraphics[scale=1]{image/04/cos-sin.pdf}
\caption{%%
  Plots of the functions $y = \cos x$~(the blue line) and
  $y = \sin x$~(the red line).  The value of $x$ is in radians.  The
  graphs of the sine and cosine functions look like waves.  For now,
  do not worry about how to calculate the values of $\cos x$ and
  $\sin x$ for any given value of $x$.
}
\label{fig:cosine_sine}
\end{figure}

\begin{figure}[!htbp]
\centering
\subfigure[]{
  \includegraphics[scale=1]{image/04/right-triangle-30-degree.pdf}
  \label{fig:right_triangle_30_degrees}
}
%%
\qquad\qquad
%%
\subfigure[]{
  \includegraphics[scale=1]{image/04/right-triangle-45-degree.pdf}
  \label{fig:right_triangle_45_degrees}
}
\caption{%%
  (a)~A right-angled triangle with an internal angle of $\pi / 6$
  radians or $\degree{30}$.  (b)~A right-triangle with an internal
  angle of $\pi / 4$ radians or $\degree{45}$.
}
\label{fig:right_triangle_30_45_degrees}
\end{figure}

\begin{example}
Consider the right-angled triangle in
\Figure{fig:right_triangle_30_degrees}.  Calculate the exact values
of $\sin\frac{\pi}{6}$ and $\cos\frac{\pi}{6}$.
\end{example}

\begin{solution}
You can use \Equation{eqn:define_sine} to calculate
$\sin\frac{\pi}{6}$.  Note that from
\Figure{fig:right_triangle_30_degrees}, the side opposite the angle of
$\frac{\radian{\pi}}{6}$ has a length of $1$.  The hypotenuse has a
length of $2$.  Then you have the exact value
$\sin\frac{\pi}{6} = 1 / 2$.

You can use \Equation{eqn:define_cosine} to calculate
$\cos\frac{\pi}{6}$.  From \Figure{fig:right_triangle_30_degrees}, the
side that is adjacent to the angle of $\frac{\radian{\pi}}{6}$ has a
length of $\sqrt{3}$.  Since the hypotenuse has a length of $2$, you
have the exact value $\cos\frac{\pi}{6} = \frac{\sqrt{3}}{2}$.
\end{solution}

\begin{exercise}
Use the right-triangle in \Figure{fig:right_triangle_45_degrees} to
calculate the exact values of $\sin\frac{\pi}{4}$ and
$\cos\frac{\pi}{4}$.
\end{exercise}
%%
\ifbool{showSolution}{
\begin{solution}
From \Figure{fig:right_triangle_45_degrees}, the side that is opposite
to the angle of $\frac{\radian{\pi}}{4}$ has a length of $\sqrt{2}$.
Since the hypotenuse has a length of $2$, it follows that you have the
exact value $\sin\frac{\pi}{4} = \frac{\sqrt{2}}{2}$.
\Figure{fig:right_triangle_45_degrees} also shows that the side that
is adjacent to the angle of $\frac{\radian{\pi}}{4}$ has a length of
$\sqrt{2}$.  As the hypotenuse has a length of $2$, you have the exact
value $\cos\frac{\pi}{4} = \frac{\sqrt{2}}{2}$.
\end{solution}
}{}

\begin{exercise}
The three internal angles of any triangle must add up to
$\degree{180}$.  Use this fact and the right-triangle in
\Figure{fig:right_triangle_30_degrees} to calculate exact values for
$\sin\frac{\pi}{3}$ and $\cos\frac{\pi}{3}$.
\end{exercise}
%%
\ifbool{showSolution}{
\begin{solution}
\Figure{fig:right_triangle_30_degrees} shows a right-triangle with two
given internal angles.  You have the right angle, which is
$\degree{90}$ or $\pi / 2$ radians.  You also have the given angle of
$\degree{30}$ or $\pi/6$ radians.  Let $\degree{x}$ be the unknown
value of the third internal angle.  Since the sum of the three
internal angles must be $\degree{180}$, you have the equation
$\degree{90} + \degree{30} + \degree{x} = \degree{180}$, which can be
simplified to $\degree{120} + \degree{x} = \degree{180}$.  Solving the
last equation for $\degree{x}$ shows that the third internal angle is
%%
\begin{align*}
\degree{x}
&=
\degree{180} - \degree{120} \\[4pt]
&=
\degree{60}
\end{align*}
%%
or $\pi / 3$ radians.  The three internal angles are labelled in
\Figure{fig:right_triangle_60_degrees}.

\begin{figure}[!htbp]
\centering
\includegraphics[scale=1]{image/04/right-triangle-60-degree.pdf}
\caption{%%
  A right-triangle whose internal angles are
  $\degree{90}$~(or $\pi/2$ radians), $\degree{30}$~(or $\pi/6$
  radians), and $\degree{60}$~(or $\pi/3$ radians).
}
\label{fig:right_triangle_60_degrees}
\end{figure}

From the angle of $\pi/3$ radians, the side opposite the angle has a
length of $\sqrt{3}$.  Since the hypotenuse has a length of $2$, you
have the exact value $\sin\frac{\pi}{3} = \frac{\sqrt{3}}{2}$.  From
the angle of $\pi/3$ radians, the side adjacent to the angle has a
length of $1$.  Thus you have the exact value
$\cos\frac{\pi}{3} = \frac{1}{2}$.
\end{solution}
}{}


%%%%%%%%%%%%%%%%%%%%%%%%%%%%%%%%%%%%%%%%%%%%%%%%%%%%%%%%%%%%%%%%%%%%%%%%%%%

\section{Unit circle}

Consider the unit circle in \Figure{fig:point_on_unit_circle}.  A unit
circle is a circle whose radius is one unit in length.  In this
section, you will learn how a unit circle is related to the sine and
cosine functions.

Let $\tuple{a}{b}$ be a point on the unit circle of
\Figure{fig:point_on_unit_circle}.  Suppose you zoom in on the
triangle in \Figure{fig:point_on_unit_circle} and end up with the
right-triangle in \Figure{fig:right_triangle_in_unit_circle}.  The
triangle has a base of length $a$ and a height of length $b$.  The
hypotenuse is the radius of the unit circle.  Let $\varphi$ be the
angle~(in radians) that spans from the positive half of the $x$-axis
to the point $\tuple{a}{b}$, going anti-clockwise.  From the angle
$\varphi$, the side adjacent to $\varphi$ has a length of $a$.  The
hypotenuse has a length of $1$ because the hypotenuse is the radius of
the unit circle.  Then you have the equation
$\cos\varphi = a / 1 = a$.  The side opposite $\varphi$ has a length
of $b$.  As the hypotenuse has a length of $1$, you have the equation
$\sin\varphi = b / 1 = b$.  In other words, the values of $a$ and $b$
can be written in terms of sine and cosine as:
%%
\begin{equation}
\label{eqn:value_of_x_y_on_unit_circle}
a = \cos\varphi
%%
\qquad\text{and}\qquad
%%
b = \sin\varphi.
\end{equation}

\begin{figure}[!htbp]
\centering
\includegraphics[scale=1]{image/04/unit-circle-right-triangle.pdf}
\caption{%%
  The right-triangle from the unit circle of
  \Figure{fig:point_on_unit_circle}.
}
\label{fig:right_triangle_in_unit_circle}
\end{figure}

\begin{figure}[!htbp]
\centering
\includegraphics[scale=1.1]{image/04/unit-circle.pdf}
\caption{%%
  A unit circle in the Cartesian coordinate system.  The circle has a
  radius of $r = 1$ and is centred at the origin.  If $\tuple{a}{b}$
  is any point on the circle, the point makes an angle of $\varphi$
  radians going anti-clockwise from the positive half of the $x$-axis
  to the point.  The vertical dashed line is perpendicular to the
  $x$-axis.
}
\label{fig:point_on_unit_circle}
\end{figure}

\begin{example}
Use the point $\tuple{1}{0}$ in \Figure{fig:point_on_unit_circle} to
derive the exact values $\cos(0) = 1$ and $\sin(0) = 0$.
\end{example}

\begin{solution}
From the positive half of the $x$-axis, going anti-clockwise to the
point $\tuple{1}{0}$ you have an angle of $\degree{0}$ or
$\varphi = 0$ radians because the point $\tuple{1}{0}$ lies on the
$x$-axis.  Thus you have $\tuple{a}{b} = \tuple{1}{0}$ and
$\varphi = 0$.  Use \Equation{eqn:value_of_x_y_on_unit_circle} to
write $\cos(0) = 1$ and $\sin(0) = 0$.

On the other hand, you can derive the exact values as follows.  The
point $\tuple{1}{0}$ makes a right-triangle with the $x$-axis, where
the height of the triangle has length $0$, the base of the triangle
has length $1$, and the hypotenuse has length $1$ because the
hypotenuse is the radius of the unit circle.  From the angle of $0$
radians, the side adjacent to the angle is the base of the
right-triangle so you can write $\cos(0) = 1 / 1$, which simplifies to
$\cos(0) = 1$.  The side opposite the given angle has length $0$.
Since the hypotenuse has length $1$, you can write $\sin(0) = 0 / 1$,
which simplifies to $\sin(0) = 0$.
\end{solution}

\begin{example}
Use the point $\tuple{0}{1}$ in \Figure{fig:point_on_unit_circle} to
derive the exact values $\cos\frac{\pi}{2} = 0$ and
$\sin\frac{\pi}{2} = 1$.
\end{example}

\begin{solution}
From the positive half of the $x$-axis, going anti-clockwise to the
point $\tuple{0}{1}$ results in an angle of $\degree{90}$, which you
know from \Example{ex:90_degrees_to_radians} is
$\varphi = \pi / 2$ radians.   Then you have
$\tuple{a}{b} = \tuple{0}{1}$ and $\varphi = \pi / 2$.  Use
\Equation{eqn:value_of_x_y_on_unit_circle} to write
$\cos\frac{\pi}{2} = 0$ and $\sin\frac{\pi}{2} = 1$.

You can also derive the exact values as follows.  The right-triangle
that the point $\tuple{0}{1}$ makes with the $x$-axis has a base of
length $0$ and a height of length $1$.  The hypotenuse of the triangle
is the radius of the unit circle and hence is of length $1$.  From the
angle $\pi / 2$, the side adjacent to the angle is the base of the
triangle.  Since the hypotenuse has length $1$, you can write
$\cos\frac{\pi}{2} = 0 / 1$, which can be simplified to
$\cos\frac{\pi}{2} = 0$.  The side opposite the angle $\pi / 2$ is the
height of the right-triangle, which you know to have length $1$.  Use
the value of the hypotenuse to write $\sin\frac{\pi}{2} = 1 / 1$,
which simplifies to $\sin\frac{\pi}{2} = 1$.
\end{solution}

\begin{exercise}
Explain why $\cos\pi = -1$ and $\sin\pi = 0$.
\end{exercise}

\ifbool{showSolution}{
\begin{solution}
The point $\tuple{-1}{0}$ on the unit circle in
\Figure{fig:point_on_unit_circle} makes an angle of $\pi$ radians~(or
\degree{180}) with respect to the positive half of the $x$-axis.  Then
you have $\tuple{a}{b} = \tuple{-1}{0}$ and $\varphi = \pi$.  Use
\Equation{eqn:value_of_x_y_on_unit_circle} to write $\cos\pi = -1$ and
$\sin\pi = 0$.
\end{solution}
}{}

\begin{exercise}
\label{ex:cos_sin_270_degrees}
Explain why $\cos\frac{3\pi}{2} = 0$ and $\sin\frac{3\pi}{2} = -1$.
\end{exercise}

\ifbool{showSolution}{
\begin{solution}
The point $\tuple{0}{-1}$ on the unit circle in
\Figure{fig:point_on_unit_circle} makes an angle of $\degree{270}$
with respect to the positive half of the $x$-axis, which you know from
\Example{ex:convert_270_degrees_to_radians} is equivalent to
$\frac{3\pi}{2}$ radians.  Then you have
$\tuple{a}{b} = \tuple{0}{-1}$ and $\varphi = \frac{3\pi}{2}$.  Use
\Equation{eqn:value_of_x_y_on_unit_circle} to write
$\cos\frac{3\pi}{2} = 0$ and $\sin\frac{3\pi}{2} = -1$.
\end{solution}
}{}

\begin{exercise}
Explain why $\cos(-\frac{\pi}{2}) = 0$ and
$\sin(-\frac{\pi}{2}) = -1$.
\end{exercise}

\ifbool{showSolution}{
\begin{solution}
Suppose a point $\tuple{a}{b}$ on the unit circle in
\Figure{fig:point_on_unit_circle} makes an angle of $-\pi / 2$
radians.  The angle is obtained by going clockwise from the positive
half of the $x$-axis to the point $\tuple{a}{b}$.  So going by
$-\pi / 2$ radians is the same as going by $\pi / 2$ radians clockwise
from the positive half of the $x$-axis to $\tuple{a}{b}$.  In other
words, $-\pi / 2$ radians is the same as $\frac{3\pi}{2}$ radians.
Use the results from \Exercise{ex:cos_sin_270_degrees} to conclude
that $\cos(-\frac{\pi}{2}) = 0$ and $\sin(-\frac{\pi}{2}) = -1$.
\end{solution}
}{}

The angle of $\varphi$ radians tells you how much of the unit circle
to cover.  An angle of $\varphi = \radian{2\pi}$ means that you make a
complete trip around the unit circle, going anti-clockwise starting
from the point $\tuple{1}{0}$.  The angle $\varphi = \radian{\pi}$
means that your trip covers only half of the unit circle because
$\radian{\pi} = \degree{180}$, going anti-clockwise from the point
$\tuple{1}{0}$.  Finally, $\varphi = \frac{\radian{\pi}}{2}$ means
that your trip covers only one-quarter of the unit circle because
$\frac{\radian{\pi}}{2} = \degree{90}$, going anti-clockwise from the
point $\tuple{1}{0}$.

\begin{exercise}
Starting from the point $\tuple{1}{0}$ on the unit circle, suppose you
go anti-clockwise around the unit circle by an angle of $\varphi$
radians.  Explain why going around the circle by
$\varphi = \radian{2\pi}$ is equivalent to standing still at the point
$\tuple{1}{0}$.
\end{exercise}
%%
\ifbool{showSolution}{
\begin{solution}
Starting from the point $\tuple{1}{0}$ on the unit circle, going
anti-clockwise around the circle by $2\pi$ radians means that you make
a complete trip around the unit circle and end up at the starting
point.  This is the same as not making the trip at all, because you
end up where you started.
\end{solution}
}{}

\begin{exercise}
Starting from the point $\tuple{1}{0}$ on the unit circle, suppose you
go anti-clockwise around the circle by $\frac{3\pi}{2}$ radians.
Describe where you end up.
\end{exercise}
%%
\ifbool{showSolution}{
\begin{solution}
The angle of $\frac{3\pi}{2}$ radians is equivalent to
$\degree{270}$.  The reason is that $\frac{3\pi}{2}$ can be written as
%%
\begin{align*}
\frac{3\pi}{2}
&=
\frac{2\pi}{2} + \frac{\pi}{2} \\[4pt]
&=
\pi + \frac{\pi}{2}.
\end{align*}
%%
You know that $\radian{\pi}$ is equivalent to $\degree{180}$, so it
follows that $\frac{\radian{\pi}}{2}$ is equivalent to
$\frac{\degree{180}}{2} = \degree{90}$.  Thus $\frac{3\pi}{2}$ radians
is equivalent to $\degree{180} + \degree{90} = \degree{270}$.
Starting from the point $\tuple{1}{0}$ on the unit circle, going
around the circle by $\degree{270}$ would bring you to the point
$\tuple{0}{-1}$ on the unit circle.
\end{solution}
}{}


%%%%%%%%%%%%%%%%%%%%%%%%%%%%%%%%%%%%%%%%%%%%%%%%%%%%%%%%%%%%%%%%%%%%%%%%%%%

\section{Cartesian to polar}

Let's now discuss how to convert a point $\tuple{a}{b}$ in the
Cartesian coordinate system to the polar coordinate system.
\Figure{fig:convert_from_polar_to_Cartesian_coordinates} illustrates
how the point $\tuple{a}{b}$ can be represented in terms of radius and
angle.  The coordinate system that uses radius and angle is known as
the \emph{polar coordinate system}.  Suppose you want to convert the
Cartesian coordinates $\tuple{a}{b}$ to polar coordinates.  To do so,
you assume that $\tuple{a}{b}$ is a point on a circle that is centred
at the origin.  The distance from $\tuple{a}{b}$ to the origin is the
radius of the circle so the radius can be calculated as:
%%
\begin{align*}
r
&=
\sqrt{
  (b - 0)^2 + (a - 0)^2
} \\[4pt]
&=
\sqrt{
  a^2 + b^2
}.
\end{align*}

\begin{figure}[!htbp]
\centering
\includegraphics[scale=1.1]{image/04/polar-cartesian.pdf}
\caption{%%
  Any point $\tuple{a}{b}$ in the Cartesian coordinate system can be
  represented as a point $\tuple{r}{\varphi}$ on a circle.  The circle
  is centred at the origin and has a radius of $r$.  The dashed
  vertical line is perpendicular to the $x$-axis.  The value of
  $\varphi$ radians measures the angle that spans from the positive
  half of the $x$-axis to the radius of the circle, going
  anti-clockwise.  The pair $\tuple{r}{\varphi}$ is known as the
  \emph{polar coordinates} of $\tuple{a}{b}$.  In other words, the
  Cartesian coordinates $\tuple{a}{b}$ and the polar coordinates
  $\tuple{r}{\varphi}$ both describe the same point.
}
\label{fig:convert_from_polar_to_Cartesian_coordinates}
\end{figure}

To calculate the angle $\varphi$ in
\Figure{fig:convert_from_polar_to_Cartesian_coordinates}, you use a
rule called the \emph{law of sines}.  For the triangle in
\Figure{fig:law_of_sines}, the law of sines states that
\[
\frac{a}{\sin\alpha}
=
\frac{b}{\sin\beta}
=
\frac{c}{\sin\gamma}.
\]
This equation is true no matter what the triangle is.  The triangle
can be right-angled, equilateral, isosceles, or scalene.

\begin{figure}[!htbp]
\centering
\includegraphics[scale=1.1]{image/04/law-sines.pdf}
\caption{%%
  A triangle with sides $a$, $b$, and $c$.  The angle opposite side
  $a$ is $\alpha$, the angle opposite side $b$ is $\beta$, and the
  angle opposite side $c$ is $\gamma$.  This is a general triangle.
  You do not assume it is right-angled, equilateral, isosceles, or
  scalene.  However, you do assume that the sum of the internal angles
  is $\alpha + \beta + \gamma = \degree{180}$.
}
\label{fig:law_of_sines}
\end{figure}

From
\Figure{fig:convert_from_polar_to_Cartesian_coordinates} you see that
the dashed vertical line has a length of $b$ and the angle opposite
the radius is $\degree{90}$ or $\pi / 2$ radians.  Use the law of
sines to write
%%
\begin{equation}
\label{eqn:law_of_sines}
\frac{b}{\sin \varphi}
=
\frac{r}{\sin \frac{\pi}{2}}.
\end{equation}
%%
Since $\sin \frac{\pi}{2} = 1$, \Equation{eqn:law_of_sines} can be
simplified to
\[
\frac{b}{\sin \varphi}
=
r.
\]
Solve the last equation for $\sin \varphi$ and you get
$b/r = \sin \varphi$.  Now solve for the angle $\varphi$ and you get
\[
\varphi
=
\arcsin\parenthesis*{\frac{b}{r}}.
\]
The function $\arcsin x$ is the inverse of the function $\sin x$.  For
now, do not worry about how to calculate the value of $\arcsin x$ for
any given value of $x$.  The above is summarised in the following
theorem.

\begin{theorem}
\label{thm:convert_Cartesian_to_polar_coordinates}
Let $\tuple{a}{b}$ be a point in the Cartesian coordinate system.
Then $\tuple{a}{b}$ can be written in polar coordinates as
$\tuple{r}{\varphi}$, where
\[
r
=
\sqrt{a^2 + b^2}
%%
\qquad
\text{and}
\qquad
%%
\varphi
=
\arcsin\parenthesis*{\frac{b}{r}}.
\]
\end{theorem}

\begin{figure}[!htbp]
\centering
\includegraphics[scale=1]{image/04/right-triangle-60-degree.pdf}
\caption{%%
  All internal angles of the right-triangle in
  \Figure{fig:right_triangle_30_degrees} are labelled.
}
\label{fig:right_triangle_third_angle_60_degrees_labelled}
\end{figure}

\begin{example}
Verify the law of sines for the triangle in
\Figure{fig:right_triangle_30_degrees}.
\end{example}

\begin{solution}
First, you need to know the degrees of all internal angles.  In
\Figure{fig:right_triangle_30_degrees}, you have the given angles of
$\degree{90}$~(or $\pi / 2$ radians) and
$\degree{30}$~(or $\pi / 6$ radians).  If the third angle is $x$
degrees, then the three angles must add up to
%%
\begin{align*}
180
&=
90 + 30 + x \\[4pt]
&=
120 + x
\end{align*}
%%
degrees.  The reason is that the three internal angles of any triangle
must sum to $\degree{180}$.  Solve the equation $180 = 120 + x$ for
$x$ to see that you have $x = 60$.  In other words, the third internal
angle is $\degree{60}$ or $\pi / 3$ radians.  After labelling the
third internal angle, you end up with
\Figure{fig:right_triangle_third_angle_60_degrees_labelled}.

Finally, you verify the law of sines for the triangle in
\Figure{fig:right_triangle_third_angle_60_degrees_labelled}.  The side
opposite the angle $\frac{\radian{\pi}}{6}$ has length $1$ and so you
have the ratio
\[
\frac{1}{\sin\frac{\pi}{6}}
=
\frac{1}{1/2}
=
1 \times 2
=
2.
\]
The side opposite the angle $\frac{\radian{\pi}}{3}$ has length
$\sqrt{3}$ and so you have the ratio
\[
\frac{\sqrt{3}}{\sin\frac{\pi}{3}}
=
\frac{\sqrt{3}}{\sqrt{3} / 2}
=
\sqrt{3} \times \frac{2}{\sqrt{3}}
=
2.
\]
The side opposite the angle $\frac{\radian{\pi}}{2}$ has length $2$
and so you have the ratio
\[
\frac{2}{\sin\frac{\pi}{2}}
=
\frac{2}{2/2}
=
2.
\]
Therefore for the triangle in
\Figure{fig:right_triangle_third_angle_60_degrees_labelled} you have
\[
\frac{1}{\sin\frac{\pi}{6}}
=
\frac{\sqrt{3}}{\sin\frac{\pi}{3}}
=
\frac{2}{\sin\frac{\pi}{2}}
\]
and the law of sines is verified for the given triangle.
\end{solution}

\begin{exercise}
Verify the law of sines for the triangle in
\Figure{fig:right_triangle_45_degrees}.
\end{exercise}
%%
\ifbool{showSolution}{
\begin{solution}
First, you must determine the third angle in the triangle of
\Figure{fig:right_triangle_45_degrees}.  The two given angles are
$\degree{45}$~(or $\pi / 4$ radians) and
$\degree{90}$~(or $\pi / 2$ radians).  If $x$ degrees represents the
third angle, then you have
%%
\begin{align*}
180
&=
90 + 45 + x \\[4pt]
&=
135 + x.
\end{align*}
%%
Solve the equation $180 = 135 + x$ for $x$ and you obtain
$x = 45$.  In other words, the third angle is $\degree{45}$ or
$\pi / 4$ radians.  After labelling the third angle, you end up with
\Figure{fig:right_triangle_45_degrees_all_labelled}.

\begin{figure}[!htbp]
\centering
\includegraphics[scale=1]{image/04/right-triangle-45-degree-all.pdf}
\caption{%%
  All internal angles of the triangle in
  \Figure{fig:right_triangle_45_degrees} are labelled.
}
\label{fig:right_triangle_45_degrees_all_labelled}
\end{figure}

Now you can verify the law of sines for the triangle in
\Figure{fig:right_triangle_45_degrees_all_labelled}.  The side
opposite the angle $\pi / 4$ has length $\sqrt{2}$ and so you have the
ratio
\[
\frac{\sqrt{2}}{\sin\frac{\pi}{4}}
=
\frac{\sqrt{2}}{\sqrt{2} / 2}
=
\sqrt{2} \times \frac{2}{\sqrt{2}}
=
2.
\]
Furthermore, the side opposite the angle $\pi / 2$ has length $2$,
thus you have
\[
\frac{2}{\sin\frac{\pi}{2}}
=
\frac{2}{2/2}
=
\frac{2}{1}
=
2.
\]
Therefore for the triangle in
\Figure{fig:right_triangle_45_degrees_all_labelled} you have
\[
\frac{\sqrt{2}}{\sin\frac{\pi}{4}}
=
\frac{2}{\sin\frac{\pi}{2}}
\]
and the law of sines is verified for the triangle under
consideration.
\end{solution}
}{}

\begin{exercise}
\label{ex:square_root_of_a_squared}
Let $a \geq 0$ be a real number.  Explain why $(\sqrt{a})^2 = a$.
\end{exercise}

\ifbool{showSolution}{
\begin{solution}
Suppose $a \geq 0$ is any real number.  Write
%%
\begin{align*}
(\sqrt{a})^2
&=
\sqrt{a} \times \sqrt{a} \\[4pt]
&=
\sqrt{a \times a} \\[4pt]
&=
\sqrt{a^2}.
\end{align*}
%%
The square root of $a^2$ is $a$ itself because you assumed that
$a \geq 0$.  Therefore $(\sqrt{a})^2 = a$.
\end{solution}
}{}

\begin{example}
If $a > 0$ is a positive number, any point $\tuple{a}{a}$ in the
Cartesian coordinate system makes an angle of
$\degree{45}$~(or $\pi / 4$ radians) with respect to the positive half
of the $x$-axis.  Convert the Cartesian coordinates
$(\sqrt{2}\comma \sqrt{2})$ to polar coordinates.
\end{example}

\begin{solution}
You assume that the point $A = (\sqrt{2}\comma \sqrt{2})$ lies on a
circle that is centred at the origin.  The radius of the circle is
equivalent to the distance from the origin to the point $A$.  Use
\Theorem{thm:convert_Cartesian_to_polar_coordinates} to write the
radius $r$ as
\[
r
=
\sqrt{
  (\sqrt{2})^2 + (\sqrt{2})^2
}.
\]
Now use \Exercise{ex:square_root_of_a_squared} to simplify the latter
equation to
%%
\begin{align*}
r
&=
\sqrt{2 + 2} \\[4pt]
&=
2.
\end{align*}
%%
Since the point $A$ makes an angle of $\pi / 4$ radians with respect
to the positive half of the $x$-axis, you can write
$\varphi = \pi / 4$.  On the other hand, you can use
\Theorem{thm:convert_Cartesian_to_polar_coordinates} to calculate the
angle as
%%
\begin{align*}
\varphi
&=
\arcsin\frac{\sqrt{2}}{2} \\[4pt]
&=
\frac{\pi}{4}.
\end{align*}
%%
Therefore the Cartesian coordinates $(\sqrt{2}\comma \sqrt{2})$ can be
represented as the polar coordinates $(2\comma \frac{\pi}{4})$.
\end{solution}

\begin{exercise}
Refer to the unit circle in \Figure{fig:point_on_unit_circle}.
Determine the polar coordinates of each of the Cartesian coordinates
$\tuple{1}{0}$, $\tuple{0}{1}$, $\tuple{-1}{0}$, and $\tuple{0}{-1}$.
\end{exercise}

\ifbool{showSolution}{
\begin{solution}
Each of the points $\tuple{1}{0}$, $\tuple{0}{1}$, $\tuple{-1}{0}$,
and $\tuple{0}{-1}$ lie on the unit circle so in polar coordinates
each of them has a radius of $r = 1$.  The point $\tuple{1}{0}$ makes
an angle of $\varphi = 0$ radians and so $\tuple{1}{0}$ can be written
in polar coordinates as $\tuple{1}{0}$.  The point $\tuple{0}{1}$
makes an angle of $\varphi = \pi / 2$ radians and so it has the polar
representation $(1\comma \frac{\pi}{2})$.  The point $\tuple{-1}{0}$
makes an angle of $\varphi = \pi$ radians and so it can be written as
the polar coordinates $\tuple{1}{\pi}$.  Finally, the point
$\tuple{0}{-1}$ makes an angle of $\varphi = \frac{3\pi}{2}$ radians
and therefore it can be written in polar coordinates as
$(1\comma \frac{3\pi}{2})$.
\end{solution}
}{}


\newpage
%%%%%%%%%%%%%%%%%%%%%%%%%%%%%%%%%%%%%%%%%%%%%%%%%%%%%%%%%%%%%%%%%%%%%%%%%%%

\section*{Problem}

\begin{problem}
\item The point $A = (\frac{\sqrt{3}}{2}\comma \frac{1}{2})$ in the
  Cartesian coordinate system makes an angle of $\degree{30}$ with the
  positive half of the $x$-axis.  Draw the unit circle centred at the
  origin.  Mark the point $A$ on the unit circle.  Draw a straight
  line from the origin to $A$.  Label the angle that the point $A$
  makes with respect to the positive half of the $x$-axis.  Convert
  $A$ to polar coordinates.
\ifbool{showSolution}{
  \begin{solution}
  See \Figure{fig:unit_circle_30_degrees}.  Since any unit circle has
  a radius of $r = 1$, the polar coordinates of $A$ are given by
  $\tuple{1}{\frac{\pi}{6}}$.

  \begin{figure}[!htbp]
  \centering
  \includegraphics[scale=1]{image/04/unit-circle-30-degree.pdf}
  \caption{%%
    The point $(\frac{\sqrt{3}}{2}\comma \frac{1}{2})$ makes an angle
    of $\degree{30}$~(or $\pi/6$ radians) with respect to the positive
    half of the $x$-axis.
  }
  \label{fig:unit_circle_30_degrees}
  \end{figure}
  \end{solution}
}{}

\item Let $n \geq 0$ be an even integer.  Explain why
  $\cos(n\pi) = 1$ and $\sin(n\pi) = 0$.
\ifbool{showSolution}{
  \begin{solution}
  Since $n \geq 0$ is even, it can be written as $n = 2k$ for some
  integer $k \geq 0$.  Then $n\pi = 2k\pi = k(2\pi)$, where $2\pi$
  radians is equivalent to $\degree{360}$.  So $k (2\pi)$ means,
  starting from the point $\tuple{1}{0}$ on the unit circle of
  \Figure{fig:point_on_unit_circle}, you go anti-clockwise $k$ times
  around the unit circle.  One complete trip around the unit circle
  spans $2\pi$ radians, so $k (2\pi)$ radians mean you do $k$ complete
  trips around the unit circle, thus ending up at the point
  $\tuple{1}{0}$.  Therefore $\cos(n\pi) = 1$ and $\sin(n\pi) = 0$.
\end{solution}
}{}

\item Let $n > 0$ be an odd integer.  Explain why $\cos(n\pi) = -1$
  and $\sin(n\pi) = 0$.
\ifbool{showSolution}{
  \begin{solution}
  Since $n > 0$ is odd, it can be written as $n = 2k + 1$ for some
  integer $k \geq 0$.  Note that you can write
  %%
  \begin{align*}
  n\pi
  &=
  (2k + 1) \pi \\[4pt]
  &=
  2k\pi + \pi \\[4pt]
  &=
  k(2\pi) + \pi.
  \end{align*}
  %%
  In other words, starting from the point $\tuple{1}{0}$ on the unit
  circle of \Figure{fig:point_on_unit_circle}, you make $k$ complete
  trips around the unit circle~(going anti-clockwise) and end up at
  the starting point of $\tuple{1}{0}$.  Finally, you go
  anti-clockwise from $\tuple{1}{0}$ by $\pi$ radians.  Your
  destination is now the point $\tuple{-1}{0}$ and therefore
  $\cos(n\pi) = -1$ and $\sin(n\pi) = 0$.
  \end{solution}
}{}

\item\label{prob:cosine_sine_45_degrees}
  Explain why $\cos\frac{\pi}{4} = \sin\frac{\pi}{4} = \sqrt{2} / 2$.
\ifbool{showSolution}{
  \begin{solution}
  The point $(\frac{\sqrt{2}}{2}\comma \frac{\sqrt{2}}{2})$ in the
  Cartesian coordinate system makes an angle of
  $\degree{45}$~(or $\varphi = \pi / 4$ radians) with the positive
  half of the $x$-axis.  The distance from the point to the origin is
  %%
  \begin{align*}
  r
  &=
  \sqrt{
    \parenthesis*{\frac{\sqrt{2}}{2}}^2
    +
    \parenthesis*{\frac{\sqrt{2}}{2}}^2
  } \\[4pt]
  &=
  \sqrt{
    \frac{2}{4}
    +
    \frac{2}{4}
  } \\[4pt]
  &=
  1
  \end{align*}
  %%
  and so the point $(\frac{\sqrt{2}}{2}\comma \frac{\sqrt{2}}{2})$
  lies on the unit circle that is centred at the origin.  Now use
  \Equation{eqn:value_of_x_y_on_unit_circle} to write
  $\sqrt{2} / 2 = \cos\frac{\pi}{4}$ and
  $\sqrt{2} / 2 = \sin\frac{\pi}{4}$, which can also be written as
  $\cos\frac{\pi}{4} = \sin\frac{\pi}{4} = \sqrt{2} / 2$.
  \end{solution}
}{}

\item For which values of $\varphi$ would $\cos\varphi = \sin\varphi$?
\ifbool{showSolution}{
  \begin{solution}
  From \Problem{prob:cosine_sine_45_degrees} you know that the point
  $(\frac{\sqrt{2}}{2}\comma \frac{\sqrt{2}}{2})$ makes an angle of
  $\degree{45}$~(or $\pi / 4$ radians) with the positive half of the
  $x$-axis.  The diagonal reflection of
  $(\frac{\sqrt{2}}{2}\comma \frac{\sqrt{2}}{2})$ is the point
  $(-\frac{\sqrt{2}}{2}\comma -\frac{\sqrt{2}}{2})$, which makes an
  angle of $\pi / 4$ radians with the negative half of the $x$-axis.
  In other words, the point
  $(-\frac{\sqrt{2}}{2}\comma -\frac{\sqrt{2}}{2})$ makes an angle of
  \[
  \pi + \frac{\pi}{4}
  =
  \frac{5\pi}{4}
  \]
  radians with the positive half of the $x$-axis and so
  $\cos\frac{5\pi}{4} = \sin\frac{5\pi}{4} = -\sqrt{2} / 2$.  Note
  that if $k \geq 0$ is an integer, then $\frac{\pi}{4} + 2k\pi$
  radians is equivalent to $\pi / 4$ radians.  The reason is that
  $2k\pi = k(2\pi)$ radians is the same as making $k$ complete trips
  around the unit circle and ending up at where you started.  For the
  same reason $\frac{5\pi}{4} + 2k\pi$ radians is equivalent to
  $\frac{5\pi}{4}$ radians.  Therefore $\cos\varphi = \sin\varphi$
  whenever you have $\varphi = \frac{\pi}{4} + 2k\pi$ or
  $\varphi = \frac{5\pi}{4} + 2k\pi$, where $k \geq 0$ is an integer.
  \end{solution}
}{}

\item For which values of $\varphi$ would you have
  $\cos\varphi = -\sqrt{2} / 2$ and $\sin\varphi = \sqrt{2} / 2$?
\ifbool{showSolution}{
  \begin{solution}
  The point $(-\frac{\sqrt{2}}{2}\comma \frac{\sqrt{2}}{2})$ is a
  reflection of the point
  $(\frac{\sqrt{2}}{2}\comma \frac{\sqrt{2}}{2})$ along the $y$-axis.
  So the point $(-\frac{\sqrt{2}}{2}\comma \frac{\sqrt{2}}{2})$ makes
  an angle of $\frac{\pi}{2} + \frac{\pi}{4} = \frac{3\pi}{4}$ radians
  with respect to the positive half of the $x$-axis.  In other words,
  you can write $\cos\frac{3\pi}{4} = -\sqrt{2} / 2$ and
  $\sin\frac{3\pi}{4} = \sqrt{2} / 2$.  Note that if $k \geq 0$ is an
  integer, then the radians of
  \[
  \frac{3\pi}{4} + 2k\pi
  =
  \frac{3\pi}{4}
  \]
  because $2k\pi$ radians is the same as making $k$ complete trips
  around the unit circle.  After those $k$ complete trips, you end up
  where you started on the unit circle.  Therefore if $k \geq 0$ is an
  integer and you have
  \[
  \varphi
  =
  \frac{3\pi}{4} + 2k\pi
  \]
  then $\cos\varphi = -\sqrt{2} / 2$ and $\sin\varphi = \sqrt{2} / 2$.
  \end{solution}
}{}

\item The point $(\frac{1}{2}\comma \frac{\sqrt{3}}{2})$ in the
  Cartesian coordinate system makes an angle of $\degree{60}$ with
  respect to the positive half of the $x$-axis.  For which values of
  $\varphi$ would you have $\cos\varphi = 1 / 2$ and
  $\sin\varphi = \sqrt{3} / 2$?
\ifbool{showSolution}{
  \begin{solution}
  The distance from the point $(\frac{1}{2}\comma \frac{\sqrt{3}}{2})$
  to the origin is
  %%
  \begin{align*}
  r
  &=
  \sqrt{
    \parenthesis*{\frac{1}{2}}^2
    +
    \parenthesis*{\frac{\sqrt{3}}{2}}^2
  } \\[4pt]
  &=
  \sqrt{
    \frac{1}{4}
    +
    \frac{3}{4}
  } \\[4pt]
  &=
  \sqrt{\frac{4}{4}} \\[4pt]
  &=
  1
  \end{align*}
  %%
  and so the point $(\frac{1}{2}\comma \frac{\sqrt{3}}{2})$ lies on
  the unit circle.  An angle of $\degree{60}$ is the same as
  $\pi / 3$ radians.  Use \Equation{eqn:value_of_x_y_on_unit_circle}
  to write $\cos\frac{\pi}{3} = 1 / 2$ and
  $\sin\frac{\pi}{3} = \sqrt{3} / 2$.  Note also that if $k \geq 0$ is
  an integer, then the radians of
  \[
  \frac{\pi}{3} + 2k\pi
  =
  \frac{\pi}{3}
  \]
  because $2k\pi$ radians is the same as making $k$ complete trips
  around the unit circle and then ending up where you started.
  Therefore if $k \geq 0$ is an integer and
  \[
  \varphi
  =
  \frac{\pi}{3} + 2k\pi
  \]
  then $\cos\varphi = 1 / 2$ and $\sin\varphi = \sqrt{3} / 2$.
  \end{solution}
}{}

\begin{table}[!htbp]
\centering
\begin{tabular}{rccc} \toprule
degrees       & $\varphi$       & $\cos\varphi$        & $\sin\varphi$ \\\midrule
$\degree{0}$  & $0$             & $1$                  & $0$           \\[6pt]
$\degree{30}$ & $\frac{\pi}{6}$ & $\frac{\sqrt{3}}{2}$ & $\frac{1}{2}$ \\[6pt]
$\degree{45}$ & $\frac{\pi}{4}$                                        \\[6pt]
$\degree{60}$                                                          \\[6pt]
$\degree{90}$                                                          \\[6pt]
$\degree{135}$                                                         \\[6pt]
$\degree{180}$                                                         \\[6pt]
$\degree{225}$                                                         \\[6pt]
$\degree{270}$                                                         \\[6pt]
$\degree{315}$                                                         \\\bottomrule
\end{tabular}

\caption{%%
  Special values of the functions $\cos\varphi$ and $\sin\varphi$.
  This table is incomplete.  You need to fill in the missing entries.
}
\label{tab:cosine_sine_special_values_incomplete}
\end{table}
%%
\item \Table{tab:cosine_sine_special_values_incomplete} lists some
  special values of the cosine and sine functions.  Fill in the
  missing entries.
\ifbool{showSolution}{
  \begin{solution}
  See \Table{tab:cosine_sine_special_values_complete}.

  \begin{table}[!htbp]
  \centering
  \begin{tabular}{rccc}                                                               \toprule
degrees        & $\varphi$        & $\cos\varphi$         & $\sin\varphi$         \\\midrule
$\degree{0}$   & $0$              & $1$                   & $0$                   \\[6pt]
$\degree{30}$  & $\frac{\pi}{6}$  & $\frac{\sqrt{3}}{2}$  & $\frac{1}{2}$         \\[6pt]
$\degree{45}$  & $\frac{\pi}{4}$  & $\frac{\sqrt{2}}{2}$  & $\frac{\sqrt{2}}{2}$  \\[6pt]
$\degree{60}$  & $\frac{\pi}{3}$  & $\frac{1}{2}$         & $\frac{\sqrt{3}}{2}$  \\[6pt]
$\degree{90}$  & $\frac{\pi}{2}$  & $0$                   & $1$                   \\[6pt]
$\degree{135}$ & $\frac{3\pi}{4}$ & $-\frac{\sqrt{2}}{2}$ & $\frac{\sqrt{2}}{2}$  \\[6pt]
$\degree{180}$ & $\pi$            & $-1$                  & $0$                   \\[6pt]
$\degree{225}$ & $\frac{5\pi}{4}$ & $-\frac{\sqrt{2}}{2}$ & $-\frac{\sqrt{2}}{2}$ \\[6pt]
$\degree{270}$ & $\frac{3\pi}{2}$ & $0$                   & $-1$                  \\[6pt]
$\degree{315}$ & $\frac{7\pi}{4}$ & $\frac{\sqrt{2}}{2}$  & $-\frac{\sqrt{2}}{2}$ \\\bottomrule
\end{tabular}

  \caption{%%
    Some special values of the functions $\cos\varphi$ and
    $\sin\varphi$.
  }
  \label{tab:cosine_sine_special_values_complete}
  \end{table}
  \end{solution}
}{}

\item\label{prob:pythagoras_theorem_unit_circle}
  Suppose that
  \[
  \cos^2\varphi
  =
  (\cos\varphi) (\cos\varphi)
  =
  (\cos\varphi)^2
  \]
  and
  \[
  \sin^2\varphi
  =
  (\sin\varphi) (\sin\varphi)
  =
  (\sin\varphi)^2.
  \]
  Explain why $\cos^2\varphi + \sin^2\varphi = 1$.
\ifbool{showSolution}{
\begin{solution}
From the unit circle in \Figure{fig:point_on_unit_circle}, you know
that $\cos\varphi$ and $\sin\varphi$ are the lengths of two sides of a
right-angled triangle, with the hypotenuse being one unit.  Use
Pythagoras' theorem to write the hypotenuse as
%%
\begin{align*}
1
&=
(\cos\varphi)^2 + (\sin\varphi)^2 \\[4pt]
&=
\cos^2\varphi + \sin^2\varphi
\end{align*}
%%
where $1^2 = 1$.
\end{solution}
}{}

\item Let $r > 0$ be a real number.  Present two explanations as to
  why
  %%
  \begin{equation}
  \label{eqn:pythagoras_theorem_sine_cosine}
  (r \cos\varphi)^2 + (r \sin\varphi)^2
  =
  r^2.
  \end{equation}
\ifbool{showSolution}{
\begin{solution}
From \Figure{fig:convert_from_polar_to_Cartesian_coordinates}, you see
that $r \cos\varphi$ and $r \sin\varphi$ are the lengths of two sides
of a right-angled triangle.  The value of the hypotenuse is $r$.  Use
Pythagoras' theorem to write the hypotenuse as
\[
r^2
=
(r \cos\varphi)^2 + (r \sin\varphi)^2
\]
which is \Equation{eqn:pythagoras_theorem_sine_cosine}.

Here is another explanation.  Use
\Problem{prob:pythagoras_theorem_unit_circle} to write
$1 = \cos^2\varphi + \sin^2\varphi$.  Multiply both sides of the
latter equation by $r^2$ to obtain
%%
\begin{align*}
r^2
&=
r^2
\parenthesis*{
  \cos^2\varphi + \sin^2\varphi
} \\[4pt]
&=
r^2 \cos^2\varphi + r^2 \sin^2\varphi \\[4pt]
&=
(r \cos\varphi)^2 + (r \sin\varphi)^2
\end{align*}
%%
which is \Equation{eqn:pythagoras_theorem_sine_cosine}.
\end{solution}
}{}
\end{problem}

\end{document}
