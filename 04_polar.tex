%%%%%%%%%%%%%%%%%%%%%%%%%%%%%%%%%%%%%%%%%%%%%%%%%%%%%%%%%%%%%%%%%%%%%%%%%%%

\documentclass[a4paper,oneside,12pt]{article}
\usepackage{mystyle}

\begin{document}

\title{\Large\bf Polar coordinate system}
\author{%%
  Minh Van Nguyen \\
  \url{mvngu@gmx.com}
}
\date{\today}
\maketitle


%%%%%%%%%%%%%%%%%%%%%%%%%%%%%%%%%%%%%%%%%%%%%%%%%%%%%%%%%%%%%%%%%%%%%%%%%%%

\section{Degrees and radians}

Before discussing how to get a better approximation of $\pi$, you need
to know about radians.  Angular degrees, or degrees, are used to
measure angles.  Another way to measure angles is to use
\emph{radians}, which are defined as follows.  A unit circle has a
radius of $r = 1$ so its circumference is $2 \pi r = 2 \pi$.  So the
value of $2 \pi$ is the distance around the unit circle.  The value of
$2\pi$ is also used as the angle of a circle and you say that a circle
has $2\pi$ radians.  Since any circle has $360$ degrees~(or
$\degree{360}$), then $\degree{360} = 2\pi$ radians.  The last
equation can be simplified to $\degree{180} = \pi$, which means that
$\degree{180}$ is equivalent to $\pi$ radians.

\begin{exercise}
Explain how you would define one radian.
\end{exercise}

\ifbool{showSolution}{
\begin{solution}
Since $\degree{180} = \pi$ radians, divide both sides by $\pi$ to get
$\degree{180} / \pi = 1$ radian.  Therefore one radian is
approximately $57.3$ degrees.
\end{solution}
}{}

\begin{exercise}
Write $\degree{90}$ in terms of radians.
\end{exercise}

\ifbool{showSolution}{
\begin{solution}
You know that $\degree{180} = \pi$ radians.  Divide both sides of the
equation by $2$ to see that $\degree{90} = \pi / 2$ radians.
\end{solution}
}{}

\Figure{fig:convert_from_polar_to_Cartesian_coordinates} illustrates
how a point $\tuple{a}{b}$ in the Cartesian coordinate system can be
represented in terms of radius and angle.  The coordinate system that
uses radius and angle is known as the \emph{polar coordinate system}.
Suppose you want to convert the Cartesian coordinates $\tuple{a}{b}$
to polar coordinates.  To do so, you assume that $\tuple{a}{b}$ is a
point on a circle that is centred at the origin.  Then the distance
from $\tuple{a}{b}$ to the origin is the radius of the circle so the
radius can be calculated as:
%%
\begin{align*}
r
&=
\sqrt{
  (b - 0)^2 + (a - 0)^2
} \\[4pt]
&=
\sqrt{
  a^2 + b^2
}.
\end{align*}
%%
To calculate the angle $\varphi$ in
\Figure{fig:convert_from_polar_to_Cartesian_coordinates}, you use a
rule called the \emph{law of sines}.  From
\Figure{fig:convert_from_polar_to_Cartesian_coordinates} you see that
the dashed vertical line has a length of $b$ and the angle opposite
the radius is $\degree{90}$ or $\pi / 2$ radians.  Use the law of
sines to write
%%
\begin{equation}
\label{eqn:law_of_sines}
\frac{b}{\sin \varphi}
=
\frac{r}{\sin \frac{\pi}{2}}.
\end{equation}
%%
Since $\sin \frac{\pi}{2} = 1$, \Equation{eqn:law_of_sines} can be
simplified to
\[
\frac{b}{\sin \varphi}
=
r.
\]
Solve the last equation for $\sin \varphi$ and you get
$\frac{b}{r} = \sin \varphi$.  Now solve for the angle $\varphi$ and
you get
\[
\varphi
=
\arcsin\parenthesis*{\frac{b}{r}}.
\]

\begin{figure}[!htbp]
\centering
\includegraphics[scale=1.1]{image/04/polar-cartesian.pdf}
\caption{%%
  Any point $\tuple{a}{b}$ in the Cartesian coordinate system can be
  represented as a point $\tuple{r}{\varphi}$ on a circle.  The circle
  is centred at the origin and has a radius of $r$.  The value of
  $\varphi$ radians measures the angle that spans from the $x$-axis to
  the radius of the circle, going anti-clockwise.  The point
  $\tuple{r}{\varphi}$ is known as the \emph{polar coordinates} of
  $\tuple{a}{b}$.  In other words, the Cartesian coordinates
  $\tuple{a}{b}$ and the polar coordinates $\tuple{r}{\varphi}$ both
  describe the same point.
}
\label{fig:convert_from_polar_to_Cartesian_coordinates}
\end{figure}


\end{document}
