%%%%%%%%%%%%%%%%%%%%%%%%%%%%%%%%%%%%%%%%%%%%%%%%%%%%%%%%%%%%%%%%%%%%%%%%%%%

\documentclass[a4paper,oneside,12pt]{article}
\usepackage{mystyle}

\begin{document}

\title{\Large\bf Trigonometric functions}
\author{%%
  Minh Van Nguyen \\
  \url{mvngu@gmx.com}
}
\date{\today}
\maketitle

\noindent
By now, you should be familiar with the sine function $\sin x$ and the
cosine function $\cos x$, where $x$ is an angle in radians.  In this
document, you will further investigate properties of the sine and
cosine functions.  Trigonometric functions are found in many physical
phenomena.  A common example is sound waves, which can be explained in
terms of a slinky.\footnote{
  See the video at
  \url{https://youtu.be/kxQj-wPePBU}.  Refer also to the slinky
  simulation at
  \url{http://www.physicsclassroom.com/Physics-Interactives/Waves-and-Sound/Slinky-Lab/Slinky-Lab-Interactive}.
}

\begin{itemize}
\item Application: equations of parabolic motion for projectiles.
\end{itemize}


%%%%%%%%%%%%%%%%%%%%%%%%%%%%%%%%%%%%%%%%%%%%%%%%%%%%%%%%%%%%%%%%%%%%%%%%%%%

\section{Vertical shift of sine function}

\begin{figure}[!htbp]
\centering
\includegraphics[scale=1.1]{image/13/a-sin-b.pdf}
\caption{%%
  A graph of the general sine function
  $f(x) = a \sin x + b$ from $x = -2\pi$ to $x = 2\pi$.  The midline
  of $f(x)$ is the dashed horizontal line $y = b$.  The amplitude of
  $f(x)$ is the absolute value $\absoluteValue{a}$.
}
\label{fig:trigonometric:general_sine}
\end{figure}

Let $a$ and $b$ be fixed real numbers and let $x$ be a real variable
that represents an angle in radians.  The value of $a$ is often
assumed to be $a \neq 0$.  The sine function can be written
as
\[
f(x)
=
a \sin x + b
\]
which is graphed in \Figure{fig:trigonometric:general_sine}.  The
number $b$ is called the \emph{midline} of $f(x)$ and the absolute
value $\absoluteValue{a}$ is called the \emph{amplitude} of $f(x)$.
The amplitude $\absoluteValue{a}$ measures how high and how low the
value of the sine function $f(x)$ can be.\footnote{
  See the video at
  \url{https://youtu.be/2Kos5VrtTtA}.
}
As you can see in \Figure{fig:trigonometric:general_sine}, given the
sine function $f(x) = a \sin x + b$ the highest value of $f(x)$ is
$b + \absoluteValue{a}$, which is the value of the midline plut the
amplitude.  Furthermore, the lowest value of $f(x)$ is
$b - \absoluteValue{a}$, which is the value of the midline minus the
amplitude.  The number $b$ is the midline of the sine function $f(x)$
because the horizontal line $y = b$ is midway between the highest and
lowest values of $f(x)$.  In fact, if $h$ is the highest value of
$f(x)$ and $\ell$ is the lowest value of $f(x)$, then the value of the
midline is the average $\frac{h + \ell}{2}$.  If you draw the
horizontal line $y = b$ on a graph of $f(x)$~(e.g.~the dashed
horizontal line in \Figure{fig:trigonometric:general_sine}) you will
see that $f(x) = b$ whenever $a \sin x = 0$.  The above is summarised
in \Definition{def:trigonometric:sine_amplitude_and_midline}.

\begin{definition}
\label{def:trigonometric:sine_amplitude_and_midline}
\textbf{Amplitude and midline.}
Let $a$ and $b$ be real constants such that $a \neq 0$.  Given the
sine function $f(x) = a \sin x + b$, the number $\absoluteValue{a}$ is
called the \emph{amplitude} of $f(x)$ and the horizontal line $y = b$
is called the \emph{midline} of $f(x)$.
\end{definition}

\begin{figure}[!htbp]
\centering
\includegraphics[scale=1.1]{image/13/sin-vertical-shift.pdf}
\caption{%%
  The value of the midline has the effect of vertically shifting the
  graph of the sine function.
}
\label{fig:trigonometric:sine_vertical_shift}
\end{figure}

How does the value of the midline affect the graph of the sine
function?  The value of the midline has the effect of shifting the
graph of the sine function vertically.  If the value of the midline is
positive, i.e.~$b > 0$, then the graph of the sine function will be
shifted upward.  However, if the value of the midline is negative,
i.e.~$b < 0$, then the graph of the sine function will be shifted
downward.  The vertical shift of the sine function is illustrated in
\Figure{fig:trigonometric:sine_vertical_shift}.

\begin{figure}[!htbp]
\centering
\includegraphics[scale=1.1]{image/13/3-sin-1.pdf}
\caption{%%
  A graph of the sine function $f(x) = 3 \sin x + 1$ from $x = 0$ to
  $x = 2\pi$.
}
\label{fig:trigonometric:sine_3_sin_1}
\end{figure}

\begin{example}
Consider the sine function $f(x) = 3 \sin x + 1$.  Determine the
midline and amplitude of $f(x)$.  Calculate the highest and lowest
values of the function.  Draw a graph of $f(x)$ from $x = 0$ to
$x = 2\pi$.
\end{example}

\begin{solution}
The first thing you should do is work out the midline and amplitude of
the function $f(x)$.  The midline of the function is $y = 1$.  The
amplitude is $3$.  The highest value of $f(x)$ is obtained by adding
the amplitude $3$ to the value of the midline.  Doing so gives you the
value $1 + 3 = 4$.  The lowest value of $f(x)$ is obtained by
subtracting the amplitude $3$ from the value of the midline.  Doing so
results in $1 - 3 = -2$.  Thus the highest value of $f(x)$ is $4$ and
the lowest value of $f(x)$ is $-2$.
\Figure{fig:trigonometric:sine_3_sin_1} shows the midline $y = 1$ as a
dashed horizontal line.

Next, determine all values of $x$ within the range
$0 \leq x \leq 2\pi$ such that the graph of $f(x)$ intersects the
midline.  This is equivalent to determining all values of $x$ such
that $\sin x = 0$.  The reason is that if $\sin x = 0$, then the
function $f(x) = 3 \sin x + 1$ simplifies to $f(x) = 1$, which is the
midline.  You know that $\sin x = 0$ when
$x = \triple{0}{\pi}{2\pi}$ and so you have
$f(0) = f(\pi) = f(2\pi) = 3 \times 0 + 1 = 1$.  In other words, the
graph of $f(x)$ intersects the midline $y = 1$ at the points
$\tuple{0}{1}$, $\tuple{\pi}{1}$, and $\tuple{2\pi}{1}$.
\Figure{fig:trigonometric:sine_3_sin_1} shows the points of
intersection as black dots.

Now determine the highest and lowest points of $f(x)$, where
$0 \leq x \leq 2\pi$.  To determine the highest point of $f(x)$, you
determine the highest value of $\sin x$.  You know that the highest
value of $\sin x$ is $1$, which occurs when $x = \frac{\pi}{2}$.  Then
the highest value of $f(x)$ is $f(\pi/2) = 3 \times 1 + 1 = 4$ so that
the highest point of $f(x)$ is $\tuple{\frac{\pi}{2}}{4}$.  Similarly,
the lowest value of $\sin x$ is $-1$, which occurs when
$x = \frac{3\pi}{2}$.  Then the lowest value of $f(x)$ is
$f(3\pi/2) = 3(-1) + 1 = -2$ so that the lowest point of $f(x)$ is
$\tuple{\frac{3\pi}{2}}{-2}$.  The highest and lowest points of $f(x)$
are shown in \Figure{fig:trigonometric:sine_3_sin_1} as red dots.

Finally, draw a wave through the above five points and you obtain the
graph in \Figure{fig:trigonometric:sine_3_sin_1}.
\end{solution}

\begin{exercise}
Consider the function $f(x) = 4 \sin x + 2$.  Determine the midline
and amplitude of $f(x)$.  Calculate the highest and lowest values of
the function.  Draw a graph of $f(x)$ from $x = 0$ to $x = 2\pi$.
\end{exercise}

\ifbool{showSolution}{
\begin{solution}
The midline of $f(x)$ is the horizontal line $y = 2$.  The amplitude
of $f(x)$ is $4$.  To obtain the highest value of $f(x)$, you add the
amplitude $4$ to the value of the midline and get $2 + 4 = 6$.  To
obtain the lowest value of $f(x)$, you subtract the amplitude $4$ from
the value of the midline and get $2 - 4 = -2$.  That is, the highest
and lowest values of $f(x)$ are $6$ and $-2$, respectively.

To graph $f(x)$ from $x = 0$ to $x = 2\pi$, you should first determine
those values of $x$ for which the graph of $f(x)$ intersects the
midline.  This is the same as determining all values of $x$ such that
$\sin x = 0$.  You know that $\sin 0 = \sin \pi = \sin 2\pi = 0$ so
that when $x = \triple{0}{\pi}{2\pi}$ you have
$f(0) = f(\pi) = f(2\pi) = 4 \times 0 + 2 = 2$.  Thus the graph of
$f(x)$ intersects the midline at the points $\tuple{0}{2}$,
$\tuple{\pi}{2}$, and $\tuple{2\pi}{2}$.

Next, you determine the highest and lowest points of $f(x)$ within the
range $0 \leq x \leq 2\pi$.  You know that the highest value of
$\sin x$ is $1$, which occurs when $x = \frac{\pi}{2}$.  Substitute
the latter value into $f(x)$ and you obtain
%%
\begin{align*}
f(\pi/2)
&=
4 \sin \frac{\pi}{2} + 2 \\[4pt]
&=
4 \times 1 + 2 \\[4pt]
&=
6.
\end{align*}
%%
The lowest value of $\sin x$ is $-1$, which occurs when
$x = \frac{3\pi}{2}$.  Substitute the latter value into the function
$f(x)$ and simplify to obtain
%%
\begin{align*}
f(3\pi / 2)
&=
4 \sin \frac{3\pi}{2} + 2 \\[4pt]
&=
4 (-1) + 2 \\[4pt]
&=
-4 + 2 \\[4pt]
&=
-2.
\end{align*}
%%
In other words, the highest point of $f(x)$ is
$\tuple{\frac{\pi}{2}}{6}$ and the lowest point of $f(x)$ is
$\tuple{\frac{3\pi}{2}}{-2}$.

\begin{figure}[!htbp]
\centering
\includegraphics[scale=1.1]{image/13/4-sin-2.pdf}
\caption{%%
  A graph of the function $f(x) = 4 \sin x + 2$ from $x = 0$ to
  $x = 2\pi$.
}
\label{fig:trigonometric:4_sinx_2}
\end{figure}

Finally, draw a wave through the above points and you obtain the graph
shown in \Figure{fig:trigonometric:4_sinx_2}.  The black dots show the
points at which the graph of $f(x)$ intersects the midline $y = 2$.
The red dots show the highest and lowest points of $f(x)$.
\end{solution}
}{}

\begin{exercise}
Consider the function $f(x) = -2 \sin x + 6$.  Determine the midline
and amplitude of $f(x)$.  Calculate the highest and lowest values of
the function.  Draw a graph of $f(x)$ from $x = 0$ to $x = 2\pi$.
\end{exercise}

\ifbool{showSolution}{
\begin{solution}
The midline of the function $f(x)$ is $y = 6$ and the amplitude is
$\absoluteValue{-2} = 2$.  \Figure{fig:trigonometric:minus2_sin_6}
shows the midline as a dashed horizontal line.  The highest value of
$f(x)$ is obtained by adding the amplitude to the value of the
midline.  Then the required highest value is $6 + 2 = 8$.  Similarly,
the lowest value of $f(x)$ is obtained by subtracting the amplitude
from the value of the midline.  Thus the lowest value of $f(x)$ is
$6 - 2 = 4$.

\begin{figure}[!htbp]
\centering
\includegraphics[scale=1.1]{image/13/minus2-sin-6.pdf}
\caption{%%
  A graph of the function $f(x) = -2 \sin x + 6$ from $x = 0$ to
  $x = 2\pi$.
}
\label{fig:trigonometric:minus2_sin_6}
\end{figure}

To draw a graph of the function $f(x)$ within the range
$0 \leq x \leq 2\pi$, you should first determine all points of $f(x)$
at which the graph of $f(x)$ intersects the midline.  This is the same
as determining all values of $x$ such that $\sin x = 0$.  You already
know that $\sin x = 0$ whenever $x = \triple{0}{\pi}{2\pi}$.  Then you
have $f(0) = f(\pi) = f(2\pi) = 6$.  In other words, the graph of
$f(x)$ intersects the midline at the points $\tuple{0}{6}$,
$\tuple{\pi}{6}$, and $\tuple{2\pi}{6}$.  The points of intersection
are shown in \Figure{fig:trigonometric:minus2_sin_6} as black dots.

Next, determine the highest and lowest points of $f(x)$.  Note that
the highest value of $\sin x$ is $1$, which occurs when
$x = \frac{\pi}{2}$.  Then $f(\pi/2) = -2 \times 1 + 6 = 4$, which is
the lowest value of $f(x)$.  That is, the lowest point of $f(x)$ is
$\tuple{\frac{\pi}{2}}{4}$.  Furthermore, the lowest value of $\sin x$
is $-1$, which occurs when $x = \frac{3\pi}{2}$.  Then you have
$f(3\pi/2) = -2 (-1) + 6 = 8$, which is the highest value of $f(x)$.
The highest point of $f(x)$ is $\tuple{\frac{3\pi}{2}}{8}$.  The
highest and lowest points are shown in
\Figure{fig:trigonometric:minus2_sin_6} as red dots.

Finally, draw a wave through the five points to obtain the graph shown
in \Figure{fig:trigonometric:minus2_sin_6}.
\end{solution}
}{}

\begin{figure}[!htbp]
\centering
\includegraphics[scale=1.1]{image/13/5-sin-minus2.pdf}
\caption{%%
  A graph of a sine function of the form $f(x) = a \sin x + b$ from
  $x = 0$ to $x = 2\pi$.
}
\label{fig:trigonometric:5_sin_minus2}
\end{figure}

\begin{exercise}
\Figure{fig:trigonometric:5_sin_minus2} shows a graph of a sine
function that has a midline of $y = -2$.  The function has
$\tuple{\frac{\pi}{2}}{3}$ as one of its highest points.  If the
function is of the form $f(x) = a \sin x + b$, determine the values of
$a$ and $b$.
\end{exercise}

\ifbool{showSolution}{
\begin{solution}
Since the midline is $y = -2$, then $b = -2$.  The highest value of
$f(x)$ is obtained by adding the amplitude to the value of the
midline.  If the amplitude is $a$, then you have the equation
$-2 + a = 3$.  Solving the latter equation for $a$ yields
%%
\begin{align*}
a
&=
3 + 2 \\[4pt]
&=
5.
\end{align*}
%%
In other words, the amplitude is $a = 5$.  Therefore,
\Figure{fig:trigonometric:5_sin_minus2} shows a graph of the function
$f(x) = 5 \sin x - 2$.
\end{solution}
}{}

\begin{exercise}
A sine function $f(x) = a \sin x + b$ has
$\tuple{\frac{\pi}{2}}{\frac{15}{2}}$ and
$\tuple{\frac{3\pi}{2}}{-\frac{1}{2}}$ as some of its highest and
lowest points, respectively.  Determine the amplitude and midline of
$f(x)$.  Sketch a graph of $f(x)$ from $x = 0$ to $x = 2\pi$.
\end{exercise}

\ifbool{showSolution}{
\begin{solution}
Since $\frac{15}{2}$ is the highest value of $f(x)$ and $-\frac{1}{2}$
is the lowest value of $f(x)$, then the value of the midline is the
average of the highest and lowest values.  Thus the midline of $f(x)$
is
%%
\begin{align*}
y
&=
\frac{1}{2} \parenthesis*{\frac{15}{2} - \frac{1}{2}} \\[4pt]
&=
\frac{1}{2} \parenthesis*{\frac{15 - 1}{2}} \\[4pt]
&=
\frac{1}{2} \times 7 \\[4pt]
&=
\frac{7}{2}.
\end{align*}
%%
The amplitude of $f(x)$ is the difference between the highest value of
$f(x)$ and the value of the midline.  The amplitude is also the
difference between the value of the midline and the lowest value of
$f(x)$.  That is, the amplitude of $f(x)$ is
%%
\begin{align*}
a
&=
\frac{15}{2} - \frac{7}{2} \\[4pt]
&=
\frac{15 - 7}{2} \\[4pt]
&=
\frac{8}{2} \\[4pt]
&=
4.
\end{align*}
%%
Thus the function $f(x)$ can be written as
$f(x) = 4 \sin x + \frac{7}{2}$ and is graphed in
\Figure{fig:trigonometric:4_sin_7half}.

\begin{figure}[!htbp]
\centering
\includegraphics[scale=1.1]{image/13/4-sin-7half.pdf}
\caption{%%
  A graph of the function $f(x) = 4 \sin x + \frac{7}{2}$ from
  $x = 0$ to $x = 2\pi$.
}
\label{fig:trigonometric:4_sin_7half}
\end{figure}

\end{solution}
}{}

\begin{exercise}
Consider the function $f(x) = a \sin x$.  Graph the function when
$a = \triple{1}{\frac{3}{2}}{2}$.  Also graph the function when
$a = \triple{1}{\frac{1}{2}}{\frac{3}{4}}$.  Describe the effect of
the value of the amplitude on the graph of the sine function.
\end{exercise}

\ifbool{showSolution}{
\begin{solution}
\Figures{fig:trigonometric:sine_vertical_stretch}{fig:trigonometric:sine_vertical_compress}
show graphs of the function $f(x) = a \sin x$ when the values of the
amplitude are $a = \triple{1}{\frac{3}{2}}{2}$ and
$a = \triple{1}{\frac{3}{4}}{\frac{1}{2}}$.  As the figures show, when
the amplitude is $a > 1$, the graph of the sine function is stretched
in the vertical direction.  When the amplitude is $0 < a < 1$, the
graph of the sine function is compressed in the vertical direction.

\begin{figure}[!htbp]
\centering
\includegraphics[scale=1.1]{image/13/sin-vertical-stretch.pdf}
\caption{%%
  Graphs of the function $f(x) = a \sin x$ for
  $a = \triple{1}{\frac{3}{2}}{2}$.
}
\label{fig:trigonometric:sine_vertical_stretch}
\end{figure}

\begin{figure}[!htbp]
\centering
\includegraphics[scale=1.1]{image/13/sin-vertical-compress.pdf}
\caption{%%
  Graphs of the function $f(x) = a \sin x$ for
  $a = \triple{1}{\frac{3}{4}}{\frac{1}{2}}$.
}
\label{fig:trigonometric:sine_vertical_compress}
\end{figure}

\end{solution}
}{}


%%%%%%%%%%%%%%%%%%%%%%%%%%%%%%%%%%%%%%%%%%%%%%%%%%%%%%%%%%%%%%%%%%%%%%%%%%%

\section{Vertical shift of cosine function}

Let $a$ and $b$ be fixed real numbers such that $a \neq 0$ and let $x$
be a real variable.  The cosine function can be written as
\[
f(x)
=
a \cos x + b
\]
and is graphed in \Figure{fig:trigonometric:a_cos_b} from $x = -2\pi$
to $x = 2\pi$.  The \emph{midline} of $f(x)$ is the horizontal line
$y = b$.  The \emph{amplitude} of $f(x)$ is the absolute value
$\absoluteValue{a}$.  The highest value of $f(x)$ is obtained by
adding the amplitude to the value of the midline.  Similarly, the
lowest value of $f(x)$ is obtained by subtracting the amplitude from
the value of the midline. Thus the highest value of $f(x)$ is
$b + \absoluteValue{a}$ and the lowest value of $f(x)$ is
$b - \absoluteValue{a}$.  The value $b$ of the midline is halfway
between the highest and lowest values of $f(x)$.  If $h$ is the
highest value of $f(x)$ and $\ell$ is the lowest value of $f(x)$, then
the value of the midline is the average $\frac{h + \ell}{2}$.  The
above is summarised in
\Definition{def:trigonometric:cosine_amplitude_and_midline}.

\begin{definition}
\label{def:trigonometric:cosine_amplitude_and_midline}
\textbf{Amplitude and midline.}
Let $a$ and $b$ be real constants such that $a \neq 0$.  Given the
cosine function $f(x) = a \cos x + b$, the number $\absoluteValue{a}$
is called the \emph{amplitude} of $f(x)$ and the horizontal line $y =
b$ is called the \emph{midline} of $f(x)$.
\end{definition}

\begin{figure}[!htbp]
\centering
\includegraphics[scale=1.1]{image/13/a-cos-b.pdf}
\caption{%%
  A graph of the general cosine function $f(x) = a \cos x + b$ from
  $x = -2\pi$ to $x = 2\pi$.  The midline of $f(x)$ is the dashed
  horizontal line $y = b$.  The amplitude of $f(x)$ is the absolute
  value $\absoluteValue{a}$.
}
\label{fig:trigonometric:a_cos_b}
\end{figure}

Just as the value of the midline vertically shifts the graph of a sine
function, the value of the midline also vertically shifts the graph of
a cosine function.  If the value of the midline is positive,
i.e.~$b > 0$, then the graph of the cosine function will be shifted
upward.  On the other hand, if the value of the midline is negative,
i.e.~$b < 0$, the graph of the cosine function will be shifted
downward.  \Figure{fig:trigonometric:cos_vertical_shift} illustrates
the vertical shift of the graph of the cosine function.

\begin{figure}[!htbp]
\centering
\includegraphics[scale=1.1]{image/13/cos-vertical-shift.pdf}
\caption{%%
  The value of the midline has the effect of vertically shifting the
  graph of the cosine function.
}
\label{fig:trigonometric:cos_vertical_shift}
\end{figure}

\begin{figure}[!htbp]
\centering
\includegraphics[scale=1.1]{image/13/2-cos-3.pdf}
\caption{%%
  A graph of the function $f(x) = 2 \cos x + 1$ from $x = 0$ to
  $x = 2\pi$.
}
\label{fig:trigonometric:2_cos_1}
\end{figure}

\begin{example}
Consider the function $f(x) = 2 \cos x + 1$.  Determine the midline
and amplitude of $f(x)$.  Calculate the highest and lowest values of
the function.  Sketch a graph of $f(x)$ from $x = 0$ to $x = 2\pi$.
\end{example}

\begin{solution}
The midline of the function $f(x)$ is $y = 1$ and the amplitude is
$2$.  The midline is shown in \Figure{fig:trigonometric:2_cos_1} as a
dashed horizontal line.  The highest value of $f(x)$ is obtained by
adding the amplitude to the value of the midline.  Thus the highest
value of $f(x)$ is $1 + 2 = 3$.  The lowest value of $f(x)$ is
obtained by subtracting the amplitude from the value of the midline.
Then the lowest value of $f(x)$ is $1 - 2 = -1$.

To sketch a graph of $f(x)$ from $x = 0$ to $x = 2\pi$, you should
first determine all points of $f(x)$ at which the graph of $f(x)$
intersects the midline $y = 1$.  That is, you want all values of $x$
such that $\cos x = 0$.  The reason is that if $\cos x = 0$ then
$f(x)$ simplifies to $f(x) = 1$, which is $y$-coordinate of the
midline.  You know that $\cos x = 0$ whenever
$x = \pair{\frac{\pi}{2}}{\frac{3\pi}{2}}$.  Then the graph of $f(x)$
intersects the midline at the points $\tuple{\frac{\pi}{2}}{1}$ and
$\tuple{\frac{3\pi}{2}}{1}$.  The points of intersection are shown in
\Figure{fig:trigonometric:2_cos_1} as black dots.

Next, you should determine the highest and lowest points of $f(x)$.
The highest value of $\cos x$ is $1$, which occurs whenever
$x = \pair{0}{2\pi}$.  Then you have
$f(0) = f(2\pi) = 2 \times 1 + 1 = 3$, which is the highest value of
$f(x)$.  Thus the highest points of $f(x)$ are $\tuple{0}{3}$ and
$\tuple{2\pi}{3}$.  Furthermore, the lowest value of $\cos x$ is $-1$,
which occurs whenever $x = \pi$.  Then you have
$f(\pi) = 2 (-1) + 1 = -1$, which is the lowest value of $f(x)$.  Thus
the lowest point of $f(x)$ is $\tuple{\pi}{-1}$.  The highest and
lowest points of $f(x)$ are shown in
\Figure{fig:trigonometric:2_cos_1} as red dots.

Finally, draw a wave through the five points above to obtain the graph
shown in \Figure{fig:trigonometric:2_cos_1}.
\end{solution}

\begin{exercise}
Consider the function $f(x) = \frac{3}{2} \cos x - 1$.  Determine the
midline and amplitude of the function.  Calculate the highest and
lowest values of $f(x)$.  Sketch a graph of $f(x)$ from $x = 0$ to
$x = 2\pi$.
\end{exercise}

\ifbool{showSolution}{
\begin{solution}
The function $f(x)$ has a midline of $y = -1$ and an amplitude of
$3/2$.  \Figure{fig:trigonometric:3half_cos_minus1} shows the midline
as a dashed horizontal line.  The highest value of $f(x)$ is obtained
by adding the amplitude to the value of the midline.  That is, the
highest value of $f(x)$ is
%%
\begin{align*}
-1 + \frac{3}{2}
&=
\frac{-2}{2} + \frac{3}{2} \\[4pt]
&=
\frac{-2 + 3}{2} \\[4pt]
&=
\frac{1}{2}.
\end{align*}
%%
The lowest value of $f(x)$ is obtained by subtracting the amplitude
from the value of the midline.  In other words, the lowest value of
$f(x)$ is
%%
\begin{align*}
-1 - \frac{3}{2}
&=
\frac{-2}{2} - \frac{3}{2} \\[4pt]
&=
\frac{-2 - 3}{2} \\[4pt]
&=
-\frac{5}{2}.
\end{align*}

\begin{figure}[!htbp]
\centering
\includegraphics[scale=1.1]{image/13/3half-cos-minus1.pdf}
\caption{%%
  A graph of the function $f(x) = \frac{3}{2} \cos x - 1$ from $x = 0$
  to $x = 2\pi$.
}
\label{fig:trigonometric:3half_cos_minus1}
\end{figure}

To sketch a graph of $f(x)$ from $x = 0$ to $x = 2\pi$, first you
should determine all points at which the graph of $f(x)$ intersects
the midline.  This is the same as determining all values of $x$ such
that $\cos x = 0$.  You know that $\cos x = 0$ whenever
$x = \pair{\frac{\pi}{2}}{\frac{3\pi}{2}}$ and so you have
$f(\pi/2) = f(3\pi/2) = \frac{3}{2} \times 0 - 1 = -1$.  Thus the
graph of $f(x)$ intersects the midline at the points
$\tuple{\frac{\pi}{2}}{-1}$ and $\tuple{\frac{3\pi}{2}}{-1}$.  The
points of intersection are shown in
\Figure{fig:trigonometric:3half_cos_minus1} as black dots.

Next, you determine the highest and lowest points of $f(x)$.  The
highest value of $\cos x$ is $1$, which occurs whenever
$x = \pair{0}{2\pi}$.  Then you have
$f(0) = f(2\pi) = \frac{3}{2} \times 1 - 1 = \frac{1}{2}$.  The lowest
value of $\cos x$ is $-1$, which occurs whenever $x = \pi$.  Then you
have $f(\pi) = \frac{3}{2} (-1) - 1 = -\frac{5}{2}$.  In other words,
the highest points of $f(x)$ are $\tuple{0}{\frac{1}{2}}$ and
$\tuple{2\pi}{\frac{1}{2}}$.  The lowest point of $f(x)$ is
$\tuple{\pi}{-\frac{5}{2}}$.  The highest and lowest points are shown
in \Figure{fig:trigonometric:3half_cos_minus1} as red dots.

Finally, draw a wave through the above five points and you obtain the
graph in \Figure{fig:trigonometric:3half_cos_minus1}.
\end{solution}
}{}

\begin{exercise}
Consider the function $f(x) = -4 \cos x + 3$.  Determine the midline
and amplitude of $f(x)$.  Calculate the highest and lowest values of
the function.  Draw a graph of $f(x)$ from $x = 0$ to $x = 2\pi$.
\end{exercise}

\ifbool{showSolution}{
\begin{solution}
The midline of the function $f(x)$ is $y = 3$, which is shown in
\Figure{fig:trigonometric:minus4_cos_3} as a dashed horizontal line.
The amplitude of the function is the absolute value
$\absoluteValue{-4} = 4$.  The highest value of $f(x)$ is obtained by
adding the amplitude to the value of the midline.  Thus the highest
value of $f(x)$ is $3 + 4 = 7$.  The lowest value of $f(x)$ is
obtained by subtracting the amplitude from the value of the midline.
That is, the lowest value of $f(x)$ is $3 - 4 = -1$.

\begin{figure}[!htbp]
\centering
\includegraphics[scale=1.1]{image/13/minus4-cos-3.pdf}
\caption{%%
  A graph of the function $f(x) = -4 \cos x + 3$ from $x = 0$ to
  $x = 2\pi$.
}
\label{fig:trigonometric:minus4_cos_3}
\end{figure}

To draw a graph of $f(x)$ from $x = 0$ to $x = 2\pi$, you should first
determine all points at which the graph of $f(x)$ intersects the
midline.  This occurs for all values of $x$ such that $\cos x = 0$
because if $\cos x = 0$ then the function $f(x)$ simplifies to
$f(x) = 3$, which is the value of the midline.  You know that
$\cos x = 0$ whenever $x = \pair{\frac{\pi}{2}}{\frac{3\pi}{2}}$.
Then you have $f(\pi/2) = f(3\pi/2) = -4 \times 0 + 3 = 3$ and so the
graph of $f(x)$ intersects the midline at the points
$\tuple{\frac{\pi}{2}}{3}$ and $\tuple{\frac{3\pi}{2}}{3}$.  The
points of intersection are shown in
\Figure{fig:trigonometric:minus4_cos_3} as black dots.

Next, you should determine the highest and lowest points of $f(x)$.
Note that the highest value of $\cos x$ is $1$, which occurs whenever
$x = \pair{0}{2\pi}$.  Then you have
$f(0) = f(2\pi) = -4 \times 1 + 3 = -1$, which is the lowest value of
$f(x)$.  Thus the lowest points of $f(x)$ are $\tuple{0}{-1}$ and
$\tuple{2\pi}{-1}$.  Furthermore, the lowest value of $\cos x$ is
$-1$, which occurs whenever $x = \pi$.  Then you have
$f(\pi) = -4 (-1) + 3 = 7$, which is the highest value of $f(x)$.  In
other words, the highest point of $f(x)$ is $\tuple{\pi}{7}$.  The
highest and lowest points of $f(x)$ are shown in
\Figure{fig:trigonometric:minus4_cos_3} as red dots.

Finally, draw a wave through the above five points and you obtain the
graph shown in \Figure{fig:trigonometric:minus4_cos_3}.
\end{solution}
}{}

\begin{figure}[!htbp]
\centering
\includegraphics[scale=1.1]{image/13/half-cos-5half.pdf}
\caption{%%
  A graph of a cosine function of the form $f(x) = a \cos x + b$ from
  $x = 0$ to $x = 2\pi$.
}
\label{fig:trigonometric:half_cos_5half}
\end{figure}

\begin{exercise}
\Figure{fig:trigonometric:half_cos_5half} shows a graph of a cosine
function, whose midline is $y = 5/2$.  The function has $\tuple{0}{3}$
as one of its highest points.  If the function is of the form
$f(x) = a \cos x + b$, determine the values of $a$ and $b$.
\end{exercise}

\ifbool{showSolution}{
\begin{solution}
Since the midline is $y = 5/2$, then you have $b = 5/2$.  To get the
highest value of the function, you add the amplitude to the value of
the midline.  If the amplitude is $a$, then you have the equation
$\frac{5}{2} + a = 3$.  Solving the latter equation for $a$ shows that
%%
\begin{align*}
a
&=
3 - \frac{5}{2} \\[4pt]
&=
\frac{6}{2} - \frac{5}{2} \\[4pt]
&=
\frac{6 - 5}{2} \\[4pt]
&=
\frac{1}{2}.
\end{align*}
%%
In other words, the amplitude is $1/2$.  Thus
\Figure{fig:trigonometric:half_cos_5half} shows a graph of the
function $f(x) = \frac{1}{2} \cos x + \frac{5}{2}$.
\end{solution}
}{}

\begin{exercise}
A cosine function $f(x) = a \cos x + b$ has $\tuple{0}{\frac{11}{2}}$
and $\tuple{\pi}{\frac{1}{2}}$ as some of its highest and lowest
points, respectively.  Determine the amplitude and midline of $f(x)$.
Sketch a graph of $f(x)$ from $x = 0$ to $x = 2\pi$.
\end{exercise}

\ifbool{showSolution}{
\begin{solution}
The highest and lowest values of $f(x)$ are $\frac{11}{2}$ and
$\frac{1}{2}$, respectively.  The value of the midline of $f(x)$ is
the average of the highest and lowest values of $f(x)$.  Thus the
midline of $f(x)$ is
%%
\begin{align*}
y
&=
\frac{1}{2} \parenthesis*{\frac{11}{2} + \frac{1}{2}} \\[4pt]
&=
\frac{1}{2} \parenthesis*{\frac{11 + 1}{2}} \\[4pt]
&=
\frac{1}{2} \times 6 \\[4pt]
&=
3.
\end{align*}
%%
The amplitude of $f(x)$ is the difference between the highest value of
$f(x)$ and the value of the midline.  The amplitude can also be
calculated as the difference between the value of the midline and the
lowest value of $f(x)$.  That is, the amplitude of $f(x)$ is
%%
\begin{align*}
\frac{11}{2} - 3
&=
\frac{11}{2} - \frac{6}{2} \\[4pt]
&=
\frac{11 - 6}{2} \\[4pt]
&=
\frac{5}{2}.
\end{align*}
%%
Then the function $f(x)$ can be written as
$f(x) = \frac{5}{2} \cos x + 3$, which is graphed in
\Figure{fig:trigonometric:5half_cos_3}.

\begin{figure}[!htbp]
\centering
\includegraphics[scale=1.1]{image/13/5half-cos-3.pdf}
\caption{%%
  A graph of the function $f(x) = \frac{5}{2} \cos x + 3$ from $x = 0$
  to $x = 2\pi$.
}
\label{fig:trigonometric:5half_cos_3}
\end{figure}

\end{solution}
}{}

\begin{exercise}
Consider the function $f(x) = a \cos x$.  Graph the function for
$a = \triple{1}{\frac{3}{2}}{2}$.  Also graph the function for
$a = \triple{1}{\frac{1}{2}}{\frac{3}{4}}$.  Describe the effect of
the value of the amplitude on the graph of the cosine function.
\end{exercise}

\ifbool{showSolution}{
\begin{solution}
\Figures{fig:trigonometric:cos_vertical_stretch}{fig:trigonometric:cos_vertical_compress}
show graphs of the function $f(x) = a \cos x$ for
$a = \triple{1}{\frac{3}{2}}{2}$ and for
$a = \triple{1}{\frac{1}{2}}{\frac{3}{4}}$.  When the amplitude is
$a > 1$, the graph of the cosine function is stretched vertically.
However, when the amplitude is $0 < a < 1$, the graph of the cosine
function is compressed vertically.  That is, the value of the
amplitude has the effect of stretching or compressing in the vertical
direction the graph of the cosine function.

\begin{figure}[!htbp]
\centering
\includegraphics[scale=1.1]{image/13/cos-vertical-stretch.pdf}
\caption{%%
  Graphs of the function $f(x) = a \cos x$ for
  $a = \triple{1}{\frac{3}{2}}{2}$.
}
\label{fig:trigonometric:cos_vertical_stretch}
\end{figure}

\begin{figure}[!htbp]
\centering
\includegraphics[scale=1.1]{image/13/cos-vertical-compress.pdf}
\caption{%%
  Graphs of the function $f(x) = a \cos x$ for
  $a = \triple{1}{\frac{1}{2}}{\frac{3}{4}}$.
}
\label{fig:trigonometric:cos_vertical_compress}
\end{figure}

\end{solution}
}{}


%%%%%%%%%%%%%%%%%%%%%%%%%%%%%%%%%%%%%%%%%%%%%%%%%%%%%%%%%%%%%%%%%%%%%%%%%%%

\section{Period and frequency}

Let $\triple{a}{b}{c}$ be real constants such that $a \neq 0$ and
$c \neq 0$ and let $x$ be a real variable.  Consider the sine and
cosine functions of the form
\[
f(x)
=
a \sin(cx) + b
%%
\qquad
\text{and}
\qquad
%%
g(x)
=
a \cos(cx) + b.
\]
You already know that the amplitude is the absolute value
$\absoluteValue{a}$ and the midline is $y = b$.  What effect does the
value of $c$ have on the graph of $f(x)$ and $g(x)$?  To understand
how the value of $c$ affects the graph of the sine and cosine
functions, consider the simpler functions $F(x) = \sin(cx)$ and
$G(x) = \cos(cx)$.
\Figure{subfig:trigonometric:sine_horizontal_stretch} shows graphs of
$F(x)$ for $c = \triple{1}{\frac{3}{4}}{\frac{1}{2}}$ and
\Figure{subfig:trigonometric:sine_horizontal_compress} shows some
graphs of $F(x)$ for $c = \triple{1}{\frac{3}{2}}{2}$.  The figures
show that when $0 < c < 1$ the value of $c$ stretches the graph of the
sine function along the horizontal direction.  However, when $c > 1$
the value of $c$ horizontally compresses the graph of the sine
function.  Thus the value of $c > 0$ has the effect of stretching or
compressing along the horizontal axis the graph of the sine function.

\begin{figure}[!htbp]
\centering
\subfigure[]{
  \includegraphics[scale=1.1]{image/13/sin-horizontal-stretch.pdf}
  \label{subfig:trigonometric:sine_horizontal_stretch}
}
%%
%%
\subfigure[]{
  \includegraphics[scale=1.1]{image/13/sin-horizontal-compress.pdf}
  \label{subfig:trigonometric:sine_horizontal_compress}
}
\caption{%%
  Various graphs of the sine function $F(x) = \sin(cx)$ for
  (a)~$c = \triple{1}{\frac{3}{4}}{\frac{1}{2}}$ and
  (b)~$c = \triple{1}{\frac{3}{2}}{2}$.
}
\label{fig:trigonometric:sine_horizontal_stretch_compress}
\end{figure}

\begin{exercise}
Graph the cosine function $G(x) = \cos(cx)$ for
$c = \triple{1}{\frac{3}{4}}{\frac{1}{2}}$.  Also graph $G(x)$ for
$c = \triple{1}{\frac{4}{3}}{2}$.  If $c > 0$, describe the effect of
the value of $c$ on the graph of the cosine function.
\end{exercise}

\ifbool{showSolution}{
\begin{solution}
\Figures{subfig:trigonometric:cosine_horizontal_stretch}{subfig:trigonometric:cosine_horizontal_compress}
show various graphs of the cosine function $G(x) = \cos(cx)$ for
$c = \triple{1}{\frac{3}{4}}{\frac{1}{2}}$ and
$c = \triple{1}{\frac{4}{3}}{2}$.  As can be seen from the figures,
when $0 < c < 1$ the value of $c$ effectively stretches along the
horizontal direction the graph of the cosine function.  However, when
$c > 1$ the value of $c$ compresses along the horizontal direction the
graph of the cosine function.  In effect, the value of $c > 0$
stretches or compresses the graph of the cosine function along the
horizontal axis.

\begin{figure}[!htbp]
\centering
\subfigure[]{
  \includegraphics[scale=1.1]{image/13/cos-horizontal-stretch.pdf}
  \label{subfig:trigonometric:cosine_horizontal_stretch}
}
%%
%%
\subfigure[]{
  \includegraphics[scale=1.1]{image/13/cos-horizontal-compress.pdf}
  \label{subfig:trigonometric:cosine_horizontal_compress}
}
\caption{%%
  Various graphs of the cosine function $G(x) = \cos(cx)$ for
  (a)~$c = \triple{1}{\frac{3}{4}}{\frac{1}{2}}$ and
  (b)~$c = \triple{1}{\frac{4}{3}}{2}$.
}
\label{fig:trigonometric:cosine_horizontal_stretch_compress}
\end{figure}

\end{solution}
}{}

\begin{figure}[!htbp]
\centering
\includegraphics[scale=1.1]{image/13/sin-cos-cycle-default.pdf}
\caption{%%
  The functions $f(x) = a \sin x + b$ and $g(x) = a \cos x + b$ each
  requires $2\pi$ units of time to complete one cycle.
}
\label{fig:trigonometric:sine_cosine_one_cycle}
\end{figure}

The effect of the factor $c$ can also be understood in terms of the
amount of time required for the graph of a sine or cosine function to
complete one cycle.  Let $a \neq 0$ and $b$ be fixed real numbers and
suppose that $x$ is a real variable.
\Figure{fig:trigonometric:sine_cosine_one_cycle} shows graphs of the
functions $f(x) = a \sin x + b$ and $g(x) = a \cos x + b$.  If you
interpret the variable $x$ as representing time, then
\Figure{fig:trigonometric:sine_cosine_one_cycle} shows that the graph
of each of the functions $f(x)$ and $g(x)$ requires $2\pi$ units of
time to complete one cycle.  Now suppose that $c > 0$ is a fixed real
number and consider the functions $F(x) = a \sin(cx) + b$ and
$G(x) = a \cos(cx) + b$.  The effect of the factor $c$ is to change
the amount of time required for the graph of a sine or cosine function
to complete one cycle.  In many applications of trigonometric
functions, the amount of time required for the graph of a sine or
cosine function to complete one cycle is also called the
\emph{wavelength}.\footnote{
  See the video at
  \url{https://youtu.be/E-SPpUhzYZY}.
}

\begin{figure}[!htbp]
\centering
\includegraphics[scale=1.1]{image/13/sin-cycle.pdf}
\caption{%%
  Graphs of the sine function $F(x) = a \sin(cx) + b$ for
  $c = \triple{\frac{1}{2}}{1}{2}$.
}
\label{fig:trigonometric:sine_cycle}
\end{figure}

For example, \Figure{fig:trigonometric:sine_cycle} shows various
graphs of $F(x)$ for $c = \triple{\frac{1}{2}}{1}{2}$.  When $c = 1$
you have $F_1(x) = f(x) = a \sin x + b$, the graph of which requires
$2\pi$ units of time to complete one cycle.  When $c = 2$ you have
$F_2(x) = a \sin(2x) + b$, whose graph requires $\pi$ units of time to
complete one cycle.  Finally, when $c = \frac{1}{2}$ you have
$F_3(x) = a \sin(\frac{x}{2}) + b$, whose graph requires $4\pi$ units
of time to complete one cycle.  As can be seen from
\Figure{fig:trigonometric:sine_cycle}, when $0 < c < 1$ the graph of
$F(x)$ requires more time than the graph of $f(x)$ to complete one
cycle, which is due to the fact that $c$ horizontally stretches the
graph of the sine function.  However, when $c > 1$ the graph of $F(x)$
requires less time than the graph of $f(x)$ to complete one cycle, the
reason being that $c$ horizontally compresses the graph of the sine
function.  In other words, the value of $c > 0$ not only stretches or
compresses the graph of a sine function along the horizontal axis, the
number also shortens or lengthens the amount of time required for the
graph of a sine function to complete one cycle.
\Definition{def:trigonometric:sine_cosine_period} captures the idea of
the amount of time required for the graph of a sine or cosine function
to complete one cycle.

\begin{definition}
\label{def:trigonometric:sine_cosine_period}
\textbf{Period.}
Let $\triple{a}{b}{c}$ be real constants such that $a \neq 0$ and
$c \neq 0$.  Let $x$ be a real variable and consider the functions
\[
f(x)
=
a \sin(cx) + b
%%
\qquad
\text{and}
\qquad
%%
g(x)
=
a \cos(cx) + b.
\]
The amount of time required for the graphs of $f(x)$ and $g(x)$ to
complete one cycle is called the \emph{period} of the functions.  The
period of $f(x)$ and $g(x)$ is defined as $2\pi / \absoluteValue{c}$.
\end{definition}

\begin{exercise}
Calculate the period of $f(x) = \sin(3x)$.
\end{exercise}

\ifbool{showSolution}{
\begin{solution}
In the function $f(x) = \sin(3x)$, your $c$ value is $c = 3$.  Then
$f(x)$ has a period of $2\pi / 3$.
\end{solution}
}{}

\begin{exercise}
Consider the function $f(x) = \cos(cx)$.  Graph the function for
$c = \triple{\frac{1}{2}}{1}{\frac{4}{3}}$, where
$0 \leq x \leq 2\pi$.  When $c = 1$, how long does the graph of $f(x)$
take to complete one cycle?  Determine the period of $f(x)$ when
$c = \frac{4}{3}$.
\end{exercise}

\ifbool{showSolution}{
\begin{solution}
See \Figure{fig:trigonometric:cosine_cycle}.  When $c = 1$, the graph
of $f(x)$ requires $2\pi$ units of time to complete one cycle.  When
$c = \frac{4}{3}$, the period of $f(x)$ is
%%
\begin{align*}
\frac{2\pi}{4/3}
&=
2\pi \div \frac{4}{3} \\[4pt]
&=
2\pi \times \frac{3}{4} \\[4pt]
&=
\frac{3\pi}{2}.
\end{align*}
%%
In other words, when $c = \frac{4}{3}$ the graph of $f(x)$ requires
$\frac{3\pi}{2}$ units of time to complete one cycle.

\begin{figure}[!htbp]
\centering
\includegraphics[scale=1.1]{image/13/cos-cycle.pdf}
\caption{%%
  Graphs of the function $f(x) = \cos(cx)$ for
  $c = \triple{\frac{1}{2}}{1}{\frac{4}{3}}$.
}
\label{fig:trigonometric:cosine_cycle}
\end{figure}

\end{solution}
}{}

\begin{exercise}
Let $c > 0$ be a fixed real number and consider the functions
$f(x) = \sin(cx)$ and $g(x) = \cos(-cx)$.  Prove that $f(x)$ and
$g(x)$ have the same period.
\end{exercise}

\ifbool{showSolution}{
\begin{solution}
The period of $f(x)$ is $2\pi / \absoluteValue{c}$, which simplifies
to $2\pi / c$ because $c$ is assumed to be positive.  The period of
$g(x)$ is $2\pi / \absoluteValue{-c}$.  Since $c$ is positive, then
$-c$ is negative and you have $\absoluteValue{-c} = c$.  Thus the
period of $g(x)$ is $2\pi / \absoluteValue{-c} = 2\pi / c$, which is
the same as the period of $f(x)$.
\end{solution}
}{}

\begin{exercise}
Let $\triple{a}{b}{c}$ be real constants such that $a \neq 0$ and
consider the function $f(x) = a \sin(cx) + b$.  Determine all values
of $c$ such that the period of $f(x)$ is $\frac{\pi}{2}$.
\end{exercise}

\ifbool{showSolution}{
\begin{solution}
The period of $f(x)$ can be written as the equation
$\frac{2\pi}{\absoluteValue{c}} = \frac{\pi}{2}$.  Multiply both sides
by $\absoluteValue{c}$ to get
$2\pi = \frac{\pi}{2} \absoluteValue{c}$, which upon solving for
$\absoluteValue{c}$ results in
%%
\begin{align*}
\absoluteValue{c}
&=
\frac{2}{\pi} \times 2\pi \\[4pt]
&=
4.
\end{align*}
%%
This means that when $c = 4$ the function $f(x)$ has a period of
$\frac{\pi}{2}$ because
\[
\frac{2\pi}{\absoluteValue{4}}
=
\frac{2\pi}{4}
=
\frac{\pi}{2}.
\]
Furthermore, when $c = -4$ the function $f(x)$ also has a period of
$\frac{\pi}{2}$.  The reason is that
\[
\frac{2\pi}{\absoluteValue{-4}}
=
\frac{2\pi}{4}
=
\frac{\pi}{2}.
\]
Therefore when $c = \pm 4$ the function $f(x)$ has a period of
$\frac{\pi}{2}$.
\end{solution}
}{}

\begin{figure}[!htbp]
\centering
\includegraphics[scale=1.1]{image/13/half-sin-3pi.pdf}
\caption{%%
  A trigonometric function that has a period of $3\pi$.
}
\label{fig:trigonometric:half_sine_3pi}
\end{figure}

\begin{example}
Determine a formula for the function graphed in
\Figure{fig:trigonometric:half_sine_3pi}.
\end{example}

\begin{solution}
\Figure{fig:trigonometric:half_sine_3pi} shows a graph of a sine
function whose midline is $y = 0$, which means that the figure
possibly shows a function of the form $f(x) = a \sin(cx)$.  The
highest and lowest values of the function are $\frac{1}{2}$ and
$-\frac{1}{2}$, respectively, so that the amplitude is
$a = \frac{1}{2}$.  The graph of the function requires $3\pi$ units of
time to complete one cycle, thus the function has a period of $3\pi$.
The value of $c$ is related to the period by the equation
$2\pi / \absoluteValue{c} = 3\pi$.  Solve the latter equation for
$\absoluteValue{c}$ to get
\[
\absoluteValue{c}
=
\frac{2\pi}{3\pi}
=
\frac{2}{3}.
\]
You may assume that $c > 0$.  Therefore it is possible that
\Figure{fig:trigonometric:half_sine_3pi} shows a graph of the function
$f(x) = \frac{1}{2} \sin(\frac{2}{3} x)$.
\end{solution}

\begin{figure}[!htbp]
\centering
\includegraphics[scale=1.1]{image/13/3quarter-cos-5pi.pdf}
\caption{%%
  A trigonometric function whose period is $5\pi$.
}
\label{fig:trigonometric:3quarter_cos_5pi}
\end{figure}

\begin{exercise}
Determine a formula for the function graphed in
\Figure{fig:trigonometric:3quarter_cos_5pi}.
\end{exercise}

\ifbool{showSolution}{
\begin{solution}
\Figure{fig:trigonometric:3quarter_cos_5pi} possibly shows a graph of
a cosine function whose midline is $y = 0$.  Then the function shown
in the figure might be of the form $f(x) = a \cos(cx)$.  The highest
and lowest values of the function are $\frac{3}{4}$ and
$-\frac{3}{4}$, so that the amplitude is $a = \frac{3}{4}$.  If you
assume that $c > 0$, then the value of $c$ is related to the period by
the equation $\frac{2\pi}{c} = 5\pi$.  Solving the latter equation for
$c$ shows that
\[
c
=
\frac{2\pi}{5\pi}
=
\frac{2}{5}.
\]
Therefore the function whose graph is shown in
\Figure{fig:trigonometric:3quarter_cos_5pi} can be described by the
formula $f(x) = \frac{3}{4} \cos(\frac{2}{5} x)$.
\end{solution}
}{}

The \emph{frequency} of a sine or cosine function is a concept that is
closely related to the period.  The frequency counts the number of
cycles per unit time.  For example, if the graph of a sine function
$f(x)$ requires one second to complete one cycle, you would say that
the frequency of the function is one cycle per second.  In various
applications of the sine and cosine functions, the number of cycles
per second is usually expressed in terms of \emph{hertz} and written
as Hz.  In the above example, you would say that the function $f(x)$
has a frequency of $1$~Hz.  As another example, suppose that the graph
of a cosine function $g(x)$ completes one cycle in $0.5$ seconds.
Then in one second the graph of $g(x)$ would complete two cycles and
hence you would say that the frequency of $g(x)$ is $2$~Hz.
\Definition{def:trigonometric:sine_cosine_frequency} provides a way to
calculate the frequencies of sine and cosine functions.

\begin{definition}
\label{def:trigonometric:sine_cosine_frequency}
\textbf{Frequency.}
Let $\triple{a}{b}{c}$ be real constants such that $a \neq 0$ and
$c \neq 0$ and let $x$ be a real variable.  Consider the functions
\[
f(x)
=
a \sin(cx) + b
%%
\qquad
\text{and}
\qquad
%%
g(x)
=
a \cos(cx) + b
\]
each of which has a period of $p = 2 \pi / \absoluteValue{c}$.  The
\emph{frequency} of $f(x)$ and $g(x)$ is the number
%%
\begin{equation}
\label{eqn:trigonometric:frequency_sin_cos}
\frac{1}{p}
=
\frac{\absoluteValue{c}}{2\pi}.
\end{equation}
\end{definition}

\begin{exercise}
Read about hertz and Heinrich Hertz at the following websites:
%%
\begin{packeditem}
\item \url{https://en.wikipedia.org/wiki/Hertz}

\item \url{https://en.wikipedia.org/wiki/Heinrich_Hertz}
\end{packeditem}
\end{exercise}

\begin{example}
Let $c \neq 0$ be a real constant and consider the function
$f(x) = \cos(cx)$.  Suppose that the graph of $f(x)$ completes one
cycle in $0.25$ seconds.
%%
\begin{packedenum}
\item\label{subeg:trigonometric:cos_1_cycle_0.25_seconds_period}
  Determine the period and frequency of $f(x)$.

\item\label{subeg:trigonometric:cos_1_cycle_0.25_seconds_c_factor}
  Calculate the possible values for $c$.
\end{packedenum}
\end{example}

\begin{solution}
\solutionpart{subeg:trigonometric:cos_1_cycle_0.25_seconds_period}
The period of $f(x)$ is the amount of time required for the graph of
$f(x)$ to complete one cycle.  Since it is assumed that the graph of
$f(x)$ completes one cycle in $0.25 = 1/4$ seconds, then $f(x)$ has a
period of $p = 1/4$ seconds.  The frequency of $f(x)$ is
%%
\begin{align*}
\frac{1}{p}
&=
1 \div \frac{1}{4} \\[4pt]
&=
1 \times \frac{4}{1} \\[4pt]
&=
4
\end{align*}
%%
cycles per second or $4$~Hz.

\solutionpart{subeg:trigonometric:cos_1_cycle_0.25_seconds_c_factor}
Let $p$ be the period of $f(x)$ so that the frequency of $f(x)$ is
$1 / p$.  From \Equation{eqn:trigonometric:frequency_sin_cos}, you
know that the frequency of $f(x)$ is related to the value
$\absoluteValue{c}$ as follows:
\[
\frac{1}{p}
=
\frac{\absoluteValue{c}}{2\pi}.
\]
From \Part{subeg:trigonometric:cos_1_cycle_0.25_seconds_period} you
know that $f(x)$ has a frequency of $4$~Hz, which can be used to
simplify the latter equation to $4 = \frac{\absoluteValue{c}}{2\pi}$.
Solve the last equation for $\absoluteValue{c}$ to get
$\absoluteValue{c} = 4 \times 2\pi = 8\pi$.  If $c = 8\pi$ then you
would get $\absoluteValue{c} = \absoluteValue{8\pi} = 8\pi$.
Furthermore, if $c = -8\pi$ then you would have
$\absoluteValue{c} = \absoluteValue{-8\pi} = 8\pi$.  Therefore the
possible values for $c$ are $c = \pm 8\pi$.
\end{solution}

\begin{exercise}
Consider the function $f(x) = \sin \frac{x}{2}$.
%%
\begin{packedenum}
\item\label{subeg:trigonometric:sin_half_period}
  Calculate the period of $f(x)$.

\item\label{subeg:trigonometric:sin_half_frequency}
  Use the period to determine the frequency of $f(x)$.
\end{packedenum}
\end{exercise}

\ifbool{showSolution}{
\begin{solution}
\solutionpart{subeg:trigonometric:sin_half_period}
Your $c$ value is $c = \frac{1}{2}$ and so the required period is
%%
\begin{align*}
\frac{2\pi}{1/2}
&=
2\pi \div \frac{1}{2} \\[4pt]
&=
2\pi \times 2 \\[4pt]
&=
4\pi.
\end{align*}

\solutionpart{subeg:trigonometric:sin_half_frequency}
Since the period is $p = 4\pi$, then the frequency is
$1/p = \frac{1}{4\pi}$.
\end{solution}
}{}

\begin{exercise}
A trigonometric function $f(x)$ has an amplitude of $3$, a midline of
$y = 4$, and a period of $\frac{4\pi}{3}$.  Determine two possible
formulae for $f(x)$.  Calculate the frequency of $f(x)$.
\end{exercise}

\ifbool{showSolution}{
\begin{solution}
You are given an amplitude of $\absoluteValue{a} = 3$ and the midline
value $b = 4$.  If $c$ is a real constant, then $f(x)$ is possibly a
sine function that can be written as $f(x) = \pm 3 \sin(cx) + 4$ or a
cosine function of the form $f(x) = \pm 3 \cos(cx) + 4$.  The
remaining problem is to determine a value for $c$.

The period is related to the value of $c$ via the equation
$\frac{2\pi}{\absoluteValue{c}} = \frac{4\pi}{3}$.  Multiply both
sides by $\absoluteValue{c}$ and simplify to get
$2\pi = \frac{4\pi}{3} \absoluteValue{c}$.  To solve for
$\absoluteValue{c}$, first multiply both sides by $3$, then divide
both sides by $4\pi$, and finally simplify to obtain
%%
\begin{align*}
\absoluteValue{c}
&=
\frac{3}{4\pi} \times 2\pi \\[4pt]
&=
\frac{3 \times 2\pi}{4\pi} \\[4pt]
&=
\frac{3}{2}.
\end{align*}
%%
The possible values for $c$ are $c = \pm \frac{3}{2}$.  If you assume
that $c > 0$, then you have $c = \frac{3}{2}$ so that $f(x)$ is
possibly the sine function $f(x) = \pm 3 \sin(\frac{3}{2}x) + 4$ or
the cosine function $f(x) = \pm 3 \cos(\frac{3}{2}x) + 4$.

If $p$ is the period of $f(x)$, then the frequency of $f(x)$ is
defined as the number $1 / p$.  Since $f(x)$ has a period of
$p = \frac{4\pi}{3}$, then $f(x)$ has a frequency of
%%
\begin{align*}
\frac{1}{p}
&=
\frac{1}{4\pi/3} \\[4pt]
&=
1 \times \frac{3}{4\pi} \\[4pt]
&=
\frac{3}{4\pi}.
\end{align*}
\end{solution}
}{}


%%%%%%%%%%%%%%%%%%%%%%%%%%%%%%%%%%%%%%%%%%%%%%%%%%%%%%%%%%%%%%%%%%%%%%%%%%%

\section{Horizontal shift}

In this section, you will investigate another property of the graphs
of sine and cosine functions.  Up to this point, you have learnt about
how to stretch the graph of a sine or cosine function along the
horizontal and vertical axes.  You have also learnt about how to
vertically shift the graph of a sine or cosine function.  What remains
to be discussed is how to shift the graph of a sine or cosine function
along the horizontal axis.

\begin{figure}[!htbp]
\centering
\includegraphics[scale=1.1]{image/13/sin-cos.pdf}
\caption{%%
  Graphs of the trigonometric functions $f(x) = \sin x$ and
  $g(x) = \cos x$ from $x = 0$ to $x = 3\pi$.
}
\label{fig:trigonometric:standard_sine_cosine}
\end{figure}

What does it mean to horizontally shift the graph of a trigonometric
function?  To answer this question, consider
\Figure{fig:trigonometric:standard_sine_cosine} which shows graphs of
the functions $f(x) = \sin x$ and $g(x) = \cos x$.  Note that you can
obtain the graph of the cosine function $g(x)$ by shifting the graph
of the sine function $f(x)$ to the left.  Looking at the graphs in
\Figure{fig:trigonometric:standard_sine_cosine}, the required amount
of horizontal shift is approximately $\frac{\pi}{2}$ units to the
left.  Thus it is reasonable to assume that the cosine function $g(x)$
can be written in terms of the sine function as
$h(x) = \sin(x + \frac{\pi}{2})$ and therefore you might conclude that
$\cos x = \sin(x + \frac{\pi}{2})$.

\begin{exercise}
You will verify that the equation $\cos x = \sin(x + \frac{\pi}{2})$
is true, especially for some special values of $x$.
%%
\begin{packedenum}
\item\label{subex:trigonometric:sin_left_shift_0}
  Let $x = 0$.  Calculate the exact values of $\cos x$ and
  $\sin(x + \frac{\pi}{2})$.

\item\label{subex:trigonometric:sin_left_shift_pi_half}
  Let $x = \frac{\pi}{2}$.  Calculate the exact values of $\cos x$ and
  $\sin(x + \frac{\pi}{2})$.

\item\label{subex:trigonometric:sin_left_shift_pi}
  Let $x = \pi$.  Calculate the exact values of $\cos x$ and
  $\sin(x + \frac{\pi}{2})$.

\item\label{subex:trigonometric:sin_left_shift_3pi_half}
  Let $x = \frac{3\pi}{2}$.  Calculate the exact values of $\cos x$
  and $\sin(x + \frac{\pi}{2})$.
\end{packedenum}
\end{exercise}

\ifbool{showSolution}{
\begin{solution}
\solutionpart{subex:trigonometric:sin_left_shift_0}
If $x = 0$, then $\sin(0 + \frac{\pi}{2}) = \sin\frac{\pi}{2} = 1$.
You also know that $\cos 0 = 1$.

\solutionpart{subex:trigonometric:sin_left_shift_pi_half}
If $x = \frac{\pi}{2}$, then
$\sin(\frac{\pi}{2} + \frac{\pi}{2}) = \sin \pi = 0$.  Furthermore,
you have $\cos \frac{\pi}{2} = 0$.

\solutionpart{subex:trigonometric:sin_left_shift_pi}
If $x = \pi$, then
$\sin(\pi + \frac{\pi}{2}) = \sin \frac{3\pi}{2} = -1$.  You also have
$\cos \pi = -1$.

\solutionpart{subex:trigonometric:sin_left_shift_3pi_half}
If $x = \frac{3\pi}{2}$, then
$\sin(\frac{3\pi}{2} + \frac{\pi}{2}) = \sin 2\pi = 0$.  You also have
$\cos \frac{3\pi}{2} = 0$.
\end{solution}
}{}

\Figure{fig:trigonometric:standard_sine_cosine} shows that you can
also obtain the graph of the function $f(x) = \sin x$ in terms of the
function $g(x) = \cos x$.  To do so, you can shift the graph of the
cosine function $g(x)$ along the horizontal axis.  The amount of
horizontal shift is approximately $\frac{\pi}{2}$ units to the right.
In other words, it is reasonable to assume that the sine function
$f(x)$ can be written in terms of the cosine function as
$h(x) = \cos(x - \frac{\pi}{2})$.  Therefore you might conclude that
$\sin x = \cos(x - \frac{\pi}{2})$.
\Definition{def:trigonometric:horizontal_shift} summarised the idea of
the horizontal shift of the graph of a trigonometric function.

\begin{exercise}
You will verify that the equation $\sin x = \cos(x - \frac{\pi}{2})$
is true, especially for certain special values of $x$.
%%
\begin{packedenum}
\item\label{subex:trigonometric:cos_right_shift_0}
  If $x = 0$, calculate the exact values of $\sin x$ and
  $\cos(x - \frac{\pi}{2})$.

\item\label{subex:trigonometric:cos_right_shift_pi_half}
  If $x = \frac{\pi}{2}$, calculate the exact values of $\sin x$ and
  $\cos(x - \frac{\pi}{2})$.

\item\label{subex:trigonometric:cos_right_shift_pi}
  If $x = \pi$, calculate the exact values of $\sin x$ and
  $\cos(x - \frac{\pi}{2})$.

\item\label{subex:trigonometric:cos_right_shift_3pi_half}
  If $x = \frac{3\pi}{2}$, calculate the exact values of $\sin x$ and
  $\cos(x - \frac{\pi}{2})$.
\end{packedenum}
\end{exercise}

\ifbool{showSolution}{
\begin{solution}
\solutionpart{subex:trigonometric:cos_right_shift_0}
If $x = 0$, then $\sin 0 = 0$.  Note that
$0 - \frac{\pi}{2} = -\frac{\pi}{2}$ radians is equivalent to
$\frac{3\pi}{2}$ radians.  Then
$\cos(0 - \frac{\pi}{2}) = \cos \frac{3\pi}{2} = 0$.

\solutionpart{subex:trigonometric:cos_right_shift_pi_half}
If $x = \frac{\pi}{2}$, then $\sin \frac{\pi}{2} = 1$.  Note that you
have $\frac{\pi}{2} - \frac{\pi}{2} = 0$.  Then you have
$\cos(\frac{\pi}{2} - \frac{\pi}{2}) = \cos 0 = 1$.

\solutionpart{subex:trigonometric:cos_right_shift_pi}
If $x = \pi$, then $\sin \pi = 0$.  Note that
$\pi - \frac{\pi}{2} = \frac{\pi}{2}$, which can be used to write
$\cos(\pi - \frac{\pi}{2}) = \cos \frac{\pi}{2} = 0$.

\solutionpart{subex:trigonometric:cos_right_shift_3pi_half}
If $x = \frac{3\pi}{2}$, then $\sin \frac{3\pi}{2} = -1$.  Note that
you have $\frac{3\pi}{2} - \frac{\pi}{2} = \pi$.  Use the latter
result to write
$\cos(\frac{3\pi}{2} - \frac{\pi}{2}) = \cos \pi = -1$.
\end{solution}
}{}

\begin{definition}
\label{def:trigonometric:horizontal_shift}
\textbf{Horizontal shift.}
Let $\quadruple{a}{b}{c}{h}$ be fixed real numbers such that
$a \neq 0$ and $c \neq 0$.  Let $x$ be a real variable and consider
the functions
\[
f(x)
=
a \sin\bigparen{c (x - h)} + b
%%
\qquad
\text{and}
\qquad
%%
g(x)
=
a \cos\bigparen{c (x - h)} + b.
\]
The graphs of $f(x)$ and $g(x)$ are the graphs of
$F(x) = a \sin(cx) + b$ and $G(x) = a \cos(cx) + b$, respectively,
that have been shifted horizontally by $h$ units.
\end{definition}

\begin{example}
Consider the function
$f(x) = \frac{3}{2} \sin(2x - \frac{\pi}{5}) + \frac{1}{2}$.
%%
\begin{packedenum}
\item\label{subeg:trigonometric:sin_horizontal_shift_standard_form}
  Write the function $f(x)$ in the form
  $f(x) = a \sin\bigparen{c (x - h)} + b$.

\item\label{subeg:trigonometric:sin_horizontal_shift_describe}
  Describe the graph of the function $f(x)$.

\item\label{subeg:trigonometric:sin_horizontal_shift_graph_unshifted}
  Let $g(x) = a \sin(cx) + b$ be the function $f(x)$ that has not been
  shifted horizontally.  On the same set of axes, graph $f(x)$ and
  $g(x)$ from $x = 0$ to $x = p$, where $p$ is the period of both
  $f(x)$ and $g(x)$.
\end{packedenum}
\end{example}

\begin{solution}
\solutionpart{subeg:trigonometric:sin_horizontal_shift_standard_form}
To write the function $f(x)$ in the form
$f(x) = a \sin\bigparen{c (x - h)} + b$, you must factorise the $2$ in
the expression $2x - \frac{\pi}{5}$.  Then you have the factorisation
\[
2x - \frac{\pi}{5}
=
2
\parenthesis*{x - \frac{\pi}{10}}
\]
which can be used to write $f(x)$ as
%%
\begin{align*}
f(x)
&=
\frac{3}{2} \sinp{2x - \frac{\pi}{5}} + \frac{1}{2} \\[4pt]
&=
\frac{3}{2} \sinp{2 \parenthesis*{x - \frac{\pi}{10}}} + \frac{1}{2}.
\end{align*}
%%
Here, you have $h = \frac{\pi}{10}$.

\solutionpart{subeg:trigonometric:sin_horizontal_shift_describe}
Let $g(x) = \frac{3}{2} \sin(2x) + \frac{1}{2}$.  The graph of the
function $f(x)$ is obtained by shifting the graph of $g(x)$ along the
horizontal axis.  The amount of horizontal shift is $\frac{\pi}{10}$
units to the right.

\begin{figure}[!htbp]
\centering
\includegraphics[scale=1.1]{image/13/sin-right-shift-pi-10.pdf}
\caption{%%
  Graphs of the sine functions
  $f(x) = \frac{3}{2} \sin(2x - \frac{\pi}{5}) + \frac{1}{2}$ and
  $g(x) = \frac{3}{2} \sin(2x) + \frac{1}{2}$.  The graph of $f(x)$ is
  obtained by shifting the graph of $g(x)$ along the horizontal axis
  by $\pi/10$ units to the right.
}
\label{fig:trigonometric:sin_right_shift_pi_10}
\end{figure}

\solutionpart{subeg:trigonometric:sin_horizontal_shift_graph_unshifted}
Your $c$ value is $c = 2$.  The period of both $f(x)$ and $g(x)$ is
$2\pi / \absoluteValue{c} = 2\pi/2 = \pi$.
\Figure{fig:trigonometric:sin_right_shift_pi_10} shows graphs of
$f(x)$ and $g(x)$ from $x = 0$ to $x = \pi$.
\end{solution}

\begin{exercise}
Consider the function $f(x) = \cos(\frac{3}{2}x - \frac{5\pi}{2})$.
%%
\begin{packedenum}
\item\label{subex:trigonometric:cos_right_shift_standard_form}
  Write $f(x)$ in the form $f(x) = \cos\bigparen{c (x - h)}$.

\item\label{subex:trigonometric:cos_right_shift_5pi_3}
  Describe the graph of the function $f(x)$.

\item\label{subex:trigonometric:cos_right_shift_graph}
  Let $g(x) = \cos(cx)$ be the function $f(x)$ that has not been
  shifted horizontally.  On the same set of axes, graph $f(x)$ and
  $g(x)$ from $x = 0$ to $x = p$, where $p$ is the period of both
  $f(x)$ and $g(x)$.
\end{packedenum}
\end{exercise}

\ifbool{showSolution}{
\begin{solution}
\solutionpart{subex:trigonometric:cos_right_shift_standard_form}
To write $f(x) = \cos(\frac{3}{2}x - \frac{5\pi}{2})$ in the form
$f(x) = \cos\bigparen{c (x - h)}$, you must factorise the ratio
$\frac{3}{2}$.  Doing so results in the expression
\[
\frac{3}{2}x - \frac{5\pi}{2}
=
\frac{3}{2} \parenthesis*{x - \frac{5\pi}{3}}
\]
which can be used to write $f(x)$ as
%%
\begin{align*}
f(x)
&=
\cosp{\frac{3}{2}x - \frac{5\pi}{2}} \\[4pt]
&=
\cosp{\frac{3}{2} \parenthesis*{x - \frac{5\pi}{3}}}
\end{align*}
%%
where $h = \frac{5\pi}{3}$.

\solutionpart{subex:trigonometric:cos_right_shift_5pi_3}
Let $g(x) = \cos(\frac{3}{2} x)$.  The graph of $f(x)$ is obtained
by shifting the graph of $g(x)$ along the horizontal axis.  The amount
of horizontal shift is $\frac{5\pi}{3}$ units to the right.

\begin{figure}[!htbp]
\centering
\includegraphics[scale=1.1]{image/13/cos-right-shift-5pi-3.pdf}
\caption{%%
  Graphs of the functions $f(x) = \cos(\frac{3}{2}x - \frac{5\pi}{2})$
  and $g(x) = \cos(\frac{3}{2} x)$ from $x = 0$ to
  $x = \frac{4\pi}{3}$.  The graph of $f(x)$ is obtained by shifting
  the graph of $g(x)$ by $\frac{5\pi}{3}$ units to the right.
}
\label{fig:trigonometric:cos_right_shift_5pi_3}
\end{figure}

\solutionpart{subex:trigonometric:cos_right_shift_graph}
Your $c$ value is $c = \frac{3}{2}$.  The period of both $f(x)$ and
$g(x)$ can be written as
%%
\begin{align*}
\frac{2\pi}{\absoluteValue{c}}
&=
\frac{2\pi}{3/2} \\[4pt]
&=
2\pi \times \frac{2}{3} \\[4pt]
&=
\frac{4\pi}{3}.
\end{align*}
%%
\Figure{fig:trigonometric:cos_right_shift_5pi_3} shows graphs of
$f(x)$ and $g(x)$ from $x = 0$ to $x = \frac{4\pi}{3}$.
\end{solution}
}{}


\newpage
%%%%%%%%%%%%%%%%%%%%%%%%%%%%%%%%%%%%%%%%%%%%%%%%%%%%%%%%%%%%%%%%%%%%%%%%%%%

\section*{Problem}

\begin{problem}
\item Read the following paper by David M. Bressoud:
  \emph{Historical Reflections on Teaching Trigonometry}.\footnote{
    The paper is available at
    \url{https://www.jstor.org/stable/20876798}.
  }

\item High and low tides in Geelong.  See also page~94 of the
  following book: W.~T.~Griffith and J.~W.~Brosing.  \emph{The Physics
    of Everyday Phenomena: A Conceptual Introduction to Physics}.
\ifbool{showSolution}{
\begin{solution}

\end{solution}
}{}

\item Sound waves, CPU speed, radio signal~(AM and FM).
\ifbool{showSolution}{
\begin{solution}

\end{solution}
}{}

\item Uniform circular motion.  See pages~131 and~241 of the book:
  N.~J.~Giordano.  \emph{College Physics: Reasoning and Relationships}.
\ifbool{showSolution}{
\begin{solution}

\end{solution}
}{}

\item Rolling motion.  See pages~267--269 of the book:
  N.~J.~Giordano.  \emph{College Physics: Reasoning and Relationships}.
\ifbool{showSolution}{
\begin{solution}

\end{solution}
}{}

\item The physics of waves.  See chapter~15 of the book:
  W.~T.~Griffith and J.~W.~Brosing.  \emph{The Physics of Everyday
    Phenomena: A Conceptual Introduction to Physics}.
\ifbool{showSolution}{
\begin{solution}

\end{solution}
}{}

\item Simple harmonic motion and damped motion.  See page~287 of the
  book: M.~Sullivan.  \emph{Trigonometry: A Unit Circle Approach}.
  See also chapter~11 of the book:  N.~J.~Giordano.
  \emph{College Physics: Reasoning and Relationships}.
\ifbool{showSolution}{
\begin{solution}

\end{solution}
}{}

\item Sound.  See chapter~13 of the book:  N.~J.~Giordano.
  \emph{College Physics: Reasoning and Relationships}.
\ifbool{showSolution}{
\begin{solution}

\end{solution}
}{}
\end{problem}

\end{document}
