%%%%%%%%%%%%%%%%%%%%%%%%%%%%%%%%%%%%%%%%%%%%%%%%%%%%%%%%%%%%%%%%%%%%%%%%%%%

\documentclass[a4paper,oneside,12pt]{article}
\usepackage{mystyle}

\begin{document}

\title{\Large\bf Trigonometric functions}
\author{%%
  Minh Van Nguyen \\
  \url{mvngu@gmx.com}
}
\date{\today}
\maketitle

\begin{itemize}
\item Application: equations of parabolic motion for projectiles.
\end{itemize}


%%%%%%%%%%%%%%%%%%%%%%%%%%%%%%%%%%%%%%%%%%%%%%%%%%%%%%%%%%%%%%%%%%%%%%%%%%%

\section{The sine and cosine functions}

By now, you should be familiar with the sine function $\sin x$ and the
cosine function $\cos x$, where $x$ is an angle in radians.  In this
section, you will further investigate properties of the sine and
cosine functions.

\begin{figure}[!htbp]
\centering
\includegraphics[scale=1.1]{image/13/a-sin-b.pdf}
\caption{%%
  A graph of the general sine function
  $f(x) = a \sin x + b$ from $x = -2\pi$ to $x = 2\pi$.  The midline
  of $f(x)$ is the dashed horizontal line $y = b$.  The amplitude of
  $f(x)$ is the absolute value $\absoluteValue{a}$.
}
\label{fig:trigonometric:general_sine}
\end{figure}

Let $a$ and $b$ be fixed real numbers and let $x$ be a real variable
that represents an angle in radians.  The general sine and cosine
functions can be written as
%%
\begin{equation}
\label{eqn:trigonometric:general_sine_cosine}
f(x)
=
a \sin x + b
%%
\qquad
\text{and}
\qquad
%%
g(x)
=
a \cos x + b.
\end{equation}
%%
In the case of the sine function $f(x) = a \sin x + b$, the number $b$
is called the \emph{midline} of $f(x)$ and the absolute value
$\absoluteValue{a}$ is called the \emph{amplitude} of $f(x)$.
\Figure{fig:trigonometric:general_sine} shows a graph of the general
sine function $f(x)$.  The amplitude $\absoluteValue{a}$ measures how
high and how low the value of the sine function $f(x)$ can be.  As you
can see in \Figure{fig:trigonometric:general_sine}, given the sine
function $f(x) = a \sin x + b$ the highest value of $f(x)$ is
$b + \absoluteValue{a}$ and the lowest value of $f(x)$ is
$b - \absoluteValue{a}$.  The number $b$ is the midline of the sine
function $f(x)$ because the horizontal line $y = b$ is midway between
the highest and lowest values of $f(x)$.  If you draw the horizontal
line $y = b$ on a graph of $f(x)$~(e.g.~the dashed horizontal line in
\Figure{fig:trigonometric:general_sine}) you will see that $f(x) = b$
whenever $a \sin x = 0$.

\begin{figure}[!htbp]
\centering
\includegraphics[scale=1.1]{image/13/3-sin-1.pdf}
\caption{%%
  A graph of the sine function $f(x) = 3 \sin x + 1$ from
  $x = -2\pi$ to $x = 2\pi$.
}
\label{fig:trigonometric:sine_3_sin_1}
\end{figure}

\begin{example}
Consider the sine function $f(x) = 3 \sin x + 1$.  Draw a graph of the
function from $x = -2\pi$ to $x = 2\pi$.  Determine the midline and
amplitude of $f(x)$.  Calculate the highest and lowest values of the
function.
\end{example}

\begin{solution}
The first thing you should do is work out the midline and amplitude of
the function $f(x)$.  The midline of the function is $y = 1$.  The
amplitude is $3$, which means that the highest value of $f(x)$ is
$1 + 3 = 4$ and the lowest value of $f(x)$ is $1 - 3 = -2$.
\Figure{fig:trigonometric:sine_3_sin_1} shows the midline
$y = 1$ as a dashed horizontal line.

Next, consider the values of $\sin x$ for
$x = \quintuple{0}{\frac{\pi}{2}}{\pi}{\frac{3\pi}{2}}{2\pi}$.  At
$x = \triple{0}{\pi}{2\pi}$, you have
$\sin 0 = \sin \pi = \sin 2\pi = 0$ and so
$f(0) = f(\pi) = f(2\pi) = 3 \times 0 + 1 = 1$.  That is, the points
$\tuple{0}{1}$, $\tuple{\pi}{1}$, and $\tuple{2\pi}{1}$ lie on the
midline $y = 1$ and are shown in
\Figure{fig:trigonometric:sine_3_sin_1} as black dots.  At
$x = \frac{\pi}{2}$, you have $\sin \frac{\pi}{2} = 1$ and so
$f(\pi/2) = 3 \times 1 + 1 = 4$, which is the highest value of
$f(x)$.  At $x = \frac{3\pi}{2}$, you have
$\sin \frac{3\pi}{2} = -1$ so that
$f(3\pi / 2) = 3 \times (-1) + 1 = -2$, which is the lowest value of
$f(x)$.  That is, the point $\tuple{\frac{\pi}{2}}{4}$ is the peak~(or
highest value) of $f(x)$ and the point
$\tuple{\frac{3\pi}{2}}{-2}$ is the valley~(or lowest point) of
$f(x)$.  \Figure{fig:trigonometric:sine_3_sin_1} shows the highest and
lowest points as red dots.

Finally, note that the graph of $f(x)$ from $x = -2\pi$ to
$x = 0$ is a copy of the graph of $f(x)$ from $x = 0$ to $x = 2\pi$.
That is, the angle of $-\pi$ randians is the same as
$2\pi - \pi = \pi$ radians.  Furthermore, $-2\pi$ radians is the same
as $2\pi - 2\pi = 0$ radians.  In other words, the points
$\tuple{-2\pi}{1}$ and $\tuple{-\pi}{1}$ lie on the midline
$y = 1$.  You know that $-\frac{3\pi}{2}$ radians is the same as
$2\pi - \frac{3\pi}{2} = \frac{\pi}{2}$ radians, which means that
$\tuple{-\frac{3\pi}{2}}{4}$ is also the highest point of $f(x)$.  You
also know that $-\frac{\pi}{2}$ radians is the same as
$2\pi - \frac{\pi}{2} = \frac{3\pi}{2}$ radians and consequently the
point $\tuple{-\frac{\pi}{2}}{-2}$ is also the lowest point of
$f(x)$.  Draw a wave through the points and you obtain the graph in
\Figure{fig:trigonometric:sine_3_sin_1}.
\end{solution}

\end{document}
