%%%%%%%%%%%%%%%%%%%%%%%%%%%%%%%%%%%%%%%%%%%%%%%%%%%%%%%%%%%%%%%%%%%%%%%%%%%

\documentclass[a4paper,oneside,12pt]{article}
\usepackage{mystyle}

\begin{document}

\title{\Large\bf Exponential functions}
\author{%%
  Minh Van Nguyen \\
  \url{mvngu@gmx.com}
}
\date{\today}
\maketitle


%%%%%%%%%%%%%%%%%%%%%%%%%%%%%%%%%%%%%%%%%%%%%%%%%%%%%%%%%%%%%%%%%%%%%%%%%%%

\section{Exponential growth}

Exponential functions are commonly used to understand the change in a
population over time.  The population can be the residents of a
country, the ants in an ant colony, the bacteria in a Petri dish, or
the amount of carbon-14 in the bone of an animal that lived thousands
of years ago.  In particular, what you want to know is the
\emph{percentage rate} of change over time, where time can be measured
in terms of seconds, minutes, hours, days, months, or years.  If you
know the \emph{initial population} and the percentage rate of change,
then you can use the exponential function to calculate the population
in the next year or the next three years.  The following example
should help you to understand the ideas presented above.

\begin{example}
\label{eg:Australian_population_2017}
\textbf{Population of Australia.}
According to the Australian Bureau of Statistics~(ABS), at the end of
September 2017 the population of Australia grew by $1.6\%$ since the
previous year.\footnote{
  ``3101.0 - Australian Demographic Statistics, Sep 2017'',
  \url{http://web.archive.org/web/20180420062713/http://www.abs.gov.au/AUSSTATS/abs@.nsf/Lookup/3101.0Main+Features1Sep\%202017},
  accessed 2018-04-20.
}
The population by the end of the given period was estimated to be
$24.7029$ million people.  Assume that for the next few years the
population of Australia will grow by a rate of $1.6\%$ per annum.
%%
\begin{packedenum}
\item\label{subeg:Australian_population_2017_growth_rate}
  Identify the rate of change.

\item\label{subeg:Australian_population_2017_initial_population}
  Determine the initial population.

\item\label{subeg:Australian_population_2017_table_graph}
  Provide a table of values for the projected population of Australia
  from 2017 to 2021, inclusive.  Graph the data in the table as a
  scatter plot.
\end{packedenum}
\end{example}

\begin{solution}
\solutionpart{subeg:Australian_population_2017_growth_rate}
By the end of September 2017, the population of Australia grew by
$1.6\%$ from the previous year so the percentage rate of change is
$1.6\%$.  Since you are concerned with population growth, the rate of
change in this example is also called the \emph{growth rate}.  The
growth rate can be written as a percentage or as a fraction.  To
express the percentage of $1.6\%$ as a fraction, you divide the number
$1.6$ by $100$ to get the decimal representation
\[
\frac{1.6}{100}
=
0.016.
\]
That is, you can also say that the growth rate is $0.016$.

\solutionpart{subeg:Australian_population_2017_initial_population}
The example does not explicitly state the value of the initial
population.  However, the example does say that the population of
Australia by the end of September 2017 was $24.7029$ million people so
you can consider this number as the initial population.  The reason
why you require an initial population is so that the population can
grow or decline.  You need to start somewhere.

\solutionpart{subeg:Australian_population_2017_table_graph}
You are required to predict the population of Australia from $2018$ to
$2021$, inclusive.  To calculate the projected~(or predicted)
population in $2018$, you must know two numbers: the population in
$2017$ and the growth rate.
From \Part{subeg:Australian_population_2017_growth_rate} you know that
the growth rate is $0.016$ per annum and
from \Part{subeg:Australian_population_2017_initial_population} you
know that the population of Australia for 2017 was $24.7029$ million
people.  By $2018$, the population would have increased by
%%
\begin{equation}
\label{eqn:Australian_population_2018}
24.7029 \times 0.016
=
0.3952464
\end{equation}
%%
million people.  To calculate the population for $2018$, you must add
the number in~\eqref{eqn:Australian_population_2018} to the population
in $2017$.  The projected population for $2018$ is given by the
expression $24.7029 + 24.7029 \times 0.016$.  Factoring out the number
$24.7029$ and simplifying shows that the required population number is
%%
\begin{equation}
\label{eqn:Australian_population_2018_calculation}
\begin{aligned}
24.7029 + 24.7029 \times 0.016
&=
24.7029 (1 + 0.016) \\[4pt]
&=
24.7029 \times 1.016 \\[4pt]
&=
25.0981464
\end{aligned}
\end{equation}
%%
million people.

Now use the same technique as explained above to calculate the
projected population for $2019$.  You already know that the growth
rate is $0.016$ per annum and that the population in $2018$ would be
$25.0981464$ million people.  By $2019$, the population would have
increased by
%%
\begin{equation}
\label{eqn:Australian_population_2019}
25.0981464 \times 0.016
=
0.4015703424
\end{equation}
%%
million people.  The projected population for $2019$ is the number
in~\eqref{eqn:Australian_population_2019} added to the population in
$2018$.  By $2019$ Australia is projected to have a population of
%%
\begin{align*}
25.0981464 + 25.0981464 \times 0.016
&=
25.0981464 (1 + 0.016) \\[4pt]
&=
25.0981464 \times 1.016 \\[4pt]
&=
25.4997167424
\end{align*}
%%
million people.  Use the same technique as explained above to
calculate the projected population in $2020$ and $2021$.  The
projected populations through to the year $2021$ are shown in
\Table{tab:Australian_population_2017}, a graph of which is
illustrated in \Figure{fig:Australian_population_2017}.
\end{solution}

\begin{table}[!htbp]
\centering
\begin{tabular}{cl}           \toprule
Year   & Population         \\\midrule
$2017$ & $24.7029$          \\[4pt]
$2018$ & $25.0981464$       \\[4pt]
$2019$ & $25.4997167424$    \\[4pt]
$2020$ & $25.9077122102784$ \\[4pt]
$2021$ & $26.3222356056429$ \\\bottomrule
\end{tabular}

\caption{%%
  The projected population of Australia through to the year $2021$.
  The population numbers are in terms of millions.  For example, the
  population in $2017$ was estimated to be $24.7029$ million people.
  The population is assumed to grow at a rate of $1.6\%$ per annum.
}
\label{tab:Australian_population_2017}
\end{table}

\begin{figure}[!htbp]
\centering
\includegraphics[scale=1]{image/11/australian-population.pdf}
\caption{%%
  The projected population~(in millions) of Australia through to the
  year $2021$.  The population is assumed to grow at a rate of $1.6\%$
  per annum.  Data are from \Table{tab:Australian_population_2017}.
}
\label{fig:Australian_population_2017}
\end{figure}

The next question you might ask is:  Is there a formula to calculate
the population of Australia for any number of years since $2017$?  The
answer is yes, but you must assume that the population grows by a
constant percentage rate each year.  Assume that you have an initial
population of $24.7029$ million people in $2017$ and that the growth
rate is $1.6\%$ or $0.016$ per annum.  According to
\Equation{eqn:Australian_population_2018_calculation}, in
$2018$~(i.e.~one year from $2017$) the population can be written as
the expression
\[
24.7029 + 24.7029 \times 0.016
=
24.7029 \times 1.016.
\]
In $2019$~(i.e.~two years from $2017$) the population can be written
as
%%
\begin{align*}
&24.7029 \times 1.016 + 24.7029 \times 1.016 \times 0.016 \\[4pt]
&=
24.7029 \times 1.016 (1 + 0.016) \\[4pt]
&=
24.7029 \times 1.016 \times 1.016 \\[4pt]
&=
24.7029 \times 1.016^2.
\end{align*}
%%
In $2020$~(i.e.~three years from $2017$) the population can be written
as
%%
\begin{align*}
&24.7029 \times 1.016^2 + 24.7029 \times 1.016^2 \times 0.016 \\[4pt]
&=
24.7029 \times 1.016^2 (1 + 0.016) \\[4pt]
&=
24.7029 \times 1.016^2 \times 1.016 \\[4pt]
&=
24.7029 \times 1.016^3.
\end{align*}
%%
In $2021$~(i.e.~four years from $2017$) the population can be written
as
%%
\begin{align*}
&24.7029 \times 1.016^3 + 24.7029 \times 1.016^3 \times 0.016 \\[4pt]
&=
24.7029 \times 1.016^3 (1 + 0.016) \\[4pt]
&=
24.7029 \times 1.016^3 \times 1.016 \\[4pt]
&=
24.7029 \times 1.016^4.
\end{align*}
%%
From the above equations, you can see that the only number that
changes is the exponent or power of $1.016$.  If $t$ is the number of
years since $2017$, then the population~(in millions) in $t$ years
since $2017$ can be written as
%%
\begin{equation}
\label{eqn:Australian_population_formula}
24.7029 \times 1.016^t.
\end{equation}
%%
You can see from \Expression{eqn:Australian_population_formula} that
the initial population is $24.7029$ million people.  The number
$1.016$ is called the \emph{growth factor} and can be written in terms
of the growth rate as $1.016 = 1 + 0.016$.

\begin{exercise}
Use \Expression{eqn:Australian_population_formula} to verify the
population numbers in \Table{tab:Australian_population_2017}.
\end{exercise}

\ifbool{showSolution}{
\begin{solution}
In \Table{tab:Australian_population_2017}, the year $2017$ is $t = 0$
years since $2017$ so according to
\Expression{eqn:Australian_population_formula} the population in
$2017$ is
\[
24.7029 \times 1.016^0
=
24.7029
\]
million people.  The year $2018$ is $t = 1$ year since $2017$ so the
population in $2018$ is
\[
24.7029 \times 1.016^1
=
25.0981464
\]
million people.  The year $2019$ is $t = 2$ years since $2017$ so the
population in $2019$ is
\[
24.7029 \times 1.016^2
=
25.4997167424
\]
million people.  The year $2020$ is $t = 3$ years since $2017$ so the
population in $2020$ is approximately
\[
24.7029 \times 1.016^3
=
25.9077122102784
\]
million people.  Finally, the year $2021$ is $t = 4$ years since
$2017$ so the population in $2021$ is approximately
\[
24.7029 \times 1.016^4
=
26.3222356056429
\]
million people.
\end{solution}
}{}

Let's derive a general formula to model situations such as those
presented in \Example{eg:Australian_population_2017}.  Denote by $a$
the initial value of the population and assume that $a \neq 0$.  Let
$r$ be the decimal representation of the percentage rate of change and
define the growth factor by $b = 1 + r$.  For example, if the
population grows by a constant percentage of $2.3\%$, the decimal
representation of this percentage rate of change is
$r = 2.3 / 100 = 0.023$ and the growth factor is
$b = 1 + 0.023 = 1.023$.  The amount of time since the initial
population is represented by $t$.  So you assume that $t \geq 0$ and
at time $t = 0$ you have the initial population of $a$.  At time
$t = 1$, the population would have changed by an amount of $ar$ since
time $t = 0$.  You add this number to the population at time $t = 0$
to get the population at $t = 1$.  Then the population at time $t = 1$
is given by
\[
a + ar
=
a(1 + r)
=
ab.
\]
At time $t = 2$, the population would have changed by an amount of
$abr$ since time $t = 1$.  You add this number to the population at
time $t = 1$ to get the population at time $t = 2$.  That is, the
population at time $t = 2$ is given by
\[
ab + abr
=
ab(1 + r)
=
abb
=
ab^2.
\]
Now you use the same technique as explained above to obtain the
population at time $t = 3$.  At time $t = 3$, the population would
have changed by an amount of $ab^2r$ since time $t = 2$.  Adding this
number to the population at time $t = 2$ shows that the population at
time $t = 3$ is given by
\[
ab^2 + ab^2r
=
ab^2 (1 + r)
=
ab^2b
=
ab^3.
\]
Use the same technique as explained above to see that the population
at time $t = 4$ is given by the expression
\[
ab^3 + ab^3r
=
ab^3 (1 + r)
=
ab^3b
=
ab^4.
\]
From the above equations, you see that in general the population
$Q(t)$ at time $t \geq 0$ is given by the formula
%%
\begin{align*}
Q(t)
&=
a(1 + r)^t \\[4pt]
&=
ab^t.
\end{align*}
The above discussion is summarised in the next theorem.

\begin{theorem}
\label{thm:exponential_growth}
\textbf{Exponential growth.}
Let $a \neq 0$ be the initial value of a population and assume that
the population changes by a constant decimal rate of $r$ per unit
time.  If $b = 1 + r$ is the growth factor and $Q(t)$ represents the
population number at time $t$, then $Q(t) = ab^t$.
\end{theorem}

\begin{packeditem}
\item {\color{red}
    Graphs of what the exponential function looks like.}
\end{packeditem}


%%%%%%%%%%%%%%%%%%%%%%%%%%%%%%%%%%%%%%%%%%%%%%%%%%%%%%%%%%%%%%%%%%%%%%%%%%%

\section{Growth and decay}

This section presents various examples of exponential growth and decay.

\begin{example}
\textbf{Bacterial growth.}
A Petri dish initially contains ten cells of a type of bacteria.  The
bacteria population is known to have a constant percentage growth rate
of $56\%$ per hour.
%%
\begin{packedenum}
\item\label{subeg:bacteria_growth_rate}
  Determine the growth rate and growth factor of the bacteria
  population.

\item\label{subeg:bacteria_initial_population}
  Determine the initial population of bacteria in the Petri dish.

\item\label{subeg:bacteria_growth_1_2_hours}
  Use the same technique as in
  \Expression{eqn:Australian_population_2018_calculation} to calculate
  the number of bacteria in the Petri dish after one hour.  How many
  bacteria would there be in the Petri dish after two hours?

\item\label{subeg:bacteria_growth_formula}
  Derive a formula for the number of bacteria in the Petri dish after
  $t$ hours.  Use the formula to verify your results
  from \Part{subeg:bacteria_growth_1_2_hours}.  Produce a graph of the
  growth of the bacteria population within $24$ hours.
\end{packedenum}
\end{example}

\begin{solution}
\solutionpart{subeg:bacteria_growth_rate}
Since the bacteria population grows at a constant percentage rate of
$56\%$ per hour, the growth rate is $r = 56 / 100 = 0.56$ per hour and
hence the growth factor is $b = 1 + 0.56 = 1.56$.

\solutionpart{subeg:bacteria_initial_population}
The initial population of bacteria is $a = 10$ cells.

\solutionpart{subeg:bacteria_growth_1_2_hours}
You have an initial bacteria population of $10$ cells.  At time
$t = 1$ hour, the bacteria population would have increased by
$10 \times 0.56$ cells.  Add this number to the initial population to
see that after one hour the bacteria population would be
%%
\begin{align*}
10 + 10 \times 0.56
&=
10 (1 + 0.56) \\[4pt]
&=
10 \times 1.56 \\[4pt]
&=
15.6
\end{align*}
%%
cells.  After two hours, the bacteria population would have increased
by $15.6 \times 0.56$ cells.  Adding this number to the population at
time $t = 1$ hour and you have
%%
\begin{align*}
15.6 + 15.6 \times 0.56
&=
15.6 (1 + 0.56) \\[4pt]
&=
15.6 \times 1.56 \\[4pt]
&=
24.336
\end{align*}
%%
cells at time $t = 2$ hours.

\solutionpart{subeg:bacteria_growth_formula}
From \Parts{subeg:bacteria_growth_rate}{subeg:bacteria_initial_population},
you know the initial population to be $a = 10$ cells and the growth
factor is $b = 1.56$.  If $Q(t)$ represents the number of bacteria
cells in the Petri dish after $t$ hours, then use
\Theorem{thm:exponential_growth} to write
\[
Q(t)
=
10 \times 1.56^t.
\]
Let's use the above formula to verify your results
from \Part{subeg:bacteria_growth_1_2_hours}.  At time $t = 1$ hour,
you have
\[
Q(1)
=
10 \times 1.56^1
=
15.6
\]
and at time $t = 2$ hours, you have
\[
Q(2)
=
10 \times 1.56^2
=
24.336.
\]
These are the same as the numbers you obtained
in \Part{subeg:bacteria_growth_1_2_hours}.
\Figure{fig:bacteria_population_24_hours} shows a plot of the bacteria
population within $24$ hours.
\end{solution}

\begin{figure}[!htbp]
\centering
\includegraphics[scale=1.1]{image/11/bacteria.pdf}
\caption{%%
  The bacteria population in a Petri dish within $24$ hours.
  Initially, the Petri dish had a population of $10$ bacteria cells
  and the bacteria population is assumed to grow at a constant
  percentage rate of $56\%$ per hour.
}
\label{fig:bacteria_population_24_hours}
\end{figure}

\begin{exercise}
\textbf{Interest.}
At the start of a new year, you opened a savings account and deposited
$\$100$ into the account.  The account earns an interest of $3\%$ per
annum, where interest is compounded monthly.  Assume that once you
have deposited the $\$100$, you leave the account alone to accrue
interest and no longer deposit any more money into the account.
%%
\begin{packedenum}
\item\label{subex:interest_initial_balance}
  Determine the initial balance of the account.

\item\label{subex:interest_growth_rate}
  Determine the monthly interest rate.  Use this value to calculate
  the monthly growth rate and hence the monthly growth factor of
  your savings account.

\item\label{subex:interest_balance_1month_2months}
  Use the same technique as in
  \Expression{eqn:Australian_population_2018_calculation} to calculate
  the balance of your account at the end of the first month.
  Calculate the balance of your account at the end of the second
  month.

\item\label{subex:interest_balance_formula}
  Derive a formula for the balance of your account after $t$ months.
  Use the formula to verify your results
  from \Part{subex:interest_balance_1month_2months}.  Produce a graph
  of the balance of your account within the first two years since you
  opened the account.

\item\label{subex:interest_200_dollars}
  How long must you wait for the balance of your account to be at
  least $\$200$?
\end{packedenum}
\end{exercise}

\ifbool{showSolution}{
\begin{solution}
\solutionpart{subex:interest_initial_balance}
The account initially has a balance of $\$100$ so the initial balance
of your account is $\$100$.

\solutionpart{subex:interest_growth_rate}
Since the yearly interest rate is $3\%$, divide the number $3$ by $12$
months to see that the monthly interest rate is $3 / 12 = 0.25$ or
$0.25\%$, which is less than one percent.  The monthly growth rate is
obtained by dividing $0.25$ by $100$.  Doing so gives you a monthly
growth rate of $r = 0.25 / 100 = 0.0025$ and hence the monthly growth
factor is $b = 1 + 0.0025 = 1.0025$.

\solutionpart{subex:interest_balance_1month_2months}
At the end of the first month, your account would have accrued an
interest of $100 \times 0.0025$ dollars.  Add this amount to the
initial balance to see that by the end of the first month your account
balance would be
%%
\begin{align*}
100 + 100 \times 0.0025
&=
100 (1 + 0.0025) \\[4pt]
&=
100 \times 1.0025 \\[4pt]
&=
100.25
\end{align*}
%%
dollars.  At the end of the second month, your account would have
accrued $100.25 \times 0.0025$ dollars in interest.  Add this amount
to the balance at the end of the first month to get the balance at the
end of the second month.  By the end of the second month, your account
balance would be
%%
\begin{align*}
100.25 + 100.25 \times 0.0025
&=
100.25 (1 + 0.0025) \\[4pt]
&=
100.25 \times 1.0025 \\[4pt]
&=
100.500625
\end{align*}
%%
dollars.

\solutionpart{subex:interest_balance_formula}
By \Parts{subex:interest_initial_balance}{subex:interest_growth_rate},
you know that the initial balance is $a = 100$ dollars and the monthly
growth factor is $b = 1.0025$.  If $B(t)$ represents the balance of
your account at the end of $t$ months, use
\Theorem{thm:exponential_growth} to write $B(t)$ as
%%
\begin{equation}
\label{eqn:interest_formula_monthly_balance}
B(t)
=
100 \times 1.0025^t.
\end{equation}
%%
Using \Formula{eqn:interest_formula_monthly_balance}, you have a
balance of $B(1) = 100 \times 1.0025^1 = 100.25$ dollars at the end of
month $t = 1$.  Furthermore, by the end of month $t = 2$ your account
balance would be $B(2) = 100 \times 1.0025^2 = 100.500625$ dollars.
These values are the same as those obtained
in \Part{subex:interest_balance_1month_2months}.
\Figure{fig:interest_balance_24_months} shows a graph of the account
balance within the first two years since the account was opened.

\begin{figure}[!htbp]
\centering
\includegraphics[scale=1.1]{image/11/interest.pdf}
\caption{%%
  A graph of the balance of a savings account for the first two years
  since the account was opened.  The account has an initial balance of
  $\$100$ and a monthly interest rate of $0.25\%$.
}
\label{fig:interest_balance_24_months}
\end{figure}

\solutionpart{subex:interest_200_dollars}
After $t = 277$ months, your account balance would be approximately
\[
B(277)
=
100 \times 1.0025^{277}
=
199.6980
\]
dollars, rounded to four decimal places.  At the end of $t = 278$
months, your account balance would be approximately
\[
B(278)
=
100 \times 1.0025^{278}
=
200.1972
\]
dollars, rounded to four decimal places.  Therefore you must wait
$278$ months, or approximately $23$ years, in order for your account
balance to be at least $\$200$.
\end{solution}
}{}

When a population or the amount of something decrease exponentially,
instead of a growth rate you now have a \emph{decay rate} $r$.
Furthermore, instead of a growth factor you now have a
\emph{decay factor} $b$.  The decay factor is defined as one minus the
decay rate so $b = 1 - r$.  You can think of the decay factor as the
fraction of something remaining after unit time.  If $a$ is the
initial population and $Q(t)$ represents the population at time $t$,
then $Q(t)$ can be written as
%%
\begin{equation}
\label{eqn:exponential_decay}
Q(t)
=
ab^t.
\end{equation}
%%
Note that \Equation{eqn:exponential_decay} is almost identical to the
equation for exponential growth in \Theorem{thm:exponential_growth}.
The next example should clarify the idea of exponential decay.

\begin{example}
\textbf{Radiocarbon dating.}
The amount of carbon-$14$ found in a fossil~(e.g.~an ancient bone or
wood) can be used to determine the age of the fossil.  Assume that the
amount of carbon-$14$ decreases by a constant percentage rate of
$11.4\%$ every $1000$ years.\footnote{
  The most accurate estimation is that the amount of carbon-$14$ in a
  fossil decreases by $50\%$ approximately every $5,730$ years.  See
  the following paper for further details:
  \url{http://doi.org/10.1038/195984a0}.
}
Suppose that you start with $300$ micrograms of carbon-$14$.
%%
\begin{packedenum}
\item\label{subeg:carbon14_decay_rate}
  Determine the decay rate.  Provide an interpretation of the decay
  rate.

\item\label{subeg:carbon14_decay_factor}
  Calculate the decay factor.  Provide an interpretation of the decay
  factor.

\item\label{subeg:carbon14_after_1000_and_2000_years}
  Determine the amount of carbon-$14$ after $1000$ years and after
  $2000$ years.

\item\label{subeg:carbon14_formula_graph}
  Derive a formula for the amount of carbon-$14$ remaining.  Produce a
  graph of the amount of carbon-$14$ remaining up to and including the
  first $50,000$ years.
\end{packedenum}
\end{example}

\begin{solution}
\solutionpart{subeg:carbon14_decay_rate}
The constant percentage rate of decay is $11.4\%$ every $1000$ years.
Divide the number $11.4$ by $100$ to get the constant decay rate of
$r = 11.4 / 100 = 0.114$.  The decay rate of $0.114$ tells you that
you lose $11.4\%$ of the amount of carbon-$14$ every $1000$ years.

\solutionpart{subeg:carbon14_decay_factor}
The decay factor is $b = 1 - 0.114 = 0.886$, which can be converted to
a percentage as $0.886 \times 100 = 88.6\%$.  In other words, after
every $1000$ years you lose $11.4\%$ of the amount of carbon-$14$ or
that only about $88.6\%$ of the amount of carbon-$14$ remains.  You
might lose $11.4\%$ of the amount, but you still have $88.6\%$ left.

\solutionpart{subeg:carbon14_after_1000_and_2000_years}
After $1000$ years, you would have lost $300 \times 0.114$ micrograms
of carbon-$14$.  Subtract this number from the original amount to get
%%
\begin{align*}
300 - 300 \times 0.114
&=
300 (1 - 0.114) \\[4pt]
&=
300 \times 0.886 \\[4pt]
&=
265.8.
\end{align*}
%%
That is, after $1000$ years you would have $265.8$ micrograms of
carbon-$14$ left.  At the end of $2000$ years, you would have lost
$265.8 \times 0.114$ micrograms of carbon-$14$.  Subtract this amount
from $265.8$ to get
%%
\begin{align*}
265.8 - 265.8 \times 0.114
&=
265.8 (1 - 0.114) \\[4pt]
&=
265.8 \times 0.886 \\[4pt]
&=
235.4988.
\end{align*}
%%
In other words, after $2000$ years you would have only $235.4988$
micrograms of carbon-$14$ left.

\solutionpart{subeg:carbon14_formula_graph}
You have an initial amount of $300$ micrograms of carbon-$14$ and a
decay factor of $0.886$.  If $A(t)$ represents the amount of
carbon-$14$ remaining after $t \times 1000$ years, then
$A(t) = 300 \times 0.886^t$.  Note that $t$ is in terms of $1000$
years.  So $t = 1$ means $1 \times 1000 = 1000$ years, $t = 2$ means
$2 \times 1000 = 2000$ years, $t = 3$ means
$3 \times 1000 = 3000$ years, and so on.

To produce a graph of the amount of carbon-$14$ left after $50,000$
years, you must work out the minimum and maximum values for $t$.  The
value of $t = 0$ means the time at which you have the initial amount
of carbon-$14$ so you take $t = 0$ as the minimum value.  Since $t$ is
in terms of $1000$ years, divide $50,000$ by $1000$ to get
$t = 50000 / 1000 = 50$ so that the maximum value of $t$ is
$t = 50$.  \Figure{fig:carbon14_decay} shows a graph of the amount of
carbon-$14$ remaining up to $50,000$ years later.
\end{solution}

\begin{figure}[!htbp]
\centering
\includegraphics[scale=1.1]{image/11/carbon14.pdf}
\caption{%%
  The amount~(in micrograms) of carbon-$14$ remainining after
  $t \times 1000$ years.  You initially have $300$ micrograms of
  carbon-$14$.  It is assumed that the constant percentage decay
  rate of carbon-$14$ is $11.4\%$ per $1000$ years.
}
\label{fig:carbon14_decay}
\end{figure}


%%%%%%%%%%%%%%%%%%%%%%%%%%%%%%%%%%%%%%%%%%%%%%%%%%%%%%%%%%%%%%%%%%%%%%%%%%%

\section{Exponential and linear growths}


%%%%%%%%%%%%%%%%%%%%%%%%%%%%%%%%%%%%%%%%%%%%%%%%%%%%%%%%%%%%%%%%%%%%%%%%%%%

\section{Compound interest}


%%%%%%%%%%%%%%%%%%%%%%%%%%%%%%%%%%%%%%%%%%%%%%%%%%%%%%%%%%%%%%%%%%%%%%%%%%%

\section{The number $e$}

\end{document}
