%%%%%%%%%%%%%%%%%%%%%%%%%%%%%%%%%%%%%%%%%%%%%%%%%%%%%%%%%%%%%%%%%%%%%%%%%%%

\documentclass[a4paper,oneside,12pt]{article}
\usepackage{mystyle}

\begin{document}

\title{\Large\bf Exponential functions}
\author{%%
  Minh Van Nguyen \\
  \url{mvngu@gmx.com}
}
\date{\today}
\maketitle


%%%%%%%%%%%%%%%%%%%%%%%%%%%%%%%%%%%%%%%%%%%%%%%%%%%%%%%%%%%%%%%%%%%%%%%%%%%

\section{Modelling growth}

Exponential functions are commonly used to understand the change in a
population over time.  The population can be the residents of a
country, the ants in an ant colony, the bacteria in a petri dish, or
the amount of carbon-14 in a dinosaur bone.  In particular, what you
want to know is the \emph{percentage rate} of change over time, where
time can be measured in terms of seconds, minutes, hours, days,
months, or years.  If you know the \emph{initial population} and the
percentage rate of change, then you can use the exponential function
to calculate the population in the next year or the next three years.
The next example should help you to understand the ideas presented
above.


%%%%%%%%%%%%%%%%%%%%%%%%%%%%%%%%%%%%%%%%%%%%%%%%%%%%%%%%%%%%%%%%%%%%%%%%%%%

\section{Exponential and linear growths}


%%%%%%%%%%%%%%%%%%%%%%%%%%%%%%%%%%%%%%%%%%%%%%%%%%%%%%%%%%%%%%%%%%%%%%%%%%%

\section{Compound interest}


%%%%%%%%%%%%%%%%%%%%%%%%%%%%%%%%%%%%%%%%%%%%%%%%%%%%%%%%%%%%%%%%%%%%%%%%%%%

\section{The number $e$}

\end{document}
