%%%%%%%%%%%%%%%%%%%%%%%%%%%%%%%%%%%%%%%%%%%%%%%%%%%%%%%%%%%%%%%%%%%%%%%%%%%

\documentclass[a4paper,oneside,12pt]{article}
\usepackage{mystyle}

\begin{document}

\title{\Large\bf Factoring a quadratic function}
\author{%%
  Minh Van Nguyen \\
  \url{mvngu@gmx.com}
}
\date{\today}
\maketitle


%%%%%%%%%%%%%%%%%%%%%%%%%%%%%%%%%%%%%%%%%%%%%%%%%%%%%%%%%%%%%%%%%%%%%%%%%%%

\section{Factoring}

This document will show you how to factorise a quadratic function.
One reason why you would factorise a quadratic function is so that you
can determine the roots of the function without using the quadratic
formula.  To factorise an expression means to write the expression as
the product of two or more expressions.  In the case of the number
$6$, you can factorise $6$ by writing it as the product of $2$ and
$3$.  Hence the integer $6$ can be written in factorised form as
\[
6
=
2 \times 3
\]
and you say that $2$ and $3$ are factors of $6$.  As another example,
you can write $60$ in factorised form as $60 = 3 \times 20$.  You can
also factorise $20$ to get $20 = 4 \times 5$ and therefore $60$ can be
factorised as
\[
60
=
3 \times 4 \times 5.
\]
As can be seen from the above examples, factorising an integer
involves writing the integer as the product of two or more integers.
The factors are usually prime integers.  A factorisation of the form
$5 = 1 \times 5$ is correct because $5$ is a prime and has no factors
other than $1$ and $5$.  The factorised form
$5 = 1 \times 5 \times 1$ is also correct, but that's cheating.

What about factoring quadratic functions?  Let's start with quadratic
functions of the form $f(x) = ax^2 + bx$, where $a$ and $b$ are any
real numbers such that $a \neq 0$.  Use the distributive laws to write
$f(x)$ in the factorised form
%%
\begin{equation}
\label{eqn:factorise_axx_bx}
f(x)
=
(ax + b)x
\end{equation}
%%
to see that $f(x)$ has the factors $x$ and $ax + b$.  Looking at the
factored form~\eqref{eqn:factorise_axx_bx}, the roots of $f(x)$ are
$x = 0$ and $x = -b/a$.  But how did you get those numbers?  To
calculate the roots of $f(x)$ means to determine all values of $x$
such that the expression $f(x) = 0$ is true.  In other words, you want
to determine all values of $x$ such that the expression
%%
\begin{equation}
\label{eqn:roots_of_axx_bx}
(ax + b)x
=
0
\end{equation}
%%
is true.  Expression~\eqref{eqn:roots_of_axx_bx} tells you that there
are two numbers, i.e.~$x$ and $ax + b$, whose product is zero.  You
have two cases:
%%
\begin{packedenumeral}
\item If $x = 0$, then \Expression{eqn:roots_of_axx_bx} is true
  because zero multiplied by another number is zero.  So one root of
  $f(x)$ is $x = 0$.

\item If $ax + b = 0$, then \Expression{eqn:roots_of_axx_bx} is also
  true.  But for which values of $x$ would you have $ax + b = 0$?
  Solving the latter expression for $x$ shows that $x = -b / a$.
  Substitute the last expresssion into~\eqref{eqn:roots_of_axx_bx}
  produces
  %%
  \begin{align*}
  \squarebracket*{
    a \parenthesis*{-\frac{b}{a}}
    +
    b
  }
  \parenthesis*{-\frac{b}{a}}
  &=
  (-b + b)
  \parenthesis*{-\frac{b}{a}} \\[4pt]
  &=
  0 \times \parenthesis*{-\frac{b}{a}} \\[4pt]
  &=
  0
  \end{align*}
  %%
  which is true.  Hence you have found that $x = -b / a$ is another
  root of $f(x)$.
\end{packedenumeral}
%%
By writing $f(x)$ as the factored form~\eqref{eqn:factorise_axx_bx}
you can easily determine the roots of $f(x)$ without using the
quadratic formula.  The above discussion is summarised in the
following theorem.

\begin{theorem}
Consider the quadratic function $f(x) = ax^2 + bx$, where $a$ and $b$
are any real numbers such that $a \neq 0$.  The roots of $f(x)$ are
\[
x = 0
%%
\qquad
\text{and}
\qquad
%%
x = -\frac{b}{a}.
\]
\end{theorem}


%%%%%%%%%%%%%%%%%%%%%%%%%%%%%%%%%%%%%%%%%%%%%%%%%%%%%%%%%%%%%%%%%%%%%%%%%%%

\section{Completing the square}

\begin{figure}[!htbp]
\centering
\includegraphics[scale=1.1]{image/10/complete-square-a1-c0.pdf}
\caption{%%
  The quadratic function $f(x) = x^2 + bx$ can be visualised as a
  square plus a rectangle.  The square has a side length of $x$, hence
  the area of the square is $x^2$.  The rectangle has a width of $b$
  and a height of $x$, so the rectangle has an area of $bx$.  Thus
  $f(x)$ can be interpreted as the area of a square plus the area of a
  rectangle.  The rectangle can be cut in half along the dashed line
  as shown.  Each half has a width of $b/2$ and a height of $x$.
}
\label{fig:special_complete_square_square_plus_rectangle}
\end{figure}

\begin{figure}[!htbp]
\centering
\includegraphics[scale=1.1]{image/10/complete-square-a1-c0_halfb.pdf}
\caption{%%
  The quadratic function $f(x) = x^2 + bx$ can be visualised as a
  square plus a rectangle.  The rectangle is cut in half.  One half is
  arranged to the right of the square.  The other half is arranged
  underneath the square.  Now you have a shape that is nearly like a
  square.  The small dashed square in the lower right is what is
  missing to make a complete square.
}
\label{fig:}
\end{figure}

\end{document}
