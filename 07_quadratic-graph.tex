%%%%%%%%%%%%%%%%%%%%%%%%%%%%%%%%%%%%%%%%%%%%%%%%%%%%%%%%%%%%%%%%%%%%%%%%%%%

\documentclass[a4paper,oneside,12pt]{article}
\usepackage{mystyle}

\begin{document}

\title{\Large\bf Graphs of quadratic functions}
\author{%%
  Minh Van Nguyen \\
  \url{mvngu@gmx.com}
}
\date{\today}
\maketitle


%%%%%%%%%%%%%%%%%%%%%%%%%%%%%%%%%%%%%%%%%%%%%%%%%%%%%%%%%%%%%%%%%%%%%%%%%%%

\section{General form}
\label{sec:general_form}

A \emph{quadratic function} is an equation of the form
%%
\begin{equation}
\label{eqn:general_quadratic_function}
f(x)
=
ax^2 + bx + c
\end{equation}
%%
where $\triple{a}{b}{c} \in \RR$ are known constants such that
$a \neq 0$ and $x$ is a variable that can be any real number.  What
does the function $f(x)$ look like?  You will explore a number of ways
to sketch the graph of a quadratic function.

This section will show you a graph drawing technique that works for
any quadratic function.  As an example, set $a = 1$ and $b = c = 0$ in
\Equation{eqn:general_quadratic_function} so that you have
$f(x) = x^2$.  Let's calculate the values of the function
$f(x) = x^2$ for $x = \quadruple{-3}{-2}{2}{3}$.  Note that you
have
\[
f(2) = f(-2) = 4
%%
\qquad
\text{and}
\qquad
%%
f(3) = f(-3) = 9.
\]
Plot these points on one set of coordinate axes and draw a line
through the points to get the graph in \Figure{fig:quadratic_a_1}.
Note that $f(0) = 0$ so the graph of $f(x) = x^2$ touches the origin.
The general shape of the graph of
\Equation{eqn:general_quadratic_function} looks like the beak of a
duck~(or a nose) and the usual name for this graph is a
\emph{parabola}.  Note the tip or \emph{vertex} of the parabola.  The
vertex can be either the highest or lowest point of the parabola.

\begin{figure}[!htbp]
\centering
\includegraphics[scale=1.2]{image/07/a-1.pdf}
\caption{%%
  A graph of the function $f(x) = x^2$.  The general shape of the
  graph is called a \emph{parabola}.  The \emph{vertex} of the
  parabola is its highest or lowest point.  In this case, the vertex
  of $f(x) = x^2$ is the origin $\tuple{0}{0}$.
}
\label{fig:quadratic_a_1}
\end{figure}

\begin{exercise}
For the quadratic function $f(x) = x^2$, verify that you have the
following equalities:
\[
f(2) = f(-2)
%%
\qquad
\text{and}
\qquad
%%
f(3) = f(-3).
\]
\end{exercise}
%%
\ifbool{showSolution}{
\begin{solution}
If $x = 2$ or $x = -2$, then you have
%%
\begin{align*}
f(2)
&=
2^2 \\[4pt]
&=
(-1)^2 2^2 \\[4pt]
&=
(-2)^2 \\[4pt]
&=
f(-2).
\end{align*}
%%
Finally, if $x = 3$ or $x = -3$ then you can write
%%
\begin{align*}
f(3)
&=
3^2 \\[4pt]
&=
(-1)^2 3^2 \\[4pt]
&=
(-3)^2 \\[4pt]
&=
f(-3).
\end{align*}
\end{solution}
}{}

Generally speaking, how would you draw the graph of
\Equation{eqn:general_quadratic_function}?  One way is to start at the
vertex of the parabola and choose a few values of $x$ that are equally
spaced.  If the vertex of the parabola is $\tuple{a}{b}$, the
following values of $x$
\[
\quintuple{a-2}{a-1}{a}{a+1}{a+2}
\]
should usually be good enough for a rough sketch of the graph of
\Equation{eqn:general_quadratic_function}.  Of course, you still need
to calculate the function values $f(a-2)$, $f(a-1)$, $f(a+1)$, and
$f(a+2)$.  Note that the graph of the quadratic function $f(x)$ is
symmetric about the vertex $\tuple{a}{b}$.  This means that
\[
f(a+1) = f(a-1)
%%
\qquad
\text{and}
\qquad
%%
f(a+2) = f(a-2).
\]
Consequently, you need only to calculate two function values, not
four.  The above strategy is illustrated in
\Figure{fig:sketch_parabola}.

\begin{figure}[!htbp]
\centering
\includegraphics[scale=1.2]{image/07/a1-bminus4-c10.pdf}
\caption{%%
  Sketching the graph of a quadratic function.  First, locate the
  vertex $\tuple{a}{b}$ of the function.  From there, spread outward
  with a few points.  Finally, you connect the dots.
}
\label{fig:sketch_parabola}
\end{figure}

The problem now is: How do you calculate the vertex of the parabola?
If you have a quadratic function of the form $f(x) = ax^2 + bx + c$,
the vertex~(i.e.~the highest or lower point) of the function is
located at the $x$-coordinate
%%
\begin{equation}
\label{eqn:parabola_tip_x_coordinate}
x
=
-\frac{b}{2a}.
\end{equation}
%%
Substitute \Expression{eqn:parabola_tip_x_coordinate} into the
function $f(x)$ to obtain the corresponding $y$-coordinate.  For now,
do not worry about how \Expression{eqn:parabola_tip_x_coordinate} was
derived.  This will be shown later on.

\begin{example}
\label{ex:quadratic_graph_a1_b1_c1}
Sketch the graph of the quadratic function $f(x) = x^2 + x + 1$.
\end{example}

\begin{solution}
The first thing you should do is determine the vertex of the parabola.
You have the values $a = 1$ and $b = 1$.  Use
\Expression{eqn:parabola_tip_x_coordinate} to see that the
$x$-coordinate of the vertex is
%%
\begin{align*}
x
&=
-\frac{1}{2 \times 1} \\[4pt]
&=
-\frac{1}{2}
\end{align*}
%%
and the $y$-coordinate of the vertex is
%%
\begin{align*}
y
&=
f(-1/2) \\[4pt]
&=
\parenthesis*{-\frac{1}{2}}^2 + \parenthesis*{-\frac{1}{2}} + 1 \\[4pt]
&=
\frac{1}{4} - \frac{1}{2} + 1 \\[4pt]
&=
\frac{1}{4} - \frac{2}{4} + \frac{4}{4} \\[4pt]
&=
\frac{1 - 2 + 4}{4} \\[4pt]
&=
\frac{3}{4}.
\end{align*}
%%
Thus the vertex of the parabola is the point
$\tuple{-\frac{1}{2}}{\frac{3}{4}}$.  Next, choose the $x$-coordinates
%%
\begin{equation}
\label{eqn:a1_b1_c1_x_coordinates}
\frac{1}{2} = -\frac{1}{2} + 1
%%
\qquad
\text{and}
\qquad
%%
\frac{3}{2} = -\frac{1}{2} + 2
\end{equation}
%%
whose corresponding $y$-coordinates are $\frac{7}{4}$ and
$\frac{19}{4}$, respectively.  These three $y$-coordinates also
correspond to the $x$-coordinates $-\frac{3}{2}$ and $-\frac{5}{2}$,
respectively, and so you have
\[
f(1/2) = f(-3/2)
%%
\qquad
\text{and}
\qquad
%%
f(3/2) = f(-5/2).
\]
Plot the above five points, connect the dots, and you get the graph
in \Figure{fig:quadratic_graph_a1_b1_c1}.
\end{solution}

\begin{figure}[!htbp]
\centering
\includegraphics[scale=1.2]{image/07/a1-b1-c1.pdf}
\caption{%%
  A graph of the quadratic function $f(x) = x^2 + x + 1$.  The vertex
  of the function is at the point
  $\tuple{-\frac{1}{2}}{\frac{3}{4}}$.
}
\label{fig:quadratic_graph_a1_b1_c1}
\end{figure}

\begin{exercise}
For the $x$-coordinates~\eqref{eqn:a1_b1_c1_x_coordinates} in
\Example{ex:quadratic_graph_a1_b1_c1}, verify that the corresponding
$y$-coordinates are $\frac{7}{4}$ and $\frac{19}{4}$.
\end{exercise}
%%
\ifbool{showSolution}{
\begin{solution}
The quadratic function is $f(x) = x^2 + x + 1$.  For $x = 1/2$ you have
%%
\begin{align*}
f(1/2)
&=
\parenthesis*{\frac{1}{2}}^2 + \frac{1}{2} + 1 \\[4pt]
&=
\frac{1}{4} + \frac{1}{2} + 1 \\[4pt]
&=
\frac{1}{4} + \frac{2}{4} + \frac{4}{4} \\[4pt]
&=
\frac{7}{4}.
\end{align*}
%%
Finally, for $x = 3/2$ you have
%%
\begin{align*}
f(3/2)
&=
\parenthesis*{\frac{3}{2}}^2 + \frac{3}{2} + 1 \\[4pt]
&=
\frac{9}{4} + \frac{3}{2} + 1 \\[4pt]
&=
\frac{9}{4} + \frac{6}{4} + \frac{4}{4} \\[4pt]
&=
\frac{19}{4}.
\end{align*}
\end{solution}
}{}

\begin{exercise}
Sketch the graph of the function $f(x) = x^2 - 2x + 2$.
\end{exercise}
%%
\ifbool{showSolution}{
\begin{solution}
First, you determine the vertex of the function $f(x)$.  You have
$a = 1$ and $b = -2$.  Use \Expression{eqn:parabola_tip_x_coordinate}
to see that the $x$-coordinate of the vertex is
%%
\begin{align*}
x
&=
-\frac{-2}{2 \times 1} \\[4pt]
&=
\frac{2}{2} \\[4pt]
&=
1.
\end{align*}
%%
The $y$-coordinate of the vertex is
%%
\begin{align*}
f(1)
&=
1^2 - 2(1) + 2 \\[4pt]
&=
1 - 2 + 2 \\[4pt]
&=
1.
\end{align*}
%%
Hence the vertex of the function $f(x)$ is the point $\tuple{1}{1}$.
Next, choose the following values for $x$:
\[
2 = 1 + 1
%%
\qquad
\text{and}
\qquad
%%
3 = 1 + 2.
\]
The corresponding function values are
\[
f(2) = f(0) = 2
%%
\qquad
\text{and}
\qquad
%%
f(3) = f(-1) = 5.
\]
Plot the above five points to obtain the graph shown in
\Figure{fig:graph_a1_bminus2_c2}.

\begin{figure}[!htbp]
\centering
\includegraphics[scale=1.2]{image/07/a1-bminus2-c2.pdf}
\caption{%%
  A graph of the function $f(x) = x^2 - 2x + 2$.  The vertex of the
  function is the point $\tuple{1}{1}$.
}
\label{fig:graph_a1_bminus2_c2}
\end{figure}
\end{solution}
}{}


%%%%%%%%%%%%%%%%%%%%%%%%%%%%%%%%%%%%%%%%%%%%%%%%%%%%%%%%%%%%%%%%%%%%%%%%%%%

\section{Quadratic formula}
\label{sec:quadratic_formula}

Given a quadratic function $f(x) = ax^2 + bx + c$, sometimes a problem
requires you to determine the values of $x$ such that $f(x) = 0$.
Those values of $x$ for which the equation $f(x) = 0$ is true are
called the \emph{roots} of $f(x)$.  When you need to calculate the
roots of $f(x)$, the \emph{quadratic formula} is guaranteed to provide
you with at least one root.  But what is the quadratic formula and how
is it derived?

\begin{exercise}
\label{ex:completing_the_square}
Let $\pair{a}{b} \in \RR$ be constants such that $a \neq 0$ and let
$x$ be a real variable.  Show that
\[
\parenthesis*{x + \frac{b}{2a}}^2
=
x^2 + \frac{b}{a} x + \parenthesis*{\frac{b}{2a}}^2.
\]
\end{exercise}
%%
\ifbool{showSolution}{
\begin{solution}
Use the distributive laws to expand
$\parenthesis*{x + \frac{b}{2a}}^2$ and you get
%%
\begin{align*}
\parenthesis*{x + \frac{b}{2a}}^2
&=
\parenthesis*{x + \frac{b}{2a}} \parenthesis*{x + \frac{b}{2a}} \\[4pt]
&=
x \parenthesis*{x + \frac{b}{2a}}
+
\frac{b}{2a} \parenthesis*{x + \frac{b}{2a}} \\[4pt]
&=
x^2 + \frac{b}{2a} x + \frac{b}{2a} x + \parenthesis*{\frac{b}{2a}}^2 \\[4pt]
&=
x^2 + 2 \times \frac{b}{2a} x + \parenthesis*{\frac{b}{2a}}^2 \\[4pt]
&=
x^2 + \frac{b}{a} x + \parenthesis*{\frac{b}{2a}}^2
\end{align*}
%%
as required.
\end{solution}
}{}

\begin{exercise}
\label{ex:completing_the_square_redux}
Let $x$ be a real variable and suppose $\triple{a}{b}{c} \in \RR$ are
constants such that $a \neq 0$.  Show that the expression
\[
\parenthesis*{
  x + \frac{b}{2a}
}^2
=
\frac{
  b^2 - 4ac
}{
  4a^2
}
\]
can also be written as $(2ax + b)^2 = b^2 - 4ac$.
\end{exercise}
%%
\ifbool{showSolution}{
\begin{solution}
The statement of the problem assumes that the expression
\[
\parenthesis*{
  x + \frac{b}{2a}
}^2
=
\frac{
  b^2 - 4ac
}{
  4a^2
}
\]
is true.  For the sake of argument, you can assume that the expression
is true.  Multiplying both sides by $4a^2$ produces the expression
%%
\begin{equation}
\label{eqn:complete_the_square_no_denominator}
4a^2 \parenthesis*{x + \frac{b}{2a}}^2
=
b^2 - 4ac
\end{equation}
%%
where the left-hand side can be factored as
%%
\begin{equation}
\label{eqn:complete_the_square_factored}
\begin{aligned}
4a^2 \parenthesis*{x + \frac{b}{2a}}^2
&=
(2a)^2 \parenthesis*{x + \frac{b}{2a}}^2 \\[4pt]
&=
\squarebracket*{2a \parenthesis*{x + \frac{b}{2a}}}^2 \\[4pt]
&=
\parenthesis*{2ax + 2a \times \frac{b}{2a}}^2 \\[4pt]
&=
\parenthesis*{2ax + b}^2.
\end{aligned}
\end{equation}
%%
Conclude from
\Expressions{eqn:complete_the_square_no_denominator}{eqn:complete_the_square_factored}
that $(2ax + b)^2 = b^2 - 4ac$.
\end{solution}
}{}

\begin{exercise}
Let $a \in \RR$ and consider the numbers $\pm\sqrt{a}$.  Here the
symbol ``$\pm$'' means that you have both of $\sqrt{a}$ and
$-\sqrt{a}$.  If $x = \pm\sqrt{a}$, prove that $x^2 = a$.
\end{exercise}
%%
\ifbool{showSolution}{
\begin{solution}
You must prove two statements:
%%
\begin{packedenumeral}
\item\label{case:x_sqrt_a_implies_x_squared_a}
  If $x = \sqrt{a}$ then $x^2 = a$.

\item\label{case:x_minus_sqrt_a_implies_x_squared_a}
  If $x = -\sqrt{a}$ then $x^2 = a$.
\end{packedenumeral}
%%
First, let's prove \Statement{case:x_sqrt_a_implies_x_squared_a}.  If
$x = \sqrt{a}$, then you can square both sides of the equation to get
$x^2 = (\sqrt{a})^2$.  Since $\sqrt{a} = a^{1/2}$, then you can write
the expression $x^2 = (\sqrt{a})^2$ as
%%
\begin{equation}
\label{eqn:x_sqrt_a_implies_x_squared_a}
\begin{aligned}
x^2
&=
(\sqrt{a})^2 \\[4pt]
&=
(a^{1/2})^2 \\[4pt]
&=
a^{2/2} \\[4pt]
&=
a.
\end{aligned}
\end{equation}
%%
Finally, let's prove
\Statement{case:x_minus_sqrt_a_implies_x_squared_a}.  If
$x = -\sqrt{a}$, then squaring both sides of the equation results in
%%
\begin{align*}
x^2
&=
\bigparen{(-1)\sqrt{a}}^2 \\[4pt]
&=
(-1)^2 (\sqrt{a})^2 \\[4pt]
&=
(\sqrt{a})^2.
\end{align*}
%%
The latter equation can be simplied to $x^2 = a$ by using
\Expression{eqn:x_sqrt_a_implies_x_squared_a}.  Therefore if
$x = \pm\sqrt{a}$ then you have $x^2 = a$.
\end{solution}
}{}

\begin{exercise}
\label{ex:x_squared_a_implies_plus_minus_sqrt_a}
If $a$ is a real number such that $x^2 = a$, prove that
$x = \pm\sqrt{a}$.
\end{exercise}
%%
\ifbool{showSolution}{
\begin{solution}
You have two statements to prove:
%%
\begin{packedenumeral}
\item\label{case:x_squared_a_sqrt_a}
  If $a \in \RR$ such that $x^2 = a$, then $x = \sqrt{a}$.

\item\label{case:x_squared_a_minus_sqrt_a}
  If $a \in \RR$ such that $x^2 = a$, then $x = -\sqrt{a}$.
\end{packedenumeral}
%%
First, let's prove \Statement{case:x_squared_a_sqrt_a}.  You know that
$x^2 = a$ and the square root of any number $c \in \RR$ can be written
as $\sqrt{c} = c^{1/2}$.  Taking the square root of both sides of the
equation $x^2 = a$ and you get $(x^2)^{1/2} = a^{1/2}$, which can also
be written as $x^{2/2} = a^{1/2}$.  The latter equation simplifies to
$x = \sqrt{a}$ because $x^{2/2} = x^1 = x$.

Finally, let's prove \Statement{case:x_squared_a_minus_sqrt_a}.  You
know that $x^2 = a$ and since $a = (-1)^2 a$, you can also write
$x^2 = (-1)^2 a$.  Take the square root of both sides of the last
equation to get $(x^2)^{1/2} = \bigparen{(-1)^2 a}^{1/2}$, which can
be written as $x^{2/2} = (-1)^{2/2} a^{1/2}$.  Use the fact that
$(-1)^{2/2} = (-1)^1 = -1$ to write $x = -\sqrt{a}$.
\end{solution}
}{}

First, let's derive the quadratic formula.  When you write $f(x) = 0$,
it is the same as writing $ax^2 + bx + c = 0$.  When you require a
value of $x$ such that $f(x) = 0$, i.e.~a root of $f(x)$, what you
really want is to solve the equation $ax^2 + bx + c = 0$ for $x$.  In
the last equation, dividing each term by $a$ results in the equivalent
expression
\[
x^2 + \frac{b}{a} x + \frac{c}{a}
=
0.
\]
Now move the term $c/a$ to the right-hand side to obtain
%%
\begin{equation}
\label{eqn:quadratic_formula_factor_LHS}
x^2 + \frac{b}{a} x
=
-\frac{c}{a}.
\end{equation}
%%
The problem now is to factor the left-hand side of
\Equation{eqn:quadratic_formula_factor_LHS}.  From
\Exercise{ex:completing_the_square} you know that the square
$\parenthesis*{x + \frac{b}{2a}}^2$ can be expanded to become
\[
\parenthesis*{x + \frac{b}{2a}}^2
=
x^2 + \frac{b}{a} x + \parenthesis*{\frac{b}{2a}}^2
\]
where the expression $x^2 + \frac{b}{a} x$ is the same as the
left-hand side of \Equation{eqn:quadratic_formula_factor_LHS}.  Adding
$\parenthesis*{\frac{b}{2a}}^2$ to both sides of
\Equation{eqn:quadratic_formula_factor_LHS} results in
%%
\begin{align*}
x^2 + \frac{b}{a} x + \parenthesis*{\frac{b}{2a}}^2
&=
-\frac{c}{a} + \parenthesis*{\frac{b}{2a}}^2 \\[4pt]
&=
-\frac{c}{a} + \frac{b^2}{4a^2} \\[4pt]
&=
-\frac{c}{a} \times \frac{4a}{4a} + \frac{b^2}{4a^2} \\[4pt]
&=
-\frac{4ac}{4a^2} + \frac{b^2}{4a^2} \\[4pt]
&=
\frac{
  b^2 - 4ac
}{
  4a^2
}.
\end{align*}
%%
Now use \Exercise{ex:completing_the_square} to factor the left-hand
side of the latter expression and you can write
\Equation{eqn:quadratic_formula_factor_LHS} as
\[
\parenthesis*{x + \frac{b}{2a}}^2
=
\frac{
  b^2 - 4ac
}{
  4a^2
}.
\]
By \Exercise{ex:completing_the_square_redux} this expression is the
same as the equation $(2ax + b)^2 = b^2 - 4ac$.  Use
\Exercise{ex:x_squared_a_implies_plus_minus_sqrt_a} to write the last
equation as $2ax + b = \pm\sqrt{b^2 - 4ac}$.  Solve for $x$ and you
obtain
%%
\begin{align*}
x
&=
\frac{
  -b \pm \sqrt{b^2 - 4ac}
}{
  2a
}
\end{align*}
%%
which is called the \emph{quadratic formula}.  The above can be
summarised as follows.

\begin{theorem}
\label{thm:quadratic_formula}
\textbf{Quadratic formula.}
Let $\triple{a}{b}{c} \in \RR$ such that $a \neq 0$ and consider the
quadratic function $f(x) = ax^2 + bx + c$.  The roots of $f(x)$ can be
written as
%%
\begin{equation}
\label{eqn:quadratic_formula}
x
=
\frac{
  -b \pm \sqrt{b^2 - 4ac}
}{
  2a
}.
\end{equation}
\end{theorem}

What does the quadratic \Formula{eqn:quadratic_formula} mean?  How do
you make sense of the equation?  Let's assume you have a quadratic
function $f(x) = ax^2 + bx + c$ with $\triple{a}{b}{c}$ being real
numbers such that $a \neq 0$.
Expression~\eqref{eqn:quadratic_formula} states that there are values
of $x$ such that $f(x) = 0$ and those values of $x$ can be written as
\[
x
=
\frac{
  -b + \sqrt{b^2 - 4ac}
}{
  2a
}
%%
\qquad
\text{and}
\qquad
%%
x
=
\frac{
  -b - \sqrt{b^2 - 4ac}
}{
  2a
}.
\]
How can you use the quadratic formula to help you sketch the graph of
a quadratic function? In the graph of $f(x)$, the $x$-intercept is
obtained by setting $f(x) = 0$ and solving the last equation for $x$.
The whole point of \Theorem{thm:quadratic_formula} is to help you
calculate the $x$-intercepts of a quadratic function.
\Theorem{thm:quadratic_formula} also says that a quadratic function
has at most two $x$-intercepts.  The next example should help to
clarify the theory.

\begin{example}
Sketch the graph of the function $f(x) = x^2 - x - 2$.
\end{example}

\begin{solution}
You have the values $a = 1$, $b = -1$, and $c = -2$.  You can use
three points to draw the graph of $f(x)$.  The first point is the
vertex of the parabola.  The other two points are the $x$-intercepts
of $f(x)$.

First, let's determine the vertex of $f(x)$.  From
\Equation{eqn:parabola_tip_x_coordinate} you know that the vertex of
the parabola occurs at the $x$-coordinate
%%
\begin{align*}
x
&=
-\frac{-1}{2 \times 1} \\[4pt]
&=
\frac{1}{2}.
\end{align*}
%%
Then the $y$-coordinate of the vertex is
%%
\begin{align*}
y
&=
f(1/2) \\[4pt]
&=
\parenthesis*{\frac{1}{2}}^2 - \frac{1}{2} - 2 \\[4pt]
&=
\frac{1}{4} - \frac{2}{4} - \frac{8}{4} \\[4pt]
&=
\frac{1 - 2 - 8}{4} \\[4pt]
&=
-\frac{9}{4}.
\end{align*}
%%
Thus the quadratic function $f(x)$ has its vertex at the point
$\tuple{\frac{1}{2}}{-\frac{9}{4}}$.

Next, let's calculate the $x$-intercepts~(or roots) of $f(x)$.  From
\Equation{eqn:quadratic_formula} you know that one of the
$x$-intercepts occurs at
%%
\begin{align*}
x
&=
\frac{
  -(-1) + \sqrt{(-1)^2 - 4(1)(-2)}
}{
  2(1)
} \\[4pt]
&=
\frac{
  1 + \sqrt{1 + 8}
}{
  2
} \\[4pt]
&=
\frac{
  1 + \sqrt{9}
}{
  2
} \\[4pt]
&=
\frac{
  1 + 3
}{
  2
} \\[4pt]
&=
2.
\end{align*}
%%
The other $x$-intercept occurs at
%%
\begin{align*}
x
&=
\frac{
  -(-1) - \sqrt{(-1)^2 - 4(1)(-2)}
}{
  2(1)
} \\[4pt]
&=
\frac{
  1 - \sqrt{9}
}{
  2
} \\[4pt]
&=
\frac{
  1 - 3
}{
  2
} \\[4pt]
&=
-1.
\end{align*}
%%
In other words, you have two different $x$-intercepts that occur at
the points $\tuple{2}{0}$ and $\tuple{-1}{0}$.  Plot the vertex and
the two $x$-intercepts on one set of coordinate axes, draw a line
through the points, and you obtain the graph in
\Figure{fig:a1_bminus1_cminus2}.
\end{solution}

\begin{figure}[!htbp]
\centering
\includegraphics[scale=1]{image/07/a1-bminus1-cminus2.pdf}
\caption{%%
  Graph of the function $f(x) = x^2 - x - 2$ through three points.
  The three points are the vertex and roots of $f(x)$.  In this case,
  the vertex of $f(x)$ is also the lowest point in the graph of
  $f(x)$.
}
\label{fig:a1_bminus1_cminus2}
\end{figure}

\begin{exercise}
Sketch the graph of $f(x) = -2x^2 + x + 3$.
\end{exercise}
%%
\ifbool{showSolution}{
\begin{solution}
You have the values $a = -2$, $b = 1$, and $c = 3$.  You can draw the
graph of $f(x) = -2x^2 + x + 3$ by using three points: the tip of the
parabola and the two $x$-intercepts.

First, let's calculate the tip point of the parabola.  Use
\Equation{eqn:parabola_tip_x_coordinate} to see that the tip point
occurs at the $x$-coordinate
%%
\begin{align*}
x
&=
-\frac{1}{2(-2)} \\[4pt]
&=
\frac{1}{4}.
\end{align*}
%%
The $y$-coordinate of the tip point is
%%
\begin{align*}
y
&=
f(1/4) \\[4pt]
&=
-2\parenthesis*{\frac{1}{4}}^2 + \frac{1}{4} + 3 \\[4pt]
&=
-2 \times \frac{1}{16} + \frac{4}{16} + 3 \\[4pt]
&=
\frac{4 - 2}{16} + 3 \\[4pt]
&=
\frac{1}{8} + 3 \\[4pt]
&=
\frac{1}{8} + \frac{24}{8} \\[4pt]
&=
\frac{25}{8}.
\end{align*}
%%
Thus the tip of the parabola is at the point
$\tuple{\frac{1}{4}}{\frac{25}{8}}$.

Next, you calculate the $x$-intercepts.  Using the quadratic
\Formula{eqn:quadratic_formula}, you see that an $x$-intercept has the
$x$-coordinate
%%
\begin{align*}
x
&=
\frac{
  -1 + \sqrt{1^2 - 4(-2)(3)}
}{
  2(-2)
} \\[4pt]
&=
\frac{
  -1 + \sqrt{1 + 24}
}{
  -4
} \\[4pt]
&=
\frac{
  -1 + 5
}{
  -4
} \\[4pt]
&=
-1.
\end{align*}
%%
The other $x$-intercept has the $x$-coordinate
%%
\begin{align*}
x
&=
\frac{
  -1 - \sqrt{1^2 - 4(-2)(3)}
}{
  2(-2)
} \\[4pt]
&=
\frac{
  -1 - 5
}{
  -4
} \\[4pt]
&=
\frac{-6}{-4} \\[4pt]
&=
\frac{3}{2}.
\end{align*}
%%
You now have the $x$-intercepts $\tuple{-1}{0}$ and
$\tuple{\frac{3}{2}}{0}$.

Finally, you plot the above three points on one set of coordinate
axes.  Draw a line through the points and you get the graph in
\Figure{fig:aminus2_b1_c3}.

\begin{figure}[!htbp]
\centering
\includegraphics[scale=1]{image/07/aminus2-b1-c3.pdf}
\caption{%%
  Graph of the function $f(x) = -2x^2 + x + 3$.
}
\label{fig:aminus2_b1_c3}
\end{figure}
\end{solution}
}{}

\begin{exercise}
Sketch the graph of $f(x) = -x^2 + 1$.
\end{exercise}

\ifbool{showSolution}{
\begin{solution}
You have the values $a = -1$ and $c = 1$.  The value of $b$ is
$b = 0$ because $f(x) = -x^2 + 1 = -x^2 + 0x + 1$.  Again, you can use
three points to draw the graph of $f(x)$.  Those three points are: the
tip of the parabola and the two $x$-intercepts.

Use \Equation{eqn:parabola_tip_x_coordinate} to see that the tip point
occurs at the $x$-coordinate
\[
x
=
-\frac{0}{2(-1)}
=
0
\]
whose corresponding $y$-coordinate is
\[
y
=
f(0)
=
-0^2 + 1
=
1.
\]
Then the tip of the parabola occurs at the point $\tuple{0}{1}$.

Let's calculate the $x$-intercepts.  Use
\Equation{eqn:quadratic_formula} to obtain the $x$-intercept
%%
\begin{align*}
x
&=
\frac{
  -0 + \sqrt{0^2 - 4(-1)(1)}
}{
  2(-1)
} \\[4pt]
&=
\frac{
  \sqrt{4}
}{
  -2
} \\[4pt]
&=
-1.
\end{align*}
%%
The other $x$-intercept occurs at
%%
\begin{align*}
x
&=
\frac{
  -0 - \sqrt{0^2 - 4(-1)(1)}
}{
  2(-1)
} \\[4pt]
&=
\frac{
  -\sqrt{4}
}{
  -2
} \\[4pt]
&=
1.
\end{align*}
%%
Thus the $x$-intercepts are $\tuple{-1}{0}$ and $\tuple{1}{0}$.  Plot
the three points and connect the dots to obtain the graph in
\Figure{fig:aminus1_c1}.
\end{solution}

\begin{figure}[!htbp]
\centering
\includegraphics[scale=1]{image/07/aminus1-c1.pdf}
\caption{%%
  Graph of the function $f(x) = -x^2 + 1$ through three points.
}
\label{fig:aminus1_c1}
\end{figure}
}{}


%%%%%%%%%%%%%%%%%%%%%%%%%%%%%%%%%%%%%%%%%%%%%%%%%%%%%%%%%%%%%%%%%%%%%%%%%%%

\section{The discriminant}

You have seen in \Section{sec:quadratic_formula} how the quadratic
formula can help you to sketch the graph of a quadratic function.  The
general strategy was to determine the tip point of the parabola and
then use the quadratic formula to calculate two $x$-intercepts of the
function.  You plot the three points on one set of coordinate axes and
connect the dots to obtain a graph of the function.  But this strategy
does not always work.

To understand why the graph drawing strategy
from \Section{sec:quadratic_formula} can fail, let's consider the
example of the function $f(x) = x^2$.  According to the above
strategy, you first need to calculate the tip point.  This is easy
enough.  You have the values $a = 1$ and $b = c = 0$.  Just use
\Equation{eqn:parabola_tip_x_coordinate} to see that the
$x$-coordinate of the tip point is
%%
\begin{align*}
x
&=
-\frac{0}{2(1)} \\[4pt]
&=
0.
\end{align*}
%%
The corresponding $y$-coordinate is
%%
\begin{align*}
y
&=
f(0) \\[4pt]
&=
0^2 \\[4pt]
&=
0
\end{align*}
%%
and therefore the function $f(x) = x^2$ has its tip point at the
origin.  Now use the quadratic \Formula{eqn:quadratic_formula} to see
that the $x$-intercepts of $f(x)$ occur at the $x$-coordinate
%%
\begin{align*}
x
&=
\frac{
  -0 \pm \sqrt{0^2 - 4(1)(0)}
}{
  2(1)
} \\[4pt]
&=
\pm
\frac{
  \sqrt{0}
}{
  2
} \\[4pt]
&=
0
\end{align*}
%%
with the corresponding $y$-coordinate being $y = f(0) = 0$.  The
upshot is that the tip of $f(x)$ is also its $x$-intercept, i.e.~the
point $\tuple{0}{0}$.  The above strategy for graphing a quadratic
function results in only one point, not the three that you require.
In the case of the function $f(x) = x^2$, you should have used the
strategy from \Section{sec:general_form}.

\begin{exercise}
Consider the function $f(x) = x^2 + 2x + 1$.  Explain why the graph
drawing strategy from \Section{sec:quadratic_formula} cannot help you
with sketching the graph of $f(x)$.
\end{exercise}
%%
\ifbool{showSolution}{
\begin{solution}
First, let's calculate the tip point of the parabola.  From
\Equation{eqn:parabola_tip_x_coordinate}, the $x$-coordinate of the
tip point is
%%
\begin{align*}
x
&=
-\frac{2}{2(1)} \\[4pt]
&=
-1.
\end{align*}
%%
The corresponding $y$-coordinate is $y = f(-1) = 0$.  The tip of the
parabola is located at the point $\tuple{-1}{0}$.

Next, you use the quadratic \Formula{thm:quadratic_formula} to
determine the $x$-intercepts of $f(x)$.  The $x$-coordinates of the
$x$-intercept are
%%
\begin{align*}
x
&=
\frac{
  -2 \pm \sqrt{2^2 - 4(1)(1)}
}{
  2(1)
} \\[4pt]
&=
\frac{
  -2 \pm \sqrt{4 - 4}
}{
  2
} \\[4pt]
&=
\frac{-2}{2} \\[4pt]
&=
-1.
\end{align*}
%%
The corresponding $y$-coordinate is $y = f(-1) = 0$.  In other words,
the function $f(x)$ has its $x$-intercepts at the point
$\tuple{-1}{0}$, which is also the tip point of the parabola.  In the
case of $f(x)$, the graph sketching strategy
from \Section{sec:quadratic_formula} results in only one point, not
the three that you need to draw the graph.
\end{solution}
}{}

Is there a way to determine when to use the strategies
from \Sections{sec:general_form}{sec:quadratic_formula} for graphing a
quadratic function?  The answer is yes, but you need to understand a
number called the \emph{discriminant} of a quadratic function.

\begin{definition}
\label{def:discriminant}
\textbf{Discriminant.}
Let $\triple{a}{b}{c} \in \RR$ such that $a \neq 0$ and consider the
quadratic function $f(x) = ax^2 + bx + c$.  If the roots of $f(x)$ are
\[
x
=
\frac{
  -b \pm \sqrt{b^2 - 4ac}
}{
  2a
}
\]
then the number $\Delta = b^2 - 4ac$ is called the \emph{discriminant}
of $f(x)$.
\end{definition}

Like any real number, the discriminant can take on one of three types
of values.  The discriminant can be either negative, zero, or
positive.  Depending on the value of the discriminant, a quadratic
function will have either zero, one, or two $x$-intercepts.  Let's
consider each of the three cases separately.  Suppose $f(x)$ is a
quadratic function whose discriminant is $\Delta$.

\begin{packedenumeral}
\item If $\Delta < 0$, then $f(x)$ does not intersect the $x$-axis.
  The graph of $f(x)$ lies wholly above or below the $x$-axis.  See
  \Figure{fig:negative_discriminant}.

\item If $\Delta = 0$, then $f(x)$ intersects the $x$-axis once.  The
  point of intersection is also the tip point of the quadratic
  function.  This case is illustrated in
  \Figure{fig:zero_discriminant}.

\item If $\Delta > 0$, then $f(x)$ has two different $x$-intercepts.
  See \Figure{fig:positive_discriminant}.
\end{packedenumeral}

\begin{figure}[!htbp]
\centering
\subfigure[]{
  \includegraphics[scale=0.85]{image/07/a2-b1-c1.pdf}
}
%%
\quad
%%
\subfigure[]{
  \includegraphics[scale=0.93]{image/07/aminus2-b1-cminus2.pdf}
}
\caption{%%
  When the discriminant is negative, the graph of a quadratic function
  does not intersect the $x$-axis.
}
\label{fig:negative_discriminant}
\end{figure}

\begin{figure}[!htbp]
\centering
\subfigure[]{
  \includegraphics[scale=1]{image/07/a1-bminus2-c1.pdf}
}
%%
\qquad
%%
\subfigure[]{
  \includegraphics[scale=1]{image/07/aminus9quarter-bminus3-cminus1.pdf}
}
\caption{%%
  When the discriminant is zero, the graph of a quadratic function
  intersects the $x$-axis at exactly one point.
}
\label{fig:zero_discriminant}
\end{figure}

\begin{figure}[!htbp]
\centering
\subfigure[]{
  \includegraphics[scale=1]{image/07/a1-b2-cminus1.pdf}
}
%%
\qquad
%%
\subfigure[]{
  \includegraphics[scale=1]{image/07/aminus1-b3half-cminus2.pdf}
}
\caption{%%
  When the discriminant is positive, the graph of a quadratic function
  intersects the $x$-axis at two different points.
}
\label{fig:positive_discriminant}
\end{figure}

What all this means is that if the discriminant of a quadratic
function $f(x)$ is either negative or zero, then you should use the
strategy from \Section{sec:general_form} to sketch the graph of
$f(x)$.  If the discriminant of $f(x)$ is positive, then the strategy
from \Section{sec:quadratic_formula} can be used to draw the graph of
$f(x)$.  Or you could forget about \Section{sec:quadratic_formula}
entirely and use the technique from \Section{sec:general_form} because
the graph drawing strategy from that section will always work for any
quadratic function.  So why would you bother to learn about the
discriminant?

An answer can be found in
\Problems{prob:zero_discriminant}{prob:positive_discriminant} and can
be summarised as follows.  The discriminant of a quadratic function
$f(x)$ tells you how many roots $f(x)$ has.  You do not need to use
the quadratic \Formula{eqn:quadratic_formula} to know that $f(x)$ has
either a unique root or two different roots or that the graph of
$f(x)$ does not intersect the $x$-axis.  All you need to do is
calculate the discriminant of $f(x)$, which is much simpler to do than
calculating the actual roots of $f(x)$.

\begin{example}
Explain why the quadratic function $f(x) = 2x^2 + x + 1$ does not
intersect the $x$-axis.
\end{example}

\begin{solution}
The discriminant of $f(x)$ is
$\Delta = 1^2 - 4(2)(1) = 1 - 8 = -7$.  Since $\Delta$ is negative,
the graph of $f(x)$ does not intersect the $x$-axis.
\end{solution}

\begin{exercise}
How many unique roots does the function $f(x) = x^2 - 2x + 1$ have?
\end{exercise}
%%
\ifbool{showSolution}{
\begin{solution}
The discriminant of $f(x)$ is
$\Delta = (-2)^2 - 4(1)(1) = 4 - 4 = 0$.  Therefore $f(x)$ has a
unique root.
\end{solution}
}{}

\begin{exercise}
Explain why $f(x) = -x^2 - \frac{7}{2}x - 2$ has two different roots.
\end{exercise}
%%
\ifbool{showSolution}{
\begin{solution}
The discriminant of $f(x)$ is
$\Delta
=
\parenthesis*{-\frac{7}{2}}^2 - 4(-1)(-2)
=
\frac{49}{4} - 8$,
which is a positive number because $\Delta = 17/4$.  Therefore $f(x)$
has two different roots.
\end{solution}
}{}


%%%%%%%%%%%%%%%%%%%%%%%%%%%%%%%%%%%%%%%%%%%%%%%%%%%%%%%%%%%%%%%%%%%%%%%%%%%

\section*{Problem}

\begin{problem}
\item\label{prob:zero_discriminant}
  Let $f(x)$ be a quadratic function whose discriminant is zero.
  %%
  \begin{packedenum}
  \item\label{subprob:zero_discriminant_unique_root}
    Prove that $f(x)$ has a unique root.

  \item\label{subprob:zero_discriminant_root_is_tip_point}
    Prove that the root of $f(x)$ is also the tip point of the
    parabola.
  \end{packedenum}
\ifbool{showSolution}{
\begin{solution}
\solutionpart{subprob:zero_discriminant_unique_root}
Suppose the quadratic function $f(x)$ can be written as
$f(x) = ax^2 + bx + c$, where $\triple{a}{b}{c} \in \RR$ such that
$a \neq 0$.  You need to show that there is only one value of $x$ such
that $f(x) = 0$.  By \Theorem{thm:quadratic_formula} the roots of
$f(x)$ are
\[
x
=
\frac{
  -b \pm \sqrt{b^2 - 4ac}
}{
  2a
}
\]
where the number $\Delta = b^2 - 4ac$ is the discriminant of $f(x)$.
Since the discriminant is $\Delta = 0$, then the roots of $f(x)$ can
be written as
%%
\begin{align*}
x
&=
\frac{
  -b \pm \sqrt{0}
}{
  2a
} \\[4pt]
&=
\frac{-b \pm 0}{2a} \\[4pt]
&=
-\frac{b}{2a}
\end{align*}
%%
which is one number.  Therefore $f(x)$ has a unique root.

\solutionpart{subprob:zero_discriminant_root_is_tip_point}
From \Equation{eqn:parabola_tip_x_coordinate}, the tip of the parabola
occurs at the $x$-coordinate $x = -b/2a$, which
from \Part{subprob:zero_discriminant_unique_root} is also the unique
root of $f(x)$.
\end{solution}
}{}

\item\label{prob:positive_discriminant}
  If $f(x)$ is a quadratic function with positive discriminant, prove
  that $f(x)$ has two different roots.
\ifbool{showSolution}{
\begin{solution}
Suppose that the quadratic function $f(x)$ can be written as
$f(x) = ax^2 + bx + c$, where $\triple{a}{b}{c}$ are real numbers such
that $a \neq 0$.  Use \Theorem{thm:quadratic_formula} to write the
roots of $f(x)$ as
\[
x
=
\frac{
  -b \pm \sqrt{b^2 - 4ac}
}{
  2a
}.
\]
As the discriminant is assumed to be positive, the number
$\Delta = b^2 - 4ac$ is positive and so the square root
$\sqrt{\Delta}$ is also positive.  Thus the roots of $f(x)$ can be
written as
\[
x
=
\frac{-b + \sqrt{\Delta}}{2a}
%%
\qquad
\text{and}
\qquad
%%
x
=
\frac{-b - \sqrt{\Delta}}{2a}
\]
which are two different numbers because $\sqrt{\Delta} > 0$.
\end{solution}
}{}

\item If $a$ is any real number, its square root can be written as
  $\sqrt{a} = a^{1/2}$.  Explain what is wrong with the following
  expression:
  %%
  \begin{align*}
  1
  &=
  \sqrt{1} \\[4pt]
  &=
  \sqrt{(-1)^2} \\[4pt]
  &=
  \bigparen{(-1)^2}^{1/2} \\[4pt]
  &=
  (-1)^{2/2} \\[4pt]
  &=
  (-1)^1 \\[4pt]
  &=
  -1.
  \end{align*}
\ifbool{showSolution}{
\begin{solution}
The number $\sqrt{(-1)^2}$ is the distance between the points
$\tuple{0}{0}$ and $\tuple{-1}{0}$.  This is because you have
%%
\begin{align*}
\sqrt{
  (0 - 0)^2 + (-1 - 0)^2
}
&=
\sqrt{0^2 + (-1)^2} \\[4pt]
&=
\sqrt{(-1)^2}.
\end{align*}
%%
As the distance cannot be negative, you must first simplify the
expression $(-1)^2$ before you take its square root.
\end{solution}
}{}
\end{problem}

\end{document}
