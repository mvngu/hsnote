%%%%%%%%%%%%%%%%%%%%%%%%%%%%%%%%%%%%%%%%%%%%%%%%%%%%%%%%%%%%%%%%%%%%%%%%%%%

\documentclass{standalone}

\usepackage{mathptmx}
\usepackage{tikz}
\usetikzlibrary{external}
\tikzexternalize{square}

%% We default to Times.
\renewcommand{\rmdefault}{ptm}
\renewcommand{\ttdefault}{pcr}
%% Enable Times/Palatino main text font.
\normalfont\selectfont

\newcommand{\comma}{,\,}
\newcommand{\tuple}[2]{(#1\comma #2)}

%% The Cartesian coordinate system.
\newcommand{\cartesianCoordinate}{%%
  %% The x-axis.
  \draw[axisStyle] (xstart) -- (xend);
  \node at (xend) [right] {$x$};
  %% The y-axis.
  \draw[axisStyle] (ystart) -- (yend);
  \node at (yend) [above] {$y$};
}

%% A point on the Cartesian plane.
%%
%% #1 -- The x-coordinate of the point.
%% #2 -- The y-coordinate of the point.
%% #3 -- Label the point with this name.
%% #4 -- Where to place the label relative to the point.
\newcommand{\xyPoint}[4]{%%
  \node[nodeStyle] at (#1,#2) {};
  \node at (#1,#2) [#4] {$#3$};
}

%% Reflections of a point.

\begin{document}

\begin{tikzpicture}[%%
  axisStyle/.style={->,thick},%%
  dashStyle/.style={-,thick,dashed},%%
  nodeStyle/.style={circle,inner sep=1.5pt,fill=black,black}
]
%%
%%
\pgfmathsetmacro{\x}{3}
\pgfmathsetmacro{\xhigh}{4.5}
\pgfmathsetmacro{\xlow}{-4.5}
\pgfmathsetmacro{\y}{\x}
\pgfmathsetmacro{\yhigh}{\xhigh}
\pgfmathsetmacro{\ylow}{\xlow}
\coordinate (A) at (\x,\y);
\coordinate (B) at (-\x,\y);
\coordinate (C) at (-\x,-\y);
\coordinate (D) at (\x,-\y);
\coordinate (xend) at (\xhigh,0);
\coordinate (xstart) at (\xlow,0);
\coordinate (yend) at (0,\yhigh);
\coordinate (ystart) at (0,\ylow);
%%
%% The Cartesian coordinate system.
\cartesianCoordinate
%%
%% Reflections of the point (x,y).
%% The original point (x,y).
\xyPoint{\x}{\y}{\tuple{\frac{a}{2}}{\frac{a}{2}}}{above}
%% Reflection with respect to the x-axis.
\xyPoint{\x}{-\y}{\tuple{\frac{a}{2}}{-\frac{a}{2}}}{below}
%% Reflection with respect to the y-axis.
\xyPoint{-\x}{\y}{\tuple{-\frac{a}{2}}{\frac{a}{2}}}{above}
%% Diagonal reflection.
\xyPoint{-\x}{-\y}{\tuple{-\frac{a}{2}}{-\frac{a}{2}}}{below}
%%
%% Connect the corners to form a square.
\draw[dashStyle] (A) -- (B);
\draw[dashStyle] (B) -- (C);
\draw[dashStyle] (C) -- (D);
\draw[dashStyle] (D) -- (A);
\end{tikzpicture}

\end{document}
