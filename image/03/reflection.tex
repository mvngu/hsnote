%%%%%%%%%%%%%%%%%%%%%%%%%%%%%%%%%%%%%%%%%%%%%%%%%%%%%%%%%%%%%%%%%%%%%%%%%%%

\documentclass{standalone}

\usepackage{mathptmx}
\usepackage{tikz}
\usetikzlibrary{external}
\tikzexternalize{reflection}

%% We default to Times.
\renewcommand{\rmdefault}{ptm}
\renewcommand{\ttdefault}{pcr}
%% Enable Times/Palatino main text font.
\normalfont\selectfont

\newcommand{\comma}{,\,}
\newcommand{\tuple}[2]{(#1\comma #2)}

%% A tick on the x-axis.
%%
%% #1 -- A point on the x-axis.
\newcommand{\xtick}[1]{%%
  \draw[tickStyle] (#1,\tickdx) -- (#1,-\tickdx);
  \node at (#1,-\tickdx) [below] {$#1$};
}

%% A tick on the y-axis.
%%
%% #1 -- A point on the y-axis.
\newcommand{\ytick}[1]{%%
  \draw[tickStyle] (-\tickdx,#1) -- (\tickdx,#1);
  \node at (-\tickdx,#1) [left] {$#1$};
}

%% The Cartesian coordinate system.
\newcommand{\cartesianCoordinate}{%%
  %% The x-axis.
  \draw[axisStyle] (xstart) -- (xend);
  \foreach \x in {-4,-3,-2,-1,1,2,3,4}{
    \xtick{\x}
  };
  \node at (xend) [right] {$x$};
  %% The y-axis.
  \draw[axisStyle] (ystart) -- (yend);
  \foreach \y in {-3,-2,-1,1,2,3}{
    \ytick{\y}
  };
  \node at (yend) [above] {$y$};
}

%% A point on the Cartesian plane.
%%
%% #1 -- The x-coordinate of the point.
%% #2 -- The y-coordinate of the point.
%% #3 -- Label the point with this name.
%% #4 -- Where to place the label relative to the point.
\newcommand{\xyPoint}[4]{%%
  \node[nodeStyle] at (#1,#2) {};
  \node at (#1,#2) [#4] {$#3$};
}

%% Reflections of a point.

\begin{document}

\begin{tikzpicture}[%%
  axisStyle/.style={->,thick},%%
  nodeStyle/.style={circle,inner sep=1.5pt,fill=black,black},%%
  tickStyle/.style={-,thick}
]
%%
%%
\pgfmathsetmacro{\tickdx}{0.1}
\pgfmathsetmacro{\xhigh}{4.5}
\pgfmathsetmacro{\xlow}{-4.5}
\pgfmathsetmacro{\yhigh}{3.5}
\pgfmathsetmacro{\ylow}{-3.5}
\coordinate (xend) at (\xhigh,0);
\coordinate (xstart) at (\xlow,0);
\coordinate (yend) at (0,\yhigh);
\coordinate (ystart) at (0,\ylow);
%%
%% The Cartesian coordinate system.
\cartesianCoordinate
%%
%% Reflections of the point (3,2).
%% The original point (3,2).
\xyPoint{3}{2}{\tuple{3}{2}}{above}
%% Reflection with respect to the x-axis.
\xyPoint{3}{-2}{\tuple{3}{-2}}{below}
%% Reflection with respect to the y-axis.
\xyPoint{-3}{2}{\tuple{-3}{2}}{above}
%% Diagonal reflection.
\xyPoint{-3}{-2}{\tuple{-3}{-2}}{below}
\end{tikzpicture}

\end{document}
