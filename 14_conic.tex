%%%%%%%%%%%%%%%%%%%%%%%%%%%%%%%%%%%%%%%%%%%%%%%%%%%%%%%%%%%%%%%%%%%%%%%%%%%

\documentclass[a4paper,oneside,12pt]{article}
\usepackage{mystyle}

\begin{document}

\title{\Large\bf Conic sections}
\author{%%
  Minh Van Nguyen \\
  \url{mvngu@gmx.com}
}
\date{\today}
\maketitle


%%%%%%%%%%%%%%%%%%%%%%%%%%%%%%%%%%%%%%%%%%%%%%%%%%%%%%%%%%%%%%%%%%%%%%%%%%%

\section{Parametric equations}

{\color{red}
\begin{packeditem}
\item circle
  \url{https://en.wikipedia.org/wiki/Parametric_equation}

\item Archimedean spiral
  \url{https://en.wikipedia.org/wiki/Archimedean_spiral}

\item Lissajous curve
  \url{https://en.wikipedia.org/wiki/Lissajous_curve}

\item cycloid
  \url{https://en.wikipedia.org/wiki/Cycloid}

\item butterfly curve
  \url{https://en.wikipedia.org/wiki/Butterfly_curve_(transcendental)}

\item logarithmic spiral
  \url{https://en.wikipedia.org/wiki/Logarithmic_spiral}

\item golden spiral
  \url{https://en.wikipedia.org/wiki/Golden_spiral}

\item rose
  \url{https://en.wikipedia.org/wiki/Rose_(mathematics)}

\item quadrifolium~(or four-leaved clover)
  \url{https://en.wikipedia.org/wiki/Quadrifolium}

\item siluroid curve~(or fish)
  \url{https://siluroid.dejudicibus.it}

\item move curves at
  \url{http://xahlee.info/SpecialPlaneCurves_dir/specialPlaneCurves.html}
\end{packeditem}
}

On 04th December 1996, the National Aeronautics and Space
Administration~(NASA) of the USA launched a spacecraft called the Mars
Pathfinder.  The destination of the spacecraft was the planet Mars.
On 04th July 1997, the spacecraft landed on Mars.  The mission had one
primary objective: to study the planet Mars.  This task was performed
by a robotic rover called Sojourner, which explored the surface of
Mars for $85$ Earth days.\footnote{
  You can read about the results sent back to Earth by Sojourner in
  the paper at
  \url{http://web.archive.org/web/20180729072504/http://science.sciencemag.org/content/sci/278/5344/1758.full.pdf}.
}


%%%%%%%%%%%%%%%%%%%%%%%%%%%%%%%%%%%%%%%%%%%%%%%%%%%%%%%%%%%%%%%%%%%%%%%%%%%

\section{Ellipses}

{\color{red}
\begin{packeditem}
\item ellipse: equation and parametric form
  \url{https://en.wikipedia.org/wiki/Ellipse}

\item play with ellipse at
  \url{https://www.mathsisfun.com/geometry/ellipse.html}

\item circumference of ellipse

\item area of ellipse

\item elliptic orbit
  \url{https://en.wikipedia.org/wiki/Elliptic_orbit}

\item orbit equation
  \url{https://en.wikipedia.org/wiki/Orbit_equation}

\item orbital eccentricity
  \url{https://en.wikipedia.org/wiki/Orbital_eccentricity}

\item \url{http://www.braeunig.us/space/orbmech.htm}

\item Kepler's laws of planetary motion
  \url{https://en.wikipedia.org/wiki/Kepler\%27s_laws_of_planetary_motion}

\item sphere of influence
  \url{https://en.wikipedia.org/wiki/Sphere_of_influence_(astrodynamics)}
\end{packeditem}
}


%%%%%%%%%%%%%%%%%%%%%%%%%%%%%%%%%%%%%%%%%%%%%%%%%%%%%%%%%%%%%%%%%%%%%%%%%%%

\section{Hyperbolas}

{\color{red}
\begin{packeditem}
\item hyperbola
  \url{https://en.wikipedia.org/wiki/Hyperbola}
  \url{https://courses.lumenlearning.com/boundless-algebra/chapter/the-hyperbola/}
  \url{https://www.intmath.com/plane-analytic-geometry/6-hyperbola.php}

\item hyperbolic growth
  \url{https://en.wikipedia.org/wiki/Hyperbolic_growth}

\item hyperbolic trajectory
  \url{https://en.wikipedia.org/wiki/Hyperbolic_trajectory}
  \url{https://www.lunarplanner.com/Snippets/09.02.24-CometLulin/}
  \url{https://en.wikipedia.org/wiki/\%CA\%BBOumuamua}

\item Kepler orbit
  \url{https://en.wikipedia.org/wiki/Kepler_orbit}
\end{packeditem}
}


%%%%%%%%%%%%%%%%%%%%%%%%%%%%%%%%%%%%%%%%%%%%%%%%%%%%%%%%%%%%%%%%%%%%%%%%%%%

\section{Hyperbolic functions}

{\color{red}
\begin{packeditem}
\item \url{https://en.wikipedia.org/wiki/Hyperbolic_function}

\item \url{https://brilliant.org/wiki/hyperbolic-trigonometric-functions/}
\end{packeditem}
}


\newpage
%%%%%%%%%%%%%%%%%%%%%%%%%%%%%%%%%%%%%%%%%%%%%%%%%%%%%%%%%%%%%%%%%%%%%%%%%%%

\section*{Problem}

\begin{problem}
\item Read the following paper by George Musser and Mark Alpert:
  \emph{How to go to Mars}.\footnote{
    The paper is available at
    \url{http://web.archive.org/web/20180729065057/https://www.physics.ohio-state.edu/~kagan/phy596/Articles/IonPropulsion/HowtogoToMars.pdf}.
  }

\item Read the following paper by Tim Beardsley:
  \emph{The Way to Go in Space}.\footnote{
    The paper is available at
    \url{http://web.archive.org/web/20180729070023/https://www.physics.ohio-state.edu/~kagan/phy596/Articles/IonPropulsion/TheWayToGoInSpace.pdf}.
  }
\end{problem}

\end{document}
