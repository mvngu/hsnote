%%%%%%%%%%%%%%%%%%%%%%%%%%%%%%%%%%%%%%%%%%%%%%%%%%%%%%%%%%%%%%%%%%%%%%%%%%%

\documentclass[a4paper,oneside,12pt]{article}
\usepackage{mystyle}

\begin{document}

\title{\Large\bf Rational numbers}
\author{%%
  Minh Van Nguyen \\
  \url{mvngu@gmx.com}
}
\date{\today}
\maketitle

%%%%%%%%%%%%%%%%%%%%%%%%%%%%%%%%%%%%%%%%%%%%%%%%%%%%%%%%%%%%%%%%%%%%%%%%%%%

\section{Rational and irrational numbers}

A \emph{rational} number is a ratio of two integers.  A
rational number is also known as a fraction.  These are rational
numbers: $1/2$, $3/4$, $11/9$, and $-10/2$.  You write $5/2 \in \QQ$
to mean that $5/2$ belongs to the set of rational numbers.  The set of
all rational numbers is written as
\[
\QQ
=
\setdes{
  \frac{a}{b}
}{
  \pair{a}{b} \in \ZZ
  \text{ and }
  b \neq 0
}.
\]
This means that the set of rational numbers consists of all numbers of
the form $a/b$ such that $a$ and $b$ are integers, but $b$ cannot be
zero.  The top integer $a$ is called the \emph{numerator} and the
bottom integer $b$ is called the \emph{denominator}.  If $b = 1$, then
$a/b = a/1 = a$ and consequently any integer is also a rational
number.

\begin{exercise}
Provide an example of a rational number that is also an integer.
Explain why your number is an integer.
\end{exercise}

\ifbool{showSolution}{
\begin{solution}
The rational number $2/1$ is also an integer because it can be written
as $2/1 = 2$ and $2$ is an integer.
\end{solution}
}{}

\begin{exercise}
Give an example of a rational number that is not an integer.  Explain
why your number is not an integer.
\end{exercise}

\ifbool{showSolution}{
\begin{solution}
The ratio $1/2$ is not an integer because $1/2$ cannot be simplified
to any integer.
\end{solution}
}{}

\begin{exercise}
If $a/b$ is a rational number, why can't $b$ be zero?  Consider the
examples of $1/4$ and $1/0$.
\end{exercise}

\ifbool{showSolution}{
\begin{solution}
If in the rational number $a/b$ you have $b = 0$, then the ratio
$a / 0$ is not defined.  This means that as $b$ comes closer and
closer to zero, the value of the ratio $a/b$ does not come to a fixed
number.
\end{solution}
}{}

An \emph{irrational} number is any number that cannot be written as a
ratio of two integers.  The number $\sqrt{2}$ is an irrational number
so it cannot be an integer and therefore the expression
$\displaystyle{\frac{\sqrt{2}}{2}}$ is not a fraction.  But how did
you know that $\sqrt{2}$ is irrational?  How can you prove that
$\sqrt{2}$ is irrational?

\begin{theorem}
The number $\sqrt{2}$ is irrational.
\end{theorem}

\begin{proof}

\end{proof}

\begin{exercise}
Provide another example of an irrational number.
\end{exercise}

\ifbool{showSolution}{
\begin{solution}
The number $\pi = 3.141592\dots$ is an irrational number.
\end{solution}
}{}

The number $\pi = 3.141592\dots$ is often used to measure the area and
circumference of a circular region.  Since $\pi$ is irrational, the
number cannot be written as a ratio of integers.  So for practical
purposes, you must approximate $\pi$ as closely as you can.  The Greek
mathematician Archimedes used the fraction $22 / 7$ to approximate
$\pi$.

The set of \emph{real} numbers is made up of all rational and all
irrational numbers.  The set of real numbers is written as $\RR$.  For
example, the ratio $3/7$ is a real number, the square root $\sqrt{2}$
is a real number, and the integer $42$ is a real number.

\begin{exercise}
Explain why any integer is a real number.
\end{exercise}

\ifbool{showSolution}{
\begin{solution}
Any integer is also a rational number.  Since a rational number is
also a real number, it follows that any integer is also a real
number.
\end{solution}
}{}

\begin{exercise}
How can a number be represented as a picture?
\end{exercise}

\ifbool{showSolution}{
\begin{solution}
A real number can be represented as a point on the number line.
\end{solution}
}{}

\begin{exercise}
Simplify the expression
$\displaystyle{
  \frac{
    3 \times 2 + (11 - 5)
  }{
    2 \times (3 + 7)
  }
}$.
\end{exercise}

\ifbool{showSolution}{
\begin{solution}
The expression
$\displaystyle{
  \frac{
    3 \times 2 + (11 - 5)
  }{
    2 \times (3 + 7)
  }
}$
can be simplified as
%%
\begin{align*}
\frac{
  3 \times 2 + (11 - 5)
}{
  2 \times (3 + 7)
}
&=
\frac{
  3 \times 2 + 6
}{
  2 \times (3 + 7)
} \\[4pt]
&=
\frac{
  3 \times 2 + 6
}{
  2 \times 10
} \\[4pt]
&=
\frac{
  6 + 6
}{
  20
} \\[4pt]
&=
\frac{
  12
}{
  20
} \\[4pt]
&=
\frac{
  3
}{
  5
}.
\end{align*}
\end{solution}
}{}


%%%%%%%%%%%%%%%%%%%%%%%%%%%%%%%%%%%%%%%%%%%%%%%%%%%%%%%%%%%%%%%%%%%%%%%%%%%

\section*{Problem}

\begin{problem}
\item Give an example of something that can be represented as a
  rational number.
\ifbool{showSolution}{
  \begin{solution}
  The number of oranges you have eaten divided by the number of
  oranges you have altogether.
  \end{solution}
}{}

\item Provide an example of something that can be represented as a
  real number.
\ifbool{showSolution}{
  \begin{solution}
  The area of your house.
  \end{solution}
}{}

\item Simplify the expression
  \[
  E
  =
  \frac{
    3 m c^2
  }{
    (6 - 4) + 1
  }.
  \]
\ifbool{showSolution}{
  \begin{solution}
  \begin{align*}
  E
  &=
  \frac{
    3 m c^2
  }{
    (6 - 4) + 1
  } \\[4pt]
  &=
  \frac{
    3 m c^2
  }{
    2 + 1
  } \\[4pt]
  &=
  \frac{
    3 m c^2
  }{
    3
  } \\[4pt]
  &=
  m c^2.
  \end{align*}
  \end{solution}
}{}

\item The number $e = 2.71828\dots$ is called \emph{Euler's constant}.
  Read about $e$ on Wikipedia or search on the Internet for
  ``Euler's constant.''  Is $e$ a rational number?  Is $e$ an
  irrational number?  Is $e$ an integer?  Does $e$ belong to the set
  $\RR$ of real numbers?  Where can you find a use for the number $e$?
\ifbool{showSolution}{
  \begin{solution}
  Euler's constant $e = 2.71828\dots$ is an irrational number, which
  means that $e$ also belongs to the set of real numbers.  The number
  $e$ is used as the base of the \emph{natural logarithm},
  i.e.~logarithm to the base $e$.  The number $e$ is also used in the
  calculation of compound interest.
  \end{solution}
}{}
\end{problem}

\end{document}
