%%%%%%%%%%%%%%%%%%%%%%%%%%%%%%%%%%%%%%%%%%%%%%%%%%%%%%%%%%%%%%%%%%%%%%%%%%%

\documentclass[a4paper,oneside,12pt]{article}
\usepackage{mystyle}

\begin{document}

\title{\Large\bf Integers}
\author{%%
  Minh Van Nguyen \\
  \url{mvngu@gmx.com}
}
\date{\today}
\maketitle


%%%%%%%%%%%%%%%%%%%%%%%%%%%%%%%%%%%%%%%%%%%%%%%%%%%%%%%%%%%%%%%%%%%%%%%%%%%

\section{Even and odd}

A \emph{set} is a bunch or collection of objects.  The objects of a
set are called its \emph{elements}, which must all be different.  For
example, the set of all rubbish bins at your house, the set of all
fingers on your two hands, and the set of all whole numbers from one
to ten.  You use curly braces to represent a set.  The set of all even
whole numbers from one to ten can be written as
$\set{\quintuple{2}{4}{6}{8}{10}}$, the set of whole numbers from zero
to three can be written as $\set{\quadruple{0}{1}{2}{3}}$, and the set
of the first four prime numbers is $\set{\quadruple{2}{3}{5}{7}}$.
The set $\set{\triple{1}{2}{1}}$ is the same as the set
$\set{\pair{1}{2}}$ because the elements of a set must all be
distinct.

\begin{exercise}
Write out the set of all whole numbers from one to five.
\end{exercise}

\ifbool{showSolution}{
\begin{solution}
The set of all whole numbers from one up to and including five can be
written as $\set{\quintuple{1}{2}{3}{4}{5}}$.
\end{solution}
}{}

\begin{exercise}
Give two more examples of sets.  The sets do not have to be
collections of numbers.
\end{exercise}

\ifbool{showSolution}{
\begin{solution}
The set of all pens in your pencil case.  The set of all cars that you
own.  If you do not own a car, then this set of cars has zero
elements.  The set that has zero elements is called the \emph{empty}
set and is written as $\emptyset$.
\end{solution}
}{}

An \emph{integer} is a whole number such as $3$, $7$, $\the\year$,
$-5$, or $-9$.  The set of all integers is written as
$\ZZ = \set{\dots,\, -3,\, -2,\, -1,\, 0,\, 1,\, 2,\, 3,\dots}$.  You
write $2 \in \ZZ$ to mean that the number $2$ belongs to the set of
integers.

\begin{definition}
\textbf{Even integers.}
An integer $n$ is \emph{even} if you can write it as $n = 2k$, where
$k$ is some other integer.
\end{definition}

\begin{definition}
\textbf{Odd integers.}
An integer $n$ is \emph{odd} if you can write it as $n = 2r + 1$,
where $r$ is some other integer.
\end{definition}

For example, the integer $2$ is even so you can write
$2 = 2 \times 1$.  The integer $6$ is even so you can write
$6 = 2 \times 3$.  The integer $3$ is odd and you can write it as
$3 = 2 \times 1 + 1$.  The integer $9$ is odd because
$9 = 2 \times 4 + 1$.

\begin{exercise}
Is the integer $32$ even?  If yes, write $32$ in the form $2k$, where
$k$ is an integer.  If no, explain why $32$ is not even.
\end{exercise}

\ifbool{showSolution}{
\begin{solution}
The integer $32$ is even because it can be written as
$32 = 2 \times 16$.
\end{solution}
}{}

\begin{exercise}
Is the integer $51$ even?  If yes, explain why.  If no, explain why
not.
\end{exercise}

\ifbool{showSolution}{
\begin{solution}
The integer $51$ is odd because you can write $51$ as
$51 = 2 \times 25 + 1$.
\end{solution}
}{}


%%%%%%%%%%%%%%%%%%%%%%%%%%%%%%%%%%%%%%%%%%%%%%%%%%%%%%%%%%%%%%%%%%%%%%%%%%%

\section{Associative laws}

The following rules are very useful when you want to simplify a
complicated mathematical expression.

\begin{definition}
\textbf{Associative laws.}
If $\triple{a}{b}{c}$ are any numbers, then the
\emph{associative laws} state that you can write:
%%
\begin{packedenumeral}
\item $abc = a (bc) = (ab) c$

\item $a + b + c = a + (b + c) = (a + b) + c$
\end{packedenumeral}
\end{definition}

This means that for the product $abc$, you can first simplify $bc$.
If you want, you can simplify $ab$ first.  Similarly, for the sum
$a + b + c$, you can simplify $a + b$ first.  If you want, you can
simplify $b + c$ first.  Now let's put the associative laws to some
good use.

\begin{exercise}
Simplify the product $2 \times 3 \times 5$ in two ways.
\end{exercise}

\ifbool{showSolution}{
\begin{solution}
Using the associative laws, you can simplify $3 \times 5$ first so
you can write
%%
\begin{align*}
2 \times 3 \times 5
&=
2 \times (3 \times 5) \\[4pt]
&=
2 \times 15 \\[4pt]
&=
30.
\end{align*}
%%
On the other hand, the associative laws also allow you to simplify
$2 \times 3$ first.  In other words, you can also write
%%
\begin{align*}
2 \times 3 \times 5
&=
(2 \times 3) \times 5 \\[4pt]
&=
6 \times 5 \\[4pt]
&=
30.
\end{align*}
\end{solution}
}{}

\begin{exercise}
Simplify the sum $3 + 5 + 7$ in two ways.
\end{exercise}

\ifbool{showSolution}{
\begin{solution}
The associative laws allow you to first simplify $3 + 5$ and so you
can write
%%
\begin{align*}
3 + 5 + 7
&=
(3 + 5) + 7 \\[4pt]
&=
8 + 7 \\[4pt]
&=
15.
\end{align*}
%%
Furthermore, the associative laws also allow you to first simplify
$5 + 7$.  This means that you can also write
%%
\begin{align*}
3 + 5 + 7
&=
3 + (5 + 7) \\[4pt]
&=
3 + 12 \\[4pt]
&=
15.
\end{align*}
\end{solution}
}{}

Have you ever noticed that when you multiply together two even
integers, the result will also be even?
\Theorem{thm:product_of_two_evens_is_even} below confirms this
observation and it also shows why the observation is true.

\begin{theorem}
\label{thm:product_of_two_evens_is_even}
If $n$ and $r$ are even integers, then $nr = n \times r$ is even.
\end{theorem}

\begin{proof}
Since $n$ and $r$ are even, then you can write each of them as
$n = 2k$ and $r = 2s$, where $k$ and $s$ are integers.  Now you can
use the associative laws to write the product $nr$ as
%%
\begin{align*}
nr
&=
(2k) \times (2s) \\[4pt]
&=
2 \times k \times 2 \times s \\[4pt]
&=
2 \times (k \times 2 \times s) \\[4pt]
&=
2 (2ks).
\end{align*}
%%
Whatever the number $2ks$ is, it will be an integer because the
multiplication of any three integers is an integer.  Make the
substitution $t = 2ks$ and you have $nr = 2t$, where $t = 2ks$ is an
integer.  Therefore $nr$ is even.
\end{proof}

You can use \Theorem{thm:product_of_two_evens_is_even} to conclude
whether the result of multiplying two integers will be even.  If both
integers are even, then their product will be even.
\Theorem{thm:product_of_two_evens_is_even} also states that the
multiplication of $2$ by any integer will be even. For example, you
don't have to simplify the product $6 \times 8$ to know that the
result will be an even integer.  Since $6 = 2 \times 3$ and
$8 = 2 \times 4$, then you can write
%%
\begin{align*}
6 \times 8
&= (2 \times 3) \times (2 \times 4) \\[4pt]
&=
2 \times (3 \times 2 \times 4)
\end{align*}
%%
which is an even integer.  If you multiply $2$ by a million, you don't
have to work out the result to know that the result will be an even
number.


%%%%%%%%%%%%%%%%%%%%%%%%%%%%%%%%%%%%%%%%%%%%%%%%%%%%%%%%%%%%%%%%%%%%%%%%%%%

\section{Distributive laws}

The following rules can also help you to simplify a complicated
mathematical expression.

\begin{definition}
\textbf{Distributive laws.}
If $\quadruple{a}{b}{c}{d}$ are any numbers, the
\emph{distributive laws} state that you can write:
%%
\begin{packedenumeral}
\item $a (b + c) = ab + ac$

\item $(a + b) \times (c + d) = a(c + d) + b(c + d)$
\end{packedenumeral}
\end{definition}

The distributive laws say that if you have any three numbers
$\triple{a}{b}{c}$ then you can simplify the expression
$a(b + c)$ in two ways.  First, you can simplify the sum $b + c$
inside the parentheses and then multiply the result by $a$.  Or you
can simplify the two products $ab$ and $ac$ and then add the two
results together.  Suppose you want to simplify the expression
$3 (2 + 7)$.  You can first simplify the addition and then multiply
the result by $3$:
%%
\begin{align*}
3 (2 + 7)
&=
3 \times 9 \\[4pt]
&=
27.
\end{align*}
%%
On the other hand, the distributive laws also allow you to write the
expression $3 (2 + 7)$ as $3 (2 + 7) = (3 \times 2) + (3 \times 7)$.
Now you simplify the multiplications and then add the results
together:
%%
\begin{align*}
3 (2 + 7)
&=
(3 \times 2) + (3 \times 7) \\[4pt]
&=
6 + 21 \\[4pt]
&=
27.
\end{align*}

\begin{exercise}
How would you simplify the expression $5 (3 + 1)$ in two ways?
\end{exercise}

\ifbool{showSolution}{
\begin{solution}
First, you can simplify the expression $5 (3 + 1)$ by simplifying the
sum $3 + 1$ and then multiply the result by $5$.  This will result in
%%
\begin{align*}
5 (3 + 1)
&=
5 \times 4 \\[4pt]
&=
20.
\end{align*}
%%
Second, you can use the distributive laws to write
%%
\begin{align*}
5 (3 + 1)
&=
(5 \times 3) + (5 \times 1) \\[4pt]
&=
15 + 5 \\[4pt]
&=
20.
\end{align*}
\end{solution}
}{}

The distributive laws also say that if you have any four numbers
$\quadruple{a}{b}{c}{d}$ then you can simplify the expression
$(a + b) \times (c + d)$ in two ways.  First, you can simplify each of
$a + b$ and $c + d$ and then multiply the two results together.
Second, you can simplify each of $a(c + d)$ and $b(c + d)$ and then
add the two results.  For example, you can simplify the expression
$(2 + 3) \times (5 + 7)$ in two ways as follows.  The first way is:
%%
\begin{align*}
(2 + 3) \times (5 + 7)
&=
5 \times 12 \\[4pt]
&=
60.
\end{align*}
%%
The second way is:
%%
\begin{align*}
(2 + 3) \times (5 + 7)
&=
2(5 + 7) + 3(5 + 7) \\[4pt]
&=
(2 \times 12) + (3 \times 12) \\[4pt]
&=
24 + 36 \\[4pt]
&=
60.
\end{align*}

\begin{exercise}
How would you simplify the expression $(1 + 2) \times (3 + 4)$ in two
ways?
\end{exercise}

\ifbool{showSolution}{
\begin{solution}
First, you can simplify the sums inside the parentheses and then
multiply the results together.  This will result in
%%
\begin{align*}
(1 + 2) \times (3 + 4)
&=
3 \times 7 \\[4pt]
&=
21.
\end{align*}
%%
On the other hand, you can use the distributive laws to simplify the
expression as follows:
%%
\begin{align*}
(1 + 2) \times (3 + 4)
&=
1(3 + 4) + 2(3 + 4) \\[4pt]
&=
(1 \times 7) + (2 \times 7) \\[4pt]
&=
7 + 14 \\[4pt]
&=
21.
\end{align*}
\end{solution}
}{}

\begin{theorem}
If $n$ and $r$ are odd integers, then their product $nr = n \times r$
is also an odd integer.
\end{theorem}

\begin{proof}
You know that each of $n$ and $r$ is odd so you can write
$n = 2k + 1$ and $r = 2s + 1$, where $k$ and $s$ are some other
integers.  Now you can use the distributive laws to write $nr$ as
%%
\begin{align*}
nr
&=
(2k + 1) \times (2s + 1) \\[4pt]
&=
2k(2s + 1) + 1(2s + 1) \\[4pt]
&=
2k(2s + 1) + 2s + 1 \\[4pt]
&=
4ks + 2k + 2s + 1 \\[4pt]
&=
2(2ks + k + s) + 1.
\end{align*}
%%
If you make the substitution $t = 2ks + k + s$, then $t$ is an integer
and so you can write $nr = 2t + 1$.  This shows that $nr$ is an odd
integer.
\end{proof}


%%%%%%%%%%%%%%%%%%%%%%%%%%%%%%%%%%%%%%%%%%%%%%%%%%%%%%%%%%%%%%%%%%%%%%%%%%%

\section{Order of operations}

There are four main operations that are defined on numbers.  These
operations are summarised below:
%%
\begin{packedenumeral}
\item Addition is when you add two numbers together.  If $a$ and $b$
  are numbers, you write $a + b$ to mean that you want to sum $a$ and
  $b$.

\item Subtraction is finding the difference between one number and
  another number.  You write $a - b$ to mean that you subtract $b$
  from $a$.

\item Multiplication is when you multiply two numbers together.  The
  product of $a$ and $b$ can be written as $a \times b$ or $ab$.

\item Division is finding the ratio of two numbers.  The ratio of $a$
  over $b$ is written as the fraction $\frac{a}{b}$ or $a/b$, but $b$
  must not be zero.
\end{packedenumeral}

When you want to simplify a complicated expression, you should follow
a number of rules to help you.  These rules are called the
\emph{order of operations} and are listed below in the order in which
they should be followed.
%%
\begin{packedenumeral}
\item If numbers are grouped together within parentheses, you start
  from the innermost parentheses and work outward.

\item If there are multiplications and/or divisions, work from left to
  right to simplify the multiplications and/or division.

\item If there are additions and/or subtractions, work from left to
  right to simplify the additions and/or subtractions.
\end{packedenumeral}

\begin{exercise}
Let $p$ denote Patrick's age, $m$ be Marcus' age, and $u$ your uncle's
age.  The sum of all these ages is $p + m + u$.  Plug in the values
for each of $\triple{p}{m}{u}$.  How would you simplify the resulting
expression?
\end{exercise}

\ifbool{showSolution}{
\begin{solution}
You can use the associative laws to write $p + m + u = (p + m) + u$.
This indicates that you want to first simplify the sum $p + m$.  Use
your result to then simplify $(p + m) + u$.  On the other hand, the
associative laws also allow you to write $p + m + u = p + (m + u)$,
which shows that you want to first simplify the sum $m + u$.  You then
use this result to simplify the expression $p + (m + u)$.  Plug in the
values for each of $\triple{p}{m}{u}$, simplify the sum within the
parentheses, and finally simplify the resulting sum.
\end{solution}
}{}

\begin{exercise}
Simplify the expression $3 + 2 \times 4 - 8/4 + 9$.
\end{exercise}

\ifbool{showSolution}{
\begin{solution}
The expression $3 + 2 \times 4 - 8/4 + 9$ can be simplified as
%%
\begin{align*}
3 + 2 \times 4 - 8/4 + 9
&=
3 + 8 - 8/4 + 9 \\[4pt]
&=
3 + 8 - 2 + 9 \\[4pt]
&=
11 - 2 + 9 \\[4pt]
&=
9 + 9 \\[4pt]
&=
18.
\end{align*}
\end{solution}
}{}


%%%%%%%%%%%%%%%%%%%%%%%%%%%%%%%%%%%%%%%%%%%%%%%%%%%%%%%%%%%%%%%%%%%%%%%%%%%

\section*{Problems}

\begin{problem}
\item What sort of things can be represented as positive integers?
\ifbool{showSolution}{
  \begin{solution}
  Money, the number of hair on your head~(if you are not bald), the
  number of toes on your feet~(if you have toes), the number of houses
  in Melbourne.
  \end{solution}
}{}

\item What sort of things can be represented as negative integers?
\ifbool{showSolution}{
  \begin{solution}
  The temperature below zero can be represented as a negative
  integer.
  \end{solution}
}{}

\item Explain why the negative integer $-42$ is even.
\ifbool{showSolution}{
  \begin{solution}
  The integer $-42$ is even because the number can be written as
  $-42 = 2 \times (-21)$.
  \end{solution}
}{}

\item Write out the set of the first five prime numbers.
\ifbool{showSolution}{
  \begin{solution}
  The set of the first five prime numbers is
  $\set{\quintuple{2}{3}{5}{7}{11}}$.
  \end{solution}
}{}

\item How would you simplify the expression $2(4 + 1) + 3(5 + 2)$?
  Explain a different way to simplify the expression.
\ifbool{showSolution}{
  \begin{solution}
  One way is to simplify the sums inside the parentheses and then you
  simplify the resulting expression:
  %%
  \begin{align*}
  2(4 + 1) + 3(5 + 2)
  &=
  (2 \times 5) + (3 \times 7) \\[4pt]
  &=
  10 + 21 \\[4pt]
  &=
  31.
  \end{align*}
  %%
  On the other hand, you can use the distributive laws:
  %%
  \begin{align*}
  2(4 + 1) + 3(5 + 2)
  &=
  (2 \times 4 + 2 \times 1) + (3 \times 5 + 3 \times 2) \\[4pt]
  &=
  (8 + 2) + (15 + 6) \\[4pt]
  &=
  10 + 21 \\[4pt]
  &=
  31.
  \end{align*}
  \end{solution}
}{}

\item Explain what is wrong with the following simplification:
  %%
  \begin{align*}
  2 \times (1 + 3)
  &=
  2 \times 1 + 3 \\[4pt]
  &=
  2 + 3 \\[4pt]
  &=
  5.
  \end{align*}
\ifbool{showSolution}{
  \begin{solution}
  The addition inside the parentheses should be simplified first.  The
  simplification should be:
  %%
  \begin{align*}
  2 \times (1 + 3)
  &=
  2 \times 4 \\[4pt]
  &=
  8.
  \end{align*}
  \end{solution}
}{}

\item Explain why the square root $\sqrt{4 \times 16}$ is an integer.
\ifbool{showSolution}{
  \begin{solution}
  The square root $\sqrt{4 \times 16}$ can be simplified as
  %%
  \begin{align*}
  \sqrt{4 \times 16}
  &=
  \sqrt{4} \times \sqrt{16} \\[4pt]
  &=
  2 \times 4 \\[4pt]
  &=
  8.
  \end{align*}
  %%
  Therefore $\sqrt{4 \times 16}$ is an integer.
  \end{solution}
}{}

\item Let $x$ be any even integer and $y$ any odd integer.  Explain
  why the product $xy = x \times y$ is an even integer.
\ifbool{showSolution}{
\begin{solution}
You are given that $x$ is an even integer so you can write $x$ as
$x = 2k$, where $k$ is another integer.  You also know that $y$ is an
odd integer so you can write it as $y = 2r + 1$, where $r$ is another
integer.  You can now use the distributive and associative laws to
write the product $xy$ as
%%
\begin{align*}
xy
&=
(2k) \times (2r + 1) \\[4pt]
&=
(2k \times 2r) + (2k \times 1) \\[4pt]
&=
2 (2kr) + 2k \\[4pt]
&=
2 (2kr + k).
\end{align*}
%%
The number $2 (2kr + k)$ is even because the product of $2$ by any
integer is even.  Therefore $xy$ is even.
\end{solution}
}{}

\item Let $n$ be any even integer. Explain why the square number
  $n^2 = n \times n$ is also an even integer.
\ifbool{showSolution}{
  \begin{solution}
  Since $n$ is an even integer, then $n^2 = n \times n$ is the product
  of two even integers.  You can use
  \Theorem{thm:product_of_two_evens_is_even} to conclude that the
  square number $n^2$ is also even.
  \end{solution}
}{}

\item This project requires you to measure the amount of time that you
  spend showering.  For example, suppose you will shower this evening.
  Before you turn on the shower water, write down the current date and
  time~(e.g.~\today, start at 5:30~pm).  After you have showered and
  turned off the shower water, write down the time that you turned off
  the water~(e.g.~end at 5:37~pm).  The difference between these two
  times is the amount of time that you spent showering during that
  evening.  In the above example times, the amount of time spent
  showering would be seven minutes.  Record the amount of time you
  spent showering for each day that you shower.  It's like keeping a
  shower diary.  Bring along your shower diary next week.
\end{problem}

\end{document}
