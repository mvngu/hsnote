%%%%%%%%%%%%%%%%%%%%%%%%%%%%%%%%%%%%%%%%%%%%%%%%%%%%%%%%%%%%%%%%%%%%%%%%%%%

\documentclass[a4paper,oneside,12pt]{article}
\usepackage{mystyle}

\begin{document}

\title{\Large\bf Logarithmic functions}
\author{%%
  Minh Van Nguyen \\
  \url{mvngu@gmx.com}
}
\date{\today}
\maketitle

{\color{red}
\begin{packeditem}
\item Log plot to recognize exponential growth/decay.

\item Newton's law of cooling.

\item Half-life of exponential decay.

\item Use linear regression to estimate parameters of exponential
  model.
\end{packeditem}
}


\newpage
%%%%%%%%%%%%%%%%%%%%%%%%%%%%%%%%%%%%%%%%%%%%%%%%%%%%%%%%%%%%%%%%%%%%%%%%%%%

\section*{Problem}

\begin{problem}
\item Read the following article by Kathleen M. Clark and Clemency
  Montelle:
  \emph{Logarithms: The Early History of a Familiar Function}.\footnote{
    Available at
    \url{http://web.archive.org/web/20180509072424/https://www.maa.org/press/periodicals/convergence/logarithms-the-early-history-of-a-familiar-function},
    accessed 2018-05-09.
  }

\item In \Exercise{ex:Newtons_law_of_cooling}, you saw an approximate
  version of Newton's law of cooling at work.  In this problem, you
  will explore a more precise version of Newton's law of
  cooling.\footnote{
    The following paper contains a history of Newton's law of cooling:
    \url{https://doi.org/10.1007/s11191-010-9324-1}.
  }
  In words, Newton's law of cooling says that when a hot object is
  placed within a cooler environment, the hot object will over time
  cool down to the temperature of its surrounding.  Let $T_o$ be the
  initial temperature of an object and let $T_s$ be the temperature of
  the object's surrounding; temperatures are in units of degrees
  Celsius.
\ifbool{showSolution}{
\begin{solution}

\end{solution}
}{}

\item Exponential distribution and cricket.
\ifbool{showSolution}{
\begin{solution}

\end{solution}
}{}

\item Pressure and density above sea level.
\ifbool{showSolution}{
\begin{solution}

\end{solution}
}{}
\end{problem}

\end{document}
