%%%%%%%%%%%%%%%%%%%%%%%%%%%%%%%%%%%%%%%%%%%%%%%%%%%%%%%%%%%%%%%%%%%%%%%%%%%

\documentclass[a4paper,oneside,12pt]{article}
\usepackage{mystyle}

\begin{document}

\title{\Large\bf Logarithmic functions}
\author{%%
  Minh Van Nguyen \\
  \url{mvngu@gmx.com}
}
\date{\today}
\maketitle

{\color{red}
\begin{packeditem}
\item Log plot to recognize exponential growth/decay.

\item Newton's law of cooling.

\item Half-life of exponential decay.

\item Use linear regression to estimate parameters of exponential
  model.
\end{packeditem}
}


%%%%%%%%%%%%%%%%%%%%%%%%%%%%%%%%%%%%%%%%%%%%%%%%%%%%%%%%%%%%%%%%%%%%%%%%%%%

\section{What is logarithm?}

Exponentiation and logarithm are inverse functions of each other.
This is similar to the way addition and subtraction are inverse
operations of each other, or multiplication and division are inverse
operations of each other.  Consequently, you can use this inverse
property between exponentiation and logarithm to solve equations that
involve exponential functions.  What does all this mean?

You already know that the function $f(x) = 10^x$ is an exponential
function.  Furthermore, the function $f(x)$ represents an exponential
growth.  Suppose you were to solve the equation
%%
\begin{equation}
\label{eqn:exponential_growth_100_10_x}
100
=
10^x.
\end{equation}
%%
Take some time to think about what
\Equation{eqn:exponential_growth_100_10_x} is telling you.  The
equation tells you that you want a value of $x$ such that when $10$ is
raised to the power of $x$, you get $100$ as a result.  What would the
value of $x$ be?  Here, the number $10$ is the \emph{base} and the
unknown $x$ is the \emph{exponent}.  To solve
\Equation{eqn:exponential_growth_100_10_x} for $x$, take the logarithm
to the base $10$ of both sides to get
%%
\begin{equation}
\label{eqn:log10_100}
\log_{10} 100
=
\log_{10} (10^x).
\end{equation}
%%
The right-hand side of \Equation{eqn:log10_100} simplifies to
$\log_{10} (10^x) = x$ because the logarithm is to the base $10$ and
the exponential function $10^x$ has base $10$.  In other words,
\Equation{eqn:log10_100} simplifies to
\[
\log_{10} 100
=
x.
\]
If $x = 2$, then $10^2 = 100$.  You say that $2$ is equivalent to the
logarithm of $100$ to the base $10$.  In symbol, this is written as
\[
\log_{10}100
=
2.
\]
Thus the solution to \Equation{eqn:exponential_growth_100_10_x} is
$x = 2$.


\newpage
%%%%%%%%%%%%%%%%%%%%%%%%%%%%%%%%%%%%%%%%%%%%%%%%%%%%%%%%%%%%%%%%%%%%%%%%%%%

\section*{Problem}

\begin{problem}
\item Read the following article by Kathleen M. Clark and Clemency
  Montelle:
  \emph{Logarithms: The Early History of a Familiar Function}.\footnote{
    Available at
    \url{http://web.archive.org/web/20180509072424/https://www.maa.org/press/periodicals/convergence/logarithms-the-early-history-of-a-familiar-function},
    accessed 2018-05-09.
  }

\item In \Exercise{ex:Newtons_law_of_cooling}, you saw an approximate
  version of Newton's law of cooling at work.  In this problem, you
  will explore a more precise version of Newton's law of
  cooling.\footnote{
    The following paper contains a history of Newton's law of cooling:
    \url{https://doi.org/10.1007/s11191-010-9324-1}.
  }
  In words, Newton's law of cooling says that when a hot object is
  placed within a cooler environment, the hot object will over time
  cool down to the temperature of its surrounding.  Let $T_o$ be the
  initial temperature of an object and let $T_s$ be the temperature of
  the object's surrounding; temperatures are in units of degrees
  Celsius.
\ifbool{showSolution}{
\begin{solution}

\end{solution}
}{}

\item Exponential distribution and cricket.
\ifbool{showSolution}{
\begin{solution}

\end{solution}
}{}

\item Pressure and density above sea level.
\ifbool{showSolution}{
\begin{solution}

\end{solution}
}{}
\end{problem}

\end{document}
