%%%%%%%%%%%%%%%%%%%%%%%%%%%%%%%%%%%%%%%%%%%%%%%%%%%%%%%%%%%%%%%%%%%%%%%%%%%

\documentclass[a4paper,oneside,12pt]{article}
\usepackage{mystyle}

\begin{document}

\title{\Large\bf Approximation}
\author{%%
  Minh Van Nguyen \\
  \url{mvngu@gmx.com}
}
\date{\today}
\maketitle

%%%%%%%%%%%%%%%%%%%%%%%%%%%%%%%%%%%%%%%%%%%%%%%%%%%%%%%%%%%%%%%%%%%%%%%%%%%

\section{The number $\pi$}

The number $\pi = 3.141592\dots$ is another example of an irrational
number.  The number $\pi$ is often used to measure the area and
circumference of a circle; see \Figure{fig:general_circle}.  Since
$\pi$ is irrational, the number cannot be written as a ratio of
integers.  So for practical purposes, you must approximate $\pi$ as
closely as you can.  The Greek mathematician Archimedes used the
fraction $22 / 7$ to approximate $\pi$.  The fraction $22 / 7$ can be
written as $22 / 7 = 3.142857$, correct to six decimal digits.  This
is not a good approximation to $\pi$ because the value of $22 / 7$
differs from $\pi$ from the third decimal digit onwards.  If a number
$p \in \RR$ is to be a good approximation of $\pi$, then it should be
possible to make $p$ as close to $\pi$ as you want.

\begin{figure}[!htbp]
\centering
\includegraphics[scale=1]{image/03/circle.pdf}
\caption{%%
  A circle with radius $r$.  Since the radius cannot be negative or
  zero, you must have $r > 0$.  The \emph{radius} of a circle is
  defined as the distance from the centre of the circle to any point
  on the circle.  The \emph{diameter}, denoted $d$, is then defined as
  twice the radius, i.e.~$d = 2r$.  The distance around the circle is
  called its \emph{circumference}.
}
\label{fig:general_circle}
\end{figure}

In order to approximate $\pi$ as closely as possible, you must know
how $\pi$ is defined.  Let $c$ be the circumference of a circle and
let $d$ be the diameter of the same circle.  Then the value of $\pi$
is defined as the ratio
%%
\begin{equation}
\label{eqn:define_pi_as_ratio_of_c_over_d}
\pi
=
\frac{c}{d}.
\end{equation}
%%
But now you have two problems:
%%
\begin{packedenumeral}
\item How is the value of the diameter $d$ determined?

\item How do you calculate the value of the circumference $c$?
\end{packedenumeral}
%%
The short answer is: You approximate the values of $d$ and $c$ and
then substitute those values into
\Equation{eqn:define_pi_as_ratio_of_c_over_d} to obtain an
approximation of $\pi$.  What follows is the long answer.


%%%%%%%%%%%%%%%%%%%%%%%%%%%%%%%%%%%%%%%%%%%%%%%%%%%%%%%%%%%%%%%%%%%%%%%%%%%

\section{Coordinates}

Let's start by discussing how a pair of numbers can be represented as
a picture.  You know that any real number can be represented as a
point on the number line.  \Figure{fig:real_number_line} shows the
irrational numbers $\sqrt{2}$, $e$, and $\pi$ as points on the number
line.  What if you have a pair $\tuple{a}{b}$ of real numbers?  How
would $\tuple{a}{b}$ be represented as a point on the number line?

\begin{figure}[!htbp]
\centering
\includegraphics[scale=1.1]{image/03/number-line.pdf}
\caption{%%
  Elements from the set $\RR$ of real numbers can be represented as
  points on the number line.
}
\label{fig:real_number_line}
\end{figure}

The answer is that a pair $\tuple{a}{b}$ of real numbers cannot be
represented as a point on the number line.  You must somehow extend
the number line.  One way to extend the number line is to draw a
vertical line through the point $0$ such that the vertical line is
perpendicular to the number line.  What you then obtain is the
\emph{Cartesian coordinate system} as shown in
\Figure{fig:Cartesian_coordinate_system}.  In the Cartesian coordinate
system, a pair $\tuple{a}{b}$ of real numbers can be represented as a
point with the pair $\tuple{a}{b}$ being now called a pair of
\emph{coordinates}.  The number $a$ in the coordinates $\tuple{a}{b}$
is called the $x$-coordinate because starting from $0$ on the
$x$-axis you must move a horizontal distance of $a$ units to get to
the coordinates $\tuple{a}{0}$.  The number $b$ in the coordinates
$\tuple{a}{b}$ is called the $y$-coordinate because starting from $0$
on the $y$-axis you must move a vertical distance of $b$ units to get
to the coordinates $\tuple{0}{b}$.  Add the coordinates $\tuple{a}{0}$
and $\tuple{0}{b}$ together and you get the coordinates
$\tuple{a}{b}$, which can be drawn as a point in the Cartesian
coordinate system.

\begin{figure}[!htbp]
\centering
\includegraphics[scale=1.1]{image/03/cartesian-coordinate.pdf}
\caption{%%
  The Cartesian coordinate system consists of two perpendicular axes
  called the $x$-axis and the $y$-axis.  A pair $\tuple{a}{b}$ of real
  numbers can be represented as a point.  Starting from the origin
  $\tuple{0}{0}$, you move a distance of $a$ units along the
  horizontal axis and then move $b$ units along the vertical axis.
  Where you end up is the point $\tuple{a}{b}$.
}
\label{fig:Cartesian_coordinate_system}
\end{figure}

\begin{exercise}
Represent the pairs $\tuple{2}{4}$, $\tuple{0}{-3}$, and
$\tuple{-4}{0}$ as points in the Cartesian coordinate system.
\end{exercise}

\ifbool{showSolution}{
\begin{solution}
See \Figure{fig:some_points_in_Cartesian_coordinate_system}.
%%
\begin{figure}[!htbp]
\centering
\includegraphics[scale=1.1]{image/03/cartesian-coordinate_2-5_0-3_4-0.pdf}
\caption{%%
  The pairs $\tuple{2}{4}$, $\tuple{0}{-3}$, and $\tuple{-4}{0}$ as
  points in the Cartesian coordinate system.
}
\label{fig:some_points_in_Cartesian_coordinate_system}
\end{figure}
\end{solution}
}{}


%%%%%%%%%%%%%%%%%%%%%%%%%%%%%%%%%%%%%%%%%%%%%%%%%%%%%%%%%%%%%%%%%%%%%%%%%%%

\section{Distance}

The distance between two points $A$ and $B$ is the length of the
straight line from $A$ to $B$.  In the Cartesian coordinate system,
you use the same technique to measure the distance between two
points.  Suppose the points $A$ and $B$ have coordinates
$A = \tuple{x_1}{y_1}$ and $B = \tuple{x_2}{y_2}$; see
\Figure{fig:distance_between_two_points}.  If $A = B$, then $A$ and
$B$ are the same point so the distance from $A$ to itself is zero.  If
$A \neq B$, then there is a third point $C$ with coordinates
$C = \tuple{x_1}{y_2}$.  The line segments $AC$, $BC$, and $AB$ are
the three sides of a right-angled triangle.  The segment $AC$ has
length
\[
a
=
\absoluteValue{y_1 - y_2}
=
\absoluteValue{y_2 - y_1}
\]
and the segment $BC$ has length
\[
b
=
\absoluteValue{x_2 - x_1}
=
\absoluteValue{x_1 - x_2}
\]
but you do not yet know the length of the segment $AB$.  However,
since the points $\triple{A}{B}{C}$ are the three corners of a
right-angled triangle, you can use Pythagoras' theorem to determine
the length of the segment $AB$.

\begin{theorem}
\textbf{Pythagoras' theorem.}
Let $\triple{a}{b}{c}$ be the lengths of the three sides of a
right-angled triangle, where $c$ is the length of the hypotenuse.
Then $c$ can be written as $a^2 + b^2 = c^2$.
\end{theorem}

\begin{figure}[!htbp]
\centering
\includegraphics[scale=1.1]{image/03/distance-two-points.pdf}
\caption{%%
  Any two distinct points $A$ and $B$ are two corners of a
  right-angled triangle.  The line segment $AB$ is the hypotenuse,
  whose length can be measured by Pythagoras' theorem.
}
\label{fig:distance_between_two_points}
\end{figure}


%%%%%%%%%%%%%%%%%%%%%%%%%%%%%%%%%%%%%%%%%%%%%%%%%%%%%%%%%%%%%%%%%%%%%%%%%%%

\section{Approximating $\pi$ with a square}

Let's start by approximating the value of $\pi$ with a square.  The
square is drawn inside a unit circle such that the four corners of the
square touch the circle; see \Figure{fig:circle_inscribed_square}.

\begin{figure}[!htbp]
\centering
\includegraphics[scale=1]{image/03/circle-square.pdf}
\caption{%%
  A unit circle with an inscribed square.  A \emph{unit circle} is a
  circle whose radius is $1$.  The centre of the circle has
  coordinates $\tuple{0}{0}$.  The points $A$ and $B$ are where the
  square touches the circle.
}
\label{fig:circle_inscribed_square}
\end{figure}

\end{document}
