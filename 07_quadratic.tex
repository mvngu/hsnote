%%%%%%%%%%%%%%%%%%%%%%%%%%%%%%%%%%%%%%%%%%%%%%%%%%%%%%%%%%%%%%%%%%%%%%%%%%%

\documentclass[a4paper,oneside,12pt]{article}
\usepackage{mystyle}

\begin{document}

\title{\Large\bf Quadratic functions}
\author{%%
  Minh Van Nguyen \\
  \url{mvngu@gmx.com}
}
\date{\today}
\maketitle

\begin{packeditem}
\item Factoring quadratic function.

\item Completing the square.

\item Application: gravity and displacement of projectile.

\item Area under a quadratic function.
\end{packeditem}


%%%%%%%%%%%%%%%%%%%%%%%%%%%%%%%%%%%%%%%%%%%%%%%%%%%%%%%%%%%%%%%%%%%%%%%%%%%

\section{General form}
\label{sec:general_form}

A \emph{quadratic function} is an equation of the form
%%
\begin{equation}
\label{eqn:general_quadratic_function}
f(x)
=
ax^2 + bx + c
\end{equation}
%%
where $\triple{a}{b}{c} \in \RR$ are known constants such that
$a \neq 0$ and $x$ is a variable that can be any real number.  What
does the function $f(x)$ look like?

As an example, set $a = 1$ and $b = c = 0$ in
\Equation{eqn:general_quadratic_function} so that you have
$f(x) = x^2$.  Let's calculate the function value for
$x = \sextuple{-4}{-3}{-2}{2}{3}{4}$.  Note that you have
\[
f(2) = f(-2) = 4,
%%
\quad
%%
f(3) = f(-3) = 9,
%%
\quad
%%
f(4) = f(-4) = 16.
\]
Plot these points on one set of coordinate axes and draw a line
through the points to get the graph in \Figure{fig:quadratic_a_1}.
Note that $f(0) = 0$ so the graph of $f(x) = x^2$ touches the origin.
The general shape of the graph of
\Equation{eqn:general_quadratic_function} looks like the beak of a
duck~(or some other bird) and the usual name for this is a
\emph{parabola}.

\begin{figure}[!htbp]
\centering
\includegraphics[scale=1.2]{image/07/a-1.pdf}
\caption{%%
  A graph of the function $f(x) = x^2$.  The general shape of the
  graph is called a \emph{parabola}.
}
\label{fig:quadratic_a_1}
\end{figure}

Generally speaking, how would you draw the graph of
\Equation{eqn:general_quadratic_function}?  One way is to start at the
tip of the parabola and choose a few values of $x$ that are equally
spaced.  If the tip of the parabola is $\tuple{a}{b}$, the following
values of $x$
\[
\septuple{a-3}{a-2}{a-1}{a}{a+1}{a+2}{a+3}
\]
should usually be good enough for a rough sketch of the graph of
\Equation{eqn:general_quadratic_function}.  Of course, you still need
to calculate the function values $f(a-3)$, $f(a-2)$, $f(a-1)$,
$f(a+1)$, $f(a+2)$, and $f(a+3)$.  Note that the graph of the
quadratic function $f(x)$ is symmetric about the tip point
$\tuple{a}{b}$.  This means that
\[
f(a+1) = f(a-1),
%%
\quad
%%
f(a+2) = f(a-2),
%%
\quad
%%
f(a+3) = f(a-3)
\]
and so you only need to calculate three function values, not six.
The above strategy is illustrated
in \Figure{fig:sketch_parabola}.

\begin{figure}[!htbp]
\centering
\includegraphics[scale=1.2]{image/07/a1-bminus4-c10.pdf}
\caption{%%
  Sketching the graph of a quadratic function.  First, locate the tip
  $\tuple{a}{b}$ of the function.  From there, spread outward with a
  few points.  Finally, you connect the dots.
}
\label{fig:sketch_parabola}
\end{figure}

The problem now is: How do you calculate the tip of the parabola?  If
you have a quadratic function of the form $f(x) = ax^2 + bx + c$, the
tip of the function is located at the $x$-coordinate
%%
\begin{equation}
\label{eqn:parabola_tip_x_coordinate}
x
=
-\frac{b}{2a}.
\end{equation}
%%
Substitute \Expression{eqn:parabola_tip_x_coordinate} into the
function $f(x)$ to obtain the corresponding $y$-coordinate.  For now,
do not worry about how \Expression{eqn:parabola_tip_x_coordinate} was
derived.  This will be shown later on.

\begin{example}
\label{ex:quadratic_graph_a1_b1_c1}
Sketch the graph of the quadratic function $f(x) = x^2 + x + 1$.
\end{example}

\begin{solution}
The first thing you should do is determine the tip point.  You have
the values $a = 1$ and $b = 1$.  Use
\Expression{eqn:parabola_tip_x_coordinate} to see that the
$x$-coordinate of the tip point is
%%
\begin{align*}
x
&=
-\frac{1}{2 \times 1} \\[4pt]
&=
-\frac{1}{2}
\end{align*}
%%
and the $y$-coordinate of the tip point is
%%
\begin{align*}
y
&=
f(-1/2) \\[4pt]
&=
\parenthesis*{-\frac{1}{2}}^2 + \parenthesis*{-\frac{1}{2}} + 1 \\[4pt]
&=
\frac{1}{4} - \frac{1}{2} + 1 \\[4pt]
&=
\frac{1}{4} - \frac{2}{4} + \frac{4}{4} \\[4pt]
&=
\frac{1 - 2 + 4}{4} \\[4pt]
&=
\frac{3}{4}.
\end{align*}
%%
Thus the tip of the parabola is the point
$\tuple{-\frac{1}{2}}{\frac{3}{4}}$.  Next, choose the $x$-coordinates
%%
\begin{equation}
\label{eqn:a1_b1_c1_x_coordinates}
\frac{1}{2} = -\frac{1}{2} + 1,
%%
\qquad
%%
\frac{3}{2} = -\frac{1}{2} + 2,
%%
\qquad
%%
\frac{5}{2} = -\frac{1}{2} + 3
\end{equation}
%%
whose corresponding $y$-coordinates are
$\triple{\frac{7}{4}}{\frac{19}{4}}{\frac{39}{4}}$.  These three
$y$-coordinates also correspond to the $x$-coordinates
$\triple{-\frac{3}{2}}{-\frac{5}{2}}{-\frac{7}{2}}$ and so you have
\[
f(1/2) = f(-3/2),
%%
\quad
%%
f(3/2) = f(-5/2),
%%
\quad
%%
f(5/2) = f(-7/2).
\]
Plot the above seven points, connect the dots, and you get the graph
shown in \Figure{fig:quadratic_graph_a1_b1_c1}.
\end{solution}

\begin{figure}[!htbp]
\centering
\includegraphics[scale=1.2]{image/07/a1-b1-c1.pdf}
\caption{%%
  A graph of the quadratic function $f(x) = x^2 + x + 1$.  The tip of
  the function is at the point $\tuple{-\frac{1}{2}}{\frac{3}{4}}$.
}
\label{fig:quadratic_graph_a1_b1_c1}
\end{figure}

\begin{exercise}
For the $x$-coordinates~\eqref{eqn:a1_b1_c1_x_coordinates} in
\Example{ex:quadratic_graph_a1_b1_c1}, verify that the corresponding
$y$-coordinates are
$\triple{\frac{7}{4}}{\frac{19}{4}}{\frac{39}{4}}$.
\end{exercise}
%%
\ifbool{showSolution}{
\begin{solution}
The quadratic function is $f(x) = x^2 + x + 1$.  For $x = 1/2$ you have
%%
\begin{align*}
f(1/2)
&=
\parenthesis*{\frac{1}{2}}^2 + \frac{1}{2} + 1 \\[4pt]
&=
\frac{1}{4} + \frac{1}{2} + 1 \\[4pt]
&=
\frac{1}{4} + \frac{2}{4} + \frac{4}{4} \\[4pt]
&=
\frac{7}{4}.
\end{align*}
%%
For $x = 3/2$ you have
%%
\begin{align*}
f(3/2)
&=
\parenthesis*{\frac{3}{2}}^2 + \frac{3}{2} + 1 \\[4pt]
&=
\frac{9}{4} + \frac{3}{2} + 1 \\[4pt]
&=
\frac{9}{4} + \frac{6}{4} + \frac{4}{4} \\[4pt]
&=
\frac{19}{4}.
\end{align*}
%%
For $x = 5/2$ you have
%%
\begin{align*}
f(5/2)
&=
\parenthesis*{\frac{5}{2}}^2 + \frac{5}{2} + 1 \\[4pt]
&=
\frac{25}{4} + \frac{5}{2} + 1 \\[4pt]
&=
\frac{25}{4} + \frac{10}{4} + \frac{4}{4} \\[4pt]
&=
\frac{39}{4}.
\end{align*}
\end{solution}
}{}

\begin{exercise}
Sketch the graph of the function $f(x) = x^2 - 2x + 2$.
\end{exercise}
%%
\ifbool{showSolution}{
\begin{solution}
First, you determine the tip point.  You have $a = 1$ and $b = -2$.
Use \Expression{eqn:parabola_tip_x_coordinate} to see that the
$x$-coordinate of the tip point is
%%
\begin{align*}
x
&=
-\frac{-2}{2 \times 1} \\[4pt]
&=
\frac{2}{2} \\[4pt]
&=
1.
\end{align*}
%%
The $y$-coordinate of the tip point is
%%
\begin{align*}
f(1)
&=
1^2 - 2(1) + 2 \\[4pt]
&=
1 - 2 + 2 \\[4pt]
&=
1.
\end{align*}
%%
Then the tip of the parabola is the point $\tuple{1}{1}$.  Next,
choose the following values for $x$:
\[
2 = 1 + 1,
%%
\qquad
%%
3 = 1 + 2,
%%
\qquad
%%
4 = 1 + 3.
\]
The corresponding $y$ values are
\[
f(2) = f(0) = 2,
%%
\qquad
%%
f(3) = f(-1) = 5,
%%
\qquad
%%
f(4) = f(-2) = 10.
\]
Plot the above seven points to obtain the graph shown in
\Figure{fig:graph_a1_bminus2_c2}.

\begin{figure}[!htbp]
\centering
\includegraphics[scale=1.2]{image/07/a1-bminus2-c2.pdf}
\caption{%%
  A graph of the function $f(x) = x^2 - 2x + 2$.  The tip of the
  parabola is the point $\tuple{1}{1}$.
}
\label{fig:graph_a1_bminus2_c2}
\end{figure}
\end{solution}
}{}


%%%%%%%%%%%%%%%%%%%%%%%%%%%%%%%%%%%%%%%%%%%%%%%%%%%%%%%%%%%%%%%%%%%%%%%%%%%

\section{Quadratic formula}
\label{sec:quadratic_formula}

Given a quadratic function $f(x) = ax^2 + bx + c$, sometimes a problem
requires you to determine the values of $x$ such that $f(x) = 0$.  In
this case, the \emph{quadratic formula} is guaranteed to provide you
with at least one value of $x$ such that $f(x) = 0$.  But what is the
quadratic formula and how is it derived?

\begin{exercise}
\label{ex:completing_the_square}
Let $\pair{a}{b} \in \RR$ such that $a \neq 0$.  Show that
\[
\parenthesis*{x + \frac{b}{2a}}^2
=
x^2 + \frac{b}{a} x + \parenthesis*{\frac{b}{2a}}^2.
\]
\end{exercise}
%%
\ifbool{showSolution}{
\begin{solution}
Use the distributive laws to expand
$\parenthesis*{x + \frac{b}{2a}}^2$ and you get
%%
\begin{align*}
\parenthesis*{x + \frac{b}{2a}}^2
&=
\parenthesis*{x + \frac{b}{2a}} \parenthesis*{x + \frac{b}{2a}} \\[4pt]
&=
x \parenthesis*{x + \frac{b}{2a}}
+
\frac{b}{2a} \parenthesis*{x + \frac{b}{2a}} \\[4pt]
&=
x^2 + \frac{b}{2a} x + \frac{b}{2a} x + \parenthesis*{\frac{b}{2a}}^2 \\[4pt]
&=
x^2 + 2 \times \frac{b}{2a} x + \parenthesis*{\frac{b}{2a}}^2 \\[4pt]
&=
x^2 + \frac{b}{a} x + \parenthesis*{\frac{b}{2a}}^2
\end{align*}
%%
as required.
\end{solution}
}{}

\begin{exercise}
Let $a \in \RR$ and consider the numbers $\pm\sqrt{a}$.  Here the
symbol ``$\pm$'' means that you have both of $\sqrt{a}$ and
$-\sqrt{a}$.  If $x = \pm\sqrt{a}$, prove that $x^2 = a$.
\end{exercise}
%%
\ifbool{showSolution}{
\begin{solution}
You must prove two statements:
%%
\begin{packedenumeral}
\item\label{case:x_sqrt_a_implies_x_squared_a}
  If $x = \sqrt{a}$ then $x^2 = a$.

\item\label{case:x_minus_sqrt_a_implies_x_squared_a}
  If $x = -\sqrt{a}$ then $x^2 = a$.
\end{packedenumeral}
%%
First, let's prove \Statement{case:x_sqrt_a_implies_x_squared_a}.  If
$x = \sqrt{a}$, then you can square both sides of the equation to get
$x^2 = (\sqrt{a})^2$.  Since $\sqrt{a} = a^{1/2}$, then you can write
the expression $x^2 = (\sqrt{a})^2$ as
%%
\begin{equation}
\label{eqn:x_sqrt_a_implies_x_squared_a}
\begin{aligned}
x^2
&=
(\sqrt{a})^2 \\[4pt]
&=
(a^{1/2})^2 \\[4pt]
&=
a^{2/2} \\[4pt]
&=
a.
\end{aligned}
\end{equation}
%%
Finally, let's prove
\Statement{case:x_minus_sqrt_a_implies_x_squared_a}.  If
$x = -\sqrt{a}$, then squaring both sides of the equation results in
%%
\begin{align*}
x^2
&=
\bigparen{(-1)\sqrt{a}}^2 \\[4pt]
&=
(-1)^2 (\sqrt{a})^2 \\[4pt]
&=
(\sqrt{a})^2.
\end{align*}
%%
The latter equation can be simplied to $x^2 = a$ by using
\Expression{eqn:x_sqrt_a_implies_x_squared_a}.  Therefore if
$x = \pm\sqrt{a}$ then you have $x^2 = a$.
\end{solution}
}{}

\begin{exercise}
\label{ex:x_squared_a_implies_plus_minus_sqrt_a}
If $a$ is a real number such that $x^2 = a$, prove that
$x = \pm\sqrt{a}$.
\end{exercise}
%%
\ifbool{showSolution}{
\begin{solution}
You have two statements to prove:
%%
\begin{packedenumeral}
\item\label{case:x_squared_a_sqrt_a}
  If $a \in \RR$ such that $x^2 = a$, then $x = \sqrt{a}$.

\item\label{case:x_squared_a_minus_sqrt_a}
  If $a \in \RR$ such that $x^2 = a$, then $x = -\sqrt{a}$.
\end{packedenumeral}
%%
First, let's prove \Statement{case:x_squared_a_sqrt_a}.  You know that
$x^2 = a$ and the square root of any number $c \in \RR$ can be written
as $\sqrt{c} = c^{1/2}$.  Taking the square root of both sides of the
equation $x^2 = a$ and you get $(x^2)^{1/2} = a^{1/2}$, which can also
be written as $x^{2/2} = a^{1/2}$.  The latter equation simplifies to
$x = \sqrt{a}$ because $x^{2/2} = x^1 = x$.

Finally, let's prove \Statement{case:x_squared_a_minus_sqrt_a}.  You
know that $x^2 = a$ and since $a = (-1)^2 a$, you can also write
$x^2 = (-1)^2 a$.  Take the square root of both sides of the last
equation to get $(x^2)^{1/2} = \bigparen{(-1)^2 a}^{1/2}$, which can
be written as $x^{2/2} = (-1)^{2/2} a^{1/2}$.  Use the fact that
$(-1)^{2/2} = (-1)^1 = -1$ to write $x = -\sqrt{a}$.
\end{solution}
}{}

First, let's derive the quadratic formula.  When you write $f(x) = 0$,
it is equivalent to writing $ax^2 + bx + c = 0$.  When you require a
value of $x$ such that $f(x) = 0$, what you really want is to solve
the equation $ax^2 + bx + c = 0$ for $x$.  In the last equation, if
you divide each term by $a$ then you obtain the equivalent expression
\[
x^2 + \frac{b}{a} x + \frac{c}{a}
=
0.
\]
Now move the term $c/a$ to the right-hand side to obtain
%%
\begin{equation}
\label{eqn:quadratic_formula_factor_LHS}
x^2 + \frac{b}{a} x
=
-\frac{c}{a}.
\end{equation}
%%
The problem now is to factor the left-hand side of
\Equation{eqn:quadratic_formula_factor_LHS}.  From
\Exercise{ex:completing_the_square} you know that the square
$\parenthesis*{x + \frac{b}{2a}}^2$ can be expanded to become
\[
\parenthesis*{x + \frac{b}{2a}}^2
=
x^2 + \frac{b}{a} x + \parenthesis*{\frac{b}{2a}}^2
\]
where the expression $x^2 + \frac{b}{a} x$ is the same as the
left-hand side of \Equation{eqn:quadratic_formula_factor_LHS}.
In other words, if you add the term $\parenthesis*{\frac{b}{2a}}^2$ to
both sides of \Equation{eqn:quadratic_formula_factor_LHS}, then you
can use \Exercise{ex:completing_the_square} to factor the left-hand
side of \Equation{eqn:quadratic_formula_factor_LHS}.  Adding
$\parenthesis*{\frac{b}{2a}}^2$ to both sides of
\Equation{eqn:quadratic_formula_factor_LHS} results in
%%
\begin{align*}
x^2 + \frac{b}{a} x + \parenthesis*{\frac{b}{2a}}^2
&=
-\frac{c}{a} + \parenthesis*{\frac{b}{2a}}^2 \\[4pt]
&=
-\frac{c}{a} + \frac{b^2}{4a^2} \\[4pt]
&=
-\frac{c}{a} \times \frac{4a}{4a} + \frac{b^2}{4a^2} \\[4pt]
&=
-\frac{4ac}{4a^2} + \frac{b^2}{4a^2} \\[4pt]
&=
\frac{
  b^2 - 4ac
}{
  4a^2
}
\end{align*}
%%
where the left-hand side of the latter expression can be factored by
using \Exercise{ex:completing_the_square}.  In other words,
\Equation{eqn:quadratic_formula_factor_LHS} can also be written as
\[
\parenthesis*{x + \frac{b}{2a}}^2
=
\frac{
  b^2 - 4ac
}{
  4a^2
}.
\]
Use \Exercise{ex:x_squared_a_implies_plus_minus_sqrt_a} to write the
last equation as
%%
\begin{equation}
\label{eqn:quadratic_formula_completing_square_solve_for_x}
\begin{aligned}
x + \frac{b}{2a}
&=
\pm
\sqrt{
  \frac{
    b^2 - 4ac
  }{
    4a^2
  }
} \\[4pt]
&=
\pm
\frac{
  \sqrt{b^2 - 4ac}
}{
  2a
}
\end{aligned}
\end{equation}
%%
because the expression $\sqrt{4a^2}$ simplifies to
%%
\begin{align*}
\sqrt{4a^2}
&=
\sqrt{4} \times \sqrt{a^2} \\[4pt]
&=
2 (a^2)^{1/2} \\[4pt]
&=
2 a^{2/2} \\[4pt]
&=
2a.
\end{align*}
%%
Solve for $x$ in
\Expression{eqn:quadratic_formula_completing_square_solve_for_x} and
you obtain
%%
\begin{align*}
x
&=
-\frac{b}{2a}
\pm
\frac{
  \sqrt{b^2 - 4ac}
}{
  2a
} \\[4pt]
&=
\frac{
  -b \pm \sqrt{b^2 - 4ac}
}{
  2a
}
\end{align*}
%%
which is called the \emph{quadratic formula}.  The above can be
summarised as follows.

\begin{theorem}
\label{thm:quadratic_formula}
\textbf{Quadratic formula.}
Let $\triple{a}{b}{c} \in \RR$ such that $a \neq 0$ and consider the
quadratic function $f(x) = ax^2 + bx + c$.  Then the solutions of the
equation $f(x) = 0$ can be written as
%%
\begin{equation}
\label{eqn:quadratic_formula}
x
=
\frac{
  -b \pm \sqrt{b^2 - 4ac}
}{
  2a
}.
\end{equation}
\end{theorem}

What does the quadratic \Formula{eqn:quadratic_formula} mean?  How do
you make sense of the equation?  Let's assume you have a quadratic
function $f(x) = ax^2 + bx + c$ with $\triple{a}{b}{c}$ being real
numbers such that $a \neq 0$.
Expression~\eqref{eqn:quadratic_formula} states that there are values
of $x$ such that $f(x) = 0$ and those values of $x$ can be written as
\[
x
=
\frac{
  -b + \sqrt{b^2 - 4ac}
}{
  2a
}
%%
\qquad
\text{and}
\qquad
%%
x
=
\frac{
  -b - \sqrt{b^2 - 4ac}
}{
  2a
}.
\]
In the graph of $f(x)$, the $x$-intercept is obtained by setting
$f(x) = 0$ and solving the last equation for $x$.  The whole point of
\Theorem{thm:quadratic_formula} is to help you calculate the
$x$-intercepts of a quadratic function.
\Theorem{thm:quadratic_formula} also says that a quadratic function
has at most two $x$-intercepts.  The next example should help to
clarify the theory.

\begin{example}
Sketch the graph of the function $f(x) = x^2 - x - 2$.
\end{example}

\begin{solution}
You have the values $a = 1$, $b = -1$, and $c = -2$.  You can use
three points to draw the graph of $f(x)$.  The first point is the tip
of the parabola.  The other two points are the $x$-intercepts of
$f(x)$.

First, let's determine the tip point.  From
\Equation{eqn:parabola_tip_x_coordinate} you know that the tip of the
parabola occurs at the $x$-coordinate
%%
\begin{align*}
x
&=
-\frac{-1}{2 \times 1} \\[4pt]
&=
\frac{1}{2}.
\end{align*}
%%
Then the $y$-coordinate of the tip point is
%%
\begin{align*}
y
&=
f(1/2) \\[4pt]
&=
\parenthesis*{\frac{1}{2}}^2 - \frac{1}{2} - 2 \\[4pt]
&=
\frac{1}{4} - \frac{2}{4} - \frac{8}{4} \\[4pt]
&=
\frac{1 - 2 - 8}{4} \\[4pt]
&=
-\frac{9}{4}.
\end{align*}
%%
Thus the tip of the parabola occurs at the point
$\tuple{\frac{1}{2}}{-\frac{9}{4}}$.

Next, let's calculate the $x$-intercepts of $f(x)$.  From
\Equation{eqn:quadratic_formula} you know that one of the
$x$-intercepts occurs at
%%
\begin{align*}
x
&=
\frac{
  -(-1) + \sqrt{(-1)^2 - 4(1)(-2)}
}{
  2(1)
} \\[4pt]
&=
\frac{
  1 + \sqrt{1 + 8}
}{
  2
} \\[4pt]
&=
\frac{
  1 + \sqrt{9}
}{
  2
} \\[4pt]
&=
\frac{
  1 + 3
}{
  2
} \\[4pt]
&=
2.
\end{align*}
%%
The other $x$-intercept occurs at
%%
\begin{align*}
x
&=
\frac{
  -(-1) - \sqrt{(-1)^2 - 4(1)(-2)}
}{
  2(1)
} \\[4pt]
&=
\frac{
  1 - \sqrt{9}
}{
  2
} \\[4pt]
&=
\frac{
  1 - 3
}{
  2
} \\[4pt]
&=
-1.
\end{align*}
%%
In other words, you have two different $x$-intercepts that occur at
the points $\tuple{2}{0}$ and $\tuple{-1}{0}$.  Plot the tip point and
the two $x$-intercepts on one set of coordinate axes, draw a line
through the points, and you obtain the graph in
\Figure{fig:a1_bminus1_cminus2}.
\end{solution}

\begin{figure}[!htbp]
\centering
\includegraphics[scale=1]{image/07/a1-bminus1-cminus2.pdf}
\caption{%%
  Graph of the function $f(x) = x^2 - x - 2$ through three points.
}
\label{fig:a1_bminus1_cminus2}
\end{figure}

\begin{exercise}
Sketch the graph of $f(x) = -2x^2 + x + 3$.
\end{exercise}
%%
\ifbool{showSolution}{
\begin{solution}
You have the values $a = -2$, $b = 1$, and $c = 3$.  You can draw the
graph of $f(x) = -2x^2 + x + 3$ by using three points: the tip of the
parabola and the two $x$-intercepts.

First, let's calculate the tip point of the parabola.  Use
\Equation{eqn:parabola_tip_x_coordinate} to see that the tip point
occurs at the $x$-coordinate
%%
\begin{align*}
x
&=
-\frac{1}{2(-2)} \\[4pt]
&=
\frac{1}{4}.
\end{align*}
%%
The $y$-coordinate of the tip point is
%%
\begin{align*}
y
&=
f(1/4) \\[4pt]
&=
-2\parenthesis*{\frac{1}{4}}^2 + \frac{1}{4} + 3 \\[4pt]
&=
-2 \times \frac{1}{16} + \frac{4}{16} + 3 \\[4pt]
&=
\frac{4 - 2}{16} + 3 \\[4pt]
&=
\frac{1}{8} + 3 \\[4pt]
&=
\frac{1}{8} + \frac{24}{8} \\[4pt]
&=
\frac{25}{8}.
\end{align*}
%%
Thus the tip of the parabola is at the point
$\tuple{\frac{1}{4}}{\frac{25}{8}}$.

Next, you calculate the $x$-intercepts.  Using the quadratic
\Formula{eqn:quadratic_formula}, you see that an $x$-intercept has the
$x$-coordinate
%%
\begin{align*}
x
&=
\frac{
  -1 + \sqrt{1^2 - 4(-2)(3)}
}{
  2(-2)
} \\[4pt]
&=
\frac{
  -1 + \sqrt{1 + 24}
}{
  -4
} \\[4pt]
&=
\frac{
  -1 + 5
}{
  -4
} \\[4pt]
&=
-1.
\end{align*}
%%
The other $x$-intercept has the $x$-coordinate
%%
\begin{align*}
x
&=
\frac{
  -1 - \sqrt{1^2 - 4(-2)(3)}
}{
  2(-2)
} \\[4pt]
&=
\frac{
  -1 - 5
}{
  -4
} \\[4pt]
&=
\frac{-6}{-4} \\[4pt]
&=
\frac{3}{2}.
\end{align*}
%%
You now have the $x$-intercepts $\tuple{-1}{0}$ and
$\tuple{\frac{3}{2}}{0}$.

Finally, you plot the above three points on one set of coordinate
axes.  Draw a line through the points and you get the graph in
\Figure{fig:aminus2_b1_c3}.

\begin{figure}[!htbp]
\centering
\includegraphics[scale=1]{image/07/aminus2-b1-c3.pdf}
\caption{%%
  Graph of the function $f(x) = -2x^2 + x + 3$.
}
\label{fig:aminus2_b1_c3}
\end{figure}
\end{solution}
}{}

\begin{exercise}
Sketch the graph of $f(x) = -x^2 + 1$.
\end{exercise}

\ifbool{showSolution}{
\begin{solution}
You have the values $a = -1$ and $c = 1$.  The value of $b$ is
$b = 0$ because $f(x) = -x^2 + 1 = -x^2 + 0x + 1$.  Again, you can use
three points to draw the graph of $f(x)$.  Those three points are: the
tip of the parabola and the two $x$-intercepts.

Use \Equation{eqn:parabola_tip_x_coordinate} to see that the tip point
occurs at the $x$-coordinate
\[
x
=
-\frac{0}{2(-1)}
=
0
\]
whose corresponding $y$-coordinate is
\[
y
=
f(0)
=
-0^2 + 1
=
1.
\]
Then the tip of the parabola occurs at the point $\tuple{0}{1}$.

Let's calculate the $x$-intercepts.  Use
\Equation{eqn:quadratic_formula} to obtain the $x$-intercept
%%
\begin{align*}
x
&=
\frac{
  -0 + \sqrt{0^2 - 4(-1)(1)}
}{
  2(-1)
} \\[4pt]
&=
\frac{
  \sqrt{4}
}{
  -2
} \\[4pt]
&=
-1.
\end{align*}
%%
The other $x$-intercept occurs at
%%
\begin{align*}
x
&=
\frac{
  -0 - \sqrt{0^2 - 4(-1)(1)}
}{
  2(-1)
} \\[4pt]
&=
\frac{
  -\sqrt{4}
}{
  -2
} \\[4pt]
&=
1.
\end{align*}
%%
Thus the $x$-intercepts are $\tuple{-1}{0}$ and $\tuple{1}{0}$.  Plot
the three points and connect the dots to obtain the graph in
\Figure{fig:aminus1_c1}.
\end{solution}

\begin{figure}[!htbp]
\centering
\includegraphics[scale=1]{image/07/aminus1-c1.pdf}
\caption{%%
  Graph of the function $f(x) = -x^2 + 1$ through three points.
}
\label{fig:aminus1_c1}
\end{figure}
}{}


%%%%%%%%%%%%%%%%%%%%%%%%%%%%%%%%%%%%%%%%%%%%%%%%%%%%%%%%%%%%%%%%%%%%%%%%%%%

\section{The discriminant}

You have seen in \Section{sec:quadratic_formula} how the quadratic
formula can help you to sketch the graph of a quadratic function.  The
general strategy was to determine the tip point of the parabola and
then use the quadratic formula to calculate two $x$-intercepts of the
function.  You plot the three points on one set of coordinate axes and
connect the dots to obtain a graph of the function.  But this strategy
does not always work.

To understand why the above strategy can fail, let's consider the
example of the function $f(x) = x^2$.  According to the above
strategy, you first need to calculate the tip point.  This is easy
enough.  You have the values $a = 1$ and $b = c = 0$.  Just use
\Equation{eqn:parabola_tip_x_coordinate} to see that the
$x$-coordinate of the tip point is
%%
\begin{align*}
x
&=
-\frac{0}{2(1)} \\[4pt]
&=
0.
\end{align*}
%%
The corresponding $y$-coordinate is
%%
\begin{align*}
y
&=
f(0) \\[4pt]
&=
0^2 \\[4pt]
&=
0
\end{align*}
%%
and therefore the function $f(x) = x^2$ has its tip point at the
origin.  Now use the quadratic \Formula{eqn:quadratic_formula} to see
that the $x$-intercepts of $f(x)$ occur at the $x$-coordinate
%%
\begin{align*}
x
&=
\frac{
  -0 \pm \sqrt{0^2 - 4(1)(0)}
}{
  2(1)
} \\[4pt]
&=
\pm
\frac{
  \sqrt{0}
}{
  2
} \\[4pt]
&=
0
\end{align*}
%%
with the corresponding $y$-coordinate being $y = f(0) = 0$.  The
upshot is that the tip of $f(x)$ is also its $x$-intercept, i.e.~the
point $\tuple{0}{0}$.  The above strategy for graphing a quadratic
function results in only one point, not the three that you require.
In the case of the function $f(x) = x^2$, you should have used the
strategy from \Section{sec:general_form}.

Is there a way to determine when to use the strategies
from \Sections{sec:general_form}{sec:quadratic_formula} for graphing a
quadratic function?  The answer is yes, but you need to understand a
number called the \emph{discriminant} of a quadratic function.

\begin{definition}
\textbf{Discriminant.}
Let $\triple{a}{b}{c} \in \RR$ such that $a \neq 0$ and consider the
quadratic function $f(x) = ax^2 + bx + c$.  If the solutions of the
equation $f(x) = 0$ is given by
\[
x
=
\frac{
  -b \pm \sqrt{b^2 - 4ac}
}{
  2a
}
\]
then the number $\Delta = b^2 - 4ac$ is called the \emph{discriminant}
of the function $f(x)$.
\end{definition}

Like any real number, the discriminant can take on one of three types
of values.  The discriminant can be either negative, zero, or
positive.  Depending on the value of the discriminant, a quadratic
function will have either zero, one, or two $x$-intercepts.  Let's
consider each of the three cases separately.  Suppose $f(x)$ is a
quadratic function whose discriminant is $\Delta$.

\begin{packedenumeral}
\item If $\Delta < 0$, then $f(x)$ does not intersect the $x$-axis.
  This case is illustrated in \Figure{fig:negative_discriminant}.

\item If $\Delta = 0$, then $f(x)$ intersects the $x$-axis once.  The
  point of intersection is also the tip point of the quadratic
  function.  This case is illustrated in
  \Figure{fig:zero_discriminant}.

\item If $\Delta > 0$, then $f(x)$ has two different $x$-intercepts.
\end{packedenumeral}

\begin{figure}[!htbp]
\centering
\subfigure[]{
  \includegraphics[scale=0.85]{image/07/a2-b1-c1.pdf}
}
%%
\quad
%%
\subfigure[]{
  \includegraphics[scale=0.93]{image/07/aminus2-b1-cminus2.pdf}
}
\caption{%%
  When the discriminant is negative, the graph of a quadratic function
  does not intersect the $x$-axis.
}
\label{fig:negative_discriminant}
\end{figure}

\begin{figure}[!htbp]
\centering
\subfigure[]{
  \includegraphics[scale=1]{image/07/a1-bminus2-c1.pdf}
}
%%
\qquad
%%
\subfigure[]{
  \includegraphics[scale=1]{image/07/aminus9quarter-bminus3-cminus1.pdf}
}
\caption{%%
  When the discriminant is zero, the graph of a quadratic function
  intersects the $x$-axis at exactly one point.
}
\label{fig:zero_discriminant}
\end{figure}

\end{document}
