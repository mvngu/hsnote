%%%%%%%%%%%%%%%%%%%%%%%%%%%%%%%%%%%%%%%%%%%%%%%%%%%%%%%%%%%%%%%%%%%%%%%%%%%

\documentclass[a4paper,oneside,12pt]{article}
\usepackage{mystyle}

\begin{document}

\title{\Large\bf Real numbers}
\author{%%
  Minh Van Nguyen \\
  \url{mvngu@gmx.com}
}
\date{\today}
\maketitle

%%%%%%%%%%%%%%%%%%%%%%%%%%%%%%%%%%%%%%%%%%%%%%%%%%%%%%%%%%%%%%%%%%%%%%%%%%%

\section{Rational numbers}

A \emph{rational} number is a ratio of two integers.  A
rational number is also known as a fraction.  These are rational
numbers: $1/2$, $3/4$, $11/9$, and $-10/2$.  You write $5/2 \in \QQ$
to mean that $5/2$ belongs to the set of rational numbers.  The set of
all rational numbers is written as
\[
\QQ
=
\setdes{
  \frac{a}{b}
}{
  \pair{a}{b} \in \ZZ
  \text{ and }
  b \neq 0
}.
\]
This means that the set of rational numbers consists of all numbers of
the form $a/b$ such that $a$ and $b$ are integers, but $b$ cannot be
zero.  The top integer $a$ is called the \emph{numerator} and the
bottom integer $b$ is called the \emph{denominator}.  If $b = 1$, then
$a/b = a/1 = a$ and consequently any integer is also a rational
number.

\begin{exercise}
Provide an example of a rational number that is also an integer.
Explain why your number is an integer.
\end{exercise}

\ifbool{showSolution}{
\begin{solution}
The rational number $2/1$ is also an integer because it can be written
as $2/1 = 2$ and $2$ is an integer.
\end{solution}
}{}

\begin{exercise}
Give an example of a rational number that is not an integer.  Explain
why your number is not an integer.
\end{exercise}

\ifbool{showSolution}{
\begin{solution}
The ratio $1/2$ is not an integer because $1/2$ cannot be simplified
to any integer.
\end{solution}
}{}

\begin{exercise}
If $a/b$ is a rational number, why can't $b$ be zero?  Consider the
examples of $1/4$ and $1/0$.
\end{exercise}

\ifbool{showSolution}{
\begin{solution}
If in the rational number $a/b$ you have $b = 0$, then the ratio
$a / 0$ is not defined.  This means that as $b$ comes closer and
closer to zero, the value of the ratio $a/b$ does not come to a fixed
number.
\end{solution}
}{}

If $a/b$ is a rational number, there are many rational numbers that
have the same value as $a/b$.  For example, the ratio $1/2$ is the
same as the ratios $2/4$, $3/6$, and $4/8$.  The reason is that you
can write
%%
\begin{align*}
\frac{1}{2}
&=
\frac{1}{2} \times \frac{2}{2} \\[4pt]
&=
\frac{1}{2} \times \frac{3}{3} \\[4pt]
&=
\frac{1}{2} \times \frac{4}{4}.
\end{align*}
%%
In fact, if $c$ is any integer such that $c \neq 0$, then
\[
\frac{a}{b}
=
\frac{a}{b} \times \frac{c}{c}
=
\frac{ac}{bc}
\]
where the ratio $c/c$ can be simplified as $c/c = 1$.  The set $\ZZ$
of integers has an infinite number of elements because the sequence
$\seqi{1}{2}{3}$ of positive integers gets bigger and bigger and goes
on forever.  Furthermore, the sequence $\seqi{-1}{-2}{-3}$ of negative
integers gets smaller and smaller and also goes on forever.  In other
words, if $a/b$ is a rational number, then it is easy find an infinite
number of rational numbers that have the same value as $a/b$.  The
above is summarised as:

\begin{theorem}
\label{thm:infinitely_many_rationals_same_value}
If $a/b$ is a rational number such that $b \neq 0$, then there are
infinitely many rational numbers that have the same value as $a/b$.
\end{theorem}

It is because of \Theorem{thm:infinitely_many_rationals_same_value}
that you can simplify a fraction.  If a rational number $a/b$ can be
simplified, then there must be a rational number $c/d$ that has the
same value as $a/b$ but the numerator
$\absoluteValue{c} < \absoluteValue{a}$ and the denominator
$\absoluteValue{d} < \absoluteValue{b}$.  Given any number $x$, the
expression $\absoluteValue{x}$ is called the \emph{absolute value} of
$x$ and is defined as
\[
\absoluteValue{x}
=
\begin{cases}
x,  & \text{if $x \geq 0$}, \\[4pt]
-x, & \text{otherwise}.
\end{cases}
\]
For example, the absolute value of $0$ is $0$, the absolute value of
$3$ is $3$, and the absolute value of any positive number $y$ is $y$
itself.  You obtain the absolute value of a negative number by
multiplying the number by $-1$.  For example, the absolute value of
$-7$ is $\absoluteValue{-7} = -1 \times (-7) = 7$.

\begin{exercise}
Can the rational number $15 / 21$ be simplified any further?  If yes,
provide a simplified version of $15 / 21$.  If no, explain why not.
\end{exercise}

\ifbool{showSolution}{
\begin{solution}
The numerator $15$ and the denominator $21$ both have $3$ as a common
factor.  Then the rational number $15 / 21$ can be simplified as:
%%
\begin{align*}
\frac{15}{21}
&=
\frac{
  3 \times 5
}{
  3 \times 7
} \\[4pt]
&=
\frac{5}{7}.
\end{align*}
\end{solution}
}{}

\begin{exercise}
Simplify the fraction $-28 / 35$ as much as possible.
\end{exercise}

\ifbool{showSolution}{
\begin{solution}
The numerator and denominator both have $7$ as a common factor so the
fraction $-28 / 35$ can be simplified as:
%%
\begin{align*}
-\frac{28}{35}
&=
-\frac{
  7 \times 4
}{
  7 \times 5
} \\[4pt]
&=
-\frac{4}{5}.
\end{align*}
%%
Note that $\absoluteValue{-4} = 4$ and $\absoluteValue{-28} = 28$ and
so $\absoluteValue{-4} < \absoluteValue{-28}$ even though
$-28 < -4$.
\end{solution}
}{}

\begin{exercise}
Explain why the rational number $3/7$ cannot be simplified any
further.
\end{exercise}

\ifbool{showSolution}{
\begin{solution}
The only factor that $3$ and $7$ have in common is $1$.  Therefore the
rational number $3/7$ cannot be simplified any further.
\end{solution}
}{}


%%%%%%%%%%%%%%%%%%%%%%%%%%%%%%%%%%%%%%%%%%%%%%%%%%%%%%%%%%%%%%%%%%%%%%%%%%%

\section{Irrational numbers}

An \emph{irrational} number is any number that cannot be written as a
ratio of two integers.  The number $\sqrt{2}$ is an irrational number
so it cannot be an integer and cannot be written as a ratio of two
integers.  But how did you know that $\sqrt{2}$ is irrational?  How
would you prove that $\sqrt{2}$ is irrational?  The fact that
$\sqrt{2}$ is irrational can be demonstrated by using a technique
called \emph{proof by contradiction}.  The general idea of proof by
contradiction is to assume that a statement $S$ is true.  You then
work out a contradiction that results from assuming the truth of the
statement $S$.  Any statement is either true or false.  Since the
truth of the statement $S$ has led to a contradiction, it must be that
$S$ is false.

\begin{theorem}
The number $\sqrt{2}$ is irrational.
\end{theorem}

\begin{proof}
The number $\sqrt{2}$ is either rational or irrational.  To derive a
contradiction, you assume that $\sqrt{2}$ is a rational number.  Since
$\sqrt{2}$ is rational, it can be written as a ratio
%%
\begin{equation}
\label{eqn:root_2_as_ratio}
\sqrt{2}
=
\frac{a}{b}
\end{equation}
%%
where $a$ and $b$ are integers such that $b \neq 0$.  Square both
sides of \Equation{eqn:root_2_as_ratio} and you can write
\Equation{eqn:root_2_as_ratio} as
\[
(\sqrt{2})^2
=
\parenthesis*{
  \frac{a}{b}
}^2
\]
which simplifies to $2 = a^2 / b^2$.  Now solve for $a^2$ to see that
\Equation{eqn:root_2_as_ratio} can also be written as
%%
\begin{equation}
\label{eqn:root_2_a_square_is_even}
2b^2
=
a^2.
\end{equation}
%%
In other words, $a^2$ is even and $a$ must be even as well and so $a$
can be written as $a = 2k$, where $k$ is some other integer.
Substitute $a = 2k$ into \Equation{eqn:root_2_a_square_is_even} and
you get $2b^2 = (2k)^2$.  Simplify the equation $2b^2 = (2k)^2$ and
you see that \Equation{eqn:root_2_as_ratio} can also be written as
%%
\begin{equation}
\label{eqn:root_2_b_square_is_even}
b^2
=
2k^2.
\end{equation}
%%
Equation~\eqref{eqn:root_2_b_square_is_even} says that $b^2$ is even
and so $b$ is also even.  In other words, you can write $b = 2\ell$
for an integer $\ell$.  Substitute $b = 2\ell$ into
\Equation{eqn:root_2_b_square_is_even} and you have
$(2\ell)^2 = 2k^2$.  Simplify the last equation and you see that
\Equation{eqn:root_2_as_ratio} can also be written as
\[
2\ell^2
=
k^2.
\]
Solve the last equation for $2$ and you have $2 = (k / \ell)^2$.  Take
the square root of both sides of $2 = (k / \ell)^2$ and you get
$\sqrt{2} = k / \ell$.  Now use \Equation{eqn:root_2_as_ratio} to
write
\[
\sqrt{2}
=
\frac{a}{b}
=
\frac{k}{\ell}.
\]
The only way for the equation $a/b = k / \ell$ to be true is to have
$a = k$ and $b = \ell$.  Since you also have $a = 2k$, the only way
for the expression $a = k = 2k$ to be true is to have $a = 0$.
Substitute $a = 0$ into \Equation{eqn:root_2_as_ratio} and you have
$\sqrt{2} = 0 / b = 0$.  The contradiction is that $\sqrt{2} \neq 0$,
but you have shown that $\sqrt{2} = 0$.  The equations
$\sqrt{2} \neq 0$ and $\sqrt{2} = 0$ cannot both be true at the same
time.  So something must be wrong and that something is the assumption
that $\sqrt{2}$ is a rational number.  Therefore you conclude that
$\sqrt{2}$ must be an irrational number.
\end{proof}

\begin{exercise}
Provide another example of an irrational number.
\end{exercise}

\ifbool{showSolution}{
\begin{solution}
The number $\pi = 3.141592\dots$ is an irrational number.
\end{solution}
}{}

The number $\pi = 3.141592\dots$ is often used to measure the area and
circumference of a circular region.  Since $\pi$ is irrational, the
number cannot be written as a ratio of integers.  So for practical
purposes, you must approximate $\pi$ as closely as you can.  The Greek
mathematician Archimedes used the fraction $22 / 7$ to approximate
$\pi$.

The set of \emph{real} numbers is made up of all rational and all
irrational numbers.  The set of real numbers is written as $\RR$.  For
example, the ratio $3/7$ is a real number, the square root $\sqrt{2}$
is a real number, and the integer $42$ is a real number.

\begin{exercise}
Explain why any integer is a real number.
\end{exercise}

\ifbool{showSolution}{
\begin{solution}
Any integer is also a rational number.  Since a rational number is
also a real number, it follows that any integer is also a real
number.
\end{solution}
}{}

\begin{exercise}
How can a number be represented as a picture?
\end{exercise}

\ifbool{showSolution}{
\begin{solution}
A real number can be represented as a point on the number line.
\end{solution}
}{}

\begin{exercise}
Simplify the expression
$\displaystyle{
  \frac{
    3 \times 2 + (11 - 5)
  }{
    2 \times (3 + 7)
  }
}$.
\end{exercise}

\ifbool{showSolution}{
\begin{solution}
The expression
$\displaystyle{
  \frac{
    3 \times 2 + (11 - 5)
  }{
    2 \times (3 + 7)
  }
}$
can be simplified as
%%
\begin{align*}
\frac{
  3 \times 2 + (11 - 5)
}{
  2 \times (3 + 7)
}
&=
\frac{
  3 \times 2 + 6
}{
  2 \times (3 + 7)
} \\[4pt]
&=
\frac{
  3 \times 2 + 6
}{
  2 \times 10
} \\[4pt]
&=
\frac{
  6 + 6
}{
  20
} \\[4pt]
&=
\frac{
  12
}{
  20
} \\[4pt]
&=
\frac{
  3
}{
  5
}.
\end{align*}
\end{solution}
}{}


%%%%%%%%%%%%%%%%%%%%%%%%%%%%%%%%%%%%%%%%%%%%%%%%%%%%%%%%%%%%%%%%%%%%%%%%%%%

\section*{Problem}

\begin{problem}
\item Give an example of something that can be represented as a
  rational number.
\ifbool{showSolution}{
  \begin{solution}
  The number of oranges you have eaten divided by the number of
  oranges you have altogether.
  \end{solution}
}{}

\item Provide an example of something that can be represented as a
  real number.
\ifbool{showSolution}{
  \begin{solution}
  The area of your house.
  \end{solution}
}{}

\item Simplify the expression
  \[
  E
  =
  \frac{
    3 m c^2
  }{
    (6 - 4) + 1
  }.
  \]
\ifbool{showSolution}{
  \begin{solution}
  \begin{align*}
  E
  &=
  \frac{
    3 m c^2
  }{
    (6 - 4) + 1
  } \\[4pt]
  &=
  \frac{
    3 m c^2
  }{
    2 + 1
  } \\[4pt]
  &=
  \frac{
    3 m c^2
  }{
    3
  } \\[4pt]
  &=
  m c^2.
  \end{align*}
  \end{solution}
}{}

\item Let $a$ and $b$ be integers such that $b \neq 0$.  Suppose the
  rational number $a/b$ cannot be simplified any further.
  %%
  \begin{packedenum}
  \item\label{subprob:simplify_rational_number}
    Can the rational number $\displaystyle{\frac{6a}{12b}}$ be
    simplified any further?  If yes, provide a simplified version.  If
    no, explain why not.

  \item\label{subprob:cannot_simplify_rational_number}
    Suppose $b$ is an odd integer.  Explain why the rational number
    $\displaystyle{\frac{2}{3b}}$ cannot be simplified any further.
  \end{packedenum}
\ifbool{showSolution}{
  \begin{solution}
  \solutionpart{subprob:simplify_rational_number}
  The integers $6$ and $12$ both have $6$ as a common factor.  Then
  you can simplify $\displaystyle{\frac{6a}{12b}}$ as
  %%
  \begin{align*}
  \frac{6a}{12b}
  &=
  \frac{
    6a
  }{
    6 \times 2b
  } \\[4pt]
  &=
  \frac{a}{2b}.
  \end{align*}

  \solutionpart{subprob:cannot_simplify_rational_number}
  The integers $2$ and $3$ only have $1$ as a common factor so you
  cannot simplify the ratio $2/3$ any further.  The only way for the
  expression $\displaystyle{\frac{2}{3b}}$ to be simplified any
  further is that $3b$ be even.  Since you have assumed that $b$ is
  odd, then $3b$ is also odd because the product of two odd integers
  is also an odd integer.  In other words, $2$ and $3b$ do not have
  any factor in common other than $1$.  Therefore the rational number
  $\displaystyle{\frac{2}{3b}}$ cannot be simplified any further.
  \end{solution}
}{}

\item The number $e = 2.71828\dots$ is called \emph{Euler's constant}.
  Read about $e$ on Wikipedia or search on the Internet for
  ``Euler's constant.''  Is $e$ a rational number?  Is $e$ an
  irrational number?  Is $e$ an integer?  Does $e$ belong to the set
  $\RR$ of real numbers?  Where can you find a use for the number $e$?
\ifbool{showSolution}{
  \begin{solution}
  Euler's constant $e = 2.71828\dots$ is an irrational number, which
  means that $e$ also belongs to the set of real numbers.  The number
  $e$ is used as the base of the \emph{natural logarithm},
  i.e.~logarithm to the base $e$.  The number $e$ is also used in the
  calculation of compound interest.
  \end{solution}
}{}
\end{problem}

\end{document}
