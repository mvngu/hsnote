%%%%%%%%%%%%%%%%%%%%%%%%%%%%%%%%%%%%%%%%%%%%%%%%%%%%%%%%%%%%%%%%%%%%%%%%%%%

\documentclass[a4paper,oneside,12pt]{article}
\usepackage{mystyle}

\begin{document}

\title{\Large\bf Linear regression}
\author{%%
  Minh Van Nguyen \\
  \url{mvngu@gmx.com}
}
\date{\today}
\maketitle


%%%%%%%%%%%%%%%%%%%%%%%%%%%%%%%%%%%%%%%%%%%%%%%%%%%%%%%%%%%%%%%%%%%%%%%%%%%

\section{The mean}

The goal of this document is to use statistics for prediction.  You
will learn about a statistical model called linear regression.  Before
doing so, you need to know how to calculate the \emph{mean} of a bunch
of numbers.

Let's start with an example.  Consider the following numbers:
%%
\begin{equation}
\label{eqn:Hong_Kong_teenagers_heights}
\begin{matrix}
167, & 181, & 176, & 173, & 172, & 174, & 177, & 177, & 172, & 169.
\end{matrix}
\end{equation}
%%
These numbers are the heights~(in centimetres) of ten Hong Kong
teenagers.  To calculate the mean of the numbers in
\List{eqn:Hong_Kong_teenagers_heights}, first you add the numbers
together to get the total
\[
167 + 181 + 176 + 173 + 172 + 174 + 177 + 177 + 172 + 169
=
1738.
\]
Next, divide the total by how many numbers are in the list.  There are
ten numbers in \List{eqn:Hong_Kong_teenagers_heights} so you divide
the total by ten to get
\[
173.8
=
\frac{1738}{10}.
\]
This tells you that the ten heights in
\List{eqn:Hong_Kong_teenagers_heights} has a mean of $173.8$
centimetres.

The mean of a bunch of numbers is also called the \emph{average}.
However, the word average can have three different meanings in the
context of statistics.  When someone talks about the average of a
bunch of numbers, the person might be talking about one of three
things: the mean, median, or mode of the numbers.  So when you use the
word average, you must be careful about the sense in which you use the
word.  However, when you use the word mean, there is little confusion
about what you have in mind.

From the above example, the mean of a bunch of numbers can be defined
as follows:

\begin{definition}
\textbf{Mean.}
Let $\seq{a_1}{a_2}{a_n}$ be a sequence of $n$ numbers.  The
\emph{mean} of the sequence is defined as the number
\[
\frac{
  \sum_{i=1}^n a_i
}{
  n
}
=
\frac{
  \sumseq{a_1}{a_2}{a_n}
}{
  n
}.
\]
\end{definition}

Each of $\seq{a_1}{a_2}{a_n}$ is a number, where $a_1$ is the first
number in the sequence, $a_2$ is the second number, and so on, with
$a_n$ being the $n$-th number.  Going back to
\List{eqn:Hong_Kong_teenagers_heights}, you see that the first number
is $a_1 = 167$, the second number is $a_2 = 181$, the third is
$a_3 = 176$, and so on, with the tenth number being $a_{10} = 169$.
The list has ten numbers so you have $n = 10$.  The sigma notation
$\sum_{i=1}^n a_i$ means that you add together all numbers in the
sequence $\seq{a_1}{a_2}{a_n}$ from $i = 1$ to $i = n$.  This is the
same as writing
\[
\sum_{i=1}^n a_i
=
\sumseq{a_1}{a_2}{a_n}.
\]
If you have a sequence of one hundred numbers and you want to write
the sum of the numbers, it is easier to use the sigma
notation~(i.e.~the symbol $\sum$) to denote the sum than it is to
write out all the one hundred numbers.

\begin{example}
Use the sigma notation to write the sum of the first five Fibonacci
numbers.  Calculate the actual sum and the mean of the five numbers.
\end{example}

\begin{solution}
The Fibonacci numbers are numbers in the sequence
\[
\begin{matrix}
0, & 1, & 1, & 2, & 3, & 5, & 8, & 13, & 21, \dots
\end{matrix}
\]
The Fibonacci numbers go on forever.  The numbers form an infinite
sequence.  The first number in the sequence is $a_1 = 0$, the second
number is $a_2 = 1$, the third is $a_3 = 1$, the fourth is $a_4 = 2$,
and the fifth is $a_5 = 3$.  The sum of the first five numbers in the
Fibonacci sequence can be written as $\sum_{i=1}^5 a_i$.  The actual
sum of those five numbers is
%%
\begin{align*}
\sum_{i=1}^5 a_i
&=
\sumseq{a_1}{a_2}{a_5} \\[4pt]
&=
0 + 1 + 1 + 2 + 3 \\[4pt]
&=
7.
\end{align*}
%%
The mean of the five numbers is
\[
\frac{
  \sum_{i=1}^5 a_i
}{
  5
}
=
\frac{7}{5}
\]
which is $1.4 = 7 / 5$.
\end{solution}

\begin{exercise}
Consider the first $12$ positive integers.
%%
\begin{packedenum}
\item\label{subex:sigma_first_12_positive_integers}
  Use the sigma notation to write the sum of the first $12$ positive
  integers.

\item\label{subex:sum_first_12_positive_integers}
  Calculate the sum of the first $12$ positive integers.

\item\label{subex:mean_first_12_positive_integers}
  Calculate the mean of the first $12$ positive integers.
\end{packedenum}
\end{exercise}
%%
\ifbool{showSolution}{
\begin{solution}
\solutionpart{subex:sigma_first_12_positive_integers}
The first $12$ positive integers are $\seq{1}{2}{12}$.  The first
number is $a_1 = 1$, the second is $a_2 = 2$, and so on, with the
$12$-th number being $a_{12} = 12$.  The sum of the $12$ integers can
be written as $\sum_{i=1}^{12} a_i$.

\solutionpart{subex:sum_first_12_positive_integers}
The sum of the $12$ integers is
%%
\begin{align*}
\sum_{i=1}^{12} a_i
&=
\sumseq{1}{2}{12} \\[4pt]
&=
78.
\end{align*}

\solutionpart{subex:mean_first_12_positive_integers}
The mean of the $12$ integers is
%%
\begin{align*}
\frac{
  \sum_{i=1}^{12} a_i
}{
  12
}
&=
\frac{
  \sumseq{1}{2}{12}
}{
  12
} \\[4pt]
&=
\frac{78}{12} \\[4pt]
&=
\frac{13}{2}
\end{align*}
%%
which is $6.5 = 13 / 2$.
\end{solution}
}{}

\end{document}
