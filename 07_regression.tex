%%%%%%%%%%%%%%%%%%%%%%%%%%%%%%%%%%%%%%%%%%%%%%%%%%%%%%%%%%%%%%%%%%%%%%%%%%%

\documentclass[a4paper,oneside,12pt]{article}
\usepackage{mystyle}

\begin{document}

\title{\Large\bf Linear regression}
\author{%%
  Minh Van Nguyen \\
  \url{mvngu@gmx.com}
}
\date{\today}
\maketitle


%%%%%%%%%%%%%%%%%%%%%%%%%%%%%%%%%%%%%%%%%%%%%%%%%%%%%%%%%%%%%%%%%%%%%%%%%%%

\section{The mean}

The goal of this document is to use statistics for prediction.  You
will learn about a statistical model called linear regression.  Before
doing so, you need to know how to calculate the \emph{mean} of a bunch
of numbers.

Let's start with an example.  Consider the following numbers:
%%
\begin{equation}
\label{eqn:Hong_Kong_teenagers_heights}
\begin{matrix}
167 & 181 & 176 & 173 & 172 & 174 & 177 & 177 & 172 & 169
\end{matrix}
\end{equation}
%%
These numbers are the heights~(in centimetres) of ten Hong Kong
teenagers.  To calculate the mean of the numbers in
\List{eqn:Hong_Kong_teenagers_heights}, first you add the numbers
together to get the total
\[
167 + 181 + 176 + 173 + 172 + 174 + 177 + 177 + 172 + 169
=
1738.
\]
Next, divide the total by how many numbers are in the list.  There are
ten numbers in \List{eqn:Hong_Kong_teenagers_heights} so you divide
the total by ten to get
\[
173.8
=
\frac{1738}{10}.
\]
This tells you that the ten heights in
\List{eqn:Hong_Kong_teenagers_heights} has a mean of $173.8$
centimetres.

\end{document}
