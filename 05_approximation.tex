%%%%%%%%%%%%%%%%%%%%%%%%%%%%%%%%%%%%%%%%%%%%%%%%%%%%%%%%%%%%%%%%%%%%%%%%%%%

\documentclass[a4paper,oneside,12pt]{article}
\usepackage{mystyle}

\begin{document}

\title{\Large\bf Approximate $\pi$}
\author{%%
  Minh Van Nguyen \\
  \url{mvngu@gmx.com}
}
\date{\today}
\maketitle


%%%%%%%%%%%%%%%%%%%%%%%%%%%%%%%%%%%%%%%%%%%%%%%%%%%%%%%%%%%%%%%%%%%%%%%%%%%

\section{The number $\pi$}

The number $\pi = 3.141592\dots$ is another example of an irrational
number.  The number $\pi$ is often used to measure the area and
circumference of a circle; see \Figure{fig:general_circle}.  Since
$\pi$ is irrational, the number cannot be written as a ratio of
integers.  So for practical purposes, you must approximate $\pi$ as
closely as you can.  The Greek mathematician Archimedes used the
fraction $22 / 7$ to approximate $\pi$.  The fraction $22 / 7$ can be
written as $22 / 7 \approx 3.142857$, correct to six decimal digits.
(The symbol ``$\approx$'' means ``approximately''.)  This is not a
good approximation of $\pi$ because the value of $22 / 7$ differs from
$\pi$ from the third decimal digit onwards.  If a number $p \in \RR$
is to be a good approximation of $\pi$, then it should be possible to
make $p$ as close to $\pi$ as you want.

\begin{figure}[!htbp]
\centering
\includegraphics[scale=1]{image/04/circle.pdf}
\caption{%%
  A circle with radius $r$.  Since the radius cannot be negative, you
  must have $r \geq 0$.  The \emph{radius} of a circle is defined as
  the distance from the centre of the circle to any point on the
  circle.  The \emph{diameter}, denoted $d$, is then defined as twice
  the radius, i.e.~$d = 2r$.  The distance around the circle is called
  its \emph{circumference}.
}
\label{fig:general_circle}
\end{figure}

In order to approximate $\pi$ as closely as possible, you must know
how $\pi$ is defined.  Let $c$ be the circumference of a circle and
let $d$ be the diameter of the same circle.  Then the value of $\pi$
is defined as the ratio
%%
\begin{equation}
\label{eqn:define_pi_as_ratio_of_c_over_d}
\pi
=
\frac{c}{d}.
\end{equation}
%%
You already know that the diameter is equal to twice the radius.  The
only problem now is:
%%
\begin{packeditem}
\item How do you calculate the value of the circumference $c$?
\end{packeditem}
%%
The short answer is: You approximate the value of $c$ and then
substitute the values of $c$ and $d$ into
\Equation{eqn:define_pi_as_ratio_of_c_over_d} to obtain an
approximation of $\pi$.  What follows is the long answer.

\begin{exercise}
Describe the circle whose radius is zero.
\end{exercise}

\ifbool{showSolution}{
\begin{solution}
The circle whose radius is zero is an infinitely small point.
\end{solution}
}{}


%%%%%%%%%%%%%%%%%%%%%%%%%%%%%%%%%%%%%%%%%%%%%%%%%%%%%%%%%%%%%%%%%%%%%%%%%%%

\section{Approximating $\pi$ with a square}

Let's start by approximating the value of $\pi$ with a square.  The
square is drawn inside a unit circle such that the four corners of the
square touch the circle; see \Figure{fig:circle_inscribed_square}.
Since the unit circle has a radius of $1$, then the unit circle has a
diameter of $d = 2$.  What remains is to approximate the circumference
$c$ of the unit circle.

\begin{figure}[!htbp]
\centering
\includegraphics[scale=1]{image/04/circle-square.pdf}
\caption{%%
  A unit circle with an inscribed square.  A \emph{unit circle} is a
  circle whose radius is $1$.  The centre of the circle has
  coordinates $\tuple{0}{0}$.  The points $A$ and $B$ are where the
  square touches the circle.
}
\label{fig:circle_inscribed_square}
\end{figure}

\Figure{fig:circle_inscribed_square} suggests that the circumference
of the unit circle can approximated by the perimeter of the inscribed
square.  If a square has a side length of $\ell$, then the perimeter
of the square is the sum of all the four sides of the square.  In
other words, the square has a perimeter of
$4\ell = \ell + \ell + \ell + \ell$.  For the square in
\Figure{fig:circle_inscribed_square} the value of $\ell$ is the length
of the line segment $AB$, which can be calculated as
%%
\begin{align*}
\ell
&=
\sqrt{(0 - 1)^2 + (1 - 0)^2} \\[4pt]
&=
\sqrt{1 + 1} \\[4pt]
&=
\sqrt{2}.
\end{align*}
%%
So the square has a perimeter of $4\ell = 4\sqrt{2}$, which you can
take to be an approximation of the circumference of the unit circle.
Using \Equation{eqn:define_pi_as_ratio_of_c_over_d} you see that the
value of $\pi$ can be approximated as
%%
\begin{align*}
\pi
&\approx
\frac{4\sqrt{2}}{2} \\[4pt]
&\approx
2\sqrt{2} \\[4pt]
&\approx
2.828427
\end{align*}
%%
correct to six decimal places.  The approximation of $2\sqrt{2}$ is
worst than Archimedes' approximate value of $22/7$.  Is there a way to
get a better approximation of $\pi$?

\begin{exercise}
Determine the area of the unit circle.
\end{exercise}

\ifbool{showSolution}{
\begin{solution}
If $r$ is the radius of a circle, then the circle has an area of
$\pi r^2$.  The unit circle has a radius of $r = 1$ so the area of the
unit circle is $\pi \times 1^2 = \pi$.
\end{solution}
}{}

\begin{exercise}
Calculate the area of the square in
\Figure{fig:circle_inscribed_square}.
\end{exercise}

\ifbool{showSolution}{
\begin{solution}
If $w$ is the width of a square, then the square has an area of
$w^2$.  As the square in \Figure{fig:circle_inscribed_square} has a
width of $\sqrt{2}$, the area of the square is $(\sqrt{2})^2 = 2$.
\end{solution}
}{}


%%%%%%%%%%%%%%%%%%%%%%%%%%%%%%%%%%%%%%%%%%%%%%%%%%%%%%%%%%%%%%%%%%%%%%%%%%%

\section{Degrees and radians}

Before discussing how to get a better approximation of $\pi$, you need
to know about radians.  Angular degrees, or degrees, are used to
measure angles.  Another way to measure angles is to use
\emph{radians}, which are defined as follows.  A unit circle has a
radius of $r = 1$ so its circumference is $2 \pi r = 2 \pi$.  So the
value of $2 \pi$ is the distance around the unit circle.  The value of
$2\pi$ is also used as the angle of a circle and you say that a circle
has $2\pi$ radians.  Since any circle has $360$ degrees~(or
$\degree{360}$), then $\degree{360} = 2\pi$ radians.  The last
equation can be simplified to $\degree{180} = \pi$, which means that
$\degree{180}$ is equivalent to $\pi$ radians.

\begin{exercise}
Explain how you would define one radian.
\end{exercise}

\ifbool{showSolution}{
\begin{solution}
Since $\degree{180} = \pi$ radians, divide both sides by $\pi$ to get
$\degree{180} / \pi = 1$ radian.  Therefore one radian is
approximately $57.3$ degrees.
\end{solution}
}{}

\begin{exercise}
Write $\degree{90}$ in terms of radians.
\end{exercise}

\ifbool{showSolution}{
\begin{solution}
You know that $\degree{180} = \pi$ radians.  Divide both sides of the
equation by $2$ to see that $\degree{90} = \pi / 2$ radians.
\end{solution}
}{}

\Figure{fig:convert_from_polar_to_Cartesian_coordinates} illustrates
how a point $\tuple{a}{b}$ in the Cartesian coordinate system can be
represented in terms of radius and angle.  The coordinate system that
uses radius and angle is known as the \emph{polar coordinate system}.
Suppose you want to convert the Cartesian coordinates $\tuple{a}{b}$
to polar coordinates.  To do so, you assume that $\tuple{a}{b}$ is a
point on a circle that is centred at the origin.  Then the distance
from $\tuple{a}{b}$ to the origin is the radius of the circle so the
radius can be calculated as:
%%
\begin{align*}
r
&=
\sqrt{
  (b - 0)^2 + (a - 0)^2
} \\[4pt]
&=
\sqrt{
  a^2 + b^2
}.
\end{align*}
%%
To calculate the angle $\varphi$ in
\Figure{fig:convert_from_polar_to_Cartesian_coordinates}, you use a
rule called the \emph{law of sines}.  From
\Figure{fig:convert_from_polar_to_Cartesian_coordinates} you see that
the dashed vertical line has a length of $b$ and the angle opposite
the radius is $\degree{90}$ or $\pi / 2$ radians.  Use the law of
sines to write
%%
\begin{equation}
\label{eqn:law_of_sines}
\frac{b}{\sin \varphi}
=
\frac{r}{\sin \frac{\pi}{2}}.
\end{equation}
%%
Since $\sin \frac{\pi}{2} = 1$, \Equation{eqn:law_of_sines} can be
simplified to
\[
\frac{b}{\sin \varphi}
=
r.
\]
Solve the last equation for $\sin \varphi$ and you get
$\frac{b}{r} = \sin \varphi$.  Now solve for the angle $\varphi$ and
you get
\[
\varphi
=
\arcsin\parenthesis*{\frac{b}{r}}.
\]

\begin{figure}[!htbp]
\centering
\includegraphics[scale=1.1]{image/04/polar-cartesian.pdf}
\caption{%%
  Any point $\tuple{a}{b}$ in the Cartesian coordinate system can be
  represented as a point $\tuple{r}{\varphi}$ on a circle.  The circle
  is centred at the origin and has a radius of $r$.  The value of
  $\varphi$ radians measures the angle that spans from the $x$-axis to
  the radius of the circle, going anti-clockwise.  The point
  $\tuple{r}{\varphi}$ is known as the \emph{polar coordinates} of
  $\tuple{a}{b}$.  In other words, the Cartesian coordinates
  $\tuple{a}{b}$ and the polar coordinates $\tuple{r}{\varphi}$ both
  describe the same point.
}
\label{fig:convert_from_polar_to_Cartesian_coordinates}
\end{figure}


%%%%%%%%%%%%%%%%%%%%%%%%%%%%%%%%%%%%%%%%%%%%%%%%%%%%%%%%%%%%%%%%%%%%%%%%%%%

\section{Approximate $\pi$ with an octagon}


%%%%%%%%%%%%%%%%%%%%%%%%%%%%%%%%%%%%%%%%%%%%%%%%%%%%%%%%%%%%%%%%%%%%%%%%%%%

\section*{Problem}

\begin{problem}
\item You draw a pentagon inside the unit circle such that each of the
  five corners of the pentagon touches the unit circle.
  %%
  \begin{packedenum}
  \item Determine the coordinates of each of the five points of the
    pentagon.
  \end{packedenum}
\end{problem}

\end{document}
