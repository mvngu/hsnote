%%%%%%%%%%%%%%%%%%%%%%%%%%%%%%%%%%%%%%%%%%%%%%%%%%%%%%%%%%%%%%%%%%%%%%%%%%%

\documentclass[a4paper,oneside,12pt]{article}
\usepackage{mystyle}

\begin{document}

\title{\Large\bf Roots of quadratic functions}
\author{%%
  Minh Van Nguyen \\
  \url{mvngu@gmx.com}
}
\date{\today}
\maketitle

\noindent
This document will show you various ways to determine the roots of a
quadratic function.  You already know that the quadratic formula can
be used to determine the roots of any quadratic function.  The purpose
of this document is to show you how to determine the roots
\emph{without} using the quadratic formula.  Why would you want to
learn another technique to find the roots?  One reason is that so you
can check your results.  If using two different techniques produce the
same roots of a quadratic function, then you can be sure that your
results are correct.  Another reason is that in some cases you might
find it easier to use a technique other than the quadratic formula.
In the rest of the document, you will encounter cases where it would
be much easier to factorise a quadratic function than to use the
quadratic formula.  How would you factorise a quadratic function?


%%%%%%%%%%%%%%%%%%%%%%%%%%%%%%%%%%%%%%%%%%%%%%%%%%%%%%%%%%%%%%%%%%%%%%%%%%%

\section{Factoring when $c = 0$}

To factorise an expression means to write the expression as the
product of two or more expressions.  In the case of the number $6$,
you can factorise $6$ by writing it as the product of $2$ and $3$.
Hence the integer $6$ can be written in factored form as
\[
6
=
2 \times 3
\]
and you say that $2$ and $3$ are factors of $6$.  As another example,
you can write $60$ in factored form as $60 = 3 \times 20$.  You can
also factorise $20$ to get $20 = 4 \times 5$ and therefore $60$ can be
factorised as
%%
\begin{align*}
60
&=
3 \times 20 \\[4pt]
&=
3 \times 4 \times 5.
\end{align*}
%%
As can be seen from the above examples, factorising an integer
involves writing the integer as the product of two or more integers.
The factors are usually prime integers.  A factorisation such as
$5 = 1 \times 5$ is correct because $5$ is a prime and has no positive
factors other than $1$ and $5$.  The factored form
$5 = 1 \times 5 \times 1$ is also correct, but that's cheating because
you can also write $5 = 1 \times 5 \times 1 \times 1$ and so on.

What about factoring quadratic functions?  Let's start with quadratic
functions of the form $f(x) = ax^2 + bx$, where $a$ and $b$ are real
constants such that $a \neq 0$.  Note that $c = 0$ and so the vertical
intercept of $f(x)$ is the point $\tuple{0}{0}$.  Use the distributive
laws to write $f(x)$ in the factored form
%%
\begin{equation}
\label{eqn:quadroots:factorise_axx_bx}
f(x)
=
(ax + b)x
\end{equation}
%%
and you can see that $f(x)$ has the factors $x$ and $ax + b$.  Looking
at the factored form~\eqref{eqn:quadroots:factorise_axx_bx}, the roots
of $f(x)$ are $x = 0$ and $x = -b/a$.  But how did you get those
numbers?  To calculate the roots of $f(x)$ means to determine all
values of $x$ such that the expression $f(x) = 0$ is true.  In other
words, you want all values of $x$ such that the expression
%%
\begin{equation}
\label{eqn:quadroots:roots_of_axx_bx}
(ax + b)x
=
0
\end{equation}
%%
is true.  Expression~\eqref{eqn:quadroots:roots_of_axx_bx} tells you
that there are two numbers, i.e.~$x$ and $ax + b$, whose product is
zero.  You have two cases:
%%
\begin{packedenumeral}
\item If $x = 0$, then \Expression{eqn:quadroots:roots_of_axx_bx} is
  true because zero multiplied by another number is zero.  So one root
  of $f(x)$ is $x = 0$.

\item If $ax + b = 0$, then \Expression{eqn:quadroots:roots_of_axx_bx}
  is also true.  But for which values of $x$ would you have
  $ax + b = 0$?  Solving the latter expression for $x$ shows that
  $x = -b / a$.  Substituting the last expression
  into~\eqref{eqn:quadroots:roots_of_axx_bx} produces
  %%
  \begin{align*}
  \squarebracket*{
    a \parenthesis*{-\frac{b}{a}}
    +
    b
  }
  \parenthesis*{-\frac{b}{a}}
  &=
  (-b + b)
  \parenthesis*{-\frac{b}{a}} \\[4pt]
  &=
  0 \times \parenthesis*{-\frac{b}{a}} \\[4pt]
  &=
  0
  \end{align*}
  %%
  which is true.  Hence you have found that $x = -b / a$ is another
  root of $f(x)$.
\end{packedenumeral}
%%
By writing $f(x)$ as the factored
form~\eqref{eqn:quadroots:factorise_axx_bx} you can easily determine
the roots of $f(x)$ without using the quadratic formula.  The above
discussion is summarised in the following theorem.

\begin{theorem}
\label{thm:quadroots:quadratic_function_vertical_intercept_zero}
\textbf{Factorisation.}
Consider the quadratic function $f(x) = ax^2 + bx$, where $a$ and $b$
are real constants such that $a \neq 0$.  The roots of $f(x)$ are
\[
x = 0
%%
\qquad
\text{and}
\qquad
%%
x = -\frac{b}{a}.
\]
\end{theorem}

\begin{example}
\label{eg:quadroots:AmazingCar}
\textbf{AmazingCar.}
At a particular toy store, the sales of a toy called AmazingCar is
given by the sales function $S(x) = -3x + 40$.  Here $x$ represents
the price in dollars of each unit of AmazingCar and so $x$ is the unit
price.  The sales function $S(x)$ represents the number of units of
AmazingCar sold during a week.
%%
\begin{packedenum}
\item\label{subeg:quadroots:AmazingCar_graph_sales_function}
  Produce a graph of the sales function.  Use the graph to help you
  explain the sales function.  Identify the unit prices at which the
  number of units of AmazingCar sold is highest and lowest.

\item\label{subeg:quadroots:AmazingCar_revenue_function}
  Derive an expression for the revenue from selling units of
  AmazingCar during a particular week.  Produce a graph of the revenue
  function.

\item\label{subeg:quadroots:AmazingCar_price_zero_revenue}
  At which unit prices would the revenue from selling units of
  AmazingCar be zero dollars during a week?
\end{packedenum}
\end{example}

\begin{solution}
\solutionpart{subeg:quadroots:AmazingCar_graph_sales_function}
\Figure{fig:quadroots:AmazingCar_sales} shows a graph of the sales
function $S(x)$.  The graph shows that as the price for each unit of
AmazingCar increases the lower is the number of units sold during a
week.  In other words, the lower is the unit price the higher is the
number of units sold during the week.  This sounds reasonable because
as the price of something decreases you would expect to sell more of
the product.  Note the two points on the graph that show the lowest
and highest number of units sold.  These are the intercepts of the
axes.  The horizontal intercept is the point
$\tuple{\frac{40}{3}}{0}$, which tells you that when the unit price is
approximately $\$13.33$ zero units of AmazingCar would be sold per
week.  The lowest number of units sold per week is zero.  The vertical
intercept $\tuple{0}{40}$ tells you that when the unit price is zero
dollars~(each unit of AmazingCar is given away free of charge), the
number of units sold per week is $40$, which is also the highest
number of units sold per week.

\begin{figure}[!htbp]
\centering
\includegraphics[scale=1.1]{image/10/amazingcar-sales.pdf}
\caption{%%
  Graph of the sales function $S(x) = -3x + 40$ for AmazingCar.  Here
  $x$ represents the unit price in Australian dollars and $S(x)$
  represents the number of units of AmazingCar sold at $x$ dollars
  per unit.
}
\label{fig:quadroots:AmazingCar_sales}
\end{figure}

\solutionpart{subeg:quadroots:AmazingCar_revenue_function}
Since $x$ is the unit price and $S(x)$ represents how many units were
sold during a week, the revenue from selling AmazingCar is the unit
price multiplied by the number of units sold.  Thus you have the
expression
\[
R(x)
=
x S(x)
=
x(-3x + 40).
\]
The revenue function is graphed in
\Figure{fig:quadroots:AmazingCar_revenue}.

\begin{figure}[!htbp]
\centering
\includegraphics[scale=1.1]{image/10/amazingcar-revenue.pdf}
\caption{%%
  Graph of the revenue function $R(x) = x(-3x + 40)$ for AmazingCar.
  The function $R(x)$ represents the revenue~(in dollars) from selling
  units of AmazingCar during a week at $x$ dollars per unit.
}
\label{fig:quadroots:AmazingCar_revenue}
\end{figure}

\solutionpart{subeg:quadroots:AmazingCar_price_zero_revenue}
\Figure{fig:quadroots:AmazingCar_revenue} shows that you have two unit
prices at which the revenue from selling AmazingCar would be zero
dollars during a week.  Those particular two unit prices are the roots
of the revenue function $R(x)$.  You can use the quadratic formula to
determine the roots of $R(x)$.  However, note that you can use the
distributive laws to write the revenue function as
%%
\begin{equation}
\label{eqn:quadroots:AmazingCar_revenue_function_distributive}
R(x)
=
-3x^2 + 40x.
\end{equation}
%%
You then use
\Theorem{thm:quadroots:quadratic_function_vertical_intercept_zero} to
conclude that the roots of $R(x)$ are $x = 0$ and
\[
x
=
-\frac{40}{-3}
=
\frac{40}{3}.
\]
In other words, the revenue during a week from selling units of
AmazingCar would be zero dollars if the unit prices were either zero
dollars or approximately $\$13.33$.
\end{solution}

\begin{exercise}
\textbf{More AmazingCar.}
You will further explore \Example{eg:quadroots:AmazingCar}.
%%
\begin{packedenum}
\item\label{subeg:quadroots:AmazingCar_roots_quadratic_formula}
  Use the quadratic formula to verify the roots of the revenue
  function.

\item\label{subeg:quadroots:AmazingCar_maximum_revenue}
  Determine the unit price that would result in maximum revenue for a
  week from selling units of AmazingCar.
\end{packedenum}
\end{exercise}

\ifbool{showSolution}{
\begin{solution}
\solutionpart{subeg:quadroots:AmazingCar_roots_quadratic_formula}
The discriminant of the revenue
function~\eqref{eqn:quadroots:AmazingCar_revenue_function_distributive}
is given by the expression $\Delta = 40^2 - 4(-3)(0) = 40^2$.  Apply
the quadratic formula to the revenue
function~\eqref{eqn:quadroots:AmazingCar_revenue_function_distributive}
to see that the roots of $R(x)$ are
%%
\begin{align*}
x
&=
\frac{
  -40 \pm \sqrt{\Delta}
}{
  2(-3)
} \\[4pt]
&=
\frac{
  -40 \pm 40
}{
  -6
}.
\end{align*}
Thus the roots of $R(x)$ are
\[
x
=
\frac{-40 + 40}{-6}
=
0
\]
and
\[
x
=
\frac{-40 - 40}{-6}
=
\frac{40}{3}
\]
which are the same as what you got in \Example{eg:quadroots:AmazingCar}.

\solutionpart{subeg:quadroots:AmazingCar_maximum_revenue}
The maximum revenue is located at the vertex in the graph of $R(x)$.
The horizontal coordinate of the vertex is
\[
x
=
\frac{-40}{2(-3)}
=
\frac{20}{3}.
\]
The vertical coordinate of the vertex is
\[
R(20/3)
=
\frac{20}{3}
\parenthesis*{
  -3 \times \frac{20}{3}
  +
  40
}
=
\frac{20}{3}
(-20 + 40)
=
\frac{400}{3}.
\]
That is, the maximum revenue for a week from selling units of
AmazingCar would be approximately $\$133.33$, which would occur if the
unit price is approximately $\$6.67$.  All numbers have been rounded
to two decimal places.
\end{solution}
}{}

\begin{exercise}
Consider the function $f(x) = 5x^2 - 3x$.
%%
\begin{packedenum}
\item\label{subex:quadroots:factorise_a5_bminus3_c0}
  Factorise the function  and determine its roots.

\item\label{subex:quadroots:verify_roots_substitution_a5_bminus3_c0}
  Verify the roots from \Part{subex:quadroots:factorise_a5_bminus3_c0}
  by substituting the roots into $f(x)$ and simplify.

\item\label{subex:quadroots:verify_roots_quadratic_formula_a5_bminus3_c0}
  Verify the roots from \Part{subex:quadroots:factorise_a5_bminus3_c0}
  by using the quadratic formula.
\end{packedenum}
\end{exercise}

\ifbool{showSolution}{
\begin{solution}
\solutionpart{subex:quadroots:factorise_a5_bminus3_c0}
For the function $f(x) = 5x^2 - 3x$, you may use the distributive laws
to factor out the $x$ and obtain
\[
f(x)
=
x(5x - 3).
\]
As for the roots of $f(x)$, you may use
\Theorem{thm:quadroots:quadratic_function_vertical_intercept_zero}.
However, you may also reason as follows.  Let's set $f(x)$ to zero and
determine for which values of $x$ would the expression $f(x) = 0$ be
true.  In other words, you want all values of $x$ such that the
expression $x(5x - 3) = 0$ is true.  The latter expression is true
when $x = 0$ because $0 \times (5x - 3)$ becomes zero.  So one root of
$f(x)$ occurs at $x = 0$.  If $5x - 3 = 0$, then $x \times 0 = 0$ is
also true.  The expression $5x - 3 = 0$ can also be written as
$5x = 3$ and solving for $x$ produces $x = 3 / 5$, which is another
root of $f(x)$.  Therefore the roots of $f(x)$ are $x = 0$ and
$x = 3 / 5$.

\solutionpart{subex:quadroots:verify_roots_substitution_a5_bminus3_c0}
Let's check your results
from \Part{subex:quadroots:factorise_a5_bminus3_c0} by substitution.
If you substitute each root into $f(x)$ and simplify, you should get
zero as a function value.  Substitute the root $x = 0$ into $f(x)$ and
simplify to get
\[
f(0)
=
5(0^2) - 3(0)
=
0.
\]
Now substitute $x = 3 / 5$ into $f(x)$ and simplify to get
%%
\begin{align*}
f(3 / 5)
&=
5\parenthesis*{\frac{3}{5}}^2
-
3\parenthesis*{\frac{3}{5}} \\[4pt]
&=
5 \times \frac{3}{5} \times \frac{3}{5}
-
3 \times \frac{3}{5} \\[4pt]
&=
3 \times \frac{3}{5}
-
3 \times \frac{3}{5} \\[4pt]
&=
0
\end{align*}
%%
as required.

\solutionpart{subex:quadroots:verify_roots_quadratic_formula_a5_bminus3_c0}
The discriminant of $f(x)$ is
$\Delta = (-3)^2 - 4(5)(0) = 3^2$.  The quadratic formula shows that
the roots of $f(x)$ are
%%
\begin{align*}
x
&=
\frac{
  -(-3)
  \pm
  \sqrt{\Delta}
}{
  2(5)
} \\[4pt]
&=
\frac{
  3
  \pm
  \sqrt{3^2}
}{
  10
} \\[4pt]
&=
\frac{3 \pm 3}{10}.
\end{align*}
%%
Thus one root of $f(x)$ is
\[
x_1
=
\frac{3 + 3}{10}
=
\frac{6}{10}
=
\frac{3}{5}
\]
and the other root is
\[
x_2
=
\frac{3 - 3}{10}
=
\frac{0}{10}
=
0.
\]
These are the same as the roots you obtained
in \Part{subex:quadroots:factorise_a5_bminus3_c0}.
\end{solution}
}{}


%%%%%%%%%%%%%%%%%%%%%%%%%%%%%%%%%%%%%%%%%%%%%%%%%%%%%%%%%%%%%%%%%%%%%%%%%%%

\section{Factoring when $a = 1$}
\label{sec:quadroots:factor_quadratic_a_1}

Given the quadratic function $f(x) = ax^2 + bx + c$, if $a = 1$ then
the function simplifies to the expression
%%
\begin{equation}
\label{eqn:quadroots:monic_quadratic_function}
f(x)
=
x^2 + bx + c.
\end{equation}
%%
You have already seen examples of this function, especially when you
were considering the interpretation of a quadratic function as the
area of a rectangle.  This interpretation is illustrated in
\Figure{fig:quadroots:quadratic_as_rectangle}, which you have seen
before.  From the figure, you may write the area of the larger
rectangle as the expression
%%
\begin{equation}
\label{eqn:quadroots:monic_quadratic_function_factored}
(x + \alpha) (x + \beta)
=
x^2 + (\alpha + \beta)x + \alpha\beta.
\end{equation}
%%
How is \Expression{eqn:quadroots:monic_quadratic_function_factored}
related to the problem of factoring a quadratic function?

\begin{figure}[!htbp]
\centering
\includegraphics[scale=1.1]{image/08/quadratic-as-square.pdf}
\caption{%%
  The quadratic function $f(x) = (x + \alpha)(x + \beta)$ can be
  interpreted as the area of a rectangle whose base and height are
  $x + \alpha$ and $x + \beta$, respectively.
}
\label{fig:quadroots:quadratic_as_rectangle}
\end{figure}

The obvious answer is that
\Expression{eqn:quadroots:monic_quadratic_function_factored} gives you
the factored form of the quadratic function
%%
\begin{equation}
\label{eqn:quadroots:monic_quadratic_function_roots}
f(x)
=
x^2 + (\alpha + \beta)x + \alpha\beta.
\end{equation}
%%
One factor is $x + \alpha$, the other factor is $x + \beta$, and
therefore the roots of $f(x)$ are $x = -\alpha$ and $x = -\beta$.  In
other words, when you have a quadratic function that can be written as
in \Expression{eqn:quadroots:monic_quadratic_function}, you ask
yourself: How do I factorise the function so that I end up with
something like
expressions~\eqref{eqn:quadroots:monic_quadratic_function_factored}
or~\eqref{eqn:quadroots:monic_quadratic_function_roots}?  If you
compare
\Expressions{eqn:quadroots:monic_quadratic_function}{eqn:quadroots:monic_quadratic_function_factored},
you will notice that you have two equations:
%%
\begin{equation}
\label{eqn:quadroots:monic_quadratic_function_factors_sum_product}
\alpha + \beta
=
b
%%
\qquad
\text{and}
\qquad
%%
\alpha\beta
=
c.
\end{equation}
%%
These equations tell you that you want two numbers $\alpha$ and
$\beta$ such that when they are added together you get $b$ and when
they are multiplied together you get $c$.  The problem now is to
determine the values of $\alpha$ and $\beta$.  The examples below will
help to clarify the theory.  The above discussion is summarised in the
next theorem.

\begin{theorem}
\textbf{Factorisation.}
Consider the function $f(x) = x^2 + (\alpha + \beta)x + \alpha\beta$,
where $\alpha$ and $\beta$ are real constants.  Then $f(x)$ can be
written in factored form as $f(x) = (x + \alpha) (x + \beta)$ and the
roots of $f(x)$ are $x = -\alpha$ and $x = -\beta$.
\end{theorem}

\begin{example}
\label{eg:quadroots:factorise_monic_a1_b2_c6}
Consider the function $f(x) = x^2 + 5x + 6$.  Factorise $f(x)$ and
determine its roots.  Check your results by using the quadratic
formula.
\end{example}

\begin{solution}
The function $f(x)$ has the form similar to
\Expression{eqn:quadroots:monic_quadratic_function} so you should
consider the equations
in~\eqref{eqn:quadroots:monic_quadratic_function_factors_sum_product}.
That is, you want two numbers whose sum is $5$ and whose product is
$6$.  You need to know the integer factors of $6$.  The positive
factors of $6$ are $1$, $2$, $3$, and $6$.  The negative factors are
obtained by multiplying each positive factor by $-1$.  Then the
integer factors of $6$ are
\[
\octuple{1}{2}{3}{6}{-1}{-2}{-3}{-6}.
\]
Among these factors, choose two that sum to $5$ and have a product of
$6$.  The required factors are $2$ and $3$ because $2 + 3 = 5$ and
$2 \times 3 = 6$.  Thus $f(x)$ can be written in factored form as
$f(x) = (x + 2) (x + 3)$.  If $x = -2$, then $f(-2)$ simplifies to
zero.  If $x = -3$, then $f(-3)$ also becomes zero.  Therefore the
roots of $f(x)$ are $x = -2$ and $x = -3$.

You can verify the roots you obtained above by using the quadratic
formula.  The quadratic formula shows that the roots of $f(x)$ are
%%
\begin{align*}
x
&=
\frac{
  -5
  \pm
  \sqrt{
    5^2 - 4(1)(6)
  }
}{
  2(1)
} \\[4pt]
&=
\frac{
  -5
  \pm
  \sqrt{
    25 - 24
  }
}{
  2
} \\[4pt]
&=
\frac{
  -5 \pm 1
}{
  2
}.
\end{align*}
%%
Then one root of $f(x)$ is
\[
x_1
=
\frac{-5 + 1}{2}
=
\frac{-4}{2}
=
-2
\]
and the other root is
\[
x_2
=
\frac{-5 - 1}{2}
=
\frac{-6}{2}
=
-3.
\]
These are the same roots as obtained above.
\end{solution}

\begin{exercise}
Factorise the function $f(x) = x^2 + 7x + 12$ and determine its
roots.  Use the quadratic formula to verify your results.
\end{exercise}

\ifbool{showSolution}{
\begin{solution}
You want two numbers that sum to $7$ and have a product of $12$.  The
integer factors of $12$ are
\[
\sextuple{1}{2}{3}{4}{6}{12}
\]
among which $3$ and $4$ are the required integers.  Thus you have the
factored form $f(x) = (x + 3) (x + 4)$.  If $x = -3$, then $f(-3)$
simplifies to zero.  If $x = -4$, then $f(-4)$ also becomes zero.
Therefore the roots of $f(x)$ are $x = -3$ and $x = -4$.

Now use the quadratic formula to verify the roots that you obtained
above.  The quadratic formula shows that the roots of $f(x)$ are
%%
\begin{align*}
x
&=
\frac{
  -7
  \pm
  \sqrt{
    7^2 - 4(1)(12)
  }
}{
  2(1)
} \\[4pt]
&=
\frac{
  -7
  \pm
  \sqrt{
    49 - 48
  }
}{
  2
} \\[4pt]
&=
\frac{
  -7 \pm 1
}{
  2
}.
\end{align*}
%%
Thus one root of $f(x)$ is
\[
x_1
=
\frac{-7 + 1}{2}
=
\frac{-6}{2}
=
-3
\]
and the other root is
\[
x_2
=
\frac{-7 - 1}{2}
=
\frac{-8}{2}
=
-4.
\]
These are the same roots as obtained above.
\end{solution}
}{}

\begin{example}
Consider the function $g(x) = 2x^2 + 16x + 30$.  Factorise the
function and determine its roots.  Use the quadratic formula to verify
your results.
\end{example}

\begin{solution}
The function $g(x)$ looks like it is different from
\Expression{eqn:quadroots:monic_quadratic_function}, but not really.
The trick is to write $g(x)$ in such a way that allows you to use the
technique explained in
\Example{eg:quadroots:factorise_monic_a1_b2_c6}.  Note that each of
the three terms in $g(x)$ has a common factor, namely $2$.  Factoring
out the $2$ and $g(x)$ can be written as
%%
\begin{align*}
g(x)
&=
2 (x^2 + 8x + 15) \\[4pt]
&=
2 \cdot h(x)
\end{align*}
%%
where $h(x) = x^2 + 8x + 15$.  Note that the roots of $h(x)$ are also
the roots of $g(x)$.  The reason is that if $x = x_1$ is any root of
$h(x)$, then $h(x_1)$ simplifies to zero and hence $2 \cdot h(x_1)$
also becomes zero.  Thus $x_1$ is a root of $g(x)$.  In other words,
you need only to determine the roots of $h(x)$.

Let's factorise $h(x)$.  The function $h(x)$ has a form similar to
\Expression{eqn:quadroots:monic_quadratic_function} so you can use the
technique explained in \Example{eg:quadroots:factorise_monic_a1_b2_c6}
to factorise $h(x)$.  You want two numbers that add up to $8$ and have
a product of $15$.  The positive integer factors of $15$ are $1$, $3$,
$5$, and $15$.  Multiply each positive factor by $-1$ to get a
negative integer factor.  Then the integer factors of $15$ are
\[
\octuple{1}{3}{5}{15}{-1}{-3}{-5}{-15}.
\]
Among these factors, $3$ and $5$ are the required integers because
$3 + 5 = 8$ and $3 \times 5 = 15$.  Now write $h(x)$ in the factored
form $h(x) = (x + 3) (x + 5)$ and therefore $g(x)$ can be written in
the factored form
\[
g(x)
=
2 (x + 3) (x + 5).
\]
If $x = -3$, then $g(-3)$ simplifies to zero.  If $x = -5$, then
$g(-5)$ also becomes zero.  Therefore the roots of $g(x)$ are
$x = -3$ and $x = -5$.

Now use the quadratic formula to verify the roots that you obtained
above.  The quadratic formula shows that $g(x)$ has the roots
%%
\begin{align*}
x
&=
\frac{
  -16
  \pm
  \sqrt{16^2 - 4(2)(30)}
}{
  2(2)
} \\[4pt]
&=
\frac{
  -16
  \pm
  \sqrt{256 - 240}
}{
  4
} \\[4pt]
&=
\frac{
  -16 \pm 4
}{
  4
}.
\end{align*}
%%
So one root of $g(x)$ is
\[
x_1
=
\frac{-16 + 4}{4}
=
\frac{-12}{4}
=
-3
\]
and the other root is
\[
x_2
=
\frac{-16 - 4}{4}
=
\frac{-20}{4}
=
-5.
\]
These are the same roots as obtained above.
\end{solution}

\begin{exercise}
Factorise the function $g(x) = 3x^2 + 15x + 18$ and determine its
roots.  Use the quadratic formula to check your results.
\end{exercise}

\ifbool{showSolution}{
\begin{solution}
The trick is to write $g(x)$ in a form similar to
\Expression{eqn:quadroots:monic_quadratic_function}.  Doing so would
allow you to use the technique explained in
\Example{eg:quadroots:factorise_monic_a1_b2_c6}.  Each of the three
terms in $g(x)$ has $3$ as a common factor.  Factor out the $3$ to
write $g(x)$ as
%%
\begin{align*}
g(x)
&=
3 (x^2 + 5x + 6) \\[4pt]
&=
3 \cdot h(x)
\end{align*}
where $h(x) = x^2 + 5x + 6$.  The function $h(x)$ is the same as the
function in \Example{eg:quadroots:factorise_monic_a1_b2_c6} so you
have the factored form $h(x) = (x + 2) (x + 3)$ and therefore $g(x)$
has the factored form
\[
g(x)
=
3 (x + 2) (x + 3).
\]
Then the roots of $g(x)$ are the same roots derived in
\Example{eg:quadroots:factorise_monic_a1_b2_c6}.
\end{solution}
}{}

\begin{example}
\label{eg:quadroots:factored_form_bminus2_c1}
Factorise $f(x) = x^2 - 2x + 1$ and determine its roots.  Check your
results by means of the quadratic formula.
\end{example}

\begin{solution}
You need to be careful about the negative sign.  The function $f(x)$
has a form similar to
\Expression{eqn:quadroots:monic_quadratic_function}.  You require two
numbers that sum to $-2$ and have a product of $1$.  Since there is a
negative sign, you must consider all integer factors of $1$, both the
positive and negative factors.  The positive factor of $1$ is $1$
itself.  Now multiply each positive factor by $-1$ to get the negative
factors.  So multiplying $1$ by $-1$ results in $-1$.  Then the
factors of $1$ are
\[
\pair{-1}{1}.
\]
The required integers are $-1$ and $-1$ again because you have the sum
\[
(-1) + (-1)
=
-2
\]
and the product $(-1) \times (-1) = 1$.  Now write $f(x)$ in the
factored form
\[
f(x)
=
(x - 1)^2.
\]
If $x = 1$, then $f(1)$ simplifies to zero.  Therefore $f(x)$ has the
root $x = 1$.

Now check your result via the quadratic formula.  Using the quadratic
formula, you have the root
%%
\begin{align*}
x
&=
\frac{
  -(-2)
  \pm
  \sqrt{(-2)^2 - 4(1)(1)}
}{
  2(1)
} \\[4pt]
&=
\frac{
  2
  \pm
  \sqrt{4 - 4}
}{
  2
} \\[4pt]
&=
\frac{
  2 \pm 0
}{
  2
} \\[4pt]
&=
1
\end{align*}
%%
which is the same as what you obtained above.
\end{solution}

\begin{exercise}
\label{ex:quadroots:factorise_aminus1_c1}
Factorise $f(x) = -x^2 + 1$ and determine its roots.  Verify your
results by means of the quadratic formula.
\end{exercise}

\ifbool{showSolution}{
\begin{solution}
Note that the function $f(x)$ can be written as
\[
f(x)
=
-(x^2 - 1)
=
-(x^2 + 0 \cdot x - 1).
\]
If you define the function $g(x) = x^2 + 0 \cdot x - 1$, then $f(x)$
can be written as
\[
f(x)
=
-g(x).
\]
In other words, you need only to determine the roots of $g(x)$.

Let's factorise $g(x)$.  You want two numbers that sum to zero and
have a product of $-1$.  The integer factors of $-1$ are
\[
\pair{-1}{1}.
\]
Among these factors, $-1$ and $1$ are the required numbers because you
have the sum $-1 + 1 = 0$ and the product $(-1) \times 1 = -1$.  Now
write $g(x)$ in the factored form $g(x) = (x - 1) (x + 1)$ and
therefore $f(x)$ can be written in factored form as
\[
f(x)
=
-(x - 1) (x + 1).
\]
If $x = 1$, then $f(1)$ simplifies to zero.  If $x = -1$, then $f(-1)$
also becomes zero.  Therefore the roots of $f(x)$ are $x = -1$ and
$x = 1$.

Finally, you verify your results.  The quadratic formula shows that
the roots of $f(x)$ are
%%
\begin{align*}
x
&=
\frac{
  -0
  \pm
  \sqrt{0^2 - 4(-1)(1)}
}{
  2(-1)
} \\[4pt]
&=
\frac{
  \pm \sqrt{4}
}{
  -2
} \\[4pt]
&=
\frac{\pm 2}{-2}.
\end{align*}
%%
Then one root of $f(x)$ is
\[
x_1
=
\frac{2}{-2}
=
-1
\]
and the other root is
\[
x_2
=
\frac{-2}{-2}
=
1.
\]
These are the same roots as those you obtained above.
\end{solution}
}{}


%%%%%%%%%%%%%%%%%%%%%%%%%%%%%%%%%%%%%%%%%%%%%%%%%%%%%%%%%%%%%%%%%%%%%%%%%%%

\section{Completing the square: special case}
\label{sec:quadroots:completing_the_square_special_case}

The idea of \emph{completing the square} is to turn a quadratic
function from the form $f(x) = ax^2 + bx + c$ into the form
%%
\begin{equation}
\label{eqn:quadroots:completing_square_general_form}
f(x)
=
\alpha(x + \beta)^2 + \gamma
\end{equation}
%%
where $\alpha$, $\beta$, and $\gamma$ are numbers that can be written
in terms of $a$, $b$, and $c$.  First, let's consider the case where
$a = 1$.  That is, you want to write
\[
f(x)
=
x^2 + bx + c
\]
in a form similar to
\Expression{eqn:quadroots:completing_square_general_form}.
From \Section{sec:quadroots:factor_quadratic_a_1} you already know how
to factorise $f(x)$ and determine its roots.  However, let's see how
to determine the roots of $f(x)$ in a different way.  To understand
how this can be accomplished, consider the function $g(x) = x^2 + bx$
so that you have
%%
\begin{equation}
\label{eqn:quadroots:completing_square_special_case}
f(x)
=
g(x) + c.
\end{equation}
%%
The function $g(x)$ can be interpreted as the area that results from
adding the area of a square to the area of a rectangle; see
\Figure{fig:quadroots:special_complete_square_square_plus_rectangle}.
The square has a side length of $x$ and hence an area of $x^2$.  The
rectangle has a width and height of lengths $b$ and $x$, respectively,
and so an area of $bx$.  At this point, you might ask yourself:  What
about the constant $c$?

\begin{figure}[!htbp]
\centering
\includegraphics[scale=1.1]{image/10/complete-square-a1-c0.pdf}
\caption{%%
  The quadratic function $g(x) = x^2 + bx$ can be visualised as the
  area of a square plus the area of a rectangle.  The square has a
  side length of $x$, hence the area of the square is $x^2$.  The
  rectangle has a width of $b$ and a height of $x$, so the rectangle
  has an area of $bx$.  Thus $g(x)$ can be interpreted as the area of
  a square plus the area of a rectangle.  The rectangle can be cut in
  half along the dashed line as shown.  Each half has a width of $b/2$
  and a height of $x$.
}
\label{fig:quadroots:special_complete_square_square_plus_rectangle}
\end{figure}

\begin{figure}[!htbp]
\centering
\includegraphics[scale=1.1]{image/10/complete-square-a1-c0_halfb.pdf}
\caption{%%
  The quadratic function $g(x) = x^2 + bx$ can be visualised as the
  area of a square plus the area of a rectangle.  The rectangle is cut
  in half.  One half is arranged to the right of the square.  The
  other half is arranged underneath the square.  Now you have a shape
  that is nearly like a square.  The small dashed square in the lower
  right is what is missing to make a complete square.
}
\label{fig:quadroots:special_complete_square_nearly_square}
\end{figure}

To see how you can account for the constant $c$, consider cutting the
rectangle in half along the dashed line shown in
\Figure{fig:quadroots:special_complete_square_square_plus_rectangle}.
Move one half to the bottom of the square and the other half to the
right side of the square, as shown in
\Figure{fig:quadroots:special_complete_square_nearly_square}.  What
you end up with is a shape that looks similar to a whole square,
except for a missing small piece in the lower-right corner.  The
missing piece is a square whose side length is $b/2$ and hence the
square has an area of $\parenthesis*{\frac{b}{2}}^2$.  In other words,
adding the area $\parenthesis*{\frac{b}{2}}^2$ to $g(x)$ would result
in the area of the larger square.  Since the larger square has a side
length of $x + \frac{b}{2}$, its area is
$\parenthesis*{x + \frac{b}{2}}^2$ and thus you have the expression
\[
g(x) + \parenthesis*{\frac{b}{2}}^2
=
\parenthesis*{x + \frac{b}{2}}^2
\]
which can be solved for $g(x)$ to produce
\[
g(x)
=
\parenthesis*{x + \frac{b}{2}}^2
-
\parenthesis*{\frac{b}{2}}^2.
\]
Substitute the latter expression
into~\eqref{eqn:quadroots:completing_square_special_case} and you get
%%
\begin{equation}
\label{eqn:quadroots:completing_square_special_case_squared}
\begin{aligned}
f(x)
&=
\parenthesis*{x + \frac{b}{2}}^2
-
\parenthesis*{\frac{b}{2}}^2
+
c \\[4pt]
&=
\parenthesis*{x + \frac{b}{2}}^2
+
c
-
\parenthesis*{\frac{b}{2}}^2.
\end{aligned}
\end{equation}
%%
The \Expression{eqn:quadroots:completing_square_special_case_squared}
has a form that is similar to
\Expression{eqn:quadroots:completing_square_general_form} because you
can make the substitutions
\[
\alpha
=
1,
%%
\qquad
%%
\beta
=
\frac{b}{2},
%%
\qquad
%%
\gamma
=
c
-
\parenthesis*{\frac{b}{2}}^2.
\]

How would you use
\Expression{eqn:quadroots:completing_square_special_case_squared} to
determine the roots of $f(x)$?  That is, how can
\Expression{eqn:quadroots:completing_square_special_case_squared} help
you to determine all values of $x$ such that the expression
$f(x) = 0$ is true?  You equate
\Expression{eqn:quadroots:completing_square_special_case_squared} to
zero to get
\[
\parenthesis*{x + \frac{b}{2}}^2
+
c
-
\parenthesis*{\frac{b}{2}}^2
=
0
\]
which can also be written as
\[
\parenthesis*{x + \frac{b}{2}}^2
=
\parenthesis*{\frac{b}{2}}^2 - c.
\]
In the latter expression, taking the square root of both sides results
in
\[
x + \frac{b}{2}
=
\pm
\sqrt{
  \parenthesis*{\frac{b}{2}}^2 - c
}
\]
and solve for $x$ to get the roots
\[
x
=
-\frac{b}{2}
\pm
\sqrt{
  \parenthesis*{\frac{b}{2}}^2 - c
}.
\]
The above discussion is summarised in the next theorem.

\begin{theorem}
\label{thm:quadroots:monic_quadratic_roots}
\textbf{Completing the square.}
Let $f(x) = x^2 + bx + c$ be a quadratic function, where $b$ and $c$
are given constants.  Then $f(x)$ can be written as
\[
f(x)
=
\parenthesis*{x + \frac{b}{2}}^2
+
c
-
\parenthesis*{\frac{b}{2}}^2
\]
and the roots of $f(x)$ are given by
\[
x
=
-\frac{b}{2}
\pm
\sqrt{
  \parenthesis*{\frac{b}{2}}^2 - c
}.
\]
\end{theorem}

So far so good.  The next question is: How can you use
\Theorem{thm:quadroots:monic_quadratic_roots} to help you determine
the roots of a quadratic function?

\begin{example}
\label{eg:quadroots:completing_square_monic_bminus10_c21}
Consider the function $f(x) = x^2 - 10x + 21$.
%%
\begin{packedenum}
\item\label{subeg:quadroots:completing_square_roots_a1_bminus10_c21}
  Use the technique of completing the square to determine the roots of
  $f(x)$.

\item\label{subeg:quadroots:completing_square_verify_substitution_a1_bminus10_c21}
  Check your results by substituting the roots into $f(x)$.

\item\label{subeg:quadroots:completing_square_verify_quadform_a1_bminus10_c21}
  Verify your results by means of the quadratic formula.
\end{packedenum}
\end{example}

\begin{solution}
\solutionpart{subeg:quadroots:completing_square_roots_a1_bminus10_c21}
You can use the technique explained
in \Section{sec:quadroots:factor_quadratic_a_1} to factorise $f(x)$
and then read off its roots.  However, let's see how the roots of
$f(x)$ can be determined by applying
\Theorem{thm:quadroots:monic_quadratic_roots}.  The function
$f(x) = x^2 - 10x + 21$ has a form that is similar to the general
quadratic function in \Theorem{thm:quadroots:monic_quadratic_roots}
because you have $b = -10$ and $c = 21$.  Use
\Theorem{thm:quadroots:monic_quadratic_roots} to write $f(x)$ as the
expression
%%
\begin{align*}
f(x)
&=
\parenthesis*{x + \frac{-10}{2}}^2
+
21 - \parenthesis*{\frac{-10}{2}}^2 \\[4pt]
&=
(x - 5)^2
+
21 - (-5)^2 \\[4pt]
&=
(x - 5)^2 + 21 - 25 \\[4pt]
&=
(x - 5)^2 - 4.
\end{align*}
%%
The roots of $f(x)$ are all values of $x$ such that the expression
$f(x) = 0$ is true.  That is, you want all values of $x$ such that the
expression
\[
(x - 5)^2 - 4
=
0
\]
is true.  The latter expression can also be written as
$(x - 5)^2 = 4$ and taking the square root of both sides results in
$x - 5 = \pm\sqrt{4}$.  Now solve for $x$ to get the roots
$x = 5 \pm 2$.  Therefore one root of $f(x)$ is $x = 7$ and the other
root is $x = 3$.

\solutionpart{subeg:quadroots:completing_square_verify_substitution_a1_bminus10_c21}
You can check your results by substituting each root into $f(x)$.
Doing so should produce zero as the function value.  Substituting the
root $x = 7$ into $f(x)$ results in
%%
\begin{align*}
f(7)
&=
7^2 - 10(7) + 21 \\[4pt]
&=
49 - 70 + 21 \\[4pt]
&=
70 - 70 \\[4pt]
&=
0.
\end{align*}
%%
Substitute the root $x = 3$ into $f(x)$ to get
%%
\begin{align*}
f(3)
&=
3^2 - 10(3) + 21 \\[4pt]
&=
9 - 30 + 21 \\[4pt]
&=
30 - 30 \\[4pt]
&=
0
\end{align*}
%%
as required.

\solutionpart{subeg:quadroots:completing_square_verify_quadform_a1_bminus10_c21}
You can also use the quadratic formula to verify your results.  The
quadratic formula shows that the roots of $f(x)$ are
%%
\begin{align*}
x
&=
\frac{
  -(-10)
  \pm
  \sqrt{(-10)^2 - 4(1)(21)}
}{
  2(1)
} \\[4pt]
&=
\frac{
  10
  \pm
  \sqrt{100 - 84}
}{
  2
} \\[4pt]
&=
\frac{
  10 \pm 4
}{
  2
}.
\end{align*}
%%
One root of $f(x)$ is
\[
x_1
=
\frac{10 + 4}{2}
=
\frac{14}{2}
=
7
\]
and the other root is
\[
x_2
=
\frac{10 - 4}{2}
=
\frac{6}{2}
=
3.
\]
These are the same roots as those you derived
in \Part{subeg:quadroots:completing_square_roots_a1_bminus10_c21}.
\end{solution}

\begin{exercise}
Repeat \Example{eg:quadroots:completing_square_monic_bminus10_c21} by
using the technique explained
in \Section{sec:quadroots:factor_quadratic_a_1} to determine the roots
of $f(x) = x^2 - 10x + 21$.
\end{exercise}

\ifbool{showSolution}{
\begin{solution}
You want two numbers that sum to $-10$ and have a product of $21$.
The positive factors of $21$ are $1$, $3$, $7$, and $21$.  Multiply
each factor by $-1$ to get the negative factors of $21$.  Thus all
factors of $21$ are
\[
\octuple{1}{3}{7}{21}{-1}{-3}{-7}{-21}.
\]
Among these factors, the required integers are $-3$ and $-7$ because
you have the sum $(-3) + (-7) = -10$ and and the product
$(-3) \times (-7) = 21$.  Then $f(x)$ can be written in factored form
as $f(x) = (x - 3) (x - 7)$.  If $x = 3$, then $f(3)$ simplifies to
zero.  If $x = 7$, then $f(7)$ also simplifies to zero.  Therefore the
roots of $f(x)$ are $x = 3$ and $x = 7$.  These roots are the same as
those derived in
\Example{eg:quadroots:completing_square_monic_bminus10_c21}.
\end{solution}
}{}

\begin{exercise}
Repeat \Example{eg:quadroots:factored_form_bminus2_c1} by using
\Theorem{thm:quadroots:monic_quadratic_roots} to determine the roots
of the function $f(x) = x^2 - 2x + 1$.  Check your results by
substituting the roots into $f(x)$ and simplify.
\end{exercise}

\ifbool{showSolution}{
\begin{solution}
The function $f(x) = x^2 - 2x + 1$ has a form that is similar to the
function in \Theorem{thm:quadroots:monic_quadratic_roots} because you
have $b = -2$ and $c = 1$.  Use
\Theorem{thm:quadroots:monic_quadratic_roots} to write $f(x)$ as the
expression
%%
\begin{align*}
f(x)
&=
\parenthesis*{x + \frac{-2}{2}}^2
+
1 - \parenthesis*{\frac{-2}{2}}^2 \\[4pt]
&=
(x - 1)^2 + 1 - (-1)^2 \\[4pt]
&=
(x - 1)^2.
\end{align*}
%%
Therefore $f(x)$ has the root $x = 1$.

To check your result, substitute the root $x = 1$ into $f(x)$.  This
should produce a function value of zero.  Substituting $x = 1$ into
$f(x)$ yields
%%
\begin{align*}
f(1)
&=
1^2 - 2(1) + 1 \\[4pt]
&=
1 - 2 + 1 \\[4pt]
&=
2 - 2 \\[4pt]
&=
0
\end{align*}
as required.
\end{solution}
}{}

\begin{exercise}
Consider the function $f(x) = 4x^2 + 12x - 40$.
%%
\begin{packedenum}
\item\label{subeg:quadroots:monic_quadratic_x2_x5_completing_square}
  Use the technique of completing the square to determine the roots of
  $f(x)$.

\item\label{subeg:quadroots:monic_quadratic_x2_x5_factored_form}
  Use factorisation~(i.e.~the technique explained
  in \Section{sec:quadroots:factor_quadratic_a_1}) to determine the
  roots of $f(x)$.

\item\label{subeg:quadroots:monic_quadratic_x2_x5_quadratic_formula}
  Use the quadratic formula to determine the roots of $f(x)$.

\item\label{subeg:quadroots:monic_quadratic_x2_x5_verify}
  Use substitution to verify the roots you obtained.
\end{packedenum}
\end{exercise}

\ifbool{showSolution}{
\begin{solution}
\solutionpart{subeg:quadroots:monic_quadratic_x2_x5_completing_square}
The function $f(x)$ can be written in such a way that it resembles the
function in \Theorem{thm:quadroots:monic_quadratic_roots}.  Factoring
out the $4$ and you have
%%
\begin{align*}
f(x)
&=
4(x^2 + 3x - 10) \\[4pt]
&=
4 \cdot g(x)
\end{align*}
%%
where the function $g(x) = x^2 + 3x - 10$ is similar to the function
in \Theorem{thm:quadroots:monic_quadratic_roots} because you have
$b = 3$ and $c = -10$.  Note that the roots of $g(x)$ are also the
roots of $f(x)$.  The reason is that if $x = x_1$ is a value of $x$
such that $g(x_1) = 0$, then $4 \cdot g(x_1)$ is also zero.  That is,
you only need to determine the roots of $g(x)$, which is simpler than
the original definition of $f(x)$.

Use \Theorem{thm:quadroots:monic_quadratic_roots} to write $g(x)$ as
the expression
%%
\begin{equation}
\label{eqn:quadroots:completed_square_form_b3_cminus10}
\begin{aligned}
g(x)
&=
\parenthesis*{x + \frac{3}{2}}^2
-
10 - \parenthesis*{\frac{3}{2}}^2 \\[4pt]
&=
\parenthesis*{x + \frac{3}{2}}^2
-
10 - \frac{9}{4} \\[4pt]
&=
\parenthesis*{x + \frac{3}{2}}^2
-
\frac{40}{4} - \frac{9}{4} \\[4pt]
&=
\parenthesis*{x + \frac{3}{2}}^2
-
\frac{49}{4}.
\end{aligned}
\end{equation}
%%
Now you determine the roots of $g(x)$.  In other words, you want all
values of $x$ such that the expression $g(x) = 0$ is true.  Equate
\Expression{eqn:quadroots:completed_square_form_b3_cminus10} to zero
to get
\[
\parenthesis*{x + \frac{3}{2}}^2
-
\frac{49}{4}
=
0
\]
which can also be written as
\[
\parenthesis*{x + \frac{3}{2}}^2
=
\frac{49}{4}.
\]
Take the square root of both sides and you have
%%
\begin{align*}
x + \frac{3}{2}
&=
\pm\sqrt{\frac{49}{4}} \\[4pt]
&=
\pm\frac{\sqrt{49}}{\sqrt{4}} \\[4pt]
&=
\pm\frac{7}{2}.
\end{align*}
%%
Solving the latter expression for $x$ produces
\[
x
=
\pm\frac{7}{2} - \frac{3}{2}.
\]
Therefore one root of $f(x)$ is
\[
x
=
\frac{7}{2} - \frac{3}{2}
=
\frac{7 - 3}{2}
=
2
\]
and the other root is
\[
x
=
-\frac{7}{2} - \frac{3}{2}
=
\frac{-7 - 3}{2}
=
-5.
\]

\solutionpart{subeg:quadroots:monic_quadratic_x2_x5_factored_form}
Let's use factorisation to determine the roots of $f(x)$.  As
explained
in \Part{subeg:quadroots:monic_quadratic_x2_x5_completing_square}, you
need only to determine the roots of the function
$g(x) = x^2 + 3x - 10$.  You require two numbers that sum to $3$ and
have a product of $-10$.  The positive integer factors of $-10$ are
$1$, $2$, $5$, and $10$.  Multiply each factor by $-1$ to get the
negative factors.  Thus all integer factors of $-10$ are
\[
\octuple{1}{2}{5}{10}{-1}{-2}{-5}{-10}.
\]
Among these factors, the required numbers are $5$ and $-2$ because you
have the sum $5 + (-2) = 3$ and the product $5 \times (-2) = -10$.
Then $f(x)$ can be written in factored form as
\[
f(x)
=
4(x - 2) (x + 5).
\]
Therefore the roots of $f(x)$ are $x = 2$ and $x = -5$, which are the
same as those obtained
in \Part{subeg:quadroots:monic_quadratic_x2_x5_completing_square}.

\solutionpart{subeg:quadroots:monic_quadratic_x2_x5_quadratic_formula}
The quadratic formula shows that the roots of $f(x)$ are
%%
\begin{align*}
x
&=
\frac{
  -12
  \pm
  \sqrt{
    12^2 - 4(4)(-40)
  }
}{
  2(4)
} \\[4pt]
&=
\frac{
  -12
  \pm
  \sqrt{144 + 640}
}{
  8
} \\[4pt]
&=
\frac{
  -12 \pm 28
}{
  8
}.
\end{align*}
%%
Thus one root of $f(x)$ is
\[
x_1
=
\frac{-12 + 28}{8}
=
\frac{16}{8}
=
2
\]
and the other root is
\[
x_2
=
\frac{-12 - 28}{8}
=
\frac{-40}{8}
=
-5.
\]
These are the same roots as those you obtained
in \Part{subeg:quadroots:monic_quadratic_x2_x5_completing_square}.

\solutionpart{subeg:quadroots:monic_quadratic_x2_x5_verify}
To check your results, substitute the roots into $f(x)$ and simplify.
This should produce a function value of zero.  Substitute the root
$x = 2$ into $f(x)$ to get
%%
\begin{align*}
f(2)
&=
4(2^2) + 12(2) - 40 \\[4pt]
&=
16 + 24 - 40 \\[4pt]
&=
40 - 40 \\[4pt]
&=
0.
\end{align*}
%%
Next, substitute the root $x = -5$ into $f(x)$:
%%
\begin{align*}
f(-5)
&=
4(-5)^2 + 12(-5) - 40 \\[4pt]
&=
100 - 60 - 40 \\[4pt]
&=
100 - 100 \\[4pt]
&=
0.
\end{align*}
\end{solution}
}{}


%%%%%%%%%%%%%%%%%%%%%%%%%%%%%%%%%%%%%%%%%%%%%%%%%%%%%%%%%%%%%%%%%%%%%%%%%%%

\section{Completing the square: general case}

The general case of completing the square is very similar to the
special case explained
in \Section{sec:quadroots:completing_the_square_special_case}.  Given
a quadratic function $f(x) = ax^2 + bx + c$, where $a \neq 0$, you can
factor out the $a$ to get
%%
\begin{align*}
f(x)
&=
a
\parenthesis*{
  x^2 + \frac{b}{a}x + \frac{c}{a}
} \\[4pt]
&=
a \cdot g(x)
\end{align*}
%%
where you have the function
\[
g(x)
=
x^2 + \frac{b}{a}x + \frac{c}{a}.
\]
Note that the roots of $g(x)$ are also the roots of $f(x)$.  The
reason is because if $x = x_1$ is a root of $g(x)$, then you get
$g(x_1) = 0$ and thus the expression $a \cdot g(x_1)$ is also zero.
The upshot is that you need only to determine the roots of $g(x)$.  If
$b / a$ and $c / a$ are both integers, you can try to use the
technique explained in \Section{sec:quadroots:factor_quadratic_a_1} to
determine the roots of $g(x)$.  However, the technique of completing
the square as explained
in \Section{sec:quadroots:completing_the_square_special_case} will
work regardless of whether $b / a$ and $c / a$ are integers.  In
general, you should use the technique of completing the square to
determine the roots of $g(x)$.  To check your work, you can use~(if
possible) the factorisation technique explained
in \Section{sec:quadroots:factor_quadratic_a_1}, or you can substitute
the roots into $f(x)$ and derive a function value of zero, or you can
use the quadratic formula to see whether you get the same roots.  The
example below should help you to understand how to use the technique
of completing the square to determine the roots of any quadratic
function.

\begin{example}
\label{eg:quadroots:completing_square_a3_bminus7_c2}
Use the technique of completing the square to determine the roots of
the function $f(x) = 3x^2 - 7x + 2$.  Use substitution to check your
results.
\end{example}

\begin{solution}
First, you factor out the $3$.  Doing so results in
\[
f(x)
=
3
\parenthesis*{
  x^2 - \frac{7}{3}x + \frac{2}{3}
}.
\]
If you define the function $g(x) = x^2 - \frac{7}{3}x + \frac{2}{3}$,
then you can write
\[
f(x)
=
3 \cdot g(x).
\]
In the last expression, if you want to determine the roots of $f(x)$,
then this is the same as determining the roots of $g(x)$.

Next, you use the technique of completing the square as explained in
\Theorem{thm:quadroots:monic_quadratic_roots} to determine the roots
of $g(x)$.  Use \Theorem{thm:quadroots:monic_quadratic_roots} to write
$g(x)$ as
%%
\begin{align*}
g(x)
&=
\squarebracket*{
  x
  +
  \parenthesis*{-\frac{7}{3}} \times \frac{1}{2}
}^2
+
\frac{2}{3}
-
\squarebracket*{
  \parenthesis*{-\frac{7}{3}} \times \frac{1}{2}
}^2 \\[4pt]
&=
\parenthesis*{
  x
  -
  \frac{7}{6}
}^2
+
\frac{2}{3}
-
\parenthesis*{-\frac{7}{6}}^2 \\[4pt]
&=
\parenthesis*{
  x
  -
  \frac{7}{6}
}^2
+
\frac{2}{3}
-
\frac{49}{36} \\[4pt]
&=
\parenthesis*{
  x
  -
  \frac{7}{6}
}^2
-
\frac{25}{36}.
\end{align*}
%%
The roots of $g(x)$ are all values of $x$ such that the expression
$g(x) = 0$ is true.  In other words, you want to determine all values
of $x$ such that the expression
\[
\parenthesis*{
  x
  -
  \frac{7}{6}
}^2
-
\frac{25}{36}
=
0
\]
is true.  The last expression can also be written as
\[
\parenthesis*{
  x
  -
  \frac{7}{6}
}^2
=
\frac{25}{36}
\]
and taking the square root of both sides results in
\[
x - \frac{7}{6}
=
\pm
\sqrt{\frac{25}{36}}
=
\pm
\frac{5}{6}.
\]
Now solve for $x$ to get the roots
\[
x
=
\frac{7}{6} \pm \frac{5}{6}.
\]
Therefore one root of $g(x)$ is
\[
x_1
=
\frac{7}{6} + \frac{5}{6}
=
\frac{12}{6}
=
2
\]
and the other root is
\[
x_2
=
\frac{7}{6} - \frac{5}{6}
=
\frac{2}{6}
=
\frac{1}{3}.
\]

Finally, you check your results.  You can use the quadratic formula to
see whether you get the same roots.  However, for now you will
substitute the roots of $g(x)$ into $f(x)$ and derive the function
value of zero.  Substituting the root $x_1 = 2$ into $f(x)$ produces
%%
\begin{align*}
f(2)
&=
3(2)^2 - 7(2) + 2 \\[4pt]
&=
12 - 14 + 2 \\[4pt]
&=
14 - 14 \\[4pt]
&=
0.
\end{align*}
%%
Substitute the root $x_2 = 1 / 3$ into $f(x)$ to get
%%
\begin{align*}
f(1/3)
&=
3\parenthesis*{\frac{1}{3}}^2
-
7\parenthesis*{\frac{1}{3}}
+
2 \\[4pt]
&=
\frac{3}{9}
-
\frac{7}{3}
+
2 \\[4pt]
&=
\frac{1}{3}
-
\frac{7}{3}
+
2 \\[4pt]
&=
-\frac{6}{3} + 2 \\[4pt]
&=
-2 + 2 \\[4pt]
&=
0
\end{align*}
%%
as required.
\end{solution}

\begin{exercise}
Use the quadratic formula to verify the roots in
\Example{eg:quadroots:completing_square_a3_bminus7_c2}.
\end{exercise}

\ifbool{showSolution}{
\begin{solution}
Using the quadratic formula, the roots of $f(x) = 3x^2 - 7x + 2$ are
%%
\begin{align*}
x
&=
\frac{
  -(-7)
  \pm
  \sqrt{
    (-7)^2 - 4(3)(2)
  }
}{
  2(3)
} \\[4pt]
&=
\frac{
  7
  \pm
  \sqrt{
    49 - 24
  }
}{
  6
} \\[4pt]
&=
\frac{
  7
  \pm
  \sqrt{25}
}{
  6
} \\[4pt]
&=
\frac{
  7 \pm 5
}{
  6
}.
\end{align*}
%%
Therefore one root of $f(x)$ is
\[
x_1
=
\frac{7 + 5}{6}
=
\frac{12}{6}
=
2
\]
and the other root is
\[
x_2
=
\frac{7 - 5}{6}
=
\frac{2}{6}
=
\frac{1}{3}.
\]
These are the same roots as obtained in
\Example{eg:quadroots:completing_square_a3_bminus7_c2}.
\end{solution}
}{}

\begin{exercise}
\label{ex:quadroots:quadratic_roots:given_roots_x1_x2}
Provide two different quadratic functions each of which has the roots
$x = 1$ and $x = 2$.
\end{exercise}

\ifbool{showSolution}{
\begin{solution}
The function $f(x) = (x - 1) (x - 2)$ has the roots $x = 1$ and
$x = 2$.  This can be easily verified by substituting the roots into
$f(x)$ and simplify.  Substituting $x = 1$ into $f(x)$ and you get
$f(1) = (1 - 1) (1 - 2)$, which simplifies to zero.  Furthermore,
substituting $x = 2$ into $f(x)$ and you get
$f(2) = (2 - 1) (2 - 2)$, which also simplifies to zero.  Note that
the function
\[
g(x)
=
-f(x)
=
(x - 1) (x - 2)
\]
also has the same roots as $f(x)$.
\end{solution}
}{}


\newpage
%%%%%%%%%%%%%%%%%%%%%%%%%%%%%%%%%%%%%%%%%%%%%%%%%%%%%%%%%%%%%%%%%%%%%%%%%%%

\section*{Problem}

\begin{problem}
\item Read the following paper by Michael J. Boss\'e and
  N. R. Nandakumar: \emph{The factorability of quadratics: motivation
    for more techniques}.\footnote{
    The paper is available at
    \url{https://doi.org/10.1093/teamat/hrh018}.
  }

\item\label{prob:quadroots:given_roots_all_possible_quadratics}
  Let $\alpha$ and $\beta$ be any real numbers.  Determine all
  possible quadratic functions each of which has the roots
  $x = \alpha$ and $x = \beta$.
\ifbool{showSolution}{
\begin{solution}
The function $f(x) = (x - \alpha) (x - \beta)$ has the roots
$x = \alpha$ and $x = \beta$.  If $\gamma$ is any real number such
that $\gamma \neq 0$, then the function
%%
\begin{equation}
\label{eqn:quadroots:all_quadratics_with_given_roots}
g(x)
=
\gamma \cdot f(x)
=
\gamma (x - \alpha) (x - \beta)
\end{equation}
%%
also has the same roots as $f(x)$.  You can easily verify this by
substituting the roots of $f(x)$ into $g(x)$ and simplify.
Substituting the root $x = \alpha$ into $g(x)$ and you get
%%
\begin{align*}
g(\alpha)
&=
\gamma (\alpha - \alpha) (\alpha - \beta) \\[4pt]
&=
\gamma \times 0 \times (\alpha - \beta) \\[4pt]
&=
0.
\end{align*}
%%
Substituting the root $x = \beta$ into $g(x)$ results in
%%
\begin{align*}
g(\alpha)
&=
\gamma (\beta - \alpha) (\beta - \beta) \\[4pt]
&=
\gamma (\beta - \alpha) \times 0 \\[4pt]
&=
0.
\end{align*}
%%
Therefore if $\gamma \neq 0$ is a real number, then the function
$g(x)$ as defined
in~\eqref{eqn:quadroots:all_quadratics_with_given_roots} will have the
roots $x = \alpha$ and $x = \beta$.
\end{solution}
}{}

\item\Exercise{ex:quadroots:quadratic_roots:given_roots_x1_x2} and
  \Problem{prob:quadroots:given_roots_all_possible_quadratics} show
  that if you are given the roots of a quadratic function $f(x)$, then
  there are many other quadratic functions that have the same roots.
  However, if you are given a third and different point on the graph
  of $f(x)$, then you have enough information to determine exactly one
  quadratic function whose graph passes through the given three
  points.
  %%
  \begin{packedenum}
  \item\label{subprob:quadroots:given_roots_x1_x5}
    Let $f(x)$ be a quadratic function whose roots are $x = 1$ and
    $x = -5$.  Show that $f(x) = (x - 1) (x + 5)$ and
    $g(x) = 3(x - 1) (x + 5)$ have the same roots.

  \item\label{subprob:quadroots:given_roots_x1_x5_point_2_7}
    Let $f(x)$ be a quadratic function that has the roots $x = 1$ and
    $x = -5$.  Further suppose that $\tuple{2}{7}$ is a point on the
    graph of $f(x)$.  Prove that there is exactly one quadratic
    function that has the points $\tuple{1}{0}$, $\tuple{-5}{0}$, and
    $\tuple{2}{7}$.
  \end{packedenum}
\ifbool{showSolution}{
\begin{solution}
\solutionpart{subprob:quadroots:given_roots_x1_x5}
To verify that $x = 1$ and $x = -5$ are the roots of $f(x)$, you
substitute the roots into $f(x)$ and simplify.  Substituting $x = 1$
into $f(x)$ results in $f(1) = (1 - 1) (1 + 5)$, which simplifies to
zero.  Furthermore, substituting $x = -5$ into $f(x)$ and you get
$f(-5) = (-5 - 1) (-5 + 5)$, which also simplifies to zero.  Now
perform the same substitution for $g(x)$.  Substituting $x = 1$ into
$g(x)$ produces $g(1) = 3(1 - 1) (1 + 5)$, which simplifies to zero.
Finally, substituting $x = -5$ into $g(x)$ yields
$g(-5) = 3(-5 - 1) (-5 + 5)$, which also simplifies to zero.  Conclude
that $f(x)$ and $g(x)$ have the same roots.

\solutionpart{subprob:quadroots:given_roots_x1_x5_point_2_7}
From \Part{subprob:quadroots:given_roots_x1_x5} you know that
$f(x) = (x - 1) (x + 5)$ has the roots $x = 1$ and $x = -5$.  Since
you are also given the point $\tuple{2}{7}$, you can substitute
$x = 2$ into $f(x)$ to get $f(2) = (2 - 1) (2 + 5)$, which simplifies
to $f(2) = 7$.  Therefore the graph of the function
$f(x) = (x - 1) (x + 5)$ passes through the points $\tuple{1}{0}$,
$\tuple{-5}{0}$, and $\tuple{2}{7}$.

Next, you prove that $f(x)$ is the only quadratic function whose graph
passes through the given three points.  To do so, you let $\alpha$ and
$\beta$ be any real numbers such that the graph of the function
%%
\begin{equation}
\label{eqn:quadroots:given_roots_x1_x5_point_2_7_another_quadratic}
h(x)
=
\alpha (x - 1) (x + 5) + \beta
\end{equation}
%%
also passes through the given three points.  That is, you assume that
$h(x)$ has the roots $x = 1$ and $x = -5$.  Now you show that $f(x)$
and $h(x)$ are the same.  To do so, you determine the values of
$\alpha$ and $\beta$.  Since $x = 1$ is a root of $h(x)$, then the
expression $h(1) = 0$ must be true.  You can write $h(1)$ as
\[
h(1)
=
\alpha (1 - 1) (1 + 5) + \beta
=
\alpha \times 0 + \beta
=
\beta.
\]
Since you must have $h(1) = 0$, then $h(1) = \beta = 0$ and so $\beta$
is zero.  The function $h(x)$ now simplifies to
\[
h(x)
=
\alpha (x - 1) (x + 5).
\]
You assumed above that the graph of $h(x)$ passes through the point
$\tuple{2}{7}$.  This means that the expression $h(2) = 7$ must be
true.  You can write $h(2)$ as
\[
h(2)
=
\alpha (2 - 1) (2 + 5)
=
7 \alpha.
\]
Since you must have $h(2) = 7$, then $h(2) = 7\alpha = 7$.  In the
expression $7\alpha = 7$, solving for $\alpha$ shows that
$\alpha = 1$.  In other words, if the function $h(x)$ as defined
in~\eqref{eqn:quadroots:given_roots_x1_x5_point_2_7_another_quadratic}
is another quadratic function whose graph passes through the given
three points, then you must have $\alpha = 1$ and $\beta = 0$.  In
that case, $h(x)$ simplifies to the function $f(x)$.  Conclude that
$f(x)$ is the only quadratic function that has the three given
points.
\end{solution}
}{}

\begin{figure}[!htbp]
\centering
\includegraphics[scale=1.1]{image/10/difference-two-squares.pdf}
\caption{%%
  Visual proof of the identity called the
  \emph{difference of two squares}.  A square is nested inside
  another, larger square.  The larger square has a side length of $x$,
  while the smaller square has a side length of $a$.
}
\label{fig:quadroots:difference_two_squares}
\end{figure}

\item\label{prob:quadroots:difference_of_two_squares}
  This problem will help you to derive another technique to factorise
  a quadratic function.  The technique is called the
  \emph{difference of two squares}.
  %%
  \begin{packedenum}
  \item\label{subprob:quadroots:difference_two_squares_areas_squares}
    \Figure{fig:quadroots:difference_two_squares} shows a square
    inside a larger square.  Calculate the area $X$ of the square
    whose side length is $x$.  Calculate the area $A$ of the square
    whose side length is $a$.  Calculate the difference $X - A$ of the
    two areas.

  \item\label{subprob:quadroots:difference_two_squares_area_shaded}
    Calculate the area of the shaded region in
    \Figure{fig:quadroots:difference_two_squares}.  Prove that the
    area of the shaded region is the same as the difference $X - A$
    from \Part{subprob:quadroots:difference_two_squares_areas_squares}.

  \item\label{subprob:quadroots:difference_two_squares_identity}
    Use \Part{subprob:quadroots:difference_two_squares_area_shaded} to
    prove that $x^2 - a^2 = (x - a) (x + a)$.  This equality is called
    the \emph{difference of two squares identity}.

  \item\label{subprob:quadroots:difference_two_squares_repeat_exercise}
    Use \Part{subprob:quadroots:difference_two_squares_identity} to
    determine the roots of the function from
    \Exercise{ex:quadroots:factorise_aminus1_c1}.

  \item\label{subprob:quadroots:difference_two_squares_a_cminus}
    Consider the function $f(x) = ax^2 - c$, where $a$ and $c$ are
    any positive real numbers.
    Use \Part{subprob:quadroots:difference_two_squares_identity} to
    determine the roots of $f(x)$.
  \end{packedenum}
\ifbool{showSolution}{
\begin{solution}
\solutionpart{subprob:quadroots:difference_two_squares_areas_squares}
In \Figure{fig:quadroots:difference_two_squares}, the square whose
side length is $x$ has an area of $x^2$ and the square whose side
length is $a$ has an area of $a^2$.  Then you have the difference
$x^2 - a^2$.

\solutionpart{subprob:quadroots:difference_two_squares_area_shaded}
The shaded region in \Figure{fig:quadroots:difference_two_squares} is
made up of two rectangles both of which have the same width and
height, and a square in the lower-right corner.  Any of the shaded
rectangles has a width and height of $a$ and $x - a$ and therefore an
area of $a(x - a)$.  Since the two shaded rectangles have the same
dimensions, then the total area of these rectangles is $2a(x - a)$.
The square in the lower-right corner has a side length of $x - a$ and
thus an area of $(x - a)^2$.  Then the total area of the shaded region
is
%%
\begin{equation}
\label{eqn:quadroots:difference_two_squares}
\begin{aligned}
(x - a)^2 + 2a(x - a)
&=
x^2 - 2ax + a^2 + 2ax - 2a^2 \\[4pt]
&=
x^2 + a^2 - 2a^2 \\[4pt]
&=
x^2 - a^2.
\end{aligned}
\end{equation}
%%
This is the same as the difference obtained
in \Part{subprob:quadroots:difference_two_squares_areas_squares}.

\solutionpart{subprob:quadroots:difference_two_squares_identity}
Using \Part{subprob:quadroots:difference_two_squares_area_shaded}, in
the expression $(x - a)^2 + 2a(x - a)$ you can factor out the
$(x - a)$ to get
%%
\begin{equation}
\label{eqn:quadroots:difference_two_squares_factorised}
\begin{aligned}
(x - a)^2 + 2a(x - a)
&=
(x - a) \bigsquare{(x - a) + 2a} \\[4pt]
&=
(x - a) (x - a + 2a) \\[4pt]
&=
(x - a) (x + a).
\end{aligned}
\end{equation}
%%
Equate
\Expressions{eqn:quadroots:difference_two_squares}{eqn:quadroots:difference_two_squares_factorised}
to see that $x^2 - a^2 = (x - a) (x + a)$.

\solutionpart{subprob:quadroots:difference_two_squares_repeat_exercise}
In \Exercise{ex:quadroots:factorise_aminus1_c1}, you have the
quadratic function $f(x) = -x^2 + 1$, which can also be written as
$f(x) = -(x^2 - 1) = -g(x)$, where $g(x) = x^2 - 1$.  Note that you
can write $g(x)$ as
%%
\begin{align*}
g(x)
&=
x^2 - 1 \\[4pt]
&=
x^2 - 1^2
\end{align*}
%%
which is the difference of two squares.
Use \Part{subprob:quadroots:difference_two_squares_identity} to write
$g(x)$ in factored form as $g(x) = (x - 1) (x + 1)$ and so $f(x)$ can
be factorised as
\[
f(x)
=
-(x - 1) (x + 1).
\]
If $x = 1$, then $f(1)$ simplifies to zero.  If $x = -1$, then $f(-1)$
also simplifies to zero.  Therefore the roots of $f(x)$ are $x = 1$
and $x = -1$.

\solutionpart{subprob:quadroots:difference_two_squares_a_cminus}
Factor out the $a$ to get
%%
\begin{align*}
f(x)
&=
a
\parenthesis*{
  x^2 - \frac{c}{a}
} \\[4pt]
&=
a \cdot g(x)
\end{align*}
%%
where $g(x) = x^2 - \frac{c}{a}$.  Note that both $a$ and $c$ are
positive numbers and so the ratio $c / a$ is also positive.  Then you
have
\[
\frac{c}{a}
=
\parenthesis*{
  \sqrt{
    \frac{c}{a}
  }
}^2
\]
which can be used to write $g(x)$ as
\[
g(x)
=
x^2
-
\parenthesis*{
  \sqrt{
    \frac{c}{a}
  }
}^2
\]
which is the difference of two squares.
Use \Part{subprob:quadroots:difference_two_squares_identity} to
factorise $f(x)$ as
\[
f(x)
=
a
\parenthesis*{
  x - \sqrt{\frac{c}{a}}
}
\parenthesis*{
  x + \sqrt{\frac{c}{a}}
}
\]
and therefore the roots of $f(x)$ are $x = \sqrt{\frac{c}{a}}$ and
$x = -\sqrt{\frac{c}{a}}$.
\end{solution}
}{}

\item You can use the difference of two squares identity from
  \Problem{prob:quadroots:difference_of_two_squares} to help you
  multiply two integers.
  %%
  \begin{packedenum}
  \item\label{subprob:quadroots:difference_two_squares_7_13}
    Let's calculate the product $7 \times 13$.  You know that the
    result is $91$.  But let's do the multiplication another way.
    Show that you have the equality $7 \times 13 = (10 - 3) (10 + 3)$.
    Use the latter equality to write $7 \times 13$ as the difference
    of two squares and calculate the result of this difference.

  \item\label{subprob:quadroots:difference_two_squares_12_18}
    Use the difference of two squares identity to calculate
    $12 \times 18$.
  \end{packedenum}
\ifbool{showSolution}{
\begin{solution}
\solutionpart{subprob:quadroots:difference_two_squares_7_13}
You have $10 - 3 = 7$ and $10 + 3 = 13$ and so
\[
7 \times 13
=
(10 - 3) (10 + 3).
\]
Use
\Subproblem{prob:quadroots:difference_of_two_squares}{subprob:quadroots:difference_two_squares_identity}
to write
%%
\begin{align*}
7 \times 13
&=
(10 - 3) (10 + 3) \\[4pt]
&=
10^2 - 3^2 \\[4pt]
&=
100 - 9 \\[4pt]
&=
91.
\end{align*}

\solutionpart{subprob:quadroots:difference_two_squares_12_18}
You have $15 - 3 = 12$ and $15 + 3 = 18$.  Then the product
$12 \times 18$ can be written as the difference of two squares:
%%
\begin{align*}
12 \times 18
&=
(15 - 3) (15 + 3) \\[4pt]
&=
15^2 - 3^2.
\end{align*}
%%
The difference can be written as $225 - 9 = 216$.
\end{solution}
}{}

\item Let $n$ be any integer.
  %%
  \begin{packedenum}
  \item\label{subprob:quadroots:difference_two_consecutive_squares}
    Use \Problem{prob:quadroots:difference_of_two_squares} to prove
    that $(n + 1)^2 - n^2$ is an odd integer.

  \item\label{subprob:quadroots:difference_two_integer_squares_odd}
    If $k$ is an odd integer, use
    \Problem{prob:quadroots:difference_of_two_squares} to prove that
    $(n + k)^2 - n^2$ is odd.

  \item\label{subprob:quadroots:difference_two_integer_squares_even}
    If $k$ is even, prove that $(n + k)^2 - n^2$ is even.
  \end{packedenum}
\ifbool{showSolution}{
\begin{solution}
\solutionpart{subprob:quadroots:difference_two_consecutive_squares}
The expression $(n + 1)^2 - n^2$ is a difference of two squares.  Use
\Subproblem{prob:quadroots:difference_of_two_squares}{subprob:quadroots:difference_two_squares_identity}
to write the latter expression as
%%
\begin{align*}
(n + 1)^2 - n^2
&=
\bigsquare{(n + 1) - n}
\bigsquare{(n + 1) + n} \\[4pt]
&=
(n + 1 - n) (n + 1 + n) \\[4pt]
&=
(0 + 1) (2n + 1) \\[4pt]
&=
2n + 1.
\end{align*}
%%
The integer $2n$ is even so $2n + 1$ is odd.  Therefore
$(n + 1)^2 - n^2$ is odd.

\solutionpart{subprob:quadroots:difference_two_integer_squares_odd}
The expression $(n + k)^2 - n^2$ is a difference of two squares.  Use
\Subproblem{prob:quadroots:difference_of_two_squares}{subprob:quadroots:difference_two_squares_identity}
to write
%%
\begin{align*}
(n + k)^2 - n^2
&=
\bigsquare{(n + k) - n}
\bigsquare{(n + k) + n} \\[4pt]
&=
(n + k - n) (n + k + n) \\[4pt]
&=
k (2n + k) \\[4pt]
&=
2kn + k^2.
\end{align*}
%%
Since $k$ is odd, then $k$ can be written as $k = 2p + 1$, where $p$
is an integer.  Thus the square $k^2$ can be written as
%%
\begin{align*}
k^2
&=
(2p + 1)^2 \\[4pt]
&=
(2p + 1) (2p + 1) \\[4pt]
&=
2p(2p + 1) + 1(2p + 1) \\[4pt]
&=
2p(2p + 1) + 2p + 1.
\end{align*}
%%
Each of the integers $2p(2p + 1)$ and $2p$ is even and so $k^2$ is
odd.  The expression $2kn + k^2$ is the sum of an even integer and an
odd integer, hence the result is odd.  Therefore $(n + k)^2 - n^2$ is
odd.

\solutionpart{subprob:quadroots:difference_two_integer_squares_even}
The expression $(n + k)^2 - n^2$ is the difference of two squares.
Use
\Subproblem{prob:quadroots:difference_of_two_squares}{subprob:quadroots:difference_two_squares_identity}
to write the latter expression as
%%
\begin{align*}
(n + k)^2 - n^2
&=
\bigsquare{(n + k) - n}
\bigsquare{(n + k) + n} \\[4pt]
&=
(n + k - n) (n + k + n) \\[4pt]
&=
k (2n + k).
\end{align*}
%%
Since $k$ is an even integer, then the integer $k (2n + k)$ is also
even.  Therefore the integer $(n + k)^2 - n^2$ is even.
\end{solution}
}{}

\item Let $f(x) = ax^2 + bx + c$ be a quadratic function with
  $\triple{a}{b}{c} \in \RR$ being fixed numbers such that
  $a \neq 0$.  If $\alpha$ and $\beta$ are roots of $f(x)$, show that
  the vertex of $f(x)$ has $x$-coordinate $\frac{\alpha + \beta}{2}$.
\ifbool{showSolution}{
\begin{solution}
Let $\alpha$ and $\beta$ be roots of $f(x)$.  Suppose that the roots
can be written as
\[
\alpha
=
\frac{-b + \sqrt{\Delta}}{2a}
%%
\qquad
\text{and}
\qquad
%%
\beta
=
\frac{-b - \sqrt{\Delta}}{2a}
\]
where the discriminant is $\Delta = b^2 - 4ac$.  The sum of the roots
is
%%
\begin{align*}
\alpha + \beta
&=
\frac{-b + \sqrt{\Delta}}{2a}
+
\frac{-b - \sqrt{\Delta}}{2a} \\[4pt]
&=
\frac{
  -b + \sqrt{\Delta} - b - \sqrt{\Delta}
}{
  2a
} \\[4pt]
&=
\frac{
  -2b
}{
  2a
} \\[4pt]
&=
-
\frac{
  b
}{
  a
}.
\end{align*}
%%
Divide both sides by $2$ to see that
$\frac{\alpha + \beta}{2} = -\frac{b}{2a}$, which is the
$x$-coordinate of the vertex of $f(x)$.
\end{solution}
}{}

\item You will derive the quadratic formula via a technique that was
  known to Hindu mathematicians since around the year {\sc ad}~1025.
  % \footnote{
  %   See the following paper for further details:
  %   \url{http://www.jstor.org/stable/27965986}.
  % }
  Consider a quadratic function $f(x) = ax^2 + bx + c$, where $a$,
  $b$, and $c$ are constants such that $a \neq 0$.
  %%
  \begin{packedenum}
  \item\label{subprob:quadroots:quadratic_roots:formula_Hindu_perfect_square}
    Show that the expression $4a^2x^2 + 4abx + b^2$ can be written as
    the perfect square $(2ax + b)^2$.

  \item\label{subprob:quadroots:quadratic_roots:formula_Hindu_multiply_4}
    Multiply both sides of the equation $ax^2 + bx + c = 0$ by $4a$
    and show that the resulting equation can be written as
    $(2ax + b)^2 = b^2 - 4ac$.  Now solve the latter equation for
    $x$.
  \end{packedenum}
\ifbool{showSolution}{
\begin{solution}
\solutionpart{subprob:quadroots:quadratic_roots:formula_Hindu_perfect_square}
You can write $(2ax + b)^2$ as
%%
\begin{align*}
(2ax + b)^2
&=
(2ax + b) (2ax + b) \\[4pt]
&=
2ax(2ax + b) + b(2ax + b) \\[4pt]
&=
4a^2x^2 + 2abx + 2abx + b^2 \\[4pt]
&=
4a^2x^2 + 4abx + b^2
\end{align*}
%%
which shows that $4a^2x^2 + 4abx + b^2$ is a perfect square.

\solutionpart{subprob:quadroots:quadratic_roots:formula_Hindu_multiply_4}
Multiply the equation $ax^2 + bx + c = 0$ through by $4a$ to obtain
the equation $4a^2x^2 + 4abx + 4ac = 0$.  Subtract $4ac$ from both
sides to get $4a^2x^2 + 4abx = -4ac$ and add $b^2$ to both sides and
you have
\[
4a^2x^2 + 4abx + b^2
=
b^2 - 4ac.
\]
By \Part{subprob:quadroots:quadratic_roots:formula_Hindu_perfect_square},
you know that the left-hand side of the last equation is a perfect
square.  That is, you can write the latter equation as
$(2ax + b)^2 = b^2 - 4ac$.  Take the square root of both sides and you
have $2ax + b = \pm \sqrt{b^2 - 4ac}$, which can also be written as
$2ax = -b \pm \sqrt{b^2 - 4ac}$.  Now solve for $x$ to get the
quadratic formula.
\end{solution}
}{}

\item This problem will help you to derive~(again!) the quadratic
  formula by using a substitution technique due to Fran\c{c}ois
  Vi\`ete.
  % \footnote{
  %   See the following paper for further details:
  %   \url{http://www.jstor.org/stable/27965986}.
  % }
  %%
  \begin{packedenum}
  \item\label{subprob:quadroots:quadratic_roots:Viete_ax_square_minus_c}
    Consider the function $f(x) = ax^2 - c$, where $a$ and $c$ are any
    real numbers such that $a \neq 0$ and $c > 0$.  Determine the
    roots of $f(x)$ without using the quadratic formula, completing
    the square, or factorisation.

  \item\label{subprob:quadroots:quadratic_roots:Viete_substitution}
    Let $f(x) = ax^2 + bx + c$ be a quadratic function with
    $\triple{a}{b}{c} \in \RR$ being constants such that $a \neq 0$.
    Let $y$ be a real variable.  Use the substitution
    %%
    \begin{equation}
    \label{eqn:quadroots:quadratic_roots:Viete_substitution}
    x
    =
    y - \frac{b}{2a}
    \end{equation}
    %%
    to show that $f(x)$ can be written as
    %%
    \begin{equation}
    \label{eqn:quadroots:quadratic_roots:Viete_substitution_expand}
    g(y)
    =
    ay^2
    +
    \frac{-b^2 + 4ac}{4a}.
    \end{equation}

  \item\label{subprob:quadroots:quadratic_roots:Viete_quadratic_formula}
    Show that the quadratic function $g(y)$ as defined by
    \Expression{eqn:quadroots:quadratic_roots:Viete_substitution_expand}
    has the roots
    %%
    \begin{equation}
    \label{eqn:quadroots:quadratic_roots:Viete_quadratic_formula}
    y
    =
    \pm
    \frac{
      \sqrt{b^2 - 4ac}
    }{
      2a
    }.
    \end{equation}

  \item\label{subprob:quadroots:quadratic_roots:Viete_quadratic_formula_verify}
    From \Example{eg:quadroots:completing_square_a3_bminus7_c2}, you
    know that the function $f(x) = 3x^2 - 7x + 2$ has the roots
    $x_1 = 2$ and $x_2 = 1/3$.  Use the
    substitution~\eqref{eqn:quadroots:quadratic_roots:Viete_substitution}
    and
    \Expression{eqn:quadroots:quadratic_roots:Viete_quadratic_formula}
    to verify the given roots of $f(x)$.
  \end{packedenum}
\ifbool{showSolution}{
\begin{solution}
\solutionpart{subprob:quadroots:quadratic_roots:Viete_ax_square_minus_c}
The roots of $f(x)$ are all values of $x$ such that the expression
$f(x) = 0$ is true.  That is, you want all values of $x$ such that the
expression $ax^2 - c = 0$ is true.  The last expression can be written
as $ax^2 = c$, which in turn can be written as $x^2 = c / a$.  Take
the square root of both sides to get $x = \pm \sqrt{\frac{c}{a}}$.
Therefore the roots of $f(x)$ are
\[
x_1
=
\sqrt{\frac{c}{a}}
%%
\qquad
\text{and}
\qquad
%%
x_2
=
-\sqrt{\frac{c}{a}}.
\]

\solutionpart{subprob:quadroots:quadratic_roots:Viete_substitution}
By using the substitution $x = y - \frac{b}{2a}$, you can write $f(x)$
as
%%
\begin{align*}
a \parenthesis*{y - \frac{b}{2a}}^2
+
b \parenthesis*{y - \frac{b}{2a}}
+
c
&=
a \parenthesis*{y^2 - \frac{by}{a} + \frac{b^2}{4a^2}}
+
by - \frac{b^2}{2a}
+
c \\[4pt]
&=
ay^2 - by + \frac{b^2}{4a}
+
by - \frac{b^2}{2a}
+
c \\[4pt]
&=
ay^2 + \frac{b^2}{4a} - \frac{b^2}{2a} + c \\[4pt]
&=
ay^2
+
\frac{b^2 - 2b^2 + 4ac}{4a} \\[4pt]
&=
ay^2
+
\frac{-b^2 + 4ac}{4a}.
\end{align*}

\solutionpart{subprob:quadroots:quadratic_roots:Viete_quadratic_formula}
You want to solve the equation
\[
ay^2
+
\frac{-b^2 + 4ac}{4a}
=
0
\]
for $y$.  The latter equation can be written as
%%
\begin{align*}
ay^2
&=
-\frac{-b^2 + 4ac}{4a} \\[4pt]
&=
\frac{b^2 - 4ac}{4a}.
\end{align*}
%%
Divide both sides by $a$ to obtain the equivalent expression
\[
y^2
=
\frac{b^2 - 4ac}{4a^2}.
\]
Take the square root of both sides and you have
%%
\begin{align*}
y
&=
\pm\sqrt{
  \frac{
    b^2 - 4ac
  }{
    4a^2
  }
} \\[4pt]
&=
\pm
\frac{
  \sqrt{b^2 - 4ac}
}{
  2a
}
\end{align*}
%%
as required.

\solutionpart{subprob:quadroots:quadratic_roots:Viete_quadratic_formula_verify}
Use \Equation{eqn:quadroots:quadratic_roots:Viete_quadratic_formula}
to write
\[
y
=
\pm
\frac{
  \sqrt{(-7)^2 - 4 \times 3 \times 2}
}{
  2 \times 3
}
=
\pm\frac{5}{6}.
\]
Now use \Equation{eqn:quadroots:quadratic_roots:Viete_substitution} to
write
\[
x
=
\pm\frac{5}{6}
-
\frac{-7}{2 \times 3}
=
\frac{7 \pm 5}{6}.
\]
Hence one root of $f(x)$ is $x_1 = \frac{7 + 5}{6} = 2$ and the other
root is $x_2 = \frac{7 - 5}{6} = 1/3$.  These are the same as in
\Example{eg:quadroots:completing_square_a3_bminus7_c2}.
\end{solution}
}{}

\item Let $n$ be a positive integer such that $n \geq 2$.
  %%
  \begin{packedenum}
  \item\label{subprob:quadroots:difference_two_squares_reciprocal_product_k}
    If $k \geq 2$ is an integer, show that
    \[
    1 - \frac{1}{k^2}
    =
    \frac{k - 1}{k}
    \times
    \frac{k + 1}{k}.
    \]

  \item\label{subprob:quadroots:difference_two_squares_reciprocal_products}
    Prove that
    \[
    \productseq{
      \parenthesis*{1 - \frac{1}{2^2}}
      \parenthesis*{1 - \frac{1}{3^2}}
    }{
      \parenthesis*{1 - \frac{1}{4^2}}
    }{
      \parenthesis*{1 - \frac{1}{n^2}}
    }
    =
    \frac{n + 1}{2n}.
    \]
  \end{packedenum}
\ifbool{showSolution}{
\begin{solution}
\solutionpart{subprob:quadroots:difference_two_squares_reciprocal_product_k}
You can write
\[
1 - \frac{1}{k^2}
=
1 - \parenthesis*{\frac{1}{k}}^2
\]
which is the difference of two squares.  Use
\Subproblem{prob:quadroots:difference_of_two_squares}{subprob:quadroots:difference_two_squares_identity}
to write the latter expression as
%%
\begin{align*}
1 - \parenthesis*{\frac{1}{k}}^2
&=
\parenthesis*{1 - \frac{1}{k}}
\parenthesis*{1 + \frac{1}{k}} \\[4pt]
&=
\parenthesis*{\frac{k}{k} - \frac{1}{k}}
\parenthesis*{\frac{k}{k} + \frac{1}{k}} \\[4pt]
&=
\frac{k - 1}{k}
\times
\frac{k + 1}{k}
\end{align*}
%%
as required.

\solutionpart{subprob:quadroots:difference_two_squares_reciprocal_products}
Use \Part{subprob:quadroots:difference_two_squares_reciprocal_product_k}
to write
\[
1 - \frac{1}{2^2}
=
\frac{1}{2} \cdot \frac{3}{2},
%%
\qquad
%%
1 - \frac{1}{3^2}
=
\frac{2}{3} \cdot \frac{4}{3},
%%
\qquad
%%
1 - \frac{1}{4^2}
=
\frac{3}{4} \cdot \frac{5}{4}
\]
and in general you have
\[
1 - \frac{1}{(n - 1)^2}
=
\frac{n - 2}{n - 1} \cdot \frac{n}{n - 1},
%%
\qquad
%%
1 - \frac{1}{n^2}
=
\frac{n - 1}{n} \cdot \frac{n + 1}{n}.
\]
Thus you have the expression
%%
\begin{align*}
&\productseq{
  \parenthesis*{1 - \frac{1}{2^2}}
  \parenthesis*{1 - \frac{1}{3^2}}
}{
  \parenthesis*{1 - \frac{1}{4^2}}
}{
  \parenthesis*{1 - \frac{1}{n^2}}
} \\[4pt]
&=
\frac{1}{2} \cdot \frac{3}{2}
\cdot
\frac{2}{3} \cdot \frac{4}{3}
\cdot
\frac{3}{4} \cdot \frac{5}{4}
\cdots
\frac{n - 2}{n - 1} \cdot \frac{n}{n - 1}
\cdot
\frac{n - 1}{n} \cdot \frac{n + 1}{n}.
\end{align*}
%%
Note that all the numerators and denominators cancel out, except for
the factor $\frac{1}{2} \times \frac{n + 1}{n} = \frac{n + 1}{2n}$.
\end{solution}
}{}
\end{problem}

\end{document}
